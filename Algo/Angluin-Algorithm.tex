\subsection{Angluin Algorithm: Query Learning}

In this section we are interested in the Angluin algorithm (1992). 
\cite{angluin_learning_1992, arias_canonical_2009}.

\subsubsection{Query Learning}

Here, the method for building a minimal base is slightly different. We use 
so-called \belemp{query learning}. The idea is we formulate \belemp{queries}
to an \belemp{oracle} knowing the basis we are trying to learn. The oracle 
is assumed to provide an answer to our query in constant time. Depending on 
the query, it might also provide informations on the object we are looking for.
For the Angluin algorithm, we need 2 types of queries. Say we want to learn
a basis $\I$ over $\Sg$:
\begin{enumerate}
	\item \belemp{membership} query: is $M \subseteq \Sg$ a model of $\I$? The
	oracle may answer "yes", or "no".
	\item \belemp{equivalence} query: is a basis $\I'$ equivalent to $\I$? Again
	the answers are "yes", or "no". In the second case, the oracle provides a
	\belemp{counterexample} either positive or negative:
		\begin{itemize}
			\item[(i)] \belemp{positive}: a model $M$ of $\I$ which is not a
			model of $\I'$,
			\item[(ii)] \belemp{negative}: a non-model $M$ of $\I$ being a model
			of $\I'$. 
		\end{itemize}
\end{enumerate}
\noindent To clarify, the terms negative/positive are related to the base $\I$
we want to learn.

\subsubsection{Algorithm}

The algorithm has been proved to end up with the Duquenne-Guigue Basis, again 
we can refer to \cite{angluin_learning_1992, arias_canonical_2009} for further
explanations. The first algorithm provided by Angluin et al. relies on clauses
with unitary heads. It uses two operations allowing to reduce implications:
\begin{itemize}
	\item $refine(A \imp B, M)$: produces $M \imp \Sg$ if $B = \Sg$, 
	$M \imp B \cup A - M$ otherwise,
	\item $reduce(A \imp B, M)$: produces $A \imp M - A$ if $B = \Sg$, 
	$A \imp B \cap M$ otherwise.
\end{itemize}

\begin{algorithm}
\KwIn{$\I$}
\KwOut{$\I_c$}

\BlankLine
\BlankLine

$\I_c = \emptyset$ \;
\While{not equivalence($\I_c$)}{
	$M$ is the counterexample \;
	\If{$M$ is positive}{
		\ForEach{$A \imp B \in \I_c$ such that $M \not\models A \imp B$}{
			replace $A \imp B$ by $reduce(A \imp B, M)$ \;	
		}

	} \Else {
		\ForEach{$A \imp B \in \I_c$ such that $A \cap M \subset A$}{
			 membership($M \cap A$) \;	
		}
	
		\If{Oracle replied "no" for at least one $A \imp B$}{
			Take the first such $A \imp B$ in $\I_c$ \;	
			replace $A \imp B$ by $refine(A \imp B, A \cap M)$ \;
			
		} \Else {
			add $M \imp \Sg$ to $\I_c$ \;
			
		}

	}

}
return $\I_c$ \;
	
\caption{Angluin Algorithm}
\label{alg:Angluin}
\end{algorithm}

\paragraph{Some Observations} \begin{itemize}
	\item Proved to be polynomial (\cite{angluin_learning_1992})
	\item Proved to end up on the DG basis (\cite{arias_canonical_2009})
	\item We can see that (apart from previous remark) that we end up 
	on implications of the form $A \imp \I(A)$ (induction)
	\item .
\end{itemize}



