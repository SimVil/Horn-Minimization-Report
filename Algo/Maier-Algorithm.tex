\section{Minimum Covers in the Relational Database Model}


This part relies mainly on \cite{FD}, \cite{Maier1980} in which an algorithm for
minimizing set of functional dependencies (FD). Because FDs behave as
implications in implicational basis, we will adopt the latter terminology.

\subsection{Notations}

We consider being somehow familiar with the next notations, so that we don't 
spend time on definitions.

\begin{itemize}
	\item $\Sg$ is a set of attributes $a$, $b$, $c$, $\dots$
	\item $\I$ is a set of implications over $\Sg$
	\item given $X \subseteq \Sg$, $\I(X)$ is the closure of $X$ under $\I$
	\item 
\end{itemize}

\subsection{Definitions}

Before diving into the algorithm, we may need some definitions. Indeed, the 
minimization procedure in this case relies on a criterion for minimality using
specific objects. 

\begin{definition}[Equivalent sets] Two subsets $E, F \subseteq \Sigma$ are 
	\belemp{equivalent under $\I$} if $\I \models E \imp F$ and $\I \models F 
	\imp 
	E$. In other words, two sets are equivalent if $\I(E) = \I(F)$. It is 
	denoted 
	by $E \equiv F$. 
	
\end{definition}

\begin{definition}[Equivalence classes] Let $X \subseteq \Sigma$. The set
	
	\[ E_{\I}(X) = \{ A \imp B | A \imp B \in \I, A \equiv X \} \]
	
	\noindent is the \belemp{equivalence class} of $X$ under $\I$. In fact, it 
	is
	the set of implications of $\I$ with body equivalent to $X$. The set of all 
	non-empty such classes is denoted $\bar{E}_{\I}$.
	
\end{definition}

One should note in order to get $\bar{E}_{\I}$, there is no need to check all
the subsets of $\Sigma$. Because $\bar{E}_{\I}$ defines a partition of $\I$ 
based on implication bodies, it is sufficient to go over implications of $\I$.

\vspace{1.2em}

Next we focus on particular implications


\subsection{Algorithm}

The algorithm relies on a criterion for minimality given equivalence classes. 
The set 



\begin{algorithm}[H]
	\KwIn{An implicational basis $\I$}
	\KwOut{$\I$ with redundant implications removed}
	
	\Begin{
		\ForEach{$A \imp B \in \I$}{
			$\I^{-} := \I - \{ A \imp B \}$ \;
			\If{$\I^{-}(A) = \I(A)$}{
				$\I := \I - \{ A \imp B \}$ \;
			}
		}
	}
	
	\caption{Non-Redundant}
	\label{alg:Nonredun}
\end{algorithm}

\vspace{1.2em}

\begin{algorithm}[H]
	\KwIn{A nonredundant implicational basis $\I$}
	\KwOut{set of equivalence classes $\bar{E}_{\I}$}
	
	\Begin{
		return \;
	}	
	
	\caption{Equivalence classes}
	\label{alg:Equiv}
	
\end{algorithm}

\begin{algorithm}[H]
	\KwIn{$\phi$ an implicational basis}
	\KwOut{$\psi$ a body-minimal representation of $\phi$}
	
	\Begin{
		$\psi := \emptyset$ \;
		\While{$\exists X \in \mathcal{B}(\phi): \; \phi(X) \neq \psi(X)$}{
			$A := \min(\psi(X): \; X \in \mathcal{B}(\phi) \land 
			\phi(X) \neq \psi(X))$ \;
			$B := \phi(X)$ \;
			$\psi := \psi \cup \{ A \imp B - A \}$ \;
			
		}
		
		return $\psi$ \;	
	}
	
	\caption{BodyMinimal}
\end{algorithm}	


\subsection{Elements of proofs}

In this section we write various claims which help us prove correctness and 
equivalence of Maier/Ausiello's algorithms. Knowledge about FD-graphs is 
assumed. Equivalence of redundancy elimination is not shown (1st step of 
the algorithm). We place ourselves in a non-redundant context to match
definitions of direct determination. Non-redundancy does not affect superfluous
definition.

\begin{claim}
	A node $i$ in a FD-graph is superfluous with respect to $j$ if and only if 
	$i \equiv j$ and there exists a proper subset (non-empty) $k$ of $i$ such 
	that $\I \models k \imp j$. 
\end{claim}

\begin{proof} This is a translation of the definition of superfluous node in 
	terms of implications.	
\end{proof}

\begin{claim} the following statements are equivalent, for $A, B$ bodies of
	$\I$:
	\begin{itemize}
		\item[(i)] $A \ddv B$,
		\item[(ii)] the node $A$ is superfluous with respect to $B$, and there 
		exists 
		a dotted FD-path from $A$ to $B$ not using any outgoing full arcs of
		nodes equivalent to $A$.
	\end{itemize}
	
\end{claim}

\begin{proof} (i) $\imp$ (ii). We will use claim 1 to help us. Suppose $ A \ddv 
	B$. By 
	definition of direct determination, we have that the nodes $A$ and $B$ are 
	equivalent. By hypothesis, we have $ \I^{*} = \I - E_{\I}(A) \models A \imp 
	B$.
	We will distinguish two cases, either $B \subseteq A$, or $B \nsubseteq A$.
	
	\vspace{1.2em}
	
	\paragraph{$B \subseteq A$.} In this case, direct determination is straight 
	forward, and if $A$ and $B$ are bodies of $\I$, there exists dotted arcs 
	between
	$A$ and $B$ forming a dotted FD-path. Then $A$ is indeed superfluous with 
	respect to $B$. Furthermore, because we used only dotted arcs from $A$, we 
	did 
	not use any full arcs outgoing from $A$.
	
	\paragraph{$B \nsubseteq A$.} Since we don't use any implications with left 
	side
	equivalent to $A$ in direct determination, but still have $A \imp B$, we 
	must 
	be able to find a non-equivalent proper subset $X$ of $A$ such that $X \imp 
	B$,
	otherwise we would contradict direct determination (because we would not 
	have
	$B \subseteq \I^{~}(A)$ ). Using claim 1, we can conclude that $A$ is 
	indeed 
	superfluous with respect to $B$. Moreover, notice that using an outgoing 
	full 
	arc from a node equivalent $C$ to $A$ is exactly using an implication with 
	left
	hand side equivalent to $A$. Therefore, if there is not dotted FD-path from 
	$A$
	to $B$ not using those arcs, we would contradict direct determination.
	
	\vspace{1.2em}
	
	(ii) $\imp$ (i) Suppose $A$ is superfluous with respect to $B$ and there 
	exists
	a dotted FD-path from $A$ to $B$ not using any outgoing full arcs from 
	nodes 
	equivalent to $A$. Those full arcs represent exactly the implications 
	contained
	in $E_{\I}(A)$. Since we don't use them, the path still holds in $\I^{*}$ 
	(we 
	would remove compound nodes without outgoing full arcs of course, but this 
	would
	only make the path stops to attributes instead of compound node). Having 
	this 
	path in $\I^{~}$ means that $\I^{*} \models A \imp B$.
	
	
\end{proof} 

\begin{claim} The following two statement are equivalent, given the FD-graph of
	$\I$:
	\begin{itemize}
		\item[(i)] $A$ is a superfluous node,
		\item[(ii)] $A$ is superfluous with respect to $B$, and there exists a
		dotted path from $A$ to $B$ not using any outgoing full arcs from nodes
		equivalent to $A$.
	\end{itemize}
	
\end{claim}

\begin{proof} (ii) $\imp$ (i) is trivial. let's focus on (i) $\imp$ (ii). If 
	$A$ is superfluous, then there exists $B$ such that $A \equiv B$ and there 
	has a dotted path from $A$ to $B$.
	
	\vspace{1.2em}
	
	If $B \subseteq A$, the dotted path is straight forward. If it is not the 
	case,
	path from $A$ to $B$ uses outgoing full arcs from nodes equivalent to $A$, 
	or
	it does not. If it does not we are done. Now Suppose this
	path uses an outgoing full arc from a node $C$ equivalent to $A$. This 
	means,
	that $A \equiv B \equiv C$ by definition and therefore, there exists a
	dotted path from $A$ to $C$ (because we need to reach $C$, so to derive it,
	to use its outgoing arcs). We can reiterate this reasoning until reaching 
	$A$.
	Getting to $A$ or one of its subset would contradict our assumptions 
	meaning 
	that we stopped finding used equivalent arcs earlier.
	
\end{proof}

From the previous claims, we can yield the following one

\begin{claim} The following states are equivalent:
	\begin{itemize}
		\item[(i)] $A \ddv B$,
		\item[(ii)] $A$ is superfluous,
		\item[(iii)] $A \equiv B$ and there exists $X \subset A$ such that $\I 
		\models X \imp B$.
	\end{itemize}
\end{claim}

\begin{proof} (i) $\longleftrightarrow$ (ii) comes from claim 2 and 3. 
	(ii) $\longleftrightarrow$ (iii) comes from claim 1.
	
\end{proof}

Those claims help to see the relation between operations of the algorithms. 
Indeed, the last claim states that finding a direct determination is the same
as finding a superfluous node, therefore the two algorithms are looking for
the same structures in different terminologies. This emphasizes the fact they
work on the same computations. Remark: Maier algorithm do it in a special 
order (some kinds of minimal paths in terms of FD-graph. Can we remove them in 
any order? It seems to since this is what the Ausiello algorithm does.

\vspace{1.2em}

The next claim provides an argument in this way. It also explains in what 
extent the operation of removal in those algorithm is exact.

\begin{claim} Let $A, B$ be bodies of $\I$. If $A \equiv B$ and
	there exists $X \subset A$ such that $\I \models X \imp B$ then $\I$ is not
	minimum.
	
\end{claim}

\begin{proof} We can remove $A \imp C$ and replace $B \imp D$ by $B \imp CD$
	while keeping the same closure system.	
\end{proof}

