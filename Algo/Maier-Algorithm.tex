\subsection{Maier Algorithm: using equivalence classes}


This part relies mainly on \cite{david_minimum_1980}, \cite{maier_theory_1983} 
in which an algorithm for minimizing set of functional dependencies (FD). 
Because FDs behave as implications in implicational basis, we will adopt the 
latter terminology.

\subsubsection{Theoretical set up and definitions}

We are going to work in the FD framework, that is first notations we defined. 
Before diving into the algorithm, we may need some definitions. Indeed, the 
minimization procedure in this case relies on a criterion for minimality using
specific objects. 

\begin{definition}[\midn{Equivalence classes}] Let $X \subseteq \Sigma$. The set
	
	\[ E_{\I}(X) = \{ A \imp B | A \imp B \in \I, A \equiv X \} \]
	
\noindent is the \belemp{equivalence class} of $X$ under $\I$. In fact, it 
is the set of implications of $\I$ with body equivalent to $X$. The set of all 
non-empty such classes is denoted $\bar{E}_{\I}$.
	
\end{definition}

One should note in order to get $\bar{E}_{\I}$, there is no need to check all
the subsets of $\Sigma$. Because $\bar{E}_{\I}$ defines a partition of $\I$ 
based on implication bodies, it is sufficient to go over implications of $\I$.

\vspace{1.2em}

The second tool used by Maier in his algorithm is \belemp{direct determination}.
This notion is extensively discussed in \cite{david_minimum_1980, 
maier_theory_1983}. Thus, the definition we are going to define is more the 
result of a very useful necessary and sufficient condition than the definition
as given at the beginning. Our aim is not to (re)build all the material, but to 
explain the algorithm.

\begin{definition}[\midn{Direct Determination}] Given a basis $\I$, we say that
$A$ \belemp{directly determines} $B$ under $\I$, denoted $A \ddv B$, if $\I -
\bar{E}_{\I} \models A \imp B$. 
	
\end{definition}

More intuitively, we say that $A$ directly determines $B$ if we can reach $B$
without using any implications with body equivalent to $A$ (including $A$). 
Those definitions are sufficient to understand the algorithm. 

\subsubsection{Algorithm}

In this section we will investigate the Maier Algorithm to minimize a basis 
$\I$. The principle is short enough to describe it as follows:
\begin{enumerate}
	\item remove redundant implications
	\item remove direct determinations in this sens: if $A \ddv B$ with $A \imp 
	C$ and $B \imp D$ implications of $\I$, remove $A \imp C$ and replace by 
	$B \imp D$ by $B \imp D \cup C$. 
\end{enumerate}
\noindent Of course, proof of correctness has been given in cite 
\cite{david_minimum_1980, maier_theory_1983}. Nevertheless, we shall give later
some different elements for proving the algorithm ends with the right result. 
We do not give them here, because those elements also provide a proof for an
another algorithm (namely Ausiello and al. algorithm) through equivalence 
properties. For now, we send the reader to cited papers. The procedure is 
presented in algorithm \ref{alg:Maier-Algorithm}.


\begin{algorithm}
\KwIn{}
\KwOut{}

\BlankLine
\BlankLine

\emeemp{// redundancy elimination } \\
\ForEach{$A \imp B \in \I$}{
	$\I^{-} := \I - \{ A \imp B \}$ \;
	\If{$\I^{-}(A) = \I(A)$}{
		$\I := \I - \{ A \imp B \}$ \;
	}
}

$\bar{E}_{\I} := EquivClasses(\I)$ \;

\BlankLine

\emeemp{// direct determination elimination} \\

\ForEach{$E_{\I} \in \bar{E}_{\I}$}{
	\ForEach{$A \imp C \in E_{\I}$}{
		\If{$\exists B \imp D \in E_{\I}$ such that $A \ddv B$}{
			remove $A \imp B$ from $\I$ \;
			replace $B \imp D$ by $B \imp C \cup D$ \;	
		}
	}
}

\caption{Maier minimization algorithm}
\label{alg:Maier-Algorithm}
\end{algorithm}

\vspace{1.2em}

\paragraph{Observations} There are some observations and questions to answer
about procedure \ref{alg:Maier-Algorithm}:
\begin{itemize}
	\item How do we compute $\bar{E}_{\I}$?
	\item How do we check for direct determination?
	\item Does the operation of remove and replace alter an equivalence class 
	more than removing one of its implications?
\end{itemize}  
\noindent Let us take those questions in order. To find $\bar{E}_{\I}$, we 
will consider applying a variation of \textsc{LinClosure} to each implication
of $\I$. Thus for each $A \imp B$, instead of returning closure of $A$ under
$\I$, we will a bit vector of size $|\body{\I}|$ with the $i$-th entry being
$1$ if the body of this implication is in the closure of $A$, and $0$ otherwise.
the $i$-th bit of the vector is set if \textit{count}[$i$-th implication] = 0 in
the procedure. Running this modified version of \textsc{LinClosure} over all
implications produces a $|\body{\I}| \times |\body{\I}|$ matrix such that
the entry [$A \imp C, B \imp D$] is 1 if $\I \models A \imp B$. Consequently,
finding equivalence classes shall take no more than a run over this matrix to 
be done (after having computed closures).

\vspace{1.2em}

Direct determination can be considered as well with a modification of 
\textsc{LinClosure} on $\I - E_{\I}(A)$. In this version, for a given $A \imp 
C$, we also have to be provided with $E_{\I}(A)$. Then, direct determination 
will be exposed the first time we reach \textit{count}[$B \imp D$] = 0 in the 
computation, for $B \imp D$ different from $A \imp C$ and belonging to 
$E_{\I}(A)$. Because we stop to the first time this will ensure that we did not 
reach any other body of this equivalence class, otherwise we would have stopped 
before. Because equivalence classes partition the implications of $\I$ we will 
proceed to this operation at most once for each implication.

\vspace{1.2em}

Finally, we consider the question of whether equivalence classes are altered 
more than removal during computations. Of course we remove direct determination
is found. Does replacing $B \imp D$ by $B \imp C \cup D$ has any chances to 
add other implication to $E_{\I}(A)$? Fortunately the answer is no. Indeed, 
before removing $A \imp C$ from $E_{\I}(A)$, anything we could reach from $A$
could be reached from $B$ (because of their equivalence) and thus would have
implied $C$ from $A$ and $D$ from $B$. Thus anything reachable from $C \cup D$
would already be in $E_{\I}(A)$.

 
 
\vspace{1.2em}


\paragraph{Complexity} Again, the complexity of this algorithm has been studied
and discussed in \cite{david_minimum_1980, maier_theory_1983}. It has been 
challenged (somewhat) in \cite{ausiello_graph_1983, ausiello_minimal_1986}. We
provide a complexity analysis anyway to adapt previous material to our notations
(and ease comparison with other algorithms) and to have this report to be 
self-contained as much as possible. Let us recall few elements to analyse
complexity:
\begin{itemize}
	\item $\body{\I}$ is the number of distinct bodies in $\I$ (therefore the
	total number of implications for us), and $|\body{\I}|$ its size,
	\item $|\Sg|$ is the number of attributes in the universe $\Sg$,
	\item $|\I|$ is the size of $\I$ in memory, that is roughly $|\body{\I}| 
	\times |\Sg|$.
\end{itemize}
\noindent Let us proceed by steps. First, non-redundancy elimination. Because 
the inner \textbf{foreach} loop goes for all implications of $\I$ requiring 
$O(|\body{\I}|)$ times looping. Then, computing closure of $A$ under $\I$ and 
$\I^{-}$ requires $O(|\I|)$ operations thanks to \textsc{LinClosure}. Hence,
redundancy elimination can be done in $O(|\body{\I}| \times |\I|) = 
O(|\body{\I}|^2 \times |\Sg|)$ time. Secondly, finding equivalence classes as
mentioned previously can be done by using modified version of 
\textsc{LinClosure} performing same time as the original one. Since we have to
go over (at most) all implications to find all implications with bodies implied
by a given one, this result in $O(|\body{\I}| \times |\I|)$ complexity. Then,
having the $|\body{\I}| \times |\body{\I}|$ matrix, building equivalence classes
need a run over the matrix, that is $O(|\body{\I}|^2)$ operations. Finally,
removing direct determination can be done at most for all implications, and 
checking for this property is again made through \textsc{LinClosure}. Thus 
performing the last step of the algorithm is $O(|\body{\I}| \times |\I|) = 
O(|\body{\I}|^2 \times |\Sg|)$ time. As a conclusion, the whole algorithm works
in $O(|\body{\I}| \times |\I|)$.
