\section{"Usual" Algorithm}

Compute the canonical (or Duquenne-Guigues) basis $\I_c$ given $\I$. 
\cite{guigues_familles_1986, b._ganter_conceptual_2016}. 

\begin{algorithm}[H]
\KwIn{$\I$ an implication basis}
\KwOut{$\I_c$ the canonical basis derived from $\I$}

\BlankLine
\BlankLine

\ForEach{$A \imp B$ in $\I$}{
	$\I$ = $\I - \{A \imp B \}$ \;
	$B$ = $\I(A \cup B)$ \;
	$\I$ = $\I \cup \{A \imp B \}$ \;
}

\ForEach{$A \imp B$}{
	$\I$ = $\I - \{A \imp B \}$ \;
	$A$ = $\I(A)$ \;
	
	\If{$A \neq B$}{
		$\I$ = $\I \cup \{ A \imp B \}$ \;
	}
}

\caption{Canonical Cover}	
\end{algorithm}


\vspace{1.2em}

\noindent The first loop maximizes each heads, that is $A \imp B$ becomes $A 
\imp \I(A)$. We summarize the knowledge. The second loop maximizes left side
of implications with $\I^{-}(A)$. The second step kills redundancy.

\vspace{1.2em}

\noindent Notes: few tips to see that the resulting basis is equivalent to the 
input. By the end of the first loop, since we replaced $A \imp B$ by $A \imp 
\I(A)$, and by definition of $A \imp B$ , $A \imp B$ still holds for all $B$
($B \subseteq \I(A)$). For the second loop, observe that we omit only redundant
implications from the basis, because if $\I^{-}(A) = \I(A)$ it means that all
the implications $A \imp B$ true in $\I$ are true in $\I^{-}$. 