\section{Strong duality in Horn Minimization}

Pure Horn function (PHF). Implicates. Subsuming. Given $h$ a PHF, denote by 
$\mathcal{I}(h)$ the set of implicates of $h$. $\cal{P} \subseteq \cal{I}$ is
the set of prime implicates of $h$. Forward chaining is the closure operator. 

The essential set $\cal{E}_S$ of a subset $S$ of variables $V$ is defined as
follows:

	\[ \cal{E}_S = \{ A \imp B | A \subseteq S, B \nsubseteq S \} \]
	
It describes a subset of $h$ for which $S$ is a false point. In fact, it points
out the clauses falsified by $S$. So more than indicating whether $S$ is a model
or not of the function, it also says where does it falsify the function.

Lemma 4 states that a CNF $\phi$ represents a function $h$ if for each non-model
$S$ of $h$, $\phi$ contains at least one of the clause falsified by $h$. That is
$h$ and $\phi$ are false at the same points.

$ess(h)$ allows to partition the non-models of $h$. It is the number of maximum
pairwise disjoint essential sets of $h$. This can be viewed as the number of
subsets sufficient to disjointly describes false point of $h$. Body-disjoint 
essential set is a stronger criterion. Two essential sets $\cal{E}$, $\cal{E}'$
are body-disjoint if the intersection of the left sides they contain is empty.

sketch of proof for lemma 5. 

\begin{lemma} Let $h$ be PHF over $V$. For $P, Q \subseteq V$, we have the 
following equivalences:

\begin{itemize}
	\item[(i)] $\mathcal{E}_P$ and $\mathcal{E}_Q$ are disjoint iff $F_h(P \cap
		Q) \subseteq P \cup Q$,
	
	\item[(ii)] $\cal{E}_P$ and $\cal{E}_Q$ are body-disjoint iff $F_h(P
	 	\cap Q) \subseteq P xor Q$.
\end{itemize}
	
\end{lemma} 

\begin{proof}
(i), $\imp$. Note that for $P \subseteq V$, $F_h(P)$ is exactly the set of
variables contained in $\mathcal{E}_P \cup P$. Also, note that if $P, Q$, 
$\mathcal{E}_{P \cap Q} \subseteq \mathcal{E}_P \cap \mathcal{E}_Q$. If the
two last essential sets are disjoint, then their intersection is empty, 
leading to $\mathcal{E}_{P \cap Q} = \emptyset$. In other words, $P \cap Q$ is
a model of $h$, and by definition $F_h(P \cap Q) = P \cap Q \subseteq P \cup Q$

\vspace{1.2em}

(i), $\rimp$. Let us denote by $V(\mathcal{E}_P)$ the variables contained in 
$\mathcal{E}_P$. We reason by contraposition. Suppose $V(\mathcal{E}_{P \cap Q})
\neq \emptyset$. Then, 

\begin{align*}
	F_h(P \cap Q) & = V(\mathcal{E}_{P \cap Q}) \cup (P \cap Q) \\
	& = (V(\mathcal{E}_{P \cap Q}) \cup P) \cap
		(V(\mathcal{E}_{P \cap Q}) \cup Q) \\
	& = (A \cup X \cup P) \cap (A \cup X \cup Q) \\
	& = (P \cup X) \cap (Q \cup X) \\
	& = (P \cap Q) \cup X \nsubseteq P \cup Q
\end{align*}


\noindent where $X$ is a non-empty set derived from the definition of 
$\mathcal{E}_{P \cap Q}$ if it is non-empty. Indeed if $\mathcal{E}$ is 
non-empty then it contains at least one implication $B \imp u$ such that $B 
\subseteq P \cap Q$ and $u \notin P \cap Q$. The set of such $u$ is $X$. Note
that in this case, since $\mathcal{E}$ is derived from the essential sets of
$P$ and $Q$, $X$ does belong neither to $P$ nor $Q$. In other words $X 
\nsubseteq P \cup Q$. 

$A$ is a subset of $P \cap Q$. Indeed, it is possible not to have $P \cap Q
\subseteq V$ if the implications in $V$ only use parts of $P \cap Q$. $A$ is 
non-empty, and is be absorbed by $P$ or $Q$ in the previous derivations, 
through $P \cup A$ and $Q \cup A$. 

\vspace{1.2em}

(ii) $\imp$.
\end{proof}