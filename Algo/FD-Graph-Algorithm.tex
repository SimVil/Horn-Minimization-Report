\section{Minimal Representation of Directed Hypergraphs}

This section is dedicated to a minimization algorithm relying on Hypergraphs. 
It is derived from \cite{Hyp}, \cite{FD-graph}, \cite{Hyp-Intro}. We need to
define few notions to be able to work on the algorithm. 

\begin{definition}[\belemp{Hypergraph}] An \belemp{hypergraph} is a pair $H = 
(V, E)$ 
where $V$ is a set of vertices (as in a graph) and $E$ a set of subsets of $V$
describing hyperarcs.

\end{definition}

\noindent In fact, hypergraphs are an extension of graphs. 

\begin{definition}[\belemp{Directed Hypergraph}] A \belemp{directed hypergraph} 
$H = (V, E)$ is a pair with $V$ a set of vertices (nodes) and $E$ a set of 
elements of $2^V \times 2^V$, denoting edges going from a subset of $V$ to 
another subset of $V$.
	
\end{definition}

Directed hypergraphs are useful to graphically represent implication basis.

\paragraph{Example} Let us consider the following implication basis $\I$:

	\[ \I = \{ 1 \imp 2, 2 \imp 34, 3 \imp 12, 41 \imp 3 \} \]

\noindent Then we can define an hyper graph $L = (V, E)$ where
	\begin{itemize}
		\item $V = \{1, 2, 3, 4 \}$
		\item $E = \{ (\{ 1 \}, \{ 2\}), (\{ 2 \}, \{3, 4\}),
			(\{ 3 \}, \{ 1, 2\}), (\{1, 4\}, \{ 3\}) \}$
	\end{itemize}

\noindent To be clearer, we can see the graphical representation of this figure
in \ref{fig:DHyp1}

\begin{center}
	\begin{figure}
\begin{center}
\begin{tikzpicture}

\node[Vertex, label=left:{4}] (4) at (-1, 1) {};
\node[Vertex, label=above left:{1}] (1) at (1, 1) {};
\node[Vertex, label=below left:{2}] (2) at (-1.5, -1) {};
\node[Vertex, label=below right:{3}] (3) at (0.5, -1) {};

% 41 --> 3
\draw[<-, color=belize] (0.5, -0.8) to[bend right=18] (-0.8, 0.8);
\draw[<-, color=belize] (0.5, -0.8) to[bend left=28] (0.8, 0.8);

% 1 --> 2
\draw[<-, color=emerald] (-1.3, -0.8) -- (0.8, 0.8);

% 2 --> 34
\draw[->, color=amethyst] (-1.3, -0.8) to[bend left=45] (0.3, -0.8);
\draw[->, color=amethyst] (-1.3, -0.8) to[bend right=45] (-1, 0.8);

% 3 --> 12 
\draw[->, color=alizarine] (0.3, -0.8) to[bend right=60] (-1.3, -0.8);
\draw[->, color=alizarine] (0.3, -0.8) to[bend left=50] (0.8, 0.8);


\end{tikzpicture}
\end{center}

\caption{Example of Directed Hypergraph}
\label{fig:DHyp1}
\end{figure}



\begin{comment}
\begin{figure}[ht]
\begin{center}
\begin{tikzpicture}
\node[Input Node, minimum size=0.6cm] (x) at (-4, 0) {$x$};
\node[Hidden Node, opacity=0.65, minimum size=0.6cm] (h11) at (-2, 1) {$h_1$};
\node[Hidden Node, opacity=0.65, minimum size=0.6cm] (h12) at (-2, -1) {$h_2$};
\node[Hidden Node, opacity=0.65, minimum size=0.6cm] (h21) at (0, 1) {$h_3$};
\node[Hidden Node, minimum size=0.6cm] (h22) at (0, -1) {$h_4$};
\node[Output Node, minimum size=0.6cm] (o) at (2, 0) {$o$};
\node (y) at (4, 0) {$\hat{y}$};

\draw[->] (x) -- node[midway, above] {$u_1$} (h11);
\draw[->] (x) -- node[midway, above] {$u_2$} (h12);
\draw[->] (h11) -- node[midway, above] {$w_{1, 1}$} (h21);
\draw[->] (h11) -- node[sloped, near end, below] {$w_{2, 1}$} (h22);
\draw[->] (h12) -- node[sloped, near end, above] {$w_{1, 2}$} (h21);
\draw[->] (h12) -- node[midway, below] {$w_{2, 2}$} (h22);
\draw[->] (h21) -- node[midway, above] {$v_1$} (o);
\draw[->] (h22) -- node[midway, below] {$v_2$} (o);
\draw[->] (o) -- (y);


\draw (5,0) node{A} to[bend right=30]
node[very near start]{R}    node[pos=0.7]{S}  (9,2) node{B};

\end{tikzpicture}
\end{center}

\caption{Schéma d'un réseau de neurones pour la rétro-propagation}
\label{fig:NN-BP}
\end{figure} 


\end{comment}
\end{center}

The next definition is taken from the previous references.


\begin{definition}[\belemp{FD-Graph}]
	contenu...
\end{definition}



\begin{algorithm}[H]
	\KwIn{$\I$ an implication basis}
	\KwResult{$\I_c$ a minimal cover for $\I$}
	
	\Begin{
		Find the \emeemp{FD-Graph} of $\I$ \;
		Remove \emeemp{redundant} nodes \;
		Remove \emeemp{superfluous} nodes \;
		Remove \emeemp{redundant} arc \;
		Derive $\I_c$ from the new graph \;
	}

	\caption{Minimize (Hypergraphs)}
\end{algorithm}

\vspace{1.2em}

The result is not compelled to be the canonical basis.
\\
This document relies on \cite{FD-graph}, \cite{Hyp}. The idea is to represent
implicational basis $\I$ through directed graphs. For this document we consider 
a basis $\I$ over an attribute set $\Sg$, such that:

\begin{itemize}
	\item there is no $A \imp B$, $A' \imp B'$ in $\I$ such that $A = A'$ when
	$B \neq B'$,
	\item for all $A \imp B$ of $\I$, $A \cap B = \emptyset$
\end{itemize}

\noindent it is some kind of \midemp{reduced} form. Recall that the premises of
implications are also called \belemp{bodies} and consequences \belemp{head}.
Denote by $\mathcal{B}(\I)$ the set of $\I$'s bodies. 

\newpage

\section{FD-graphs}

FD-graph stands for \belemp{functional dependency graph}. We define the 
associated FD-graph $G_{\I} = (V, E)$ with 

\begin{itemize}
	\item $V = \Sg \cup \mathcal{B}(\I)$,
	\item $E = E_f \cup E_d$ where $E_f$ is the set of \belemp{full edges} and
	and $E_d$ the set of \belemp{dotted edges}:
	\begin{itemize}
		\item For each $A \imp B$ we draw a \belemp{full} arc from the node
		$A$ to every attribute of $B$,
		\item For each compound node $A$, we draw a \belemp{dotted} arc from
		$A$ to all of its attribute.
	\end{itemize}
\end{itemize}

\noindent Simple examples are shown in \ref{fig:FD-Graph-1}

\begin{center}
	\centering
\subfloat[FD-Graph of $a \imp bc$]{
\begin{tikzpicture}
\node[Vertex, label=left:{$a$}] (1) at (-0.5, 0) {};
\node[Vertex, label=right:{$b$}] (2) at (0.5, 0.5) {};
\node[Vertex, label=right:{$c$}] (3) at (0.5, -0.5) {};

\draw[->] (1) -- (2);
\draw[->] (1) -- (3);
\end{tikzpicture}
}\qquad
\subfloat[FD-Graph of $bc \imp a$]{
\begin{tikzpicture}
\node[Vertex, label=below:{$bc$}] (23) at (0, 0) {};
\node[Vertex, label=right:{$a$}] (1) at (1, 0) {};
\node[Vertex, label=left:{$b$}] (2) at (-1, 0.5) {};
\node[Vertex, label=left:{$c$}] (3) at (-1, -0.5) {};

\draw[->, dotted] (23) -- (2);
\draw[->, dotted] (23) -- (3);
\draw[->] (23) -- (1);
\end{tikzpicture}		
}

\subfloat[FD-Graph of $ab \imp cd$]{
\begin{tikzpicture}
\node[Vertex, label=below:{$ab$}] (12) at (0, 0) {};
\node[Vertex, label=left:{$a$}] (1) at (-1, 0.5) {};
\node[Vertex, label=left:{$b$}] (2) at (-1, -0.5) {};
\node[Vertex, label=right:{$c$}] (3) at (1, -0.5) {};
\node[Vertex, label=right:{$d$}] (4) at (1, 0.5) {};

\draw[->, dotted] (12) -- (2);
\draw[->, dotted] (12) -- (1);
\draw[->] (12) -- (3);
\draw[->] (12) -- (4);
\end{tikzpicture}	
}

\caption{Representation of some FD-graph}
\label{fig:FD-Graph-1}
\end{center}

\noindent An important notion is the one of \belemp{FD-Path}. An FD-path 
from a node $i$ to $j$ describes the implications we use to derive $i \imp j$
(with Armstrong rules, especially transitivity). Since it is not a strictly
rigorous document, we will provide examples of FD-path to give an idea. 
Intuitively, directed paths are FD-paths. But there is also one case in which
we can go "backward" in the graph. The examples relies on the figure
\ref{fig:FD-Graph-2}.

\newpage

\begin{center}
	\begin{figure}[ht]\centering
\begin{tikzpicture}
\node[Vertex, label=left:{ab}] (ab) at (-2, 0) {};
\node[Vertex, label=left:{f}] (f) at (-2, 2) {};
\node[Vertex, label=right:{af}] (af) at (-1, 3) {};
\node[Vertex, label=above:{g}] (g) at (-1, 4) {};
\node[Vertex, label=below:{a}] (a) at (-1, 1) {};
\node[Vertex, label=below:{b}] (b) at (-1, -1) {};
\node[Vertex, label=below:{c}] (c) at (0, 1) {};
\node[Vertex, label=below:{d}] (d) at (0, -1) {};
\node[Vertex, label=above:{h}] (h) at (1, 2) {};
\node[Vertex, label=left:{cd}] (cd) at (1, 0) {};
\node[Vertex, label=below:{e}] (e) at (2, 0) {};

\draw[->, dotted] (af) -- (f);
\draw[->, dotted] (af) -- (a);
\draw[->, dotted] (ab) -- (a);
\draw[->, dotted] (ab) -- (b);
\draw[->, dotted] (cd) -- (c);
\draw[->, dotted] (cd) -- (d);
\draw[->] (af) -- (g);
\draw[->] (ab) -- (f);
\draw[->] (a) -- (c);
\draw[->] (b) -- (d);
\draw[->] (c) -- (h);
\draw[->] (cd) -- (e);
\end{tikzpicture}

\caption{FD-Graph of some implicational basis}
\label{fig:FD-Graph-2}
\end{figure}
\end{center}

There are either \belemp{dotted} or \belemp{full} paths. A path $(i, j)$ is 
dotted if all arcs leaving $i$ are dotted, it is full otherwise. An FD-path is
then a \textbf{minimal} subgraph of the FD-graph.

\begin{center}
	\begin{figure}[ht]\centering
\subfloat[][dotted FD-path for $ab \imp e$]{
\begin{tikzpicture}
\node[Vertex, label=left:{ab}] (ab) at (-2, 0) {};
\node[Vertex, label=below:{a}] (a) at (-1, 1) {};
\node[Vertex, label=below:{b}] (b) at (-1, -1) {};
\node[Vertex, label=below:{c}] (c) at (0, 1) {};
\node[Vertex, label=below:{d}] (d) at (0, -1) {};
\node[Vertex, label=below:{cd}] (cd) at (1, 0) {};
\node[Vertex, label=right:{e}] (e) at (2, 0) {};

\draw[->, dotted] (ab) -- (a);
\draw[->, dotted] (ab) -- (b);
\draw[->, dotted] (cd) -- (c);
\draw[->, dotted] (cd) -- (d);
\draw[->] (a) -- (c);
\draw[->] (b) -- (d);
\draw[->] (cd) -- (e);

\end{tikzpicture}
}\qquad
\subfloat[][full FD-path for $ab \imp g$]{
\begin{tikzpicture}
\node[Vertex, label=below:{ab}] (ab) at (0, 0) {};
\node[Vertex, label=right:{a}] (a) at (1, 1) {};
\node[Vertex, label=left:{f}] (f) at (0, 2) {};
\node[Vertex, label=right:{af}] (af) at (1, 3) {};
\node[Vertex, label=right:{g}] (g) at (1, 4) {};

\draw[->, dotted] (ab) -- (a);
\draw[->, dotted] (af) -- (a);
\draw[->, dotted] (af) -- (f);
\draw[->] (ab) -- (f);
\draw[->] (af) -- (g);
\end{tikzpicture}		
}

\caption{Representation of some FD-paths}
\label{fig:FD-Graph-3}
\end{figure}
\end{center}

\newpage

\section{Algorithm}

The algorithm studied in the sources, aims to minimize an implicational basis 
in terms of bodies. As with the Duquenne-Guigues basis. It uses 3 steps. Two
to remove redundancy, the third one aims to lighten bodies and heads of 
remaining implications. We will try to express each of these steps in terms of
sets, closure, and so forth. 

\subsection{Removing redundant nodes}

The first step is about removing redundant implications (without right
maximization). In terms of FD-graphs, we remove \belemp{redundant} nodes. A
compound node (only) $i$ is said redundant if for each full arc $ij$ leaving $i$
there exists a dotted path $(i, j)$. We give an example in the figure
\ref{fig:FD-Graph-4}.

\begin{figure}[ht]\centering
\subfloat[][FD-Graph with redundant node ($ab$)]{
\begin{tikzpicture}
\node[Vertex, label=left:{ab}] (ab) at (0, 0) {};
\node[Vertex, label=below:{a}] (a) at (-1, 1) {};
\node[Vertex, label=below:{b}] (b) at (-1, -1) {};
\node[Vertex, label=below:{c}] (c) at (1, 1) {};
\node[Vertex, label=below:{d}] (d) at (1, -1) {};

\draw[->, dotted] (ab) -- (a);
\draw[->, dotted] (ab) -- (b);
\draw[->] (a) -- (c);
\draw[->] (b) -- (d);
\draw[->] (ab) -- (c);
\draw[->] (ab) -- (d);

\end{tikzpicture}
}\qquad
\subfloat[][FD-Graph with redundant node removed]{
\begin{tikzpicture}
\node[Vertex, label=below:{a}] (a) at (-1, 1) {};
\node[Vertex, label=below:{b}] (b) at (-1, -1) {};
\node[Vertex, label=below:{c}] (c) at (1, 1) {};
\node[Vertex, label=below:{d}] (d) at (1, -1) {};

\draw[->] (a) -- (c);
\draw[->] (b) -- (d);
\end{tikzpicture}		
}

\caption{Elimination of redundant nodes}
\label{fig:FD-Graph-4}
\end{figure}

In this example, the basis associated basis is $\I = { ab \imp cd \, ; \, 
	a \imp c \, ; \, b \imp d}$. Indeed, in this case, $ab \imp cd$ is 
	redundant 
because $\I - {ab \imp cd} \models ab \imp cd$. So removing a redundant node is
removing exactly one implication in $\I$ since $\I$ is reduced. In details, let
$A \imp B$ be an implication of $\I$ with $A = a_1 a_2 \dots a_n$ and $B = b_1
b_2 \dots b_m$. $A \imp B$ will be redundant in terms of FD-Graph if for each 
$b_i$ there exists $X_i \subset A$ such that $X_i \imp b_i$. That is:

\[ \bigcup_i X_i \subseteq A \imp B \]

\noindent Which may be rewritten in terms of $\I$ closure as 

\[ \bigcup_i \I(X_i) = \I(A) \]

\noindent Thinking of opposite direction, $A \imp B$ will be nonredundant if
there exists $b \in B$ such that $((b \in \I(X)) \land (X \subseteq A)) \imp 
(X = A)$.

\vspace{1.2em}

\noindent To sum up: \aliemp{the first step is about considering each $A \imp 
	B$ where $|A| > 1$, and removing it of $\I$ if $\I^{-}(A) = \I(A)$}. Note:
$\I^{-} = \I - {A \imp B}$.  


\subsection{Removing superfluous nodes}

Next, we remove from the nonredundant FD-Graph \belemp{superfluous} nodes. A 
node $i$ is \belemp{superfluous} if there is an equivalent node $j$ and a 
dotted path from $i$ to $j$. Two nodes $i, j$ are \belemp{equivalent} if there
are paths $(i, j)$ and $(j, i)$. 

\vspace{1.2em}

In terms of sets and closure, two attribute sets $A, B \subseteq \Sg$ are 
equivalent in $\I$ if $\I \models A \imp B, B \imp A$, that is, if $\I(A) = 
\I(B)$. 

\begin{figure}[ht]\centering
\subfloat[][FD-Graph with superfluous node ($ab$)]{
\begin{tikzpicture}
\node[Vertex, label=below:{ab}] (ab) at (0, 0) {};
\node[Vertex, label=above:{a}] (a) at (1, 1) {};
\node[Vertex, label=above:{b}] (b) at (1, -1) {};
\node[Vertex, label=below:{c}] (c) at (2, 1) {};
\node[Vertex, label=below:{d}] (d) at (2, -1) {};
\node[Vertex, label=right:{cd}] (cd) at (3, 0) {};
\node[Vertex, label=left:{e}] (e) at (-1, 0) {};

\draw[->, dotted] (ab) -- (a);
\draw[->, dotted] (ab) -- (b);
\draw[->, dotted] (cd) -- (c);
\draw[->, dotted] (cd) -- (d);
\draw[->] (a) -- (c);
\draw[->] (b) -- (d);
\draw[->] (cd) -- (a);
\draw[->] (cd) -- (b);
\draw[->] (ab) -- (e);

\end{tikzpicture}
}\qquad
\subfloat[][FD-Graph with superfluous node removed]{
\begin{tikzpicture}
\node[Vertex, label=above:{a}] (a) at (1, 1) {};
\node[Vertex, label=above:{b}] (b) at (1, -1) {};
\node[Vertex, label=below:{c}] (c) at (2, 1) {};
\node[Vertex, label=below:{d}] (d) at (2, -1) {};
\node[Vertex, label=right:{cd}] (cd) at (3, 0) {};
\node[Vertex, label=left:{e}] (e) at (-1, 0) {};

\draw[->, dotted] (cd) -- (c);
\draw[->, dotted] (cd) -- (d);
\draw[->] (a) -- (c);
\draw[->] (b) -- (d);
\draw[->] (cd) -- (a);
\draw[->] (cd) -- (b);
\draw[->] (cd) -- (e);
\end{tikzpicture}		
}

\caption{Elimination of superfluous node}
\label{fig:FD-Graph-5}
\end{figure}

\noindent The algorithm suggests the following:
\begin{itemize}
	\item find a superfluous node $i$, and an equivalent node $j$ with a dotted
	path from $i$ to $j$
	\item for each full arc $ik$, we add a full arc $jk$
	\item then we remove the node $i$ and all of its outgoing arcs from the 
	graph
	\item repeat until no more superfluous nodes exist
\end{itemize}
\noindent An example of this procedure is given in the figure
\ref{fig:FD-Graph-5}. In this example $\I = {ab \imp e \, ; \, a \imp c
	\, ; \, b \imp d \, ; \, cd \imp ab}$. The node $ab$ is superfluous. Since 
	our
basis are reduced, note that removing a superfluous node is removing exactly 
one implication in $\I$. In this case, the resulting $\I$ will be

\[ \I = {a \imp c, b \imp d, cd \imp abe} \] 

\noindent Now we may rewrite this operation in our terms. Let $A \imp B$ and 
for instance $C \imp D$ be part of $\I$ to be general. Then $A$ is superfluous
body if

\[ \I \models A \imp C, C \imp A \land \exists X \subset A \; s.t \;
\I \models X \imp C \]

\noindent In this case, we apply the following operations
\begin{itemize}
	\item $C \imp D$ becomes $C \imp (D \cup B)$
	\item we remove $A \imp B$ from $\I$
\end{itemize}

\noindent We exhibit some arguments to convince ourselves that this is a valid
operation. Let us call temporarily $\I^{-}$ the basis we obtain after the 
previous operations. We would like to show that $\I^{-} \models A \imp B$, i.e
that $\I^{-} \equiv \I$. We removed $A \imp B$ but we still have $\I^{-} \models
X \imp C$ and then $X \imp B$ because $C \imp D \cup B$. That is, $B \subset 
\I^{-}(X)$. Because $X \subset A$, we have $B \subset \I^{-}(A)$ which is what
we wanted. 


\subsection{Removing redundant arcs}

Finally, we remove from a minimum nonredundant FD-Graph, \belemp{redundant 
	arcs}:
\begin{itemize}
	\item dotted case: a dotted arc $ij$ is redundant if there is a dotted 
	path $(i, j)$ not using $ij$,
	\item full case: a full arc $ij$ is redundant if there is a dotted/full 
	path $(i, j)$ not using $ij$.
\end{itemize}

\noindent we can think of eliminating redundant arcs as explicit transitivity
elimination. If we remove a full arc in $A \imp B$ then we are minimizing $B$.
If we remove a dotted arc, we are minimizing $A$. We have examples in 
\ref{fig:FD-Graph-6}. 

\begin{figure}[ht]\centering
\subfloat[][FD-Graph with redundant arcs ($abc \imp a$, $f \imp d$)]{
\begin{tikzpicture}
\node[Vertex, label=above:{abc}] (abc) at (0, 1) {};
\node[Vertex, label=left:{a}] (a) at (-1, 0) {};
\node[Vertex, label=right:{b}] (b) at (0, 0) {};
\node[Vertex, label=right:{c}] (c) at (1, 0) {};
\node[Vertex, label=below:{d}] (d) at (0, -1) {};
\node[Vertex, label=below:{e}] (e) at (-1, -2) {};
\node[Vertex, label=below:{f}] (f) at (1, -2) {};

\draw[->, dotted] (abc) -- (a);
\draw[->, dotted] (abc) -- (b);
\draw[->, dotted] (abc) -- (c);
\draw[->] (b) -- (d);
\draw[->] (d) -- (a);
\draw[->] (c) -- (d);
\draw[->] (d) -- (e);
\draw[->] (e) -- (f);
\draw[->] (f) -- (d);

\end{tikzpicture}
}\qquad
\subfloat[][FD-Graph with redundant arcs removed]{
\begin{tikzpicture}
\node[Vertex, label=above:{bc}] (bc) at (0, 1) {};
\node[Vertex, label=left:{a}] (a) at (-1, 0) {};
\node[Vertex, label=right:{b}] (b) at (0, 0) {};
\node[Vertex, label=right:{c}] (c) at (1, 0) {};
\node[Vertex, label=below:{d}] (d) at (0, -1) {};
\node[Vertex, label=below:{e}] (e) at (-1, -2) {};
\node[Vertex, label=below:{f}] (f) at (1, -2) {};

\draw[->, dotted] (bc) -- (b);
\draw[->, dotted] (bc) -- (c);
\draw[->] (b) -- (d);
\draw[->] (d) -- (a);
\draw[->] (d) -- (e);
\draw[->] (e) -- (f);
\end{tikzpicture}		
}

\caption{Elimination of redundant arcs}
\label{fig:FD-Graph-6}
\end{figure}

In terms of sets, consider an implication $A \imp B$. Removing a dotted arc is
saying that given $a \in A$, and substituting $A \imp B$ by $A - a \imp B$ in
$\I$ preserves $\I \models A \imp a_i$. On the other side, removing a full arc
is, given $b \in B$ and replacing by $A \imp B - b$, we preserve $\I \models 
A \imp b$.