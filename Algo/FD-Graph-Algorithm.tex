\subsection{Ausiello Algorithm: minimality through directed graphs}

This section is dedicated to a minimization algorithm relying on hypergraphs. 
It has been set up by Ausiello and al. in \cite{ausiello_directed_2017, 
ausiello_graph_1983, ausiello_minimal_1986}. Starting from a directed hypergraph
representation, it builds a special kind of DAG, called \belemp{FD-Graph} with
which it reduces the initial hypergraph. In order, we are going to define what
is a FD-graph, provide the general idea for the algorithm as explained in 
\cite{ausiello_minimal_1986} and then go into further details and more precise
algorithms for such computations as exposed in \cite{ausiello_graph_1983}.

\subsubsection{FD-Graphs and minimum covers}

In this part, we assume that our basis $\I$ is represented through the framework
of hypergraphs (see chapter 1). Again, all the definitions we are going to state
here are exposed in \cite{ausiello_graph_1983, ausiello_minimal_1986}. First,
let us define the central object we will work on, FD-Graphs.



\begin{definition}[\midn{FD-Graph}] Given a hypergraph $\I = (\Sg, C)$ 
representing an implication basis, the directed graph $G_{\I} = (V, \  E)$ such 
that:
\begin{itemize}
	\item $V = V_0 \cup V_1$ is the set of nodes where:
		\begin{itemize}
			\item $V_0 = \Sg$ is the set of \belemp{simple} nodes (a node
			per attribute in $\Sg$),
			\item $V_1 = \{X | X \in \body{\I} \}$ is the set of 
			\belemp{compound} nodes (a node per distinct body in $\I$),
		\end{itemize}

	\item $E = E_0 \cup E_1$ is the set of arcs where:
		\begin{itemize}
			\item $E_0$ is the set of \belemp{full} arcs. We have a full arc 
			$(X, i)$ in	$E_0$ if $(X, i)$ is an hyperarc of $\I$,
			\item $E_1$ the set of \belemp{dotted} arcs. For each compound node 
			$X$ of $V^1$, we have a dotted arc $(X, \ i)$ to every attributes 
			$i$ of $X$,
		\end{itemize}

\end{itemize}
\noindent is the \belemp{Functionnal Dependency Graph} or \belemp{FD-Graph} 
associated to $\I$.
\end{definition}

In fact, hypergraph representation is not mandatory to represent a FD-Graph, 
since as we saw, hypergraphs in our context rely on "usual" implicational 
basis. To enlighten the definition of FD-Graph, simple examples are shown in 
\ref{fig:FD-Graph-1}.

\begin{center}
	\begin{figure}[ht]\centering
\subfloat[][FD-Graph of $1 \imp 23$]{
\begin{tikzpicture}
\node[Vertex, label=left:{1}] (1) at (-0.5, 0) {};
\node[Vertex, label=right:{2}] (2) at (0.5, 0.5) {};
\node[Vertex, label=right:{3}] (3) at (0.5, -0.5) {};

\draw[->] (1) -- (2);
\draw[->] (1) -- (3);
\end{tikzpicture}
}\qquad
\subfloat[][FD-Graph of $23 \imp 1$]{
\begin{tikzpicture}
\node[Vertex, label=below:{23}] (23) at (0, 0) {};
\node[Vertex, label=right:{1}] (1) at (1, 0) {};
\node[Vertex, label=left:{2}] (2) at (-1, 0.5) {};
\node[Vertex, label=left:{3}] (3) at (-1, -0.5) {};

\draw[->, dotted] (23) -- (2);
\draw[->, dotted] (23) -- (3);
\draw[->] (23) -- (1);
\end{tikzpicture}		
}

\subfloat[][FD-Graph of $12 \imp 34$]{
\begin{tikzpicture}
\node[Vertex, label=below:{12}] (12) at (0, 0) {};
\node[Vertex, label=left:{1}] (1) at (-1, 0.5) {};
\node[Vertex, label=left:{2}] (2) at (-1, -0.5) {};
\node[Vertex, label=right:{3}] (3) at (1, -0.5) {};
\node[Vertex, label=right:{4}] (4) at (1, 0.5) {};

\draw[->, dotted] (12) -- (2);
\draw[->, dotted] (12) -- (1);
\draw[->] (12) -- (3);
\draw[->] (12) -- (4);
\end{tikzpicture}	
}

\caption{Representation of some FD-graph}
\label{fig:FD-Graph-1}
\end{figure}
\end{center}

We can see in those example that drawing an FD-Graph goes this way:
\begin{itemize}
	\item For each $A \imp B$ of $\I$ we draw a \belemp{full} arc from the node
	$A$ to every attribute of $B$,
	\item For each compound node $A$, we draw a \belemp{dotted} arc from
	$A$ to all of its attribute.
\end{itemize}
\noindent which is indeed what we formally defined previously. Furthermore, for
this algorithm, we consider a basis $\I$ over an attribute set $\Sg$, such that:
\begin{itemize}
	\item there is no $A \imp B$, $A' \imp B'$ in $\I$ such that $A = A'$ when
	$B \neq B'$,
	\item for all $A \imp B$ of $\I$, $A \cap B = \emptyset$
\end{itemize}

\vspace{1.2em}

An important notion we may need for all the following is \belemp{FD-paths}.

\begin{definition}[\midn{FD-Path}] Given an FD-Graph $G_{\I} = (V, E)$, an
\belemp{FD-Path} $\langle i, \ j \rangle$ is a minimal subgraph $\bar{G}_{\I} = 
(\bar{V}, \bar{E})$ of $G_{\I}$ such that $i, j \in \bar{V}$ and either $(i, j)
\in \bar{E}$ or one of the following holds:

\begin{itemize}
	\item $j$ is a simple node and there exists $k \in \bar{V}$ such that $(k, 
	j) \in \bar{E}$ and there exists a FD-Path $\langle i, \ k \rangle$ included
	in $\bar{G}$,  
	\item $j = \bigcup_{k = 1}^n j_k$ is a compound node and there exists 
	FD-paths $\langle i, \ j_k \rangle$ included in $\bar{G}$, for all $k = 1, 
	\ \dots, \ n$.
\end{itemize}
	
\end{definition}

\noindent As is, the definition may not seems clear. Informally, a FD-path 
from a node $i$ to $j$ describes the implications we use to derive $i \imp j$
(with Armstrong rules, especially transitivity and union). Intuitively, 
directed paths are FD-paths. But there is also one case in which
we can go "backward" in the graph. For better understanding, see examples of
figure \ref{fig:FD-Graph-3} based on the basis described in figure 
\ref{fig:FD-Graph-2}.

\begin{center}
	\centering
\begin{tikzpicture}
\node[Vertex, label=left:{ab}] (ab) at (-2, 0) {};
\node[Vertex, label=left:{f}] (f) at (-2, 2) {};
\node[Vertex, label=right:{af}] (af) at (-1, 3) {};
\node[Vertex, label=above:{g}] (g) at (-1, 4) {};
\node[Vertex, label=below:{a}] (a) at (-1, 1) {};
\node[Vertex, label=below:{b}] (b) at (-1, -1) {};
\node[Vertex, label=below:{c}] (c) at (0, 1) {};
\node[Vertex, label=below:{d}] (d) at (0, -1) {};
\node[Vertex, label=above:{h}] (h) at (1, 2) {};
\node[Vertex, label=left:{cd}] (cd) at (1, 0) {};
\node[Vertex, label=below:{e}] (e) at (2, 0) {};

\draw[->, dotted] (af) -- (f);
\draw[->, dotted] (af) -- (a);
\draw[->, dotted] (ab) -- (a);
\draw[->, dotted] (ab) -- (b);
\draw[->, dotted] (cd) -- (c);
\draw[->, dotted] (cd) -- (d);
\draw[->] (af) -- (g);
\draw[->] (ab) -- (f);
\draw[->] (a) -- (c);
\draw[->] (b) -- (d);
\draw[->] (c) -- (h);
\draw[->] (cd) -- (e);
\end{tikzpicture}

\caption{FD-Graph of some implicational basis}
\label{fig:FD-Graph-2}

\end{center}

There are either \belemp{dotted} or \belemp{full} paths. A path $\langle i, j
\rangle$ is dotted if all arcs leaving $i$ are dotted, it is full otherwise.

\begin{center}
	\begin{figure}[ht]\centering
\subfloat[][dotted FD-path for $ab \imp e$]{
\begin{tikzpicture}
\node[Vertex, label=left:{ab}] (ab) at (-2, 0) {};
\node[Vertex, label=below:{a}] (a) at (-1, 1) {};
\node[Vertex, label=below:{b}] (b) at (-1, -1) {};
\node[Vertex, label=below:{c}] (c) at (0, 1) {};
\node[Vertex, label=below:{d}] (d) at (0, -1) {};
\node[Vertex, label=below:{cd}] (cd) at (1, 0) {};
\node[Vertex, label=right:{e}] (e) at (2, 0) {};

\draw[->, dotted] (ab) -- (a);
\draw[->, dotted] (ab) -- (b);
\draw[->, dotted] (cd) -- (c);
\draw[->, dotted] (cd) -- (d);
\draw[->] (a) -- (c);
\draw[->] (b) -- (d);
\draw[->] (cd) -- (e);

\end{tikzpicture}
}\qquad
\subfloat[][full FD-path for $ab \imp g$]{
\begin{tikzpicture}
\node[Vertex, label=below:{ab}] (ab) at (0, 0) {};
\node[Vertex, label=right:{a}] (a) at (1, 1) {};
\node[Vertex, label=left:{f}] (f) at (0, 2) {};
\node[Vertex, label=right:{af}] (af) at (1, 3) {};
\node[Vertex, label=right:{g}] (g) at (1, 4) {};

\draw[->, dotted] (ab) -- (a);
\draw[->, dotted] (af) -- (a);
\draw[->, dotted] (af) -- (f);
\draw[->] (ab) -- (f);
\draw[->] (af) -- (g);
\end{tikzpicture}		
}

\caption{Representation of some FD-paths}
\label{fig:FD-Graph-3}
\end{figure}
\end{center}

\vspace{1.2em}

Having explained FD-Graphs, we will now move to explanations of the algorithm
developed by Ausiello and al.

\subsubsection{Ausiello algorithm: general point of view}

The following procedure (\ref{alg:ausiello_86}) finds from a given basis its
\belemp{minimal cover}. 

\begin{algorithm}[H]
\KwIn{$\I$ an implication basis}
\KwOut{$\I_c$ a minimal cover for $\I$}

\BlankLine
\BlankLine

Find the \belemp{FD-Graph} of $\I$ \;
Remove \belemp{redundant} nodes \;
Remove \belemp{superfluous} nodes \;
Remove \belemp{redundant} arc \;
Derive $\I_c$ from the new graph \;

\caption{Ausiello algorithm (1986)}
\label{alg:ausiello_86}
\end{algorithm}

\vspace{1.2em}

Let us emphasize on the meaning of those removal steps.
The algorithm studied in the sources, aims to minimize an implicational basis 
in terms of bodies. As with the Duquenne-Guigues basis. It uses 3 steps. Two
to remove redundancy, the third one aims to lighten bodies and heads of 
remaining implications. We will try to express each of these steps in terms of
sets, closure, and so forth. 

\subsubsection{Removing redundant nodes}

The first step is about removing redundant implications (without right
maximization). In terms of FD-graphs, we remove \belemp{redundant} nodes. A
compound node (only) $i$ is said redundant if for each full arc $ij$ leaving $i$
there exists a dotted path $(i, j)$. We give an example in the figure
\ref{fig:FD-Graph-4}.

\begin{figure}[ht]\centering
\subfloat[][FD-Graph with redundant node ($ab$)]{
\begin{tikzpicture}
\node[Vertex, label=left:{ab}] (ab) at (0, 0) {};
\node[Vertex, label=below:{a}] (a) at (-1, 1) {};
\node[Vertex, label=below:{b}] (b) at (-1, -1) {};
\node[Vertex, label=below:{c}] (c) at (1, 1) {};
\node[Vertex, label=below:{d}] (d) at (1, -1) {};

\draw[->, dotted] (ab) -- (a);
\draw[->, dotted] (ab) -- (b);
\draw[->] (a) -- (c);
\draw[->] (b) -- (d);
\draw[->] (ab) -- (c);
\draw[->] (ab) -- (d);

\end{tikzpicture}
}\qquad
\subfloat[][FD-Graph with redundant node removed]{
\begin{tikzpicture}
\node[Vertex, label=below:{a}] (a) at (-1, 1) {};
\node[Vertex, label=below:{b}] (b) at (-1, -1) {};
\node[Vertex, label=below:{c}] (c) at (1, 1) {};
\node[Vertex, label=below:{d}] (d) at (1, -1) {};

\draw[->] (a) -- (c);
\draw[->] (b) -- (d);
\end{tikzpicture}		
}

\caption{Elimination of redundant nodes}
\label{fig:FD-Graph-4}
\end{figure}

In this example, the basis associated basis is $\I = { ab \imp cd \, ; \, 
	a \imp c \, ; \, b \imp d}$. Indeed, in this case, $ab \imp cd$ is 
	redundant 
because $\I - {ab \imp cd} \models ab \imp cd$. So removing a redundant node is
removing exactly one implication in $\I$ since $\I$ is reduced. In details, let
$A \imp B$ be an implication of $\I$ with $A = a_1 a_2 \dots a_n$ and $B = b_1
b_2 \dots b_m$. $A \imp B$ will be redundant in terms of FD-Graph if for each 
$b_i$ there exists $X_i \subset A$ such that $X_i \imp b_i$. That is:

\[ \bigcup_i X_i \subseteq A \imp B \]

\noindent Which may be rewritten in terms of $\I$ closure as 

\[ \bigcup_i \I(X_i) = \I(A) \]

\noindent Thinking of opposite direction, $A \imp B$ will be nonredundant if
there exists $b \in B$ such that $((b \in \I(X)) \land (X \subseteq A)) \imp 
(X = A)$.

\vspace{1.2em}

\noindent To sum up: \aliemp{the first step is about considering each $A \imp 
	B$ where $|A| > 1$, and removing it of $\I$ if $\I^{-}(A) = \I(A)$}. Note:
$\I^{-} = \I - {A \imp B}$.  


\subsubsection{Removing superfluous nodes}

Next, we remove from the nonredundant FD-Graph \belemp{superfluous} nodes. A 
node $i$ is \belemp{superfluous} if there is an equivalent node $j$ and a 
dotted path from $i$ to $j$. Two nodes $i, j$ are \belemp{equivalent} if there
are paths $(i, j)$ and $(j, i)$. 

\vspace{1.2em}

In terms of sets and closure, two attribute sets $A, B \subseteq \Sg$ are 
equivalent in $\I$ if $\I \models A \imp B, B \imp A$, that is, if $\I(A) = 
\I(B)$. 

\begin{figure}[ht]\centering
\subfloat[][FD-Graph with superfluous node ($ab$)]{
\begin{tikzpicture}
\node[Vertex, label=below:{ab}] (ab) at (0, 0) {};
\node[Vertex, label=above:{a}] (a) at (1, 1) {};
\node[Vertex, label=above:{b}] (b) at (1, -1) {};
\node[Vertex, label=below:{c}] (c) at (2, 1) {};
\node[Vertex, label=below:{d}] (d) at (2, -1) {};
\node[Vertex, label=right:{cd}] (cd) at (3, 0) {};
\node[Vertex, label=left:{e}] (e) at (-1, 0) {};

\draw[->, dotted] (ab) -- (a);
\draw[->, dotted] (ab) -- (b);
\draw[->, dotted] (cd) -- (c);
\draw[->, dotted] (cd) -- (d);
\draw[->] (a) -- (c);
\draw[->] (b) -- (d);
\draw[->] (cd) -- (a);
\draw[->] (cd) -- (b);
\draw[->] (ab) -- (e);

\end{tikzpicture}
}\qquad
\subfloat[][FD-Graph with superfluous node removed]{
\begin{tikzpicture}
\node[Vertex, label=above:{a}] (a) at (1, 1) {};
\node[Vertex, label=above:{b}] (b) at (1, -1) {};
\node[Vertex, label=below:{c}] (c) at (2, 1) {};
\node[Vertex, label=below:{d}] (d) at (2, -1) {};
\node[Vertex, label=right:{cd}] (cd) at (3, 0) {};
\node[Vertex, label=left:{e}] (e) at (-1, 0) {};

\draw[->, dotted] (cd) -- (c);
\draw[->, dotted] (cd) -- (d);
\draw[->] (a) -- (c);
\draw[->] (b) -- (d);
\draw[->] (cd) -- (a);
\draw[->] (cd) -- (b);
\draw[->] (cd) -- (e);
\end{tikzpicture}		
}

\caption{Elimination of superfluous node}
\label{fig:FD-Graph-5}
\end{figure}

\noindent The algorithm suggests the following:
\begin{itemize}
	\item find a superfluous node $i$, and an equivalent node $j$ with a dotted
	path from $i$ to $j$
	\item for each full arc $ik$, we add a full arc $jk$
	\item then we remove the node $i$ and all of its outgoing arcs from the 
	graph
	\item repeat until no more superfluous nodes exist
\end{itemize}
\noindent An example of this procedure is given in the figure
\ref{fig:FD-Graph-5}. In this example $\I = {ab \imp e \, ; \, a \imp c
	\, ; \, b \imp d \, ; \, cd \imp ab}$. The node $ab$ is superfluous. Since 
	our
basis are reduced, note that removing a superfluous node is removing exactly 
one implication in $\I$. In this case, the resulting $\I$ will be

\[ \I = {a \imp c, b \imp d, cd \imp abe} \] 

\noindent Now we may rewrite this operation in our terms. Let $A \imp B$ and 
for instance $C \imp D$ be part of $\I$ to be general. Then $A$ is superfluous
body if

\[ \I \models A \imp C, C \imp A \land \exists X \subset A \; s.t \;
\I \models X \imp C \]

\noindent In this case, we apply the following operations
\begin{itemize}
	\item $C \imp D$ becomes $C \imp (D \cup B)$
	\item we remove $A \imp B$ from $\I$
\end{itemize}

\noindent We exhibit some arguments to convince ourselves that this is a valid
operation. Let us call temporarily $\I^{-}$ the basis we obtain after the 
previous operations. We would like to show that $\I^{-} \models A \imp B$, i.e
that $\I^{-} \equiv \I$. We removed $A \imp B$ but we still have $\I^{-} \models
X \imp C$ and then $X \imp B$ because $C \imp D \cup B$. That is, $B \subset 
\I^{-}(X)$. Because $X \subset A$, we have $B \subset \I^{-}(A)$ which is what
we wanted. 


\subsubsection{Removing redundant arcs}

Finally, we remove from a minimum nonredundant FD-Graph, \belemp{redundant 
	arcs}:
\begin{itemize}
	\item dotted case: a dotted arc $ij$ is redundant if there is a dotted 
	path $(i, j)$ not using $ij$,
	\item full case: a full arc $ij$ is redundant if there is a dotted/full 
	path $(i, j)$ not using $ij$.
\end{itemize}

\noindent we can think of eliminating redundant arcs as explicit transitivity
elimination. If we remove a full arc in $A \imp B$ then we are minimizing $B$.
If we remove a dotted arc, we are minimizing $A$. We have examples in 
\ref{fig:FD-Graph-6}. 

\begin{figure}[ht]\centering
\subfloat[][FD-Graph with redundant arcs ($abc \imp a$, $f \imp d$)]{
\begin{tikzpicture}
\node[Vertex, label=above:{abc}] (abc) at (0, 1) {};
\node[Vertex, label=left:{a}] (a) at (-1, 0) {};
\node[Vertex, label=right:{b}] (b) at (0, 0) {};
\node[Vertex, label=right:{c}] (c) at (1, 0) {};
\node[Vertex, label=below:{d}] (d) at (0, -1) {};
\node[Vertex, label=below:{e}] (e) at (-1, -2) {};
\node[Vertex, label=below:{f}] (f) at (1, -2) {};

\draw[->, dotted] (abc) -- (a);
\draw[->, dotted] (abc) -- (b);
\draw[->, dotted] (abc) -- (c);
\draw[->] (b) -- (d);
\draw[->] (d) -- (a);
\draw[->] (c) -- (d);
\draw[->] (d) -- (e);
\draw[->] (e) -- (f);
\draw[->] (f) -- (d);

\end{tikzpicture}
}\qquad
\subfloat[][FD-Graph with redundant arcs removed]{
\begin{tikzpicture}
\node[Vertex, label=above:{bc}] (bc) at (0, 1) {};
\node[Vertex, label=left:{a}] (a) at (-1, 0) {};
\node[Vertex, label=right:{b}] (b) at (0, 0) {};
\node[Vertex, label=right:{c}] (c) at (1, 0) {};
\node[Vertex, label=below:{d}] (d) at (0, -1) {};
\node[Vertex, label=below:{e}] (e) at (-1, -2) {};
\node[Vertex, label=below:{f}] (f) at (1, -2) {};

\draw[->, dotted] (bc) -- (b);
\draw[->, dotted] (bc) -- (c);
\draw[->] (b) -- (d);
\draw[->] (d) -- (a);
\draw[->] (d) -- (e);
\draw[->] (e) -- (f);
\end{tikzpicture}		
}

\caption{Elimination of redundant arcs}
\label{fig:FD-Graph-6}
\end{figure}

In terms of sets, consider an implication $A \imp B$. Removing a dotted arc is
saying that given $a \in A$, and substituting $A \imp B$ by $A - a \imp B$ in
$\I$ preserves $\I \models A \imp a_i$. On the other side, removing a full arc
is, given $b \in B$ and replacing by $A \imp B - b$, we preserve $\I \models 
A \imp b$. It appears that this steps goes beyond our scope, because it aims
to minimize implication itself, not the number of implications. So, in our 
minimization context, we can stick to the first two operations to obtain a 
minimal representation in our terms. The algorithm we have to follow becomes
the procedure \ref{alg:ausiello_redux_86}

\vspace{1.2em}

\begin{algorithm}[H]
	\KwIn{$\I$ an implication basis}
	\KwOut{$\I_c$ a minimal cover for $\I$}
	
	\BlankLine
	\BlankLine
	
	Find the \belemp{FD-Graph} of $\I$ \;
	Remove \belemp{redundant} nodes \;
	Remove \belemp{superfluous} nodes \;
	Derive $\I_c$ from the new graph \;
	
	\caption{Ausiello algorithm (1986, reduced)}
	\label{alg:ausiello_redux_86}
\end{algorithm}

\vspace{1.2em}

In fact, this algorithm stands for a high-level principle of what has to be done
to minimize a given basis. The question we are going to investigate in the next
paragraphs is how to do such computations.


\subsubsection{Ausiello algorithm: effective procedure}

In this part we will focus on how the algorithm proposed by Ausiello in 
\cite{ausiello_minimal_1986} performs redundancy and superfluousness 
elimination. Actually, those operations are detailed in 
\cite{ausiello_graph_1983}. But in order to stick to our subject (reviewing 
the existing algorithms) we study it here. Also, note that we will consider
that the mapping step from $\I$ to $G_{\I}$ is already done. Also, as mentioned
in the related articles, going from one representation to another can be done
in linear time.

\vspace{1.2em}

The algorithm relies on closure of a FD-Graph. Of course the idea of closure is
similar to the one we encountered before. Nevertheless, Ausiello and al. 
introduce a notion of priority in the way they determine the closure. Because
we cannot represent to different arcs with same nodes (two arcs $(i, \ j)$), me 
must provide priority between dotted and full arcs. The authors decided to 
stress on dotted arcs since they are more likely to be a criterion for removal 
during minimization.

\paragraph{Closure of a FD-Graph} The closure is based on the following data 
structures:
\begin{itemize}
	\item $V_0$: set of \belemp{simple} nodes,
	\item $V_1$: set of \belemp{compound} nodes,
	\item $D_i$ ($\forall i \in V$): nodes from \belemp{incoming dotted} arcs
		$\{j \in V \ | \ (j, \  i) \text{ is a dotted arc} \}$,
	\item $L_{i}^0$ ($\forall i \in V$): nodes from \belemp{outgoing full} arcs
		$\{j \in V \ | \ (i, \  j) \text{ is a full arc} \}$,
	\item $L_{i}^1$ ($\forall i \in V$): nodes from \belemp{outgoing dotted} 	
		arcs $\{j \in V \ | \ (i, \  j) \text{ is a dotted arc} \}$,
	\item $L_{i}^{0+}, L_{i}^{1+}$ ($\forall i \in V$): the respective closures
		of $L_i^0, L_i^1$,
	\item $q_m$ ($\forall m \in V^1$): counter of nodes in $m$ belonging to 
	$L_i^{0+} \cup L_i^{1+}$ for some $i \in V$.
\end{itemize}

\noindent To make understanding easier, we first give pseudo-code closer from
principle than algorithms. From a general point of view, to determine the 
closure of a FD-graph, we must compute the closure of all its nodes. The 
closure of a node is described by its full and dotted outgoing arcs. Because we
put a priority on dotted possibilities, they will be computed before. Principle
are given in algorithmic/pseudo-code form so that identification between steps
of procedures and ideas of principle are easier to see.

\vspace{1.2em}

First, we introduce the procedure NodeClosure which computes the closure of a
node with respect to a type of arc. In other words, to compute the full closure
of a node, we must first apply NodeClosure to its dotted arcs, then to its full
arcs. The principle and algorithm for Nodeclosure are 
\ref{alg:FD-NodeClosure-Principle}, \ref{alg:FD-NodeClosure}. 

\vspace{1.2em}

\begin{algorithm}
\KwIn{
$L_i$: set of nodes for which there exists dotted (resp. full) arcs $(i, j)$}
\KwOut{$L_i^+$: the dotted (resp. full) closure of $i$}

\BlankLine
\BlankLine

Initialize a list of nodes to treat $S_i$ to $L_i$ \;
\While{there is a node $j$ to treat in $S_i$}{
	remove $j$ from $S_i$ \;
	\If{$j$ is simple node}{
		\ForAll{compound node $m$ \belemp{except $i$}, $j$ appears in}{
			increase $q_m$ by 1 \;
			\If {$q_m$ = number of outgoing \belemp{dotted} arcs from $m$}{
				$m$ is reachable from $i$ by \aliemp{union} \;
				$m$ must be treated, add it to $S_i$ \;
			}
		}
		
	}

	add $j$ to the closure $L_i^+$ \;
	
	\ForAll{nodes $k$ such that there is an arc $(j, \ k)$}{
		\If{$k$ is not yet in the closure $L_i^+$ or in the \belemp{dotted}
			closure $L_i^{1+}$ of $i$}{
			$k$ is reachable from $i$ by \aliemp{transitivity} \;
			$k$ must be treated, add it to $S_i$ \;
		}
	}

return $L_i^{+}$ \;
}


\caption{NodeClosure (Principle)}
\label{alg:FD-NodeClosure-Principle}
\end{algorithm}

We would like to provide some observations on top of their description. Namely 
on the \aliemp{union} step and $q_m$ counters. Say $i \imp m$ where $m$ is a 
compound node is a valid implication in a FD-graph. Furthermore say $m = 
\bigcup_i m_i$ where $m_i$'s are simple nodes. The union step models the fact 
that if we have $i \imp m_i$ for all $m_i$ in $m$, then we must have $i \imp m$ 
also. The counter $q_m$ ensures that we indeed reached all $m_i$'s in $m$. 
Also, the algorithm has access to all the structures we described above (nodes, 
sets of arcs, and so forth). Parameters are thus lists we are going to modify 
somewhat. The \textsc{NodeClosure} algorithm runs in time $O(|\I|)$. The first 
nested loop runs in at most $O(|\Sg| \times |\body{\I}|) = O(|L|)$ because 
$S_i$ contains at most $|\Sg|$ elements, and the block referring to the 
\textit{union} rule runs over compound nodes, that is bodies of $\I$. For the 
second loop (transitivity) note that we can at most consider all the edges of 
the FD-graph. In fact, the cost of transitivity operation for all $j$ is 
$O(\sum_{j = 1}^n |L_j^0 \cup L_j^1 |)$. But by definition, those sets are 
disjoints, and therefore we cannot treat more than $| E |$ arcs (the total 
number of arcs in $G$), that is $|\I|$.

\begin{algorithm}
\KwIn{
$L_i$: set of nodes for which there exists dotted (resp. full) arcs $(i, j)$}
\KwOut{$L_i^+$: the dotted (resp. full) closure of $i$}

\BlankLine
\BlankLine	

$S_i := L_i$ \;

\While{$S_i \neq \emptyset$}{
	select $j$ from $S_i$ \;
	\If{$j \in V^0$}{
		\ForAll{$m \in D_j - \{ i \}$}{
			$q_m := q_m + 1$ \;
			\If{$q_m = |L_m^1|$}{
				$S_i := S_i \cup \{ m \}$ \;
			}
			
		}
	}

	$S_i^+ := S_i^+ \cup \{ j \}$ \;
	
	\ForAll{$k \in L_j^0 \cup L_j^1$}{
		\If{$k \not\in S_i^+ \cup L_i^{1+} \cup \{ i\}$}{
			$S_i := S_i \cup \{ k \}$ \;
		}
	}
}
return $L_i^+$ \;
\caption{NodeClosure}
\label{alg:FD-NodeClosure}
\end{algorithm}

\vspace{1.2em}

Next, we present the principle and pseudo-code for the closure of an FD-graph
\ref{alg:FD-Closure-Principle}, \ref{alg:FD-Closure}. Mostly, the principle is 
the idea we described previously. There is just one observation to make about 
setting a counter $q_m$ to 1. This variable helps to see whether we can use 
union rule as we saw in procedure NodeClosure 
(\ref{alg:FD-NodeClosure-Principle}, \ref{alg:FD-NodeClosure}). We initialize it
in case $i$ is indeed part of some compound node so that we do not omit to count
it when dealing with $S_i$ (because $S_i$ does not contain $i$). In terms of
complexity, we are running \textsc{NodeClosure} on all nodes having outgoing
edges, that is $|\body{\I}|$ nodes (if a compound node is represented, it must
have at least one outgoing full arc). Since \textsc{NodeClosure} operates in
$O(|\I|)$, the whole closure algorithm must run in $O(|\body{\I}| \times |\I|)$
.

\begin{algorithm}
\KwIn{$V_0$, $V_1$ and $\forall i \in V$ $D_i$, $L_i^0$, $L_i^1$}
\KwOut{$\forall i \in V$ $L_i^{0+}$, $L_i^{1+}$}

\BlankLine
\BlankLine

\ForAll{node $i$ in $V$ with outgoing arcs}{
	
	\If{$i$ is an attribute of a compound node $m$}{
		set a counter $q_m$ to $1$ \;	
	}
	
	initialize the closure of $i$ to $\emptyset$ \;
	
	\If{$i$ is a compound node}{
		determine \belemp{dotted} arcs in the closure of $i$ \;
	}
	
	determine \belemp{full} arcs in the closure of $i$ \;
	
}

\caption{Closure (Principle)}
\label{alg:FD-Closure-Principle}
\end{algorithm}

\begin{algorithm}
\KwIn{$V_0$, $V_1$ and $\forall i \in V$ $D_i$, $L_i^0$, $L_i^1$}
\KwOut{$\forall i \in V$ $L_i^{0+}$, $L_i^{1+}$}

\BlankLine
\BlankLine

\ForAll{$i \in V$ with $L_i^0  \cup L_i^1 \neq \emptyset$}{

	\ForAll{$m \in V^1$}{
		\uIf{$m \in D_i$}{
			$q_m := 1$ \;
			
		} \Else {
			$q_m := 0$ \;
			
		}
	}

	$L_i^{1+} := \emptyset$ \;
	$L_i^{0+} := \emptyset$ \;
	
	\If{$i \in V^1$}{
		$L_i^{1+} := $ NodeClosure($L_i^{1}$) \;	
	}

	$L_i^{0+} := $ NodeClosure($L_i^{0} - L_i^{1+}$) \;

} 

\caption{Closure}
\label{alg:FD-Closure}
\end{algorithm}

\vspace{1.2em}

Now that algorithms for computing the closure of a FD-graph have been set, we
can move to the minimization part.

\paragraph{Minimization algorithm for FD-Graphs}

We have two steps in this algorithm. First, we need to remove redundant nodes, 
then superfluous nodes. To delete redundant nodes, the claim of 
\cite{ausiello_graph_1983, ausiello_minimal_1986} is that removing nodes with
dotted arcs only in the closure of an FD-Graph is sufficient. Indeed, if a node
$i$ has only dotted paths in the closure of an FD-graph, it means that for every
$i \imp j$ holding, we can find a subset of $k$ of $i$ such that $k \imp j$. In
our terms, it is saying that $\I(i) = \I^{-}$ where $\I^{-}$ denotes $\I$ from
which we removed the implication having $i$ as a body.


\begin{algorithm}
\KwIn{$G_{\I}$: the FD-Graph of some basis $\I$}
\KwOut{$G_{\I_c}$: the associated minimum FD-Graph}

\BlankLine
\BlankLine

$G_{\I}^{+} = Closure(G_{\I})$ \;

\BlankLine

\emeemp{// Redundancy Elimination} \\
\ForAll{$i \in V^1$}{
	\If{$L_i^{1+} = \emptyset$}{
		remove $i$ and its outgoing arcs from $G_{\I}$ \;	
	}
} 

\BlankLine

\emeemp{// Superfluousness Elimination} \\
\ForAll{$i \in V^1$}{
	find an equivalent node $j$ \;
	\If{$j$ exists}{
		$L_j^{0+} := L_j^{0+} \cup (L_i^{0+} \cap L_j^{1+})$ \;
		$L_j^{1+} := L_j^{1+} - (L_i^{0+} \cap L_j^{1+})$ \;
	}

	remove $i$ from the closure \;
	add $(i, j)$ to a list $L$ \; 
}

remove superfluous nodes \;
move arcs to their final destination \;	

\caption{Ausiello Minimization Algorithm}
\label{alg:ausiello-min}
\end{algorithm}

\vspace{1.2em} 

Let us try to investigate the complexity of this algorithm. According to the 
article and previous study, \textsc{Closure} is achieved in $O(|\body{\I}| 
\times |\I|)$. If the graph is
represented as adjacency lists, deleting a node and its outgoing arc can be
$O(1)$ (it consists in freeing the node and its associated two lists). Running 
over all compound nodes is $O(|\body{\I}|)$. In the case we would have to 
remove arcs one by one, note that we would have at most $O(|\Sg|)$ arcs to 
delete. This would yield a $O(|\body{\I}| \cdot |\Sg|) = O(|\I|)$
complexity. \aliemp{How to remove compound nodes that have disappeared from the
closure of other nodes ? It is time consuming}. It is $O(|\body{\I}| \cdot 
|\I^{+}|)$. We don't count for intern data structure operation. We can consider
that adding and removing can be done in $O(1)$. The first loop of 
superfluousness elimination goes over all compounds nodes, and finding an 
equivalent node is $O(V)$ Moving elements in lists associated to $j$ is also
$O(V)$ thus the first loop is $O(V^1 \cdot V)$. The second loop may need to 
go over all the full arcs of the FD-graph, which account for a $O(|\I|)$ 
complexity. As a consequence the whole algorithm may run in $O(|\body{\I}| 
\cdot |\I|)$ (because of the closure computation).



\subsection{Elements of proofs}

{\color{alizarine}
This section needs to be adjusted and corrected. In fact, statements lacks
some precision, but conclusion is unchanged: both algorithm (Maier and 
Ausiello) performs computations on same objects, but according to the 
following statement 

\begin{proposition} We have equivalence between the following:
\begin{itemize}
	\item[(i)] $A \equiv B$ and $A \ddv B$,
	\item[(ii)] $A$ is superfluous (in particular, "minimal" superfluous with 
	respect to $B$),
	\item[(iii)] third part of proposition 7. (to be adapted)
\end{itemize}
	
\end{proposition}	
	
	
}

In this section we write various claims which help us prove correctness and 
equivalence of Maier/Ausiello's algorithms. Knowledge about FD-graphs is 
assumed. Equivalence of redundancy elimination is not shown (1st step of 
the algorithm). We place ourselves in a non-redundant context to match
definitions of direct determination. Non-redundancy does not affect superfluous
definition.

\begin{proposition} \label{prop:maier.equiv_sup_sub} The following properties
are equivalent (let $A \imp C$ be the implication of $\I$ with $A$ as
body):
\begin{itemize}
	\item[(i)] A node $A$ in a FD-graph is superfluous with respect to $B$,
	\item[(ii)] $A \equiv B$ and $\I - \{A \imp C \} \models A \imp B$.
\end{itemize}
\end{proposition}

\begin{proof} (i) $\imp$ (ii). If $A$ is superfluous, we have a dotted FD-Path
$\langle A, \ B \rangle$. Since it is dotted, let us remove this node $A$ and
its outgoing arcs. Actually, none of the nodes pointed by dotted arcs of $A$
have been removed, thus we can still find the nodes $a_i$ (attributes of $a$) 
used in the dotted path $\langle A, \ B \rangle$ such that $\bigcup_i a_i 
\models B$. Because $\bigcup_i a_i \subseteq A$, we end up with $\I - \{A \imp 
C \} \models A \imp B$.

\vspace{1.2em}

(ii) $\imp$ (i). In the FD-Graph associated to $\I - \{A \imp C \}$, the node 
$A$ is not present. But still, $A \imp B$ holds. This means that we must be
able to find a list of proper subsets $A_i$ of $A$ (possibly single attributes)
such that $\bigcup_{i} A_i \models B$. Adding $A \imp C$ will add the node $A$
and in particular dotted arcs from $A$ to each attributes of $\bigcup_{i} A_i 
\subseteq A$. Thus, we will have a dotted path from $A$ to $\bigcup_{i} A_i$ and
consequently, to $B$. $A$ is indeed superfluous. Because $B$ is equivalent to 
$A$ by assumption, this property is preserved when adding a node.

\end{proof}

\begin{proposition} \label{prop:maier.equiv_dd_sub} The following statements 
are equivalent:
\begin{itemize}
	\item[(i)] $A\ddv B$,
	\item[(ii)] $\I - E_{\I}(A) \models A \imp B$.
\end{itemize}
	
\end{proposition}

\begin{proof}
Translation of notions. $A \ddv B$ means that if we remove all implications
with left sides equivalent to $A$ we can still derive $A \imp B$ using remaining
implication, that is $\I - E_{\I}(A) \models A \imp B$.
\end{proof}

\begin{proposition} \label{prop:maier.equiv_ssup_dd}
the following statements are equivalent, for $A, B$ bodies of $\I$:
\begin{itemize}
	\item[(i)] $A \ddv B$ and $B \equiv A$,
	\item[(ii)] the node $A$ is superfluous with respect to $B$, and there 
	exists 
	a dotted FD-path from $A$ to $B$ not using any outgoing full arcs of
	nodes equivalent to $A$.
\end{itemize}
	
\end{proposition}


\begin{proof} (i) $\imp$ (ii). Using propositions \ref{prop:maier.equiv_dd_sub},
\ref{prop:maier.equiv_sup_sub}, showing that there is a direct determination 
starting from $A$ implies $A$ is a superfluous node in the FD-Graph is 
straightforward. If $\I - E_{\I}(A) \subseteq \I - \{A \imp C \} \models A \imp 
B$ so does $\I - \{A \imp C \}$. This holds in particular if $B \subseteq A$. 
Moreover, notice that using an outgoing full arc from a node $D$ equivalent to 
$A$ is exactly using an implication with left hand side equivalent to $A$. 
Therefore, if there is not dotted FD-path from $A$ to $B$ not using those arcs, 
we would contradict direct determination.

\vspace{1.2em}

(ii) $\imp$ (i). Suppose $A$ is superfluous and there 
exists a dotted FD-path from $A$ to $B$ not using any outgoing full arcs 
from nodes equivalent to $A$. Those full arcs represent exactly the 
implications contained in $E_{\I}(A)$. Since we don't use them, the path still 
holds in $\I - E_{\I}(A)$ (we would remove compound nodes without outgoing full 
arcs of course, but this would only make the path stops to attributes instead 
of compound node). Having this path in $\I - E_{\I}(A)$ means that $\I - 
E_{\I}(A) \models A \imp B$.
	
\end{proof}

\begin{proposition} \label{prop:maier.equiv_sup_ssup}
	The following two statement are equivalent, given the FD-graph of $\I$:
	\begin{itemize}
		\item[(i)] $A$ is a superfluous node,
		\item[(ii)] $A$ is superfluous with respect to $B$, and there exists a
		dotted path from $A$ to an equivalent node not using any outgoing full 
		arcs from nodes	equivalent to $A$.
	\end{itemize}
	
\end{proposition}

\begin{proof} (ii) $\imp$ (i) is trivial. let's focus on (i) $\imp$ (ii). If 
	$A$ is superfluous, then there exists $B$ such that $A \equiv B$ and there 
	has a dotted path from $A$ to $B$.
	
	\vspace{1.2em}
	
	If $B \subseteq A$, the dotted path is straight forward. If it is not the 
	case, path from $A$ to $B$ uses outgoing full arcs from nodes equivalent to 
	$A$, or it does not. If it does not we are done. Now Suppose this
	path uses an outgoing full arc from a node $C$ equivalent to $A$. This 
	means, that $A \equiv B \equiv C$ by definition and therefore, there exists 
	a
	dotted path from $A$ to $C$ (because we need to reach $C$, so to derive it,
	to use its outgoing arcs). We can reiterate this reasoning until reaching 
	$A$. Getting to $A$ or one of its subset would contradict our assumptions 
	meaning that we stopped finding used equivalent arcs earlier.
	
\end{proof}

From the previous claims, we can yield the following one

\begin{proposition} \label{prop:maier.equiv_sup_dd_sub}
	The following statements are equivalent:
	\begin{itemize}
		\item[(i)] there exists $B \equiv A$ such that $A \ddv B$,
		\item[(ii)] $A$ is superfluous with respect to $B$,
		\item[(iii)] there exists $C$ such that $A \equiv C$ and $\I - \{A \imp 
		D \} \models A \imp C$. (possibly, $C = B$).
	\end{itemize}
\end{proposition}

\begin{proof} (i) $\longleftrightarrow$ (ii) comes from propositions 
	\ref{prop:maier.equiv_ssup_dd} and \ref{prop:maier.equiv_sup_ssup}. (ii) 
	$\longleftrightarrow$ (iii) comes from proposition 
	\ref{prop:maier.equiv_sup_sub}.
	
\end{proof}

Those claims help to see the relation between operations of the algorithms. 
Indeed, the last claim states that finding a direct determination is the same
as finding a superfluous node (under equivalence constraint), therefore the two 
algorithms are looking for
the same structures in different terminologies. This emphasizes the fact they
work on the same computations. Remark: Maier algorithm do it in a special 
order (some kinds of minimal paths in terms of FD-graph. Can we remove them in 
any order? It seems to since this is what the Ausiello algorithm does, and
the previous proposition states that direct determination is equivalent to the
existence of superfluous nodes.

\vspace{1.2em}

The next claim provides an argument in this way. It also explains in what 
extent the operation of removal in those algorithm is exact.

\begin{proposition} Let $A \imp C, B \imp D$ be implications of $\I$. If $A 
\equiv B$ and $\I - \{A \imp C \} \models A \imp B$, then $\I$ is not 
minimum.
	
\end{proposition}

\begin{proof} Let $\I$ be an implication basis with implications $A \imp C, B 
\imp D$, $A \equiv B$, and such that $\I - \{A \imp C \} \models A \imp B$. Let 
us build $\I_c$ by removing $A \imp C$ from $\I$, and replacing $B \imp D$ by 
$B \imp C \cup D$. Obviously, $\I_c \models A \imp B$. Moreover, $A \imp C$
still holds since $A \imp B \imp C \cup D$.
\end{proof}

\vspace{1.2em}

This proposition can be stated as its contraposition. That is, if a basis is
minimum then we cannot find equivalent bodies with one to be removed. This 
states that the second step of Ausiello and Maier algorithms end up with a 
body minimal basis, with help of equivalences proved in proposition 
\ref{prop:maier.equiv_sup_dd_sub}.














