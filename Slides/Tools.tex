% ============================================================================ %
% Section: Math notions
% ============================================================================ %

% subsection: Sets, implications, closure system, functional dependency

\subsection{Elements of set theory}

\begin{frame}
\frametitle{Closure operator and systems}

Set $\Sg$ of attributes. A map $\varphi: 2^{\Sg} \imp 2^{\Sg}$ is a 
\turemp{closure operator} if, $\forall X, Y, Z \subseteq \Sg$:
\begin{itemize}
	\item $X \subseteq \varphi(X)$ \quad \asbemp{(increasing)}
	\item $X \subseteq Y \imp \varphi(X) \subseteq \varphi(Y)$ 
	\quad \asbemp{(isotone)}
	\item $\varphi(\varphi(X)) = \varphi(X)$ \quad \asbemp{(idempotent)}
\end{itemize}

\vspace{1.2em}

Some details:
\begin{itemize}
	% \item ($\Sg$, $\varphi$) is a closure space,
	\item X is \turemp{closed} if $X = \varphi(X)$,
	\item $\Sg^\varphi$ set of closed sets: \turemp{closure system}.
	\item $\Sg^\varphi$ is closed under \turemp{intersection}, contains $\Sg$.
\end{itemize}

\end{frame}

\begin{frame}
\frametitle{Closure Example}

\begin{columns}

\column{0.5\textwidth}



\begin{itemize}
	\item Directed graph $G = (V, E)$.
	\item \turemp{Closure} $\varphi(X)$ of $X \subseteq V$: all the reachable 
	vertices 
	starting from $X$.
	\item $\varphi(\{A, B \}) = \{A, B, C, D \}$.
	\item $\varphi(\{F \}) = \{F \}$, $\{F \}$ is \turemp{closed}.
\end{itemize}


\column{0.5\textwidth}
\begin{figure}[h]
	\centering
	
	\begin{tikzpicture}
	
	\node[Vertex, label=above:{E}] (e1) at (3, 1.5) {};
	\node[Vertex, fill=alizarine, label=left:{A}] (a1) at (2, 0.5) {};
	\node[Vertex, label=right:{D}] (d1) at (4, -0.5) {};
	\node[Vertex, fill=alizarine, label=left:{B}] (b1) at (2, -0.5) {};
	\node[Vertex, label=below:{C}] (c1) at (3, -1.5) {};
	\node[Vertex, fill=emerald, label=below:{F}] (f1) at (4, 0.5) {};
	
	\draw[->] (e1) -- (a1);
	\draw[->] (e1) -- (f1);
	\draw[->] (a1) -- (b1);
	\draw[->] (a1) -- (c1);
	\draw[->] (c1) -- (d1);
	\draw[->] (c1) -- (b1);
	
	\draw[emerald, fill=emerald, fill opacity=0.2] (f1) circle (0.3);
	
	\draw[alizarine, 
		thick,
		rounded corners,
		fill=alizarine, 
		fill opacity=0.21]
	(2, 0.8) -- (2.3, 0.8) -- (2.3, -0.4) -- (3, -1.1) --
	(3.7, -0.4) -- (4, -0.1) -- (4.4, -0.5) -- (3, -1.9) --
	(1.7, -0.6) -- (1.7, 0.8) -- (2, 0.8);
	
	\end{tikzpicture}
	\caption{Closure of a vertex in a directed graph}
\end{figure}
\end{columns}

\end{frame}

\begin{frame}
\frametitle{Implications}

$A, B \subseteq \Sg$. An \turemp{implication} is:
\begin{itemize}
	\item $A \imp B$, $A$ \turemp{premise}, $B$ \turemp{conclusion}.
	\item \turemp{cause/consequence} relation: \midemp{"If we have A, we have 
	B"}.
	\item $M \subseteq \Sg$ \turemp{model} of $A \imp B$ if
		 \[B \subseteq M \lor A \nsubseteq M \]
	denoted $M \models A \imp B$, $A \imp B$ \turemp{follows} from $M$.
\end{itemize}

\vspace{0.5em}

Set of implications $\I$: \turemp{implication system}.
\begin{itemize}
	\item $M \models \I$ if each element of $\I$ follows from $M$,
	\item $\I \models A \imp B$: all models of $\I$ are models of $A \imp B$.
\end{itemize}

\end{frame}


\begin{frame}
\frametitle{Implications and closure}

Given $\I$ an implication system:
\begin{itemize}
	\item closure operator $\I(X)$: \turemp{smallest model} (inclusion wise) of 
	$\I$ containing $X$, $X \subseteq \Sg$. 
	\item Models of $\I$ form a \turemp{closure system}:
		\[ \Sg^{\I} = \{ M \subseteq \Sg \ | \ M = \I(M) \} \]
\end{itemize}

\vspace{1.2em}

\begin{lightreminder}
Important property:
\begin{itemize}
	\item $\I \models A \imp B$ iff $B \subseteq \I(A)$
\end{itemize}
\end{lightreminder}

\end{frame}

\begin{frame}
\frametitle{Small Example}

Let $\I$ be an implication system:
\begin{itemize}
	\item $\Sg = \{ a, \ b, \ c, \ d, \ e \}$,
	\item $\I = \{ ab \imp c, \ bd \imp a, ce \imp abd \}$
\end{itemize}

\vspace{1.2em}

We have:
\begin{itemize}
	\item $\I(b) = b$, $b$ is \turemp{closed}, hence a \turemp{model} of $\I$,
	\item $\I(bd) = abcd$, $bd$ is \turemp{not} a model \quad 
		\asbemp{($abcd$ is)}.
	\item $\emptyset$ is also a model.
\end{itemize}

\end{frame}

\begin{frame}
\frametitle{Applications}

\begin{itemize}
	\item Relational Databases,
	\item Formal Concept analysis,
	\item Conceptual exploration,
	\item Linguistics ?
\end{itemize}
\end{frame}

% subsection: hypergraphs

% subsection: logic

% subsection: open to lattice theory and congruences