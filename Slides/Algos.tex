% ============================================================================ %
% Section: Math notions
% ============================================================================ %

\begin{comment}
1 slide for both approach
define L' base as minimum, equivalent and nonredundant wrt L
\end{comment}

% ==== minimizing algorithms ====

\subsection{Minimizing the input}

\begin{frame}
\frametitle{Algorithms Core}

Two main ideas:
\begin{itemize}
	\item \turemp{minimizing} input system
	\begin{itemize}
		\item[\color{belize} $\triangleright$] algorithms from FCA and databases,
	\end{itemize}
	
	\item \turemp{building} a minimum system against the input
	\begin{itemize}
		\item[\color{belize} $\triangleright$] query learning interpretation, 
	\end{itemize}
\end{itemize}

\vspace{1.2em}

Some notations, given $\I$:
\begin{itemize}
	\item $|\B|$ number of implications, $|\Sg|$ number of attributes,
	\item $|\I| = O(|\B||\Sg|)$, size of $\I$. 
\end{itemize}

\end{frame}

\begin{frame}
\frametitle{First algorithm}

\textsc{MinCover}:
\begin{itemize}
	\item from Day, Wild, (80's),
	
	\vspace{0.8em}
	
	\item two steps:
		\begin{itemize}
			\item[\color{belize} 1.] \turemp{maximize} the conclusion
			\item[ ] \quad {\color{midnight}  $A \imp B$ becomes $A \imp \I(A)$}
			
			\vspace{0.5em}
			
			\item[\color{belize} 2.] \turemp{remove} redundant information
			\item[ ] \quad {\color{midnight}  if $\I(A) = \I^{-}(A)$, $A \imp \I(A)$ is \turemp{redundant}}
		\end{itemize}
	
	\vspace{0.8em}
	
	\item output: \turemp{canonical} base, complexity: $O(|\B||\I|)$.
\end{itemize}

\vspace{1em}

\begin{lightreminder}
	Recall: $\I^{-} = \I - \{ A \imp \I(A) \}$	
\end{lightreminder}

\end{frame}

\begin{frame}
\frametitle{A variation}

\textsc{DuquenneMin}:
\begin{itemize}
	\item variation of \textsc{MinCover} by Duquenne (2007),
	\item three steps:
		\begin{itemize}
			\item[\color{belize} 1.] \turemp{left-saturation} and \turemp{redundancy} elimination
			\item[ ] \quad {\color{midnight}  if $B \subseteq \I^{-}(A)$, $A \imp B$ is useless}
			\item[ ] \quad {\color{midnight}  else $A \imp B$ becomes $ \I^{-}(A) \imp (\I^{-}(A) \cup B)$}
			
			\vspace{0.5em}
			
			\item[\color{belize} 2.] sort implications in $\subseteq$-compatible order (premises)
			
			\vspace{0.5em}
			
			\item[\color{belize} 3.] \turemp{iteratively build} Duquenne-Guigues base
			\item[ ] \quad {\color{midnight} if $\I^{-}(A)$ \turemp{pseudo-closed}, add $\I^{-}(A) \imp \I(A)$ to the result}
		\end{itemize}
	
	\item output: \turemp{canonical} base, complexity: $O(|\B||\I|)$.
	
\end{itemize}
\end{frame}

\begin{frame}
\frametitle{Database approach}

\textsc{MaierMin}:
\begin{itemize}
	\item functional dependency based algorithm, Maier (80's),
	
	\item steps:
		\begin{itemize}
			\item[\color{belize} 1.] \turemp{redundancy} elimination
			\item[ ] \quad {\color{midnight}  if $B \subseteq \I^{-}(A)$,  $A \imp B$ is useless}
			
			\vspace{0.5em}
			
			\item[\color{belize} 3.] \turemp{equivalence classes} reduction
			\item[ ] \quad {\color{midnight} group implications by premises \turemp{closure}}
			\item[ ] \quad {\color{midnight} reduce those classes}
		\end{itemize}

	\item output: minimum basis, complexity: $O(|\B||\I|)$.
\end{itemize}

\end{frame}

% ==== building algorithm ====
\subsection{Building the result}

\begin{frame}
\frametitle{General principle}

Query learning and Angluin algorithm (90's):
\begin{itemize}
	\item aim: learn a theory by constructing an \turemp{hypothesis}
	\item \turemp{oracle} answering \turemp{queries} (questions)
	\item queries:
		\begin{itemize}
			\item[\color{belize} $\triangleright$]
			 {\color{midnight} \turemp{equivalence}: is our hypothesis equivalent to the target?}
			\item[\color{belize} $\triangleright$]
			{\color{midnight} \turemp{membership}: is a set a model of the target?}
		\end{itemize}
	\item improve by \turemp{counterexample}
		\begin{itemize}
			\item[\color{belize} $\triangleright$]
			{\color{midnight} \turemp{positive}: model of the target, not of the hypothesis}
			\item[\color{belize} $\triangleright$]
			{\color{midnight} \turemp{negative}: model of the hypothesis, not
			of the target}
		\end{itemize}
\end{itemize}


\end{frame}


%\begin{frame}
%\frametitle{In our case}

%\begin{itemize}
%	\item no oracle,
	
	%\vspace{1.2em}
	
	%\item no positive counter-example,
	
	%\vspace{1.2em}
	
	%\item premises are sufficient to have negative counter-example.
%\end{itemize}


%\end{frame}

\begin{frame}
\frametitle{AFP-Based Algorithm}

\textsc{AFP-Based}:
\begin{itemize}
	\item derived from Angluin, \aliemp{no proof yet},
	\item principle:
		\begin{itemize}
			\item[\color{belize} 1.] take implications one by one,
			\item[\color{belize} 2.] refine the hypothesis:
			\item[ ] \quad {\color{midnight}  use \turemp{premises} to generate possible counter-examples}
			\item[ ] \quad {\color{midnight} add right-closed implications
				or refine old ones}
			
		\end{itemize}
	
	\item expected output: \turemp{canonical} base,
	\item \aliemp{idea} of complexity: $O(|\B|^3|\I|)$.

\end{itemize}

\end{frame}

\begin{frame}
\frametitle{Using minimality constraint}

\textsc{BercziMin}:
\begin{itemize}
	\item logic based, Berczi, 2017
	\item principle:
		\begin{itemize}
			\item[\color{belize} 1.] \turemp{build} a basis $\I_c$ against the input $\I$
			
			\item[\color{belize} 2.] repeat minimality selection up to \turemp{equivalence}
			
			\item[\color{belize} 3.] \turemp{minimality selection}:
			
			\item[ ] \quad { \color{midnight} select the next minimal negative counter-example $A$},
			
			\item[ ] \quad { \color{midnight} add $A \imp \I(A)$ to $\I_c$}
			 
		\end{itemize}
	\item output: \turemp{canonical} base, complexity: $O(|\B|^2|\I|)$.
\end{itemize}

\end{frame}


% subsection: hypergraphs

% subsection: logic

% subsection: open to lattice theory and congruences