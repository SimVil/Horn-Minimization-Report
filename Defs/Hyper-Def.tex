\section{Directed graphs and Hypergraphs}

\begin{definition}[\belemp{Hypergraph}] An \belemp{hypergraph} is a pair $H = 
	(V, E)$ 
	where $V$ is a set of vertices (as in a graph) and $E$ a set of subsets of 
	$V$
	describing hyperarcs.
	
\end{definition}

\noindent In fact, hypergraphs are an extension of graphs. 


\begin{definition}[\belemp{Directed Hypergraph}] A \belemp{directed hypergraph} 
	$H = (V, E)$ is a pair with $V$ a set of vertices (nodes) and $E$ a set of 
	elements of $2^V \times 2^V$, denoting edges going from a subset of $V$ to 
	another subset of $V$.
	
\end{definition}

Directed hypergraphs are useful to graphically represent implication basis.

\paragraph{Example} Let us consider the following implication basis $\I$:

\[ \I = \{ 1 \imp 2, 2 \imp 34, 3 \imp 12, 41 \imp 3 \} \]

\noindent Then we can define an hyper graph $L = (V, E)$ where
\begin{itemize}
	\item $V = \{1, 2, 3, 4 \}$
	\item $E = \{ (\{ 1 \}, \{ 2\}), (\{ 2 \}, \{3, 4\}),
	(\{ 3 \}, \{ 1, 2\}), (\{1, 4\}, \{ 3\}) \}$
\end{itemize}

\noindent To be clearer, we can see the graphical representation of this figure
in \ref{fig:DHyp1}

\begin{center}
	\begin{figure}
\begin{center}
\begin{tikzpicture}

\node[Vertex, label=left:{4}] (4) at (-1, 1) {};
\node[Vertex, label=above left:{1}] (1) at (1, 1) {};
\node[Vertex, label=below left:{2}] (2) at (-1.5, -1) {};
\node[Vertex, label=below right:{3}] (3) at (0.5, -1) {};

% 41 --> 3
\draw[<-, color=belize] (0.5, -0.8) to[bend right=18] (-0.8, 0.8);
\draw[<-, color=belize] (0.5, -0.8) to[bend left=28] (0.8, 0.8);

% 1 --> 2
\draw[<-, color=emerald] (-1.3, -0.8) -- (0.8, 0.8);

% 2 --> 34
\draw[->, color=amethyst] (-1.3, -0.8) to[bend left=45] (0.3, -0.8);
\draw[->, color=amethyst] (-1.3, -0.8) to[bend right=45] (-1, 0.8);

% 3 --> 12 
\draw[->, color=alizarine] (0.3, -0.8) to[bend right=60] (-1.3, -0.8);
\draw[->, color=alizarine] (0.3, -0.8) to[bend left=50] (0.8, 0.8);


\end{tikzpicture}
\end{center}

\caption{Example of Directed Hypergraph}
\label{fig:DHyp1}
\end{figure}



\begin{comment}
\begin{figure}[ht]
\begin{center}
\begin{tikzpicture}
\node[Input Node, minimum size=0.6cm] (x) at (-4, 0) {$x$};
\node[Hidden Node, opacity=0.65, minimum size=0.6cm] (h11) at (-2, 1) {$h_1$};
\node[Hidden Node, opacity=0.65, minimum size=0.6cm] (h12) at (-2, -1) {$h_2$};
\node[Hidden Node, opacity=0.65, minimum size=0.6cm] (h21) at (0, 1) {$h_3$};
\node[Hidden Node, minimum size=0.6cm] (h22) at (0, -1) {$h_4$};
\node[Output Node, minimum size=0.6cm] (o) at (2, 0) {$o$};
\node (y) at (4, 0) {$\hat{y}$};

\draw[->] (x) -- node[midway, above] {$u_1$} (h11);
\draw[->] (x) -- node[midway, above] {$u_2$} (h12);
\draw[->] (h11) -- node[midway, above] {$w_{1, 1}$} (h21);
\draw[->] (h11) -- node[sloped, near end, below] {$w_{2, 1}$} (h22);
\draw[->] (h12) -- node[sloped, near end, above] {$w_{1, 2}$} (h21);
\draw[->] (h12) -- node[midway, below] {$w_{2, 2}$} (h22);
\draw[->] (h21) -- node[midway, above] {$v_1$} (o);
\draw[->] (h22) -- node[midway, below] {$v_2$} (o);
\draw[->] (o) -- (y);


\draw (5,0) node{A} to[bend right=30]
node[very near start]{R}    node[pos=0.7]{S}  (9,2) node{B};

\end{tikzpicture}
\end{center}

\caption{Schéma d'un réseau de neurones pour la rétro-propagation}
\label{fig:NN-BP}
\end{figure} 


\end{comment}
\end{center}