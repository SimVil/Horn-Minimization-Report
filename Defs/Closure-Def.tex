\section{Order theory, Closure systems pov}

In this part, we will be considering definitions of set theory, with relation to
closure systems. The reader may refer to \cite{CExp}, \cite{Lat&Ord} for more 
details and further definitions.

\begin{definition}[\belemp{Closure operator}, \belemp{closure system}] Let
$\Sigma$ be a set, and define $\phi : \Sigma \longrightarrow \Sigma$ an
application. $\phi$ is a \belemp{closure} if it has the three following
properties for all $X, Y \subseteq \Sigma$:
\begin{itemize}
	\item[(i)] $X \subseteq \phi(X)$ (\midemp{extensive})
	\item[(ii)] $X \subseteq Y \longrightarrow \phi(X) \subseteq \phi(Y)$ 
		(\midemp{monotone})
	\item[(iii)] $\phi(\phi(X)) = \phi(X)$ (\midemp{idempotent})
\end{itemize}

\noindent Then, the pair $(\Sigma, \phi)$ is called a \belemp{closure system}.
	
\end{definition}


\begin{definition}[\belemp{Closed set}] Let $(\Sigma, \phi)$ be a closure 
system. A subset $X$ of $\Sigma$ is a \belemp{closed set} (with respect to 
$\phi$) if $\phi(X) = X$. We will denote by $\Sigma_{\phi}$ the set of all 
closed sets of $(\Sigma, \phi)$, that is:
	
	\[ \Sigma_{\phi} = \left\{ X \subseteq \Sigma \; | \; \phi(X) = X 
	\right\}  \]

\noindent and $\Sigma_{\phi}$ has the following properties:
\begin{itemize}
	\item[(i)] $\Sigma \in \Sigma_{\phi}$,
	\item[(ii)] if $X, Y \in \Sigma_{\phi}$, so does $X \cap Y$ 
		(\midemp{$\Sigma_{\phi}$ is closed under intersection}).
\end{itemize}
	
\end{definition}

\noindent Note that a closure system can be characterized either by its closure
operator, or by its set of closed sets. In other words, we can derive 
$\Sigma_{\phi}$ from $\phi$, as $\phi$ from $\Sigma_{\phi}$. The closed set 
associated to $X$ is the smallest closed set containing $X$, i.e:

	\[ \phi(X) = \bigcap \{Y \in \Sigma_\phi \; | \; X \subseteq Y \} \]

\noindent Since $\Sigma_{\phi}$ is closed under intersection, the resultant of
the intersection is also a closed set. The set of closed sets ordered by 
inclusion defines a complete lattice as defined previously. 


\paragraph{Example} Let $\Sigma = \llbracket 1 \; ; \; 4 \rrbracket$ and 
$\phi(X) = X \cup \{4 \}$. The pair $(\Sigma, \phi)$ is a closure system whose
closed sets are all the subsets containing 4. Let us prove that $\phi$ is indeed
a closure operator:
\begin{itemize}
	\item[(e)] for all subsets $X$ of $\Sigma$, either $4 \in X$ and 
	$\phi(X) = X \cup \{4 \} = X$ so $X \subseteq \phi(X)$, either $4 \notin X$
	and $X \subseteq X \cup \{4 \} = \phi(X)$.
	\item[(m)]
	\item[(i)] by definition $\phi(X)$ contains 4, so it will not be
	changed by applying $\phi$ another time.
\end{itemize}

\noindent Another interesting definition of $\Sigma_{\phi}$ is:

	\[ \Sigma_{\phi} = \{ \phi(X) \; | \; X \subseteq \Sigma \} \]

\noindent Then, in our example:

	\[ \Sigma_{\phi} = \{ \{ 4\}, \{ 1, 4\}, \{ 2, 4\},
		\{ 3, 4\}, \{ 1, 2, 4\}, \{ 1, 3, 4\},\{ 2, 3, 4\}, 
		\{ 1, 2, 3, 4\} \}
	\]
	
\noindent The lattice $(\Sigma_{\phi}, \subseteq)$ is:



	 
\section{Implication Basis}

\begin{definition}[\belemp{Implication}] Let $\Sg$ be a set, and $A, B 
\subseteq 
\Sg$. An \belemp{implication} over $\Sg$ is a relation/pair between $A$ and 
$B$, denoted by $A \imp B$. $A$ is the \belemp{premise}, $B$ the 
\belemp{conclusion}. 	
\end{definition}

\begin{definition}[\belemp{Model of an implication}] A subset $M \subseteq 
\Sg$ is a model of $A \imp B$, written $M \models A 
\imp B$ if $B \subseteq M \lor A \nsubseteq M$ (think of logical 
approach).
\end{definition}

\begin{definition}[\belemp{Model of an implication basis}] Let $\I$ be
a set of implication (or an \midemp{implication basis}) over $\Sg$. A subset
$M \subseteq \Sg$ is a model of $\I$ if $M \models A \imp
B$ for each $A \imp B$ in $\I$. It is denoted by $M \models \I$.
\end{definition}

The set of models of $\I$ over $\Sg$ describes a closure system $(\Sg, \phi)$. 
Then, An implication $A \imp B$ holds in $\I$ iff $B \subseteq \phi(A)$. 
Conversely, if $(\Sg, \phi)$ is a closure system, then we can define an 
implication basis $\I$ such that the models $M(\I)$ of $\I$ are the closed sets
$\Sg_\phi$ of $(\Sg, \phi)$. An implication $A \imp B$ \belemp{semantically 
follows} from $\I$ if for all $M \subseteq \Sg$ s.t $M \models \I$, we also have
$M \models A \imp B$. Then we write $\I \models A \imp B$.

\begin{definition}[\belemp{Completeness}] $\I$ over $\Sg$ is called 
\belemp{complete} with respect to a closure system if every implication holding
in the closure system follows from $\I$.
	
\end{definition}

Given $\I$, we denote by $\I(M)$ the closure of $M$ in $\I$, that is the 
smallest model of $\I$ containing $M$.

\begin{definition}[\belemp{Properties of $\I$}] Let $\I$ be a set of 
implications over $\Sg$, and $(\Sg, \phi)$ be a closure system. $\I$ is:
\begin{itemize}
	\item[(i)] \belemp{sound} if each implications of $\I$ holds in $\Sg_\phi$
	\item[(ii)] \belemp{complete} if each implications holding in $\Sg_\phi$ 
	follows from $\I$
	\item[(iii)] \belemp{nonredundant} if no implication in $\I$ follows from 
	other implications of $\I$.
\end{itemize}
	
\end{definition}

So our goal is to find such an implication basis which concentrates all possible
knowledge in the minimal number of implications. Finding (and minimizing!) such
an implication set can be very Time-Consuming.

\begin{definition}[\belemp{Pseudo-closed set}] Let $(\Sg, \phi)$ be a closure 
system. A subset $M \subseteq \Sg$ is \belemp{pseudo-closed} iff 
\begin{itemize}
	\item $\phi(M) \neq M$ ($M$ is not closed),
	\item if $Q \subset M$ is pseudo-closed, then $\phi(Q) \subseteq M$.
\end{itemize}

\end{definition}

Note that the empty set $\emptyset$ is pseudo-closed iff it is not closed. One 
can also define pseudo-closed set with help of \midemp{quasi-closed} sets. We
can prove that a pseudo-closed set is some kind of discontinuous structure 
within a lattice.

\begin{definition}[\belemp{Duquenne-Guigues Basis}] Let $(\Sg, \phi)$ be a 
closure system. The \belemp{Duquenne-Guigues} basis or \belemp{canonical} basis
$\I$ is:

	\[ \I = \{ M \imp \phi(M) \; | \; M \subseteq \Sg, \, M \;
	 	\text{pseudo-closed} \} \]
	
\noindent and $\I$ is complete, sound and nonredundant.
\end{definition}


\begin{definition}[\belemp{Preclosed set}] Given a closure system $(\Sg, \phi)$,
a subset $M \subseteq \Sg$ is \belemp{preclosed} iff it contains the closure of
all its proper preclosed subsets.
	
\end{definition}