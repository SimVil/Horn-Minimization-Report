\section{Horn Minimization task}

Previously we settled down mathematical tools we shall use. In the next 
paragraphs we will focus on describing "with hands" the problem of Horn 
minimization in our sense. Here we suppose $\I$ is in so-called \belemp{reduced
form}, that is for all distinct implications $A \imp B, C \imp D$ of $\I$,
$A \neq C$. That is we do not have distinct implications with same bodies. This
said, we state one of the \aliemp{most important} of our problem.

\begin{definition}[\midn{Minimality of $\I$}] Let $\I$ be a set of 
implications over $\Sg$, and $(\Sg, \phi)$ be a closure system. $\I$ is:
\begin{itemize}
	\item[(i)] \belemp{sound} if each implications of $\I$ holds in 
	$\Sg_\phi$
	\item[(ii)] \belemp{complete} if each implications holding in 
	$\Sg_\phi$ follows from $\I$
	\item[(iii)] \belemp{nonredundant} if no implication in $\I$ follows 
	from other implications of $\I$.
\end{itemize}
\noindent An implicational basis with such properties is called 
\belemp{minimal}.
\end{definition}

Note that here, minimality is defined with relation to a closure system. In
our case we will just be given $\Sg$ and $\I$, without an associated $(\Sg, 
\phi)$. Therefore, our basis will automatically be sound and complete because
the closure system we will consider is the set of models of $\I$. Hence, we
can provide a simpler definition of minimality (which is in fact nonredundancy):

\begin{definition}[\midn{Mimimality}] A reduced implicational basis $\I$ is
said \belemp{minimal} if we cannot remove any implication without altering
its closure.
	
\end{definition}


As we shall see later, there exists several definition of minimality. Our 
point of view could also be called \belemp{body minimality}. More intuitively, 
our aim is to concentrate all the knowledge in a minimal number of implications.

\paragraph{Example} Let us recall our music example. We had music styles:
\textit{shoegaze, electronic, coldwave, pop, rock, dream-pop, jazz, 
experimental, atmospheric, contemporary jazz} and some implications:
\begin{itemize}
	\item[ ] \textit{jazz, experimental} $\imp$ \textit{contemporary jazz},
	\item[ ] \textit{coldwave, rock}  $\imp$ \textit{shoegaze},
	\item[ ] \textit{atmospheric} $\imp$ \textit{experimental, electronic},
	\item[ ] \textit{shoegaze, pop} $\imp$ \textit{dream-pop}.
\end{itemize}
\noindent Unfortunately, because all the bodies are disjoint, this basis is 
already minimal. But let's imagine we have the following implications instead:
\begin{itemize}
	\item[ ] \textit{coldwave} $\imp$ \textit{rock},
	\item[ ] \textit{shoegaze}  $\imp$ \textit{rock, coldwave, dream-pop},
	\item[ ] \textit{shoegaze, dream-pop} $\imp$ \textit{rock, coldwave},
	\item[ ] \textit{rock, dream-pop} $\imp$ \textit{atmospheric, shoegaze, 
	coldwave}.
\end{itemize}
\noindent Here, the third implication can be removed. Indeed, from tags 
\textit{shoegaze, dreampop} following the implications we can reach 
\textit{rock, coldwave, atmospheric}. But say we remove the third implication,
and thus keep:
\begin{itemize}
	\item[ ] \textit{coldwave} $\imp$ \textit{rock},
	\item[ ] \textit{shoegaze}  $\imp$ \textit{rock, coldwave, dream-pop},
	\item[ ] \textit{rock, dream-pop} $\imp$ \textit{atmospheric, shoegaze, 
	coldwave}.
\end{itemize}
What can we reach starting from \textit{shoegaze, dreampop}? Using the second
implication we get \textit{rock, coldwave}, and because we have \textit{rock, 
dreampop}, we also get \textit{atmospheric}. That is, we can get the same 
knowledge than in the previous case but with one less implication.

\vspace{0.5em}

Let us present a less practical but more visual example. Suppose $\Sg = \{ a, 
\  s,\  c, \  r, \  d\}$
and $\I$ as follows:

	\[ c \imp r, \; s \imp rcd, \; rd \imp asc, \; sd \imp rc \]

\noindent This basis is not minimal. Indeed, let us remove the fourth 
implication and call $\I^{-}$ the new basis. We want to know whether the 
implication we removed still holds in $\I^{-}$ so that its closure is kept (and
therefore its "knowledge"). To show that $\I^{-} \models sd \imp rc$, it is
enough to show that $rc \subseteq \I^{-}(sd)$ (see proposition 
\ref{prop:def.equiv_imp_clos}). Because we have $s \imp rcd$, then $rd \imp 
asc$ we conclude that $\I^{-}(sd) = acdrs$. Thus $\I^{-} \models sd \imp rc$ 
and $\I^{-}$ is smaller than $\I$, with $\I^{-} \equiv \I$. Also, $\I^{-}$ is 
minimal.

\vspace{0.5em}

Note that the latter example is in fact the same as the musical one. Take the 
first letter of each style and we end up with the basis described in the second
paragraph.


\vspace{1.2em}

To go in details, we will introduce some elements and objects related to body 
minimality, namely \belemp{Pseudo-closed sets} and \belemp{Duquenne-Guigues 
basis}.

\begin{definition}[\midn{Pseudo-closed set}] Let $(\Sg, \phi)$ be a closure 
	system. A subset $M \subseteq \Sg$ is \belemp{pseudo-closed} iff 
	\begin{itemize}
		\item $\phi(M) \neq M$ ($M$ is not closed),
		\item if $Q \subset M$ is pseudo-closed, then $\phi(Q) \subseteq M$.
	\end{itemize}
	
\end{definition}

Note that the empty set $\emptyset$ is pseudo-closed if and only if it is not 
closed. Also, one can note that if $P$ is pseudo-closed and $P'$ is covered
(inclusion wise, in terms of partial ordering) then $P'$ cannot be 
pseudo-closed. For the next definition, one can refer to 
\cite{b._ganter_conceptual_2016, guigues_familles_1986}.

\begin{definition}[\midn{Duquenne-Guigues Basis}] Let $(\Sg, \phi)$ be a 
closure system. The \belemp{Duquenne-Guigues} basis or \belemp{canonical} 
basis $\I$ is:

	\[ \I = \{ M \imp \phi(M) \; | \; M \subseteq \Sg, \, M \;
		\text{pseudo-closed} \} \]

\noindent and $\I$ is complete, sound and nonredundant.
\end{definition}

\noindent Thus, the Duquenne-Guigues basis is body-minimal. In this section 
we presented the minimization problem we will stick to for all this report. 
Another remark, as we shall see in the next chapter, this problem is solvable
in polynomial time
