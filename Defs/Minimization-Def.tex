\section{Horn Minimization task}

Previously we settled down mathematical tools we shall use. In the next 
paragraphs we will focus on describing "with hands" the problem of Horn 
minimization in our sense. 

\begin{itemize}
	\item rappeler closure systems tout ça
	\item notre minimalité c'est body-minimality
	\item Pseudo-closed sets
	\item Duquenne-Guigues basis
	\item exemple musique
\end{itemize}

Duquenne Guigues basis, redundancy, completeness, etc. Define our meaning of
minimality (in terms of bodies). 

\begin{definition}[\belemp{Completeness}] $\I$ over $\Sg$ is called 
	\belemp{complete} with respect to a closure system if every implication 
	holding
	in the closure system follows from $\I$.
	
\end{definition}

Given $\I$, we denote by $\I(M)$ the closure of $M$ in $\I$, that is the 
smallest model of $\I$ containing $M$.

\begin{definition}[\belemp{Properties of $\I$}] Let $\I$ be a set of 
	implications over $\Sg$, and $(\Sg, \phi)$ be a closure system. $\I$ is:
	\begin{itemize}
		\item[(i)] \belemp{sound} if each implications of $\I$ holds in 
		$\Sg_\phi$
		\item[(ii)] \belemp{complete} if each implications holding in 
		$\Sg_\phi$ 
		follows from $\I$
		\item[(iii)] \belemp{nonredundant} if no implication in $\I$ follows 
		from 
		other implications of $\I$.
	\end{itemize}
	
\end{definition}

So our goal is to find such an implication basis which concentrates all possible
knowledge in the minimal number of implications. Finding (and minimizing!) such
an implication set can be very Time-Consuming.

\begin{definition}[\belemp{Pseudo-closed set}] Let $(\Sg, \phi)$ be a closure 
	system. A subset $M \subseteq \Sg$ is \belemp{pseudo-closed} iff 
	\begin{itemize}
		\item $\phi(M) \neq M$ ($M$ is not closed),
		\item if $Q \subset M$ is pseudo-closed, then $\phi(Q) \subseteq M$.
	\end{itemize}
	
\end{definition}

Note that the empty set $\emptyset$ is pseudo-closed iff it is not closed. One 
can also define pseudo-closed set with help of \midemp{quasi-closed} sets. We
can prove that a pseudo-closed set is some kind of discontinuous structure 
within a lattice.

\begin{definition}[\belemp{Duquenne-Guigues Basis}] Let $(\Sg, \phi)$ be a 
	closure system. The \belemp{Duquenne-Guigues} basis or \belemp{canonical} 
	basis
	$\I$ is:
	
	\[ \I = \{ M \imp \phi(M) \; | \; M \subseteq \Sg, \, M \;
	\text{pseudo-closed} \} \]
	
	\noindent and $\I$ is complete, sound and nonredundant.
\end{definition}


\begin{definition}[\belemp{Preclosed set}] Given a closure system $(\Sg, \phi)$,
	a subset $M \subseteq \Sg$ is \belemp{preclosed} iff it contains the 
	closure of
	all its proper preclosed subsets.
	
\end{definition}