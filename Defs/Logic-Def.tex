\section{Propositional logic and Horn formulas}

This section is dedicated to the introduction of some propositional logic 
notations and notions. Again, we assume the reader has some background in
propositional logic anyway (we are not going to cover the meaning of 
disjunction, conjunction and so forth). The reader can refer to 
\cite{cori_mathematical_2000} for an introduction out of our scope.

\vspace{1.2em}

Before going into definitions, we set up some notations. Let us denote by
$\Sg$ the set of propositional variables. Disjunction is written with $\lor$,
conjunction with $\land$, and negation $\lnot$. Truth values are 0 (resp. 
$\bot$) and 1 (resp. $\top$).

Our aim is to build so called Horn formulas. To this purpose we must first 
introduce what we call clauses, and Horn clauses. 

\begin{definition}[clause] Let $x_1, \ dots, \ x_n$, $n \in \Ens{N}$
be variables of $\Sg$. A \belemp{clause} $\mathcal{C}$ over $x_i$'s is a 
\belemp{disjunction} of literals $p_i$:

	\[ \mathcal{C} = \bigvee_{i = 1}^{n} p_i \]

\noindent where $p_i \in \{x_i, \lnot x_i \}$.
\end{definition}

\begin{definition}[(pure) Horn clause] A clause $C$ over variables of $\Sg$
is said \belemp{Horn} (resp. \belemp{pure Horn}) if it contains at most (resp. 
exactly one) positive literal.

\end{definition}

\paragraph{Example} To clarify our idea let us give a simple example. Let $x_1,
x_2, \dots, x_n$ be boolean variables. We have the following:
\begin{itemize}
	\item $\emptyset$ is a clause (true),
	\item $(x_1 \lor x_2 \lor \lnot x_3 \lor \lnot x_4)$ is a clause,
	\item $(\lnot x_1 \lor \lnot x_2)$ is a Horn clause,
	\item $(\lnot x_1 \lor \lnot x_2 \lor x_3)$ is pure Horn.
\end{itemize}


\vspace{1.2em}

TO begin to draw a link with the previous section, one can note that we can 
write Horn clauses with disjunction, or with logical implication $\imp$. Indeed
remind that ($x_1 \lor \lnot x_2 $) can be equivalently rewritten as $x2 \imp 
x_1$. Hence, the Horn clauses defined in the previous examples become:
\begin{itemize}
	\item $(x_1 \land x_2) \imp \bot$,
	\item $(x_1 \land x_2) \imp x_3$.
\end{itemize}


\begin{definition}[\belemp{Horn formula}]
	
\end{definition}

\begin{definition}[\belemp{semantic entailment}]
	
\end{definition}