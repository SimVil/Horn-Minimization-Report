\section{Measuring performance}

This section is dedicated to explain the way we measure and compare algorithms
implementation. We give the tools we use, and the way we use them. Of course,
the main point is to compare them on time to see how do they react on various 
cases, against theoretical complexity. 

\subsection{Timer, profiler}

We use two main tools, timers and profilers. The first one aims to precisely 
time execution of some functions, as much as comparing wall-clock time and 
CPU-time spent on a program. The profiler depicts a recursive analysis of time
spent in each function, resulting in a call-graph, a tree-like structure. 
Profiling a program helps to understand bottlenecks. In our case, we can use it
to see where does an algorithm is the most time-consuming, and possibly why.

\subsubsection{Boost timer}

Since the structure of the code relies on \lil{boost} library. We use the 
\lil{boost_timer} library to time programs. Few details though. The 
\lil{boost_timer} dependency is not header-only, which means it must be compiled
to be used and linked at compilation time. see boost doc. 

\subsubsection{\lil{gperftools} CPU-profiler}

\lil{gperftools} is originally provided by Google but has been left to 
community since 2012. It contains tools such as heap-profiler, faster 
implementation of \lil{malloc} and a CPU-profiler, which is our interest. It
uses sampling method to determine time spent in functions scopes. Probably 
because of the CPU frequency (2.50 GHz), we cannot set a sampling frequency
higher than 250 samples per second, that is 1 sample each 4 millisecond. Thus,
the time results of the profiler are not more precise than 10e-3. This is the
reason why we also use timer to get precise execution time. But as far as we
are concerned, having this precision for profiling is enough for our purpose 
of examining bottlenecks. 

\lil{gperftools} is quite easy to use, and does not lead to runtime overhead
contrary to \lil{callgrind/valgrind} even though the latter one counting 
assembly instructions is more precise.