\section{Code we are starting from}

The code uses \lil{boost} library for c++. Especially bitsets. The file 
\lil{config.h} defines the underlying structure we rely on (either 
\lil{boost} or \lil{bitmagic}). Sets are represented as bitmaps. The 
corresponding class is \lil{FCA::BitSet} (we use the namespace \lil{FCA}).
Since the code is not documented, we will provide few remarks to ease
understanding.

Here is a list with short description of file construction:
\begin{itemize}
	\item \lil{config.h}: defines which library to use for bitmaps,
	\item \lil{biset.h/.cpp}: set representation as bitmaps
	\item \lil{definition.h}: two operations:
		\begin{enumerate}
			\item \lil{IsPrefixIdentical}: check equality of sets up to $n-th$
			element,
			\item \lil{Convert}: convert a bitset to its string representation
			(provided an initial attributes names set)
		\end{enumerate}
	\item \lil{implications.h/.cpp}: defines implications. One \lil{Implication}
	class and one structure \lil{ImplicationInd}. \lil{Implication} is string
	base, \lil{ImplicationInd} bitset based. Add some conversion methods (bitset
	to string, change attribute set).
	
	\item \lil{datastructure.h}: gathers implemented datastructure to lighten
	inclusions (act as a package).
	
	\item \lil{Closure.h/.cpp}: closure is a class (with only one method). In
	the \lil{apply} function, the \lil{if (false) return false} bloc after 
	modification of NewClosure (|=) stands for testing whether the given set
	(\lil{current} is closed or not). The method has thus two uses: computing
	the closure AND determining whether a set is closed or not.
	
	\item \lil{LinClosure.h/.cpp}: two methods \lil{Apply} (overloading). In the
	first one, the \lil{bool} vector \lil{use} aims to model $update \cup add$
	because they are not the same structure in code. Thus it ensures that we do
	not add an element already treated.
	
	\item \lil{LinClosure_Improved.h/.cpp}: Wild's Closure.
	
\end{itemize}