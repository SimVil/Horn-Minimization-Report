\chapter{Introduction to implications through closure systems}

In this first chapter, we will be involved in presenting our topic of 
minimization. For this ground to be understandable by as much readers as 
possible, we will heavily rely on toy examples to illustrate and provide 
intuition on the various notions we will introduce. To be more precise on the
path we are about to follow in this chapter, we are first to expose an informal
small example of the task we want to achieve. Then, we shall investigate the
history of research on our topic, to act as an exposition of the actual 
knowledge on the question and to give a context to our study. For the rest of
this chapter we will get familiar with mathematical objects called 
\midemp{closure operators} and \midemp{closure systems} modelling our problem.
As we shall observe, the topic of minimization can be described in several 
mathematical frameworks. However, even if we describe briefly other objects
in next chapters, we will stick to our closure framework in all the report in
order to have a leading light among various different terminologies.

\section{Implications and minimization: first meeting}

Let us imagine we are some specialist of flowers and plants in general. As such,
we are interested in studying \belemp{correlations} between plant 
characteristics. Some possible traits are: \textit{colourful, bloom, wither, 
aquatic, seasonal, climbing, scented, flower, perennial} and so forth. Having
observed countless plants during our studies, we are able to draw relations
among all those \belemp{attributes}. For instance, we know that a plant having
the attribute \textit{flower} is likely to have traits \textit{scent, bloom, 
wither} while a plant being \textit{perennial} (i.e: does not need a lot of
water to survive, like a cactus) is not likely to be \textit{aquatic}. 

\vspace{1.2em}

Those relations \textit{"if we have some attributes, we get those ones too"}
depict correlation between attributes (not cause/consequence!). It is important
to stress on the knowledge those relations bring. They just indicate that 
whenever we have say \textit{flower}, we have also \textit{colourful}. This is 
very different from saying that \textit{because} some plant is a flower, it 
will be colourful. We call those correlation relations \belemp{implication}
and use $ flower \imp colourful$ to denote \textit{"if we have the attribute
flower, then we have colourful"}. Now let us give some implications:

\begin{center}
	\textit{(colourful, bloom $\imp$ seasonal), (colourful, wither $\imp$ 
		seasonal), (bloom $\imp$ wither)}
\end{center}

\noindent All those implications represent a certain amount of knowledge. While
in our example they are not numerous we could imagine having tons of them. Hence
we would wonder whether there is a way to reduce the number of implications 
while keeping all the knowledge they represent. This question is 
\belemp{minimization}. Actually, in our small example we can reduce
the number of implications. Take \textit{(colourful, bloom $\imp$ seasonal)}. 
We can derive this implication relation only with the two other 
ones. Indeed, because a plant \textit{blooming} is likely to \textit{wither} 
(3rd implication), we have \textit{(colourful, bloom $\imp$ wither)}, but since 
we now have \textit{wither} and \textit{colourful} we also have 
\textit{seasonal} (2nd implication). That is,
the implication \textit{(colourful, bloom $\imp$ seasonal)} is useless (or 
\belemp{redundant}) in our context and can be removed. Our set of implications 
will then be smaller, but pointing out the same relations as before. 

\vspace{1.2em}

To summarize, we have seen that out of a set of \belemp{attributes} we can draw
several relations called \belemp{implications} providing some knowledge. We also
realized that sometimes, some implications are not necessary. Consequently, 
the set of implications we are given can be \belemp{minimized} without 
altering the information it contains. This is the topic we were interested 
during this master thesis. In the next section, we will trace back the overhaul
knowledge on this question.

\section{Research on implications theories minimization}

This section is intended to supply the reader with a general overview of the
minimization topic. After a short contextual information, we focus on some 
relevant results on the question by providing references to algorithms and 
properties dedicated to our problem. Eventually, we situate our work within this
context.

\vspace{1.2em}

The question of minimization has been discussed and developed through various 
frameworks, and several computer scientists communities. Notice that in order 
not to make this synthesis too long, we will stay within the context of 
minimization and will not trace the field of implication theories in general. 
For a survey of this domain anyway, the reader should refer to 
\cite{wild_joy_2017}. Also, note that minimality in general terms is not 
unique. Indeed, one can define several type of minimality among implication 
systems. For instance, not only we can define minimality with respect to the 
number of implication within a system (which is our interest) but also with 
respect to the number of attributes in each implications. The former one is 
called \textit{canonical} in relational database field, and \textit{hyperarc 
	minimum} within the graph context. Especially in the graph-theoretic and 
boolean logic settings, one can derive more types of minimality. For general 
introduction to boolean logic notations, we invite the reader to see 
\cite{cori_mathematical_2000}. In terms of propositional logic, implications 
are represented through Horn formulae. Interestingly, the minimization problem 
we are going to consider is the only one being polynomial time solvable. Other 
problems are proved to be NP-Complete or NP-Hard. For more discussion on other 
minimality definitions and their computational complexity, the reader should 
refer to \cite{boros_strong_2017, ausiello_directed_2017, 
	b._ganter_conceptual_2016, ausiello_minimal_1986, wild_joy_2017, 
	boros_horn_1998}. In particular for NP-Completeness in the canonical case, 
	one 
can see \cite{hammer_optimal_1993}. In subsequent explanations, we will refer 
to minimization with respect to the number of implications.

\vspace{1.2em}

To the best of our knowledge, the two first fields in which algorithms and 
properties of minimality arose are Formal Concept Analysis (FCA) (see 
\cite{ganter_formal_1999, 
	ganter_two_2010} for an introduction) and Database Theory (DB) (see 
\cite{maier_theory_1983}). Both sides were developed independently in the early 
80's. For the first domain, characterization of minimality goes to Duquenne and 
Guigues \cite{guigues_j.l_familles_1986}, in which they describe the so-called 
\textit{canonical basis} (also called \textit{Duquenne-Guigues basis} after its 
authors) relying on the notion of pseudo-closed sets. For the database part, 
study of implications is made by Maier through FD's (\cite{maier_theory_1983, 
	david_minimum_1980}). The polynomial time algorithm he gives for 
	minimization 
heavily relies on a fast subroutine discovered by Beeri and Bernstein in 
\cite{beeri_computational_1979}, 1979.

\vspace{1.2em}

From then on, knowledge increased over years and spread out over domains. 
Another algorithm based on a minimality theorem is given by Shock in 1986 
(\cite{shock_computing_1986}). Unfortunately, as we shall see and as already 
discussed by Wild in \cite{wild_computations_1995} the algorithm may not be
correct in general, even though the underlying theorem is. During the same 
period, Ausiello and al. brought the problem to graph-theoretic ground, and 
provided new structure known as \textit{FD-Graph} and algorithm to represent 
and work on implication systems in \cite{ausiello_directed_2017, 
	ausiello_graph_1983, ausiello_minimal_1986}. This approach has been seen in 
graph theory as an extension of the transitive closure in graphs 
(\cite{aho_transitive_2006}), but no consideration equivalent to minimization 
task seems to have been taken beforehand, as far as we know. Still in the 1980 
decade, Ganter expressed the canonical basis formalized by Duquenne and Guigues 
in his paper related to algorithms in FCA, \cite{ganter_two_2010} through 
closure systems, pseudo-closed and quasi-closed sets. Next, Wild 
(\cite{wild_implicational_1989, wild_theory_1994, wild_computations_1995}) 
linked within this set-theoretic framework both the relational databases, 
formal concept analysis and lattice-theoretic approach. In relating those 
fields, he describes an algorithm for minimizing a basis, similar to algorithms 
of Day and, somehow, Shock (resp. \cite{day_lattice_1992},  
\cite{shock_computing_1986}). This framework is the one we will use for our 
study, and can be found in more recent work by Ganter \& Obiedkov in 
\cite{b._ganter_conceptual_2016}. Also, the works of Maier and Duquenne-Guigues 
have been used in the lattice-theoretic context by Day in 
\cite{day_lattice_1992} to derive an algorithm based on congruence relations. 
For in-depth knowledge of implication system within lattice terminology, we can 
see \cite{davey_introduction_2002} as an introduction and 
\cite{bertet_lattices_2016} for a survey. Later, Duquenne proposed some 
variations in Day's work with another algorithm in 
\cite{duquenne_variations_2007}. More recently, Bor\`os and al. by 
working in a boolean logic framework, exhibited a theorem on the size of
canonical basis \cite{boros_exclusive_2010, boros_strong_2017}. They also gave
a general theoretic approach that algorithm should do one way or another on
reduction purpose. Out of these papers, Berczi \& al. derived a new 
minimization procedure based on hypergraphs in \cite{berczi_directed_2017}. 
Furthermore, an algorithm for computing the canonical basis starting from any 
system is given in \cite{b._ganter_conceptual_2016}.

\vspace{1.2em}

Even though the work we are going to cite is not designed to answer this 
question of minimization, it must also be exposed as the algorithm is 
intimately related to DG basis and can be used for base reduction. The paper
of Angluin and al. in query learning, see \cite{angluin_learning_1992}, provides
an algorithm for learning a Horn representation of an unknown initial formula. 
It has been shown later by Ariàs and Alcazar (\cite{arias_canonical_2009}) that
the output of Angluin algorithm was always the Duquennes-Guigues basis.

\vspace{1.2em}

Our purpose with this master thesis is to review and implement as much as 
possible the algorithms we exposed to provide a comparison. This comparison 
shall act as both theoretical and experimental statement of algorithm 
efficiency. As we already mentioned we will focus on closure theory framework.
The reason for this choice is our starting point. Because we start from the
algorithms provided by Wild and because the closure framework is the one we 
are the most familiar with, we focus on clearly explain this terminology with
examples. However, once we will be comfortable with those definitions, we will 
relate other frameworks to our main approach in the next chapter, to explain and
draw parallels with other algorithms. In the next section we will focus on 
theoretical definitions we shall need to understand the algorithms we have 
implemented.

\section{Implications and minimization: theoretic approach}

Here we will dive into mathematical representation of the task we gave
in the first section of this chapter. For the recall, our aim here is to
get familiar with the representation being closest from closure systems.  Most 
of the notions initially come from \cite{guigues_j.l_familles_1986, 
ganter_two_2010, wild_theory_1994,	ganter_formal_1999} but the reader can 
also find more than sufficient explanations in \cite{b._ganter_conceptual_2016, 
wild_joy_2017}. Readers with knowledge in relational databases will recognize 
most of functional dependency notations. The reason is close vicinity between 
implications and functional dependencies. Talking about our needs, we can 
consider them as equivalent notations. Actually, the real-life application our 
set up will be the closest from is FCA (\cite{ganter_formal_1999}) as we shall
see in the last chapter.

\subsection{Implications and closure systems}

The easiest object to project onto mathematical definitions is our attribute
set. For all the report, we fix $\Sg$ to be a set of \belemp{attributes}. 
Usually, we will denote attributes by small letters: \textit{a, b, c, \dots} 
and subsets of $\Sg$ (groups of attributes) will be denoted by capital letters: 
\textit{A, B, C, \dots} We assume the reader to have few background in 
elementary set-theoretic notations. 

\begin{definition}[\midemp{Implication, implication system}] An 
\belemp{implication} over $\Sg$ is a pair $(A, B)$ with $A, B \subseteq \Sg$. 
It is usually denoted by $A \imp B$. A set $\I$ of implications is called an 
\belemp{implication system}, \belemp{implication theory} or 
\belemp{implication(al) base(is)}.
\end{definition}

\noindent Note that given as is, this definition seems to lose the semantic
relation we depicted earlier. But we should keep in mind that in our set up, we
will be given implications more than an attribute set. Hence, implications will
make sense on their own, independently from the attribute set they are drawn 
from. Quickly, remark that implications in logical terms are expressed as
\textit{Horn formulae} giving another of its names to implication theories. 
Also, in $A \imp B$, $A$ is said to be the \belemp{premise} (or \belemp{body}) 
and $B$ the \belemp{conclusion} (\belemp{head}).

\begin{definition}[\midemp{Model}] Let $\I$ be an implication system over 
	$\Sg$, and $M \subseteq \Sg$. Then:
	\begin{itemize}
		\item[(i)] $M$ is a \belemp{model} of an implication $A \imp B$, 
		written 
		$M \models A \imp B$, if $B \subseteq M$ or $A \nsubseteq M$,
		\item[(ii)] $M$ is a \belemp{model} of $\I$ if $M \models A \imp B$ for 
		all
		$A \imp B \in \I$.
	\end{itemize}
	
\end{definition}

\noindent The notion of model may seem disarming at first sight. But $M$ being
a model of $A \imp B$ simply means that, if $A$ is included in $M$, then for
the implication $A \imp B$ to hold in $M$, we must have $B$ in $M$ too. This 
still suits the intuitive notion of premise/conclusion. Placed in the context
of $M$, $A \imp B$ says \textit{"whenever we have A, we must also have B"}.
Reader with some background in mathematical logic should be familiar with the
notation $\models$, denoting semantic entailment, as opposed to $\vdash$ for
syntactic deduction (see \cite{cori_mathematical_2000}). By a fortunate twist of
fate, semantic entailment is our next step:

\begin{definition}[\midemp{Semantic entailment}] We say that an implication 
	$A \imp B$ \belemp{semantically follows} from $\I$, denoted $\I \models A 
	\imp 
	B$, if all models $M$ of $\I$ are models of $A \imp B$.
	
\end{definition} 

\noindent Because next definitions are going to be on a slightly different 
structure, even though closely related to implication systems of course, let us
rest for a while and illustrate our definitions with an example.

\vspace{1.2em}

\paragraph{Example} Consider again our plant properties. Let $\Sg = $ 
\{\textit{colourful, bloom, wither, seasonal, aquatic, perennial, flower, 
	scented}\}. An implication could be \textit{flower $\imp$ scented}, or even
\textit{(bloom, aquatic) $\imp$ colourful} if we get rid off semantic 
interpretations. An implication basis $\I$ is for instance:

\begin{center}
	\textit{(colourful, bloom $\imp$ seasonal), (colourful, wither $\imp$ 
		seasonal), (bloom $\imp$ wither)}
\end{center}

\noindent and $M = $\textit{(colourful, bloom, seasonal)} is a model of 
\textit{colourful, bloom $\imp$ seasonal} because both the head and the body
of the implication belong to $M$. Also, $M$ is not a model of $\I$ because it
is not a model of \textit{bloom $\imp$ wither}. A model of $\I$ could be 
\textit{(bloom, wither)} or even the empty set $\emptyset$.

\vspace{1.2em}

Next definitions are about closure operators, and closure systems. We need 
to ground ourselves in those definitions before returning to implications.
$2^{\Sg}$ is the set of all subsets of $\Sg$, also named the \belemp{power set}
of $\Sg$.

\begin{definition}[\midemp{Closure operator}] Let $\Sg$ be a set and $\phi : 
	2^{\Sg} \imp 2^{\Sg}$ an application on the power set of $\Sg$. $\phi$ is
	a \belemp{closure operator} if $\forall X, Y \subseteq \Sg$:
	\begin{itemize}
		\item[(i)] $X \subseteq \phi(X)$ \midemp{(extensive)},
		\item[(ii)] $X \subseteq Y \imp \phi(X) \subseteq \phi(Y)$
		\midemp{(monotone)},
		\item[(iii)] $\phi(X) = \phi(\phi(X))$ \midemp{(idempotent)}.
	\end{itemize}
	$X \subseteq \Sg$ is called \belemp{closed} if $X = \phi(X)$.
\end{definition}

\begin{definition}[\midemp{Closure system}] Let $\Sg$ be a set, and $\Sg^{\phi}
	\subseteq 2^{\Sg}$. $\Sg^{\phi}$ is called a \belemp{closure system} if:
	\begin{itemize}
		\item[(i)] $\Sg \in \Sg^{\phi}$,
		\item[(ii)] if $\cal{S} \subseteq \Sg^{\phi}$, then $\bigcap \cal{S} 
		\in 
		\Sg^{\phi}$ \quad \midemp{(closed under intersection)}.
	\end{itemize}
	
\end{definition}

\noindent In the second definition, it is worth stressing on the fact that
$\Sg^{\phi}$ is a set of sets. Also, the notation $\Sg^{\phi}$ may seem 
surprising, but it has been chosen purposefully. Indeed, to each closure system
$\Sg^{\phi}$ over $\Sg$, we can associate a closure operator $\phi$ and 
vice-versa:
\begin{itemize}
	\item from $\phi$ to $\Sg^{\phi}$: compute all closed sets of $\phi$ to
	obtain $\Sg^{\phi}$,
	\item from $\Sg^{\phi}$ to $\phi$: define $\phi(X)$ as the smallest element
	of $\Sg^{\phi}$ (inclusion-wise) containing $X$. Observe that such a set
	always exists in $\Sg^{\phi}$ because $\Sg \in \Sg^{\phi}$.
\end{itemize}
 In any case, this notation used for clear exposition of the link
between closure systems and closure operators will be adapted to our context
of implication systems as we shall see later on. Notice that one can encounter 
another object, \belemp{closure space}, being a pair ($\Sg$, $\phi$) where
$\Sg$ is a set and $\phi$ a closure operator over $\Sg$. We are likely to find
this notation notably in \cite{wild_implicational_1989, 
	wild_theory_1994} where a general theory of closure spaces is addressed.

\vspace{1.2em}

\paragraph{Example} Let us imagine we have four people: \textit{Jezabel, Neige, 
Seraphin} and \textit{Narcisse}. Let us assume they all know each other and then
define a relation \textit{"like"} between them. For instance, say 
\textit{Séraphin likes Jezabel}. this relation is a \belemp{binary relation}: 
it relates pairs of elements. We can represent this relation by a graph where 
nodes are people and edges represent relations:

\begin{figure}[ht]
	\centering

\begin{tikzpicture}

\node[Vertex, label=above left:{\textit{Neige}}] (ng) at (-1, 1) {};
\node[Vertex, label=below left:{\textit{Seraphin}}] (s) at (-1, -1) {};
\node[Vertex, label=above right:{\textit{Narcisse}}] (na) at (1, 1) {};
\node[Vertex, label=below right:{\textit{Jezabel}}] (j) at (1, -1) {};

\draw[->] (ng) to[bend right=20] (j);
\draw[->] (s) -- (j);
\draw[->] (s) -- (ng);
\draw[->] (s) -- (na);
\draw[->] (j) -- (na);
\draw[->] (j) to[bend right=20] (ng);
\draw[->] (na) to[in=45, out=135, loop] (na);


\end{tikzpicture}

\caption{Graph of \textit{"like"} relation}
\label{fig:I-love}
\end{figure}

The arrow from \textit{Seraphin} to \textit{Jezabel} stands for 
\textit{"Seraphin likes Jezabel"} and the arrow from \textit{Narcisse} to itself
means equivalently \textit{"Narcisse likes Narcisse"}. With this clear, let 
us introduce an operation of gathering people. Starting from any group $A$ of 
persons presented here, let's add to $A$ every person liked by at least one 
element of $A$, until we can no more add people. For instance:
\begin{itemize}
	\item[-] if we start from \textit{Neige}, because \textit{Neige} likes
	\textit{Jezabel} and \textit{Jezabel} likes \textit{Narcisse} we will add
	both of them to the group of \textit{Neige},
	\item[-] because \textit{Narcisse} only likes himself, we have no people
	to add in his group.
\end{itemize}
\noindent Now observe that this operation of gathering people is in fact a
closure operator:
\begin{itemize}
	\item[(i)] it is \midemp{extensive}: starting from any group of people,
	we can only add new ones, hence either the group does not change (e.g: 
	\textit{Narcisse}) or it grows,
	\item[(ii)] it is \midemp{monotone}: if we start from a group $A$ containing
	a group $B$, it is clear that we will at least gather in $A$ all the people
	we would add with $B$,
	\item[(iii)] \midemp{idempotency}: once we added all the people we had to
	reach, then trying to find new people is useless by definition. Hence the
	group will remain the same if we apply our operation once more.
\end{itemize}

\vspace{1.2em}

We are going to get back to our main implication purpose to illustrate the 
notion of closure in our context. It turns out that given a basis
$\I$ over some set $\Sg$, the set of models of $\I$, $\Sg^{\I}$, is a closure 
system. Moreover, the operator $\I : 2^{\Sg} \imp 2^{\Sg}$ associating to a 
subset $X$ of $\Sg$ the smallest model (inclusion wise) containing $X$ is 
a closure operator. Furthermore, the closure system it defines is 
$\Sg^{\I}$. An interesting point is the mathematical computation of 
$\I(X)$ given $\I$ as a set of implications. We rely on 
\cite{wild_implicational_1989, b._ganter_conceptual_2016} to this end. Let 
us define a temporary operation $\circ : 2^{\Sg} \imp 2^{\Sg}$ as follows:

\[ X^{\circ} = 
X \cup \bigcup \{ B \; | \; A \imp B \in \I, \; A \subseteq X \} \]

\noindent Applying this operator up to stability provides $\I(X)$. In other 
words $\I(X) 
= X^{\circ \circ \dots}$. It is clear that we have a finite amount of 
iterations since $X$ cannot grow more than $\Sg$. Readers with background in
logic (see \cite{boros_strong_2017}) or graph theory 
(\cite{berczi_directed_2017}) might see this operation as the marking or 
forward chaining procedure.

\vspace{1.2em}

\paragraph{Example} Let's stick to our vegetable example, but reducing $\Sg$ to 
\{\textit{bloom, flower, colourful} \} (abbreviated \textit{b, f, c}) for the 
sake of simplicity. Furthermore, let $\I =$ \{\textit{((colourful, bloom) 
	$\imp$ flower), (flower $\imp$ bloom)}\}, abbreviated then $cb \imp f$, $f 
\imp b$. For instance, because $f \imp b \in \I$, the smallest model of $\I$ 
containing $f$ is $bf$, and $bf$ is closed. More precisely, the set of closed
sets is the following:

	\[ \Sg^{\I} = \{ \emptyset, \ b, \ c, \ bf, \ bcf \} \]
	
\noindent Pouet.

\begin{comment}
we will place $X$ under $Y$ and draw an arc from $X$ to $Y$, except we do not 
display transitive arcs. This representation is related to partially ordered 
set and is sometimes known as Hasse diagram, see \cite{davey_introduction_2002} 
for more details. We rely on figure \ref{fig:I-def-CS}.

\begin{figure}[ht]
	\centering
\begin{tikzpicture}[scale=0.6]

\node[Vertex, label=below:{$\emptyset$}] (epty) at (-4, -3) {};
\node[Vertex, label=left:{$c$}] (c) at (-6, -1) {};
\node[Vertex, label=right:{$b$}] (b) at (-4, -1) {};
\node[Vertex, label=right:{$f$}] (f) at (-2, -1) {};
\node[Vertex, label=left:{$bc$}] (bc) at (-6, 1) {};
\node[Vertex, label=left:{$cf$}] (cf) at (-4, 1) {};
\node[Vertex, label=right:{$bf$}] (bf) at (-2, 1) {};
\node[Vertex, label=above:{$bcf$}] (bcf) at (-4, 3) {};

\draw (epty) -- (c) -- (bc) -- (bcf) -- (bf) -- (f) -- (epty);
\draw (epty) -- (b);
\draw (bc) -- (b) -- (bf);
\draw (c) -- (cf) -- (f);
\draw (bcf) -- (cf);

\node[Vertex, label=below:{$\emptyset$}] (pepty) at (4, -3) {};
\node[Vertex, label=left:{$b$}] (pb) at (2, -1) {};
\node[Vertex, label=left:{$c$}] (pc) at (6, 0) {};
\node[Vertex, label=left:{$bf$}] (pbf) at (2, 1) {};
\node[Vertex, label=above:{$bcf$}] (pbcf) at (4, 3) {};

\draw (pepty) -- (pb) -- (pbf) -- (pbcf) -- (pc) --(pepty);

\draw[turquoise, fill=turquoise, opacity=0.4, thick] (epty) circle(0.5);
\draw[turquoise, fill=turquoise, opacity=0.4, thick] (c) circle(0.5);
\draw[turquoise, fill=turquoise, opacity=0.4, thick] (b) circle(0.5);
	
\draw[turquoise, fill=turquoise, fill opacity=0.3, rounded corners]
	(-2.5, -1.3) -- (-1.5, -1.3) -- (-1.5, 1.3) -- (-2.5, 1.3) -- cycle;

\draw[turquoise, rounded corners, fill=turquoise, fill opacity=0.3]
	(-3.5, 3.3) -- (-3.5, 0.5) -- (-6.5, 0.5) -- (-6.5, 1.3) -- 
	(-4.5, 3.3) -- cycle;

\draw[->, dotted, color=turquoise, very thick]
	(-3.2, 3.3) to[bend left=20] (3.7, 3.1);

\draw[->, dotted, color=turquoise, very thick]
	(-3.3, -3.3) to[bend right=20] (3.7, -3.1);

\draw[->, dotted, color=turquoise, very thick] 
	(-5.3, -1.2) to[bend right=35] (5.7, -0.2);
	
\draw[->, dotted, color=turquoise, very thick] 
	(-3.3, -1.2) to[bend right=20] (1.7, -1.2);

\draw[->, dotted, color=turquoise, very thick] 
	(-1.3, 0) to[bend right=20] (1.7, 0.8);


\end{tikzpicture}
\caption{Application of closure operator in implication context}
\label{fig:I-def-CS}
\end{figure}

\vspace{1.2em}

On the left side of the picture, one can find the boolean cube associated to 
$\Sg$, or equivalently the power set of $\Sg$ ordered by inclusion. On the 
right, the set of closed sets of $\I$. On the left side, all elements in the 
same cluster have the same closure: it defines an \belemp{equivalence class} 
under $\I$. Each dotted arc corresponds to application 
of the closure operator we described earlier.  
If a cluster contains only one element, this element is closed. This drawing 
shows the relation between a closure operator and its associated system, in 
particular in implication basis context, where the closure describes models. 
Finally, one can graphically note that the set of models is indeed closed 
under intersection. While this representation is graphically appealing, it is
clearly not tractable for larger attribute set: we have to draw
a diagram with an exponential number of elements (one for all $X \in 2^{\Sg}$). 
Thus, all Hasse diagrams we are going to draw only aim at providing some 
intuition of the various notions and not as an efficient representation.

\end{comment}

\vspace{1.2em}

Having presented the main definitions we shall need, we are to investigate 
practical computation of closures and more elaborated structures like the 
canonical basis (or Duquenne-Guigues basis) in the next section.

\begin{comment}

\gls{pseudo-closed set} pouet pouet. \gls{closure} hus hus. \gls{a} and 
\gls{z}.

In this chapter, we will focus on giving definitions of the mathematical 
objects we may need later. As we shall see, each section giving definitions is 
a different way to represent the same idea. Then, we will introduce the Horn 
Minimization task without any algorithm, to provide the reader with some 
intuition and simple examples of our problem. A more detailed description of 
the minimization task and existing solutions or studies will be given in 
chapter 2. 

\vspace{1.2em}

To be more precise, we develop first the main framework we may use to reason:
closure systems over attribute sets. We will often try to think of various 
algorithms and proofs within this framework to have a red line to follow. Then,
we approach propositional logic and few elements of graph theory. To emphasize
the links between various aspects, we will try to use the same notations on 
equivalent notions. 

% BD, Implicational Basis and stuff
\section{Introduction to implications through closure systems}

In this section, we will establish basic definitions and provide some examples
to have a sufficient (and strong enough) background to understand the topic of
Horn minimization. We assume some knowledge of set theory. For more complete
and detailed introduction to our topic, the reader may refer to 
\cite{b._ganter_conceptual_2016, davey_introduction_2002}. Some of the most 
practical applications of the following material 
dwell into artificial intelligence, with \belemp{Formal Concept Analysis} and
\belemp{Attribute Exploration}. We can find other applications within database
field (see \cite{maier_theory_1983}). For now, let us begin with an example to 
illustrate implications.

\vspace{1.2em}

\paragraph{Example} Let us imagine we are provided with a set of music 
genres: \textit{shoegaze, electronic, coldwave, pop, rock, dream-pop,
jazz, experimental, atmospheric, contemporary jazz}
 Assume we can attach various genre to a given music. Our aim is to 
draw inference of styles with other ones. In other words, say we have a music 
with tags \textit{rock, pop}, can we deduce other tag? Remind this example aims 
to 
illustrate, not to settle any musical knowledge. Let's try to give some ideas:
\begin{itemize}
	\item a music being \textit{jazz, experimental} can also be categorized
	as \textit{contemporary jazz},
	\item a music called \textit{coldwave, rock} is likely to be tagged 
	\textit{shoegaze},
	\item song with \textit{atmospheric} will probably be said to be 
	\textit{experimental, electronic},
	\item a last one, \textit{shoegaze, pop} will lead to \textit{dream-pop}.
\end{itemize}
\noindent Here we drew what we will call \belemp{implications} because we can
summarize our sentences by \textit{if a tag is present, then this one is too},
and denote them:
\begin{itemize}
	\item \textit{jazz, experimental} $\imp$ \textit{contemporary jazz},
	\item \textit{coldwave, rock}  $\imp$ \textit{shoegaze},
	\item \textit{atmospheric} $\imp$ \textit{experimental, electronic},
	\item \textit{shoegaze, pop} $\imp$ \textit{dream-pop}.
\end{itemize}
\noindent In a sense, those implications describe some possible knowledge of 
our genre set or \belemp{attributes} set.

\vspace{1.2em}

From our point of view, this example is sufficient to say that an 
\belemp{attribute set} is simply a set. We call its elements 
\belemp{attributes} to stick with the literature terminology. If not specified
in the context, we will call $\Sg$ such a set, and $a, \ b, \ c,\dots$ its 
elements. Subsets of $\Sg$ we will be denoted by capital letters $A, \ B, 
\dots$. With the example and $\Sg$, we can go over some more definitions

\begin{definition}[\midn{Implication basis}] Let $\Sg$ be an attribute set. An 
\belemp{implication basis} $\I$ over $\Sg$ is a set of \belemp{implications} 
where implications are pairs ($A$, $B$), denoted $A \imp B$, with $A, B 
\subseteq \Sg$.
	
\end{definition}

\noindent Usually, given an implication $A \imp B$, $A$ is called \belemp{body}
or \belemp{premise} while $B$ is called \belemp{head} or \belemp{consequence}.

\paragraph{Examples} Let $\Sg = \{a, \ b, \ c, \ d, \ e \}$. Some 
possible implication basis are (for the sake of readability, a subset of 
$\Sg$ we will be written as a concatenation of its elements):
\begin{itemize}
	\item $\{ab \imp de, \ a \imp c, ce \imp b \}$
	\item $\emptyset$
	\item $ \{ abc \imp ab, \ d \imp abcde \}$
\end{itemize}

Back to the musical example, the implication basis we described is of course 
$\{$ \textit{jazz, experimental} $\imp$ \textit{contemporary 
jazz}, \textit{coldwave, rock}  $\imp$ \textit{shoegaze}, \textit{atmospheric} 
$\imp$ \textit{experimental, electronic}, \textit{shoegaze, pop} $\imp$ 
\textit{dream-pop} $\}$.

\vspace{1.2em}

Each implication basis describe (possibly different!) knowledge from $\Sg$. Say
we have $\I$ over $\Sg$, at least two questions arise:
\begin{itemize}
	\item[(i)] is $A \subseteq \Sg$ \textit{"coherent"} with respect to $\I$?  
	\item[(ii)] What can we deduce from $A \subseteq \Sg$?	
\end{itemize}

\paragraph{Example} Consider again musical example. The first question would be 
"given our basis, can we imagine having a music with some tags only ?". For 
instance, consider a music with tags \textit{shoegaze, pop}. In our basis, we 
cannot have a music with these tags \textit{only} because the implication 
\textit{shoegaze, pop} $\imp$ \textit{dream-pop} states that if a music has 
\textit{shoegaze, pop} tags, it \textit{must} also have \textit{dream-pop}. On 
the other hand, taking only the tag \textit{shoegaze} is possible. 

The second question would be "given a set of tags, which tags will we obtain
in order to stay coherent with our implications?". Take \textit{coldwave, pop, 
rock}. Using our implications we will reach two other attributes: 
\textit{shoegaze, dream-pop}.

\vspace{1.2em}

In this last paragraph we described some of the \textit{most important} basic 
notions of our problem: \belemp{model} and \belemp{closure}.

\begin{definition}[\midn{Model of an implication}] Let $\Sg$ be an attribute 
set, and
$A \imp B$ an implication over $\Sg$. A subset $M \subseteq \Sg$ is a 
\belemp{model} of $A \imp B$, written $M \models A \imp B$ if $A \nsubseteq M$ 
or $B \subseteq M$.

\end{definition}

\begin{definition}[\midn{Model of an implication basis}] Let $\I$ be a basis 
over 
$\Sg$. A subset $M \subseteq \Sg$ is a \belemp{model} of $\I$, $M \models \I$, 
if $M \models A \imp B$ for each $A \imp B \in \I$.

\end{definition}

In other words, a subset $M$ of $\Sg$ will be a model of an implication if when 
it contains the body it contains also the head, or the body is not in $M$. An 
implication $A \in B$ \belemp{follows} from a basis $\I$, denoted $\I \models A 
\imp B$ if all models of $\I$ are models of $A \imp B$. In the example of 
musics we talked about, \textit{shoegaze} is a model when \textit{shoegaze, 
pop} is not. For completeness, we will define closure operators and closure 
systems in general before applying it to our context.


\begin{definition}[\midn{Closure operator, closure system}] Let
$\Sigma$ be a set, and define $\phi : \Sigma \longrightarrow \Sigma$ an
application. $\phi$ is a \belemp{closure operator} if it has the three following
properties for all $X, Y \subseteq \Sigma$:
\begin{itemize}
	\item[(i)] $X \subseteq \phi(X)$ (\midemp{extensive})
	\item[(ii)] $X \subseteq Y \longrightarrow \phi(X) \subseteq \phi(Y)$ 
		(\midemp{monotone})
	\item[(iii)] $\phi(\phi(X)) = \phi(X)$ (\midemp{idempotent})
\end{itemize}

\noindent The pair $(\Sigma, \phi)$ is called a \belemp{closure system}.
	
\end{definition}


\begin{definition}[\midn{Closed set}] Let $(\Sigma, \phi)$ be a closure 
system. A subset $X$ of $\Sigma$ is a \belemp{closed set} (with respect to 
$\phi$) if $\phi(X) = X$. We will denote by $\Sigma_{\phi}$ the set of all 
closed sets of $(\Sigma, \phi)$, that is:
	
	\[ \Sigma_{\phi} = \left\{ X \subseteq \Sigma \; | \; \phi(X) = X 
	\right\}  \]

\noindent and $\Sigma_{\phi}$ has the following properties:
\begin{itemize}
	\item[(i)] $\Sigma \in \Sigma_{\phi}$,
	\item[(ii)] if $X, Y \in \Sigma_{\phi}$, so does $X \cap Y$ 
		(\midemp{$\Sigma_{\phi}$ is closed under intersection}).
\end{itemize}
	
\end{definition}

\noindent Note that a closure system can be characterized either by its closure
operator, or by its set of closed sets. In other words, we can derive 
$\Sigma_{\phi}$ from $\phi$, as $\phi$ from $\Sigma_{\phi}$. The closed set 
associated to $X$ is the smallest closed set containing $X$, i.e:

	\[ \phi(X) = \bigcap \{Y \in \Sigma_\phi \; | \; X \subseteq Y \} \]

\noindent Since $\Sigma_{\phi}$ is closed under intersection, the resultant of
the intersection is also a closed set.


\paragraph{Example} Let $\Sigma = \llbracket 1 \; ; \; 4 \rrbracket$ and 
$\phi(X) = X \cup \{4 \}$. The pair $(\Sigma, \phi)$ is a closure system whose
closed sets are all the subsets containing 4. Another interesting definition of 
$\Sigma_{\phi}$ is:

	\[ \Sigma_{\phi} = \{ \phi(X) \; | \; X \subseteq \Sigma \} \]

\noindent Then, in our case:

	\[ \Sigma_{\phi} = \{ \{ 4\}, \{ 1, 4\}, \{ 2, 4\},
		\{ 3, 4\}, \{ 1, 2, 4\}, \{ 1, 3, 4\},\{ 2, 3, 4\}, 
		\{ 1, 2, 3, 4\} \}
	\]

\vspace{1.2em}

\noindent Now we have defined closure systems, let's get back to our 
implicational context. Let $\I$ be a basis over an attribute set $\Sg$. Let $
M \subseteq \Sg$ and define $\I(M)$ as the smallest model of $\I$ containing 
$M$. In this sens, $\I$ defines a \belemp{closure system} over $\Sg$ for 
which the closed sets are exactly the models of $\I$. The point is: how to 
define $\I(M)$ by computations? We rely on \cite{b._ganter_conceptual_2016} for 
this purpose. Let 

	\[ M^{\I} = M \cup \{ B \ | \ A \imp B \in \I, \ A \subset M \land
			B \nsubseteq M \} 
	\]

\noindent Then $\I(M)$ is obtained by repeated application of the operator 
$M^{\I}$, that is $\I(M) = M^{\I \I \dots \I}$ until the computed $M$ is 
unchanged. For readers with background in SAT-solving or graph theory, this is
equivalent to \textit{marking procedure, forward chaining}. In words, if we can 
find an implication $A \imp B$
with the body included in $M$, but not the head $B$, we add $B$ to $M$. Let us 
give an 
example to set things clear.

\paragraph{Example} As usual, let's stick to our musical example. For the 
recall, our set $\Sg$ is: \textit{shoegaze, electronic, coldwave, pop, rock, 
dream-pop, jazz, experimental, atmospheric, contemporary jazz} and the basis
we are working on $\I$ being:
\begin{itemize}
	\item[ ] \textit{jazz, experimental} $\imp$ \textit{contemporary jazz},
	\item[ ] \textit{coldwave, rock}  $\imp$ \textit{shoegaze},
	\item[ ] \textit{atmospheric} $\imp$ \textit{experimental, electronic},
	\item[ ] \textit{shoegaze, pop} $\imp$ \textit{dream-pop}.
\end{itemize}

\noindent In previous examples, we were talking about two subsets of $\Sg$:
\begin{itemize}
	\item $ A = $\textit{coldwave, pop, rock}
	\item $B = $\textit{shoegaze}
\end{itemize}

\noindent Let us try to compute their closure with respect to $\I$. $A$ is not
a model of $\I$ because of the implication \textit{coldwave, rock}  $\imp$ 
\textit{shoegaze}.  Indeed \textit{coldwave, rock} is included in $A$ but not
\textit{shoegaze} so we add it, thus $A = $\textit{coldwave, pop, 
rock, shoegaze}. The new $A$ is still not a model of $\I$, see the implication
\textit{shoegaze, pop} $\imp$ \textit{dream-pop}. We must also add 
\textit{dream-pop}. Finally, $\I(A) = $\textit{coldwave, pop, rock, 
shoegaze, dream-pop} will be the smallest model (closed set) containing $A$.

On the other hand, $B$ does respect all the implications so that it is already 
a model of $\I$, hence $\I(B) = B$. One interesting point about closure of a 
subset with respect to an implicational basis can be computed in linear time
(in the size of the basis). Of course there exists various algorithms for
computing the closure, but since it is not the object of our study we give
only the procedure we will use. To have more details on other algorithms
to compute the closure of a set with relation to an implicational basis, the
reader may refer to \cite{bazhanov_optimizations_2014, 
b._ganter_conceptual_2016}.

\begin{algorithm}
\KwIn{A basis $\I$, $X \subseteq \Sg$}
\KwOut{The closure $\I(X)$ of $X$ in $\I$}
	
\BlankLine
\BlankLine

\ForEach{$A \imp B \in \I$}{
	$count[A \imp B] := |A|$ \;
	
	\If{$|A| = 0$}{
		$X := X \cup B$ \;
	}

	\ForEach{$a \in A$}{
		$list[a] += A \imp B$ \;
	}
}

$update := X$ \;

\While{$update \neq \emptyset$}{
	choose $m \in update$ \;
	$update := update \ \{m\}$ \;
	
	\ForEach{$A \imp B \in list[m]$}{
		$count[A \imp B] -= 1$ \;
		\If{$count[A \imp B] = 0$}{
			$add := B \ X$ \;
			$X := X \cup add$ \;
			$update := update \cup add$ \;
		}
		
	}
}

return $X(\I)$ \;

	
\caption{LinClosure}
\label{alg:linclosure}
\end{algorithm}

\vspace{1.2em}

\begin{definition}[\midn{Closure of an implicational basis}] Given $\I$, the
closure $\I^+$ if $\I$ is the of all implications holding in $\I$.
	
\end{definition}

\begin{definition}[\midn{Equivalence of implicational basis}] Two implicational
basis are equivalent if they have the same closure.
	
\end{definition}

\noindent Also, two basis are equivalent if they have the same models. The 
question would be: how to determine the closure of an implicational basis? For
this purpose, one could use \belemp{Armstrong rules} (see 
\cite{b._ganter_conceptual_2016, maier_theory_1983}).

\vspace{1.2em}

Before concluding this section, we would like to introduce an useful result as
much as some remarks on the use of closure systems and implications. First 
a proposition (admitted here):

\begin{proposition} \label{prop:def.equiv_imp_clos} 
Let $\I$ be an implication basis over an attribute set $\Sg$. An implication $A 
\imp B$ follows from $\I$ if and only if $B \subseteq \I(A)$.
\end{proposition}

\begin{proof} \textit{if part}. Suppose $B \subseteq \I(A)$. $\I(A)$ is the
smallest model of $\I$ containing $A$. Therefore, for all models $M$ of $\I$
such that $A \subseteq M$ we also have $\I(A) \subseteq \I(M) = M$ (because
closed sets are models) and $B \subseteq M$ by extension. Therefore for all
models $M$ of $\I$, $A \nsubseteq M$ or $B \subseteq M$ holds. That is 
$\I \models A \imp B$.

\vspace{1.2em}

\textit{only if part}. Suppose $\I \models A \imp B$. By definition, all models
$M$ of $\I$ satisfy $A \nsubseteq M$ or $B \subseteq M$. In particular, it must
hold for the smallest model (inclusion wise) containing $A$ being $\I(A)$ which
yields $A \nsubseteq \I(A) \lor B \subseteq \I(A)$. Because this formula holds 
also for $\I(A)$ (as it is a model) and A $\nsubseteq \I(A)$ being a 
contradiction with respect to the definition of a closure operator, we conclude 
that necessarily $B \subseteq \I(A)$ must be true.

\end{proof}


\vspace{1.2em}

To conclude this short introduction on closure and implications terminology, we 
would like to say that these objects model an intuitive way of representing
knowledge and inference (see music example). More than this, they are mainly 
used because they can be checked easily, namely in linear time in the size of
our implication basis (see algorithm linclosure,  
\cite{b._ganter_conceptual_2016, david_minimum_1980,maier_theory_1983}).

	In this section we went over general definitions of closure systems, and 
implication basis. The reader may find a more exhaustive presentation in 
\cite{b._ganter_conceptual_2016, davey_introduction_2002}. Note that these are 
not all the definitions from closure terminology we shall use. Nevertheless, we 
prefer to give them when required as they are more closely related to specific 
problems, while the aim of this section is to be general. The next part is 
dedicated to a review of propositional logic in accordance to our needs.

% Personal: find the algorithm of bernstein. 











% Logical POV
\section{Propositional logic and Horn formulas}

This section is dedicated to the introduction of some propositional logic 
notations and notions. Again, we assume the reader has some background in
propositional logic anyway (we are not going to cover the meaning of 
disjunction, conjunction and so forth). The reader can refer to 
\cite{cori_mathematical_2000} for an introduction out of our scope.

\vspace{1.2em}

Before going into definitions, we set up some notations. Let us denote by
$\Sg$ the set of propositional variables. Disjunction is written with $\lor$,
conjunction with $\land$, and negation $\lnot$. Truth values are 0 (resp. 
$\bot$) and 1 (resp. $\top$). 

Our aim is to build so called Horn formulas. To this purpose we must first 
introduce what we call clauses, and Horn clauses. 


\begin{definition}[\midn{clause}] Let $x_1, \ dots, \ x_n$, $n \in \Ens{N}$
be variables of $\Sg$. A \belemp{clause} $\mathcal{C}$ over $x_i$'s is a 
\belemp{disjunction} of literals $p_i$:

	\[ \mathcal{C} = \bigvee_{i = 1}^{n} p_i \]

\noindent where $p_i \in \{x_i, \lnot x_i \}$.
\end{definition}

\begin{definition}[\midn{(pure) Horn clause}] A clause $C$ over variables of 
$\Sg$
is said \belemp{Horn} (resp. \belemp{pure Horn}) if it contains at most (resp. 
exactly one) positive literal.

\end{definition}

\paragraph{Example} To clarify our idea let us give a simple example. Let $x_1,
x_2, \dots, x_n$ be boolean variables. We have the following:
\begin{itemize}
	\item $\emptyset$ is a clause (true),
	\item $(x_1 \lor x_2 \lor \lnot x_3 \lor \lnot x_4)$ is a clause,
	\item $(\lnot x_1 \lor \lnot x_2)$ is a Horn clause,
	\item $(\lnot x_1 \lor \lnot x_2 \lor x_3)$ is pure Horn.
\end{itemize}


\vspace{1.2em}

To begin to draw a link with the previous section, one can note that we can 
write Horn clauses with disjunction, or with logical implication $\imp$. Indeed
remind that ($x_1 \lor \lnot x_2 $) can be equivalently rewritten as $x2 \imp 
x_1$. Hence, the Horn clauses defined in the previous examples become:
\begin{itemize}
	\item $(x_1 \land x_2) \imp \bot$,
	\item $(x_1 \land x_2) \imp x_3$.
\end{itemize}
\noindent We can use terms body, head the same way as with implicational basis. 
Before going any further, remind that any boolean formula $h$ can be 
rewritten in terms of \belemp{Conjunctive Normal Form} (or CNF). 

\begin{definition}[\midn{(pure) Horn CNF}] A \belemp{Horn CNF} (resp. 
\belemp{pure 
Horn CNF}) $\I$ over $\Sg$ is a conjunction of Horn clauses (resp. pure Horn
clauses) $\mathcal{C}_i$, $i \in \Ens{N}$:

	\[ \I = \bigwedge_{i} \mathcal{C}_i\]
	
\end{definition}

\begin{definition}[\midn{(pure) Horn formula}] A boolean function $h$ over 
$\Sg$
is a \belemp{Horn formula} (resp. \belemp{pure Horn formula}) if it can be 
represented with a Horn CNF $\I$ (resp. pure Horn CNF).
	
\end{definition}

Here is the link with our implication basis. A Horn CNF $\I$ can be seen as an
implicational basis. This relies on some notes:
\begin{itemize}
	\item for $P, Q, R$ propositional formulas, $(P \imp Q) \land (P \imp R)$ 
	is equivalent to $P \imp (Q \land R)$,
	\item representing sets with their \belemp{characteristic vectors} (or 
	bitmaps), translating $\cup$ by $\lor$ and $\cap$ by $\land$ give almost
	a one-to-one correspondence between sets and boolean formulas and their 
	models.
\end{itemize}

\paragraph{Example} To clarify let's think of short example taken from the 
previous section: $\Sg = \{ a, b, c, d, e \}$ and 

\[ \I = \{ ab \imp de, \ a \imp c, \ ce \imp b  \} \]

\noindent If we think of \textit{a, b, c, d, e} as boolean variables indicating
their presence into a set, we can derive the following translation of $\I$:

	\[ ((a \land b) \imp (d \land e)) \land (a \imp c) \land
		 ((c \land e) \imp b) \]

\noindent which can be represented by a (pure) Horn CNF, say $\I_h$:

\begin{align*}
	\I_h & = ((a \land b) \imp d) \land ((a \land b) \imp e) \land (a \imp c) 
	\land ((c \land e) \imp b) \\
	& \equiv (d \lor \lnot a \lor \lnot b) \land (e \lor \lnot a \lor \lnot b) 
	\land 
	(c \lor \lnot a) \land (b \lor \lnot e \lor \lnot c) 
\end{align*}

\noindent We could also have gone from $\I_h$ to $\I$.

\vspace{1.2em}

For this reason, we can also represent a Horn CNF as a set of clauses and 
clauses as pairs (body, head).

\begin{definition}[\midn{semantic entailment}] A formula $Q$ 
\belemp{semantically follows} from a formula $F$, denoted $F \models Q$ if 
every model of $F$ satisfies $Q$.
	
\end{definition}

Since the meaning of semantic entailment is the same as encountered with sets,
we can use the $\models$ the same way for both. This concludes our overview of
logical needs. We have seen the strong link between sets and logic which will 
allow us to go from one representation to another without any requirements. The
next section is dedicated to a short presentation of hypergraphs.





















% (Hyper)-graphs POV
\section{Directed graphs and Hypergraphs}

\begin{definition}[\belemp{Hypergraph}] An \belemp{hypergraph} is a pair $H = 
	(V, E)$ 
	where $V$ is a set of vertices (as in a graph) and $E$ a set of subsets of 
	$V$
	describing hyperarcs.
	
\end{definition}

\noindent In fact, hypergraphs are an extension of graphs. 


\begin{definition}[\belemp{Directed Hypergraph}] A \belemp{directed hypergraph} 
	$H = (V, E)$ is a pair with $V$ a set of vertices (nodes) and $E$ a set of 
	elements of $2^V \times 2^V$, denoting edges going from a subset of $V$ to 
	another subset of $V$.
	
\end{definition}

Directed hypergraphs are useful to graphically represent implication basis.

\paragraph{Example} Let us consider the following implication basis $\I$:

\[ \I = \{ 1 \imp 2, 2 \imp 34, 3 \imp 12, 41 \imp 3 \} \]

\noindent Then we can define an hyper graph $L = (V, E)$ where
\begin{itemize}
	\item $V = \{1, 2, 3, 4 \}$
	\item $E = \{ (\{ 1 \}, \{ 2\}), (\{ 2 \}, \{3, 4\}),
	(\{ 3 \}, \{ 1, 2\}), (\{1, 4\}, \{ 3\}) \}$
\end{itemize}

\noindent To be clearer, we can see the graphical representation of this figure
in \ref{fig:DHyp1}

\begin{center}
	\begin{figure}
\begin{center}
\begin{tikzpicture}

\node[Vertex, label=left:{4}] (4) at (-1, 1) {};
\node[Vertex, label=above left:{1}] (1) at (1, 1) {};
\node[Vertex, label=below left:{2}] (2) at (-1.5, -1) {};
\node[Vertex, label=below right:{3}] (3) at (0.5, -1) {};

% 41 --> 3
\draw[<-, color=belize] (0.5, -0.8) to[bend right=18] (-0.8, 0.8);
\draw[<-, color=belize] (0.5, -0.8) to[bend left=28] (0.8, 0.8);

% 1 --> 2
\draw[<-, color=emerald] (-1.3, -0.8) -- (0.8, 0.8);

% 2 --> 34
\draw[->, color=amethyst] (-1.3, -0.8) to[bend left=45] (0.3, -0.8);
\draw[->, color=amethyst] (-1.3, -0.8) to[bend right=45] (-1, 0.8);

% 3 --> 12 
\draw[->, color=alizarine] (0.3, -0.8) to[bend right=60] (-1.3, -0.8);
\draw[->, color=alizarine] (0.3, -0.8) to[bend left=50] (0.8, 0.8);


\end{tikzpicture}
\end{center}

\caption{Example of Directed Hypergraph}
\label{fig:DHyp1}
\end{figure}



\begin{comment}
\begin{figure}[ht]
\begin{center}
\begin{tikzpicture}
\node[Input Node, minimum size=0.6cm] (x) at (-4, 0) {$x$};
\node[Hidden Node, opacity=0.65, minimum size=0.6cm] (h11) at (-2, 1) {$h_1$};
\node[Hidden Node, opacity=0.65, minimum size=0.6cm] (h12) at (-2, -1) {$h_2$};
\node[Hidden Node, opacity=0.65, minimum size=0.6cm] (h21) at (0, 1) {$h_3$};
\node[Hidden Node, minimum size=0.6cm] (h22) at (0, -1) {$h_4$};
\node[Output Node, minimum size=0.6cm] (o) at (2, 0) {$o$};
\node (y) at (4, 0) {$\hat{y}$};

\draw[->] (x) -- node[midway, above] {$u_1$} (h11);
\draw[->] (x) -- node[midway, above] {$u_2$} (h12);
\draw[->] (h11) -- node[midway, above] {$w_{1, 1}$} (h21);
\draw[->] (h11) -- node[sloped, near end, below] {$w_{2, 1}$} (h22);
\draw[->] (h12) -- node[sloped, near end, above] {$w_{1, 2}$} (h21);
\draw[->] (h12) -- node[midway, below] {$w_{2, 2}$} (h22);
\draw[->] (h21) -- node[midway, above] {$v_1$} (o);
\draw[->] (h22) -- node[midway, below] {$v_2$} (o);
\draw[->] (o) -- (y);


\draw (5,0) node{A} to[bend right=30]
node[very near start]{R}    node[pos=0.7]{S}  (9,2) node{B};

\end{tikzpicture}
\end{center}

\caption{Schéma d'un réseau de neurones pour la rétro-propagation}
\label{fig:NN-BP}
\end{figure} 


\end{comment}
\end{center}


% Horn minimization task
\section{Horn Minimization task}

Previously we settled down mathematical tools we shall use. In the next 
paragraphs we will focus on describing "with hands" the problem of Horn 
minimization in our sense. Here we suppose $\I$ is in so-called \belemp{reduced
form}, that is for all distinct implications $A \imp B, C \imp D$ of $\I$,
$A \neq C$. That is we do not have distinct implications with same bodies. This
said, we state one of the \aliemp{most important} of our problem.

\begin{definition}[\midn{Minimality of $\I$}] Let $\I$ be a set of 
implications over $\Sg$, and $(\Sg, \phi)$ be a closure system. $\I$ is:
\begin{itemize}
	\item[(i)] \belemp{sound} if each implications of $\I$ holds in 
	$\Sg_\phi$
	\item[(ii)] \belemp{complete} if each implications holding in 
	$\Sg_\phi$ follows from $\I$
	\item[(iii)] \belemp{nonredundant} if no implication in $\I$ follows 
	from other implications of $\I$.
\end{itemize}
\noindent An implicational basis with such properties is called 
\belemp{minimal}.
\end{definition}

Note that here, minimality is defined with relation to a closure system. In
our case we will just be given $\Sg$ and $\I$, without an associated $(\Sg, 
\phi)$. Therefore, our basis will automatically be sound and complete because
the closure system we will consider is the set of models of $\I$. Hence, we
can provide a simpler definition of minimality (which is in fact nonredundancy):

\begin{definition}[\midn{Mimimality}] A reduced implicational basis $\I$ is
said \belemp{minimal} if we cannot remove any implication without altering
its closure.
	
\end{definition}


As we shall see later, there exists several definition of minimality. Our 
point of view could also be called \belemp{body minimality}. More intuitively, 
our aim is to concentrate all the knowledge in a minimal number of implications.

\paragraph{Example} Let us recall our music example. We had music styles:
\textit{shoegaze, electronic, coldwave, pop, rock, dream-pop, jazz, 
experimental, atmospheric, contemporary jazz} and some implications:
\begin{itemize}
	\item[ ] \textit{jazz, experimental} $\imp$ \textit{contemporary jazz},
	\item[ ] \textit{coldwave, rock}  $\imp$ \textit{shoegaze},
	\item[ ] \textit{atmospheric} $\imp$ \textit{experimental, electronic},
	\item[ ] \textit{shoegaze, pop} $\imp$ \textit{dream-pop}.
\end{itemize}
\noindent Unfortunately, because all the bodies are disjoint, this basis is 
already minimal. But let's imagine we have the following implications instead:
\begin{itemize}
	\item[ ] \textit{coldwave} $\imp$ \textit{rock},
	\item[ ] \textit{shoegaze}  $\imp$ \textit{rock, coldwave, dream-pop},
	\item[ ] \textit{shoegaze, dream-pop} $\imp$ \textit{rock, coldwave},
	\item[ ] \textit{rock, dream-pop} $\imp$ \textit{atmospheric, shoegaze, 
	coldwave}.
\end{itemize}
\noindent Here, the third implication can be removed. Indeed, from tags 
\textit{shoegaze, dreampop} following the implications we can reach 
\textit{rock, coldwave, atmospheric}. But say we remove the third implication,
and thus keep:
\begin{itemize}
	\item[ ] \textit{coldwave} $\imp$ \textit{rock},
	\item[ ] \textit{shoegaze}  $\imp$ \textit{rock, coldwave, dream-pop},
	\item[ ] \textit{rock, dream-pop} $\imp$ \textit{atmospheric, shoegaze, 
	coldwave}.
\end{itemize}
What can we reach starting from \textit{shoegaze, dreampop}? Using the second
implication we get \textit{rock, coldwave}, and because we have \textit{rock, 
dreampop}, we also get \textit{atmospheric}. That is, we can get the same 
knowledge than in the previous case but with one less implication.

\vspace{0.5em}

Let us present a less practical but more visual example. Suppose $\Sg = \{ a, 
\  s,\  c, \  r, \  d\}$
and $\I$ as follows:

	\[ c \imp r, \; s \imp rcd, \; rd \imp asc, \; sd \imp rc \]

\noindent This basis is not minimal. Indeed, let us remove the fourth 
implication and call $\I^{-}$ the new basis. We want to know whether the 
implication we removed still holds in $\I^{-}$ so that its closure is kept (and
therefore its "knowledge"). To show that $\I^{-} \models sd \imp rc$, it is
enough to show that $rc \subseteq \I^{-}(sd)$ (see proposition 
\ref{prop:def.equiv_imp_clos}). Because we have $s \imp rcd$, then $rd \imp 
asc$ we conclude that $\I^{-}(sd) = acdrs$. Thus $\I^{-} \models sd \imp rc$ 
and $\I^{-}$ is smaller than $\I$, with $\I^{-} \equiv \I$. Also, $\I^{-}$ is 
minimal.

\vspace{0.5em}

Note that the latter example is in fact the same as the musical one. Take the 
first letter of each style and we end up with the basis described in the second
paragraph.


\vspace{1.2em}

To go in details, we will introduce some elements and objects related to body 
minimality, namely \belemp{Pseudo-closed sets} and \belemp{Duquenne-Guigues 
basis}.

\begin{definition}[\midn{Pseudo-closed set}] Let $(\Sg, \phi)$ be a closure 
system. A subset $M \subseteq \Sg$ is \belemp{pseudo-closed} if and only if 
\begin{itemize}
	\item $\phi(M) \neq M$ ($M$ is not closed),
	\item if $Q \subset M$ is pseudo-closed, then $\phi(Q) \subseteq M$.
\end{itemize}
	
\end{definition}

Note that the empty set $\emptyset$ is pseudo-closed if and only if it is not 
closed. Also, one can note that if $P$ is pseudo-closed and $P'$ is covered
(inclusion wise, in terms of partial ordering) then $P'$ cannot be 
pseudo-closed. For the next definition, one can refer to 
\cite{b._ganter_conceptual_2016, guigues_familles_1986}.

\begin{definition}[\midn{Duquenne-Guigues Basis}] Let $(\Sg, \phi)$ be a 
closure system. The \belemp{Duquenne-Guigues} basis or \belemp{canonical} 
basis $\I$ is:

	\[ \I = \{ M \imp \phi(M) \; | \; M \subseteq \Sg, \, M \;
		\text{pseudo-closed} \} \]

\noindent and $\I$ is complete, sound and nonredundant.
\end{definition}

\noindent Thus, the Duquenne-Guigues basis is body-minimal. In this section 
we presented the minimization problem we will worked on during this thesis. 
Another remark, as we shall see in the next chapter, this problem is solvable
in polynomial time.

\space{1.2em}

To conclude this first chapter, we introduced in general our problem of 
minimizing implicational basis as our way of working. We developed an overview 
of the mathematical tools we needed to 


\end{comment}