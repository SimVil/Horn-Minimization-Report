\section{Introduction to implications through closure systems}

In this section, we will establish basic definitions and provide some examples
to have a sufficient (and strong enough) background to understand the topic of
Horn minimization. We assume some knowledge of set theory. For more complete
and detailed introduction to our topic, the reader may refer to 
\cite{b._ganter_conceptual_2016, davey_introduction_2002}. Some of the most 
practical applications of the following material 
dwell into artificial intelligence, with \belemp{Formal Concept Analysis} and
\belemp{Attribute Exploration}. We can find other applications within database
field (see \cite{maier_theory_1983}). For now, let us begin with an example to 
illustrate implications.

\vspace{1.2em}

\paragraph{Example} Let us imagine we are provided with a set of music 
genres: \textit{shoegaze, electronic, coldwave, pop, rock, dream-pop,
jazz, experimental, atmospheric, contemporary jazz}
 Assume we can attach various genre to a given music. Our aim is to 
draw inference of styles with other ones. In other words, say we have a music 
with tags \textit{rock, pop}, can we deduce other tag? Remind this example aims 
to 
illustrate, not to settle any musical knowledge. Let's try to give some ideas:
\begin{itemize}
	\item a music being \textit{jazz, experimental} can also be categorized
	as \textit{contemporary jazz},
	\item a music called \textit{coldwave, rock} is likely to be tagged 
	\textit{shoegaze},
	\item song with \textit{atmospheric} will probably be said to be 
	\textit{experimental, electronic},
	\item a last one, \textit{shoegaze, pop} will lead to \textit{dream-pop}.
\end{itemize}
\noindent Here we drew what we will call \belemp{implications} because we can
summarize our sentences by \textit{if a tag is present, then this one is too},
and denote them:
\begin{itemize}
	\item \textit{jazz, experimental} $\imp$ \textit{contemporary jazz},
	\item \textit{coldwave, rock}  $\imp$ \textit{shoegaze},
	\item \textit{atmospheric} $\imp$ \textit{experimental, electronic},
	\item \textit{shoegaze, pop} $\imp$ \textit{dream-pop}.
\end{itemize}
\noindent In a sense, those implications describe some possible knowledge of 
our genre set or \belemp{attributes} set.

\vspace{1.2em}

From our point of view, this example is sufficient to say that an 
\belemp{attribute set} is simply a set. We call its elements 
\belemp{attributes} to stick with the literature terminology. If not specified
in the context, we will call $\Sg$ such a set, and $a, \ b, \ c,\dots$ its 
elements. Subsets of $\Sg$ we will be denoted by capital letters $A, \ B, 
\dots$. With the example and $\Sg$, we can go over some more definitions

\begin{definition}[\midn{Implication basis}] Let $\Sg$ be an attribute set. An 
\belemp{implication basis} $\I$ over $\Sg$ is a set of \belemp{implications} 
where implications are pairs ($A$, $B$), denoted $A \imp B$, with $A, B 
\subseteq \Sg$.
	
\end{definition}

\noindent Usually, given an implication $A \imp B$, $A$ is called \belemp{body}
or \belemp{premise} while $B$ is called \belemp{head} or \belemp{consequence}.

\paragraph{Examples} Let $\Sg = \{a, \ b, \ c, \ d, \ e \}$. Some 
possible implication basis are (for the sake of readability, a subset of 
$\Sg$ we will be written as a concatenation of its elements):
\begin{itemize}
	\item $\{ab \imp de, \ a \imp c, ce \imp b \}$
	\item $\emptyset$
	\item $ \{ abc \imp ab, \ d \imp abcde \}$
\end{itemize}

Back to the musical example, the implication basis we described is of course 
$\{$ \textit{jazz, experimental} $\imp$ \textit{contemporary 
jazz}, \textit{coldwave, rock}  $\imp$ \textit{shoegaze}, \textit{atmospheric} 
$\imp$ \textit{experimental, electronic}, \textit{shoegaze, pop} $\imp$ 
\textit{dream-pop} $\}$.

\vspace{1.2em}

Each implication basis describe (possibly different!) knowledge from $\Sg$. Say
we have $\I$ over $\Sg$, at least two questions arise:
\begin{itemize}
	\item[(i)] is $A \subseteq \Sg$ \textit{"coherent"} with respect to $\I$?  
	\item[(ii)] What can we deduce from $A \subseteq \Sg$?	
\end{itemize}

\paragraph{Example} Consider again musical example. The first question would be 
"given our basis, can we imagine having a music with some tags only ?". For 
instance, consider a music with tags \textit{shoegaze, pop}. In our basis, we 
cannot have a music with these tags \textit{only} because the implication 
\textit{shoegaze, pop} $\imp$ \textit{dream-pop} states that if a music has 
\textit{shoegaze, pop} tags, it \textit{must} also have \textit{dream-pop}. On 
the other hand, taking only the tag \textit{shoegaze} is possible. 

The second question would be "given a set of tags, which tags will we obtain
in order to stay coherent with our implications?". Take \textit{coldwave, pop, 
rock}. Using our implications we will reach two other attributes: 
\textit{shoegaze, dream-pop}.

\vspace{1.2em}

In this last paragraph we described some of the \textit{most important} basic 
notions of our problem: \belemp{model} and \belemp{closure}.

\begin{definition}[\midn{Model of an implication}] Let $\Sg$ be an attribute 
set, and
$A \imp B$ an implication over $\Sg$. A subset $M \subseteq \Sg$ is a 
\belemp{model} of $A \imp B$, written $M \models A \imp B$ if $A \nsubseteq M$ 
or $B \subseteq M$.

\end{definition}

\begin{definition}[\midn{Model of an implication basis}] Let $\I$ be a basis 
over 
$\Sg$. A subset $M \subseteq \Sg$ is a \belemp{model} of $\I$, $M \models \I$, 
if $M \models A \imp B$ for each $A \imp B \in \I$.

\end{definition}

In other words, a subset $M$ of $\Sg$ will be a model of an implication if when 
it contains the body it contains also the head, or the body is not in $M$. An 
implication $A \in B$ \belemp{follows} from a basis $\I$, denoted $\I \models A 
\imp B$ if all models of $\I$ are models of $A \imp B$. In the example of 
musics we talked about, \textit{shoegaze} is a model when \textit{shoegaze, 
pop} is not. For completeness, we will define closure operators and closure 
systems in general before applying it to our context.


\begin{definition}[\midn{Closure operator, closure system}] Let
$\Sigma$ be a set, and define $\phi : \Sigma \longrightarrow \Sigma$ an
application. $\phi$ is a \belemp{closure operator} if it has the three following
properties for all $X, Y \subseteq \Sigma$:
\begin{itemize}
	\item[(i)] $X \subseteq \phi(X)$ (\midemp{extensive})
	\item[(ii)] $X \subseteq Y \longrightarrow \phi(X) \subseteq \phi(Y)$ 
		(\midemp{monotone})
	\item[(iii)] $\phi(\phi(X)) = \phi(X)$ (\midemp{idempotent})
\end{itemize}

\noindent The pair $(\Sigma, \phi)$ is called a \belemp{closure system}.
	
\end{definition}


\begin{definition}[\midn{Closed set}] Let $(\Sigma, \phi)$ be a closure 
system. A subset $X$ of $\Sigma$ is a \belemp{closed set} (with respect to 
$\phi$) if $\phi(X) = X$. We will denote by $\Sigma_{\phi}$ the set of all 
closed sets of $(\Sigma, \phi)$, that is:
	
	\[ \Sigma_{\phi} = \left\{ X \subseteq \Sigma \; | \; \phi(X) = X 
	\right\}  \]

\noindent and $\Sigma_{\phi}$ has the following properties:
\begin{itemize}
	\item[(i)] $\Sigma \in \Sigma_{\phi}$,
	\item[(ii)] if $X, Y \in \Sigma_{\phi}$, so does $X \cap Y$ 
		(\midemp{$\Sigma_{\phi}$ is closed under intersection}).
\end{itemize}
	
\end{definition}

\noindent Note that a closure system can be characterized either by its closure
operator, or by its set of closed sets. In other words, we can derive 
$\Sigma_{\phi}$ from $\phi$, as $\phi$ from $\Sigma_{\phi}$. The closed set 
associated to $X$ is the smallest closed set containing $X$, i.e:

	\[ \phi(X) = \bigcap \{Y \in \Sigma_\phi \; | \; X \subseteq Y \} \]

\noindent Since $\Sigma_{\phi}$ is closed under intersection, the resultant of
the intersection is also a closed set.


\paragraph{Example} Let $\Sigma = \llbracket 1 \; ; \; 4 \rrbracket$ and 
$\phi(X) = X \cup \{4 \}$. The pair $(\Sigma, \phi)$ is a closure system whose
closed sets are all the subsets containing 4. Another interesting definition of 
$\Sigma_{\phi}$ is:

	\[ \Sigma_{\phi} = \{ \phi(X) \; | \; X \subseteq \Sigma \} \]

\noindent Then, in our case:

	\[ \Sigma_{\phi} = \{ \{ 4\}, \{ 1, 4\}, \{ 2, 4\},
		\{ 3, 4\}, \{ 1, 2, 4\}, \{ 1, 3, 4\},\{ 2, 3, 4\}, 
		\{ 1, 2, 3, 4\} \}
	\]

\vspace{1.2em}

\noindent Now we have defined closure systems, let's get back to our 
implicational context. Let $\I$ be a basis over an attribute set $\Sg$. Let $
M \subseteq \Sg$ and define $\I(M)$ as the smallest model of $\I$ containing 
$M$. In this sens, $\I$ defines a \belemp{closure system} over $\Sg$ for 
which the closed sets are exactly the models of $\I$. The point is: how to 
define $\I(M)$ by computations? We rely on \cite{b._ganter_conceptual_2016} for 
this purpose. Let 

	\[ M^{\I} = M \cup \{ B \ | \ A \imp B \in \I, \ A \subset M \land
			B \nsubseteq M \} 
	\]

\noindent Then $\I(M)$ is obtained by repeated application of the operator 
$M^{\I}$, that is $\I(M) = M^{\I \I \dots \I}$ until the computed $M$ is 
unchanged. For readers with background in SAT-solving or graph theory, this is
equivalent to \textit{marking procedure, forward chaining}. In words, if we can 
find an implication $A \imp B$
with the body included in $M$, but not the head $B$, we add $B$ to $M$. Let us 
give an 
example to set things clear.

\paragraph{Example} As usual, let's stick to our musical example. For the 
recall, our set $\Sg$ is: \textit{shoegaze, electronic, coldwave, pop, rock, 
dream-pop, jazz, experimental, atmospheric, contemporary jazz} and the basis
we are working on $\I$ being:
\begin{itemize}
	\item[ ] \textit{jazz, experimental} $\imp$ \textit{contemporary jazz},
	\item[ ] \textit{coldwave, rock}  $\imp$ \textit{shoegaze},
	\item[ ] \textit{atmospheric} $\imp$ \textit{experimental, electronic},
	\item[ ] \textit{shoegaze, pop} $\imp$ \textit{dream-pop}.
\end{itemize}

\noindent In previous examples, we were talking about two subsets of $\Sg$:
\begin{itemize}
	\item $ A = $\textit{coldwave, pop, rock}
	\item $B = $\textit{shoegaze}
\end{itemize}

\noindent Let us try to compute their closure with respect to $\I$. $A$ is not
a model of $\I$ because of the implication \textit{coldwave, rock}  $\imp$ 
\textit{shoegaze}.  Indeed \textit{coldwave, rock} is included in $A$ but not
\textit{shoegaze} so we add it, thus $A = $\textit{coldwave, pop, 
rock, shoegaze}. The new $A$ is still not a model of $\I$, see the implication
\textit{shoegaze, pop} $\imp$ \textit{dream-pop}. We must also add 
\textit{dream-pop}. Finally, $\I(A) = $\textit{coldwave, pop, rock, 
shoegaze, dream-pop} will be the smallest model (closed set) containing $A$.

On the other hand, $B$ does respect all the implications so that it is already 
a model of $\I$, hence $\I(B) = B$. One interesting point about closure of a 
subset with respect to an implicational basis can be computed in linear time
(in the size of the basis). Of course there exists various algorithms for
computing the closure, but since it is not the object of our study we give
only the procedure we will use. To have more details on other algorithms
to compute the closure of a set with relation to an implicational basis, the
reader may refer to \cite{bazhanov_optimizations_2014, 
b._ganter_conceptual_2016}.

\begin{algorithm}
\KwIn{A basis $\I$, $X \subseteq \Sg$}
\KwOut{The closure $\I(X)$ of $X$ in $\I$}
	
\BlankLine
\BlankLine

\ForEach{$A \imp B \in \I$}{
	$count[A \imp B] := |A|$ \;
	
	\If{$|A| = 0$}{
		$X := X \cup B$ \;
	}

	\ForEach{$a \in A$}{
		$list[a] += A \imp B$ \;
	}
}

$update := X$ \;

\While{$update \neq \emptyset$}{
	choose $m \in update$ \;
	$update := update \ \{m\}$ \;
	
	\ForEach{$A \imp B \in list[m]$}{
		$count[A \imp B] -= 1$ \;
		\If{$count[A \imp B] = 0$}{
			$add := B \ X$ \;
			$X := X \cup add$ \;
			$update := update \cup add$ \;
		}
		
	}
}

return $X(\I)$ \;

	
\caption{LinClosure}
\label{alg:linclosure}
\end{algorithm}

\vspace{1.2em}

\begin{definition}[\midn{Closure of an implicational basis}] Given $\I$, the
closure $\I^+$ if $\I$ is the of all implications holding in $\I$.
	
\end{definition}

\begin{definition}[\midn{Equivalence of implicational basis}] Two implicational
basis are equivalent if they have the same closure.
	
\end{definition}

\noindent Also, two basis are equivalent if they have the same models. The 
question would be: how to determine the closure of an implicational basis? For
this purpose, one could use \belemp{Armstrong rules} (see 
\cite{b._ganter_conceptual_2016, maier_theory_1983}).

\vspace{1.2em}

Before concluding this section, we would like to introduce an useful result as
much as some remarks on the use of closure systems and implications. First 
a proposition (admitted here):

\begin{proposition} \label{prop:def.equiv_imp_clos} 
Let $\I$ be an implication basis over an attribute set $\Sg$. An implication $A 
\imp B$ follows from $\I$ if and only if $B \subseteq \I(A)$.
\end{proposition}

\begin{proof} \textit{if part}. Suppose $B \subseteq \I(A)$. $\I(A)$ is the
smallest model of $\I$ containing $A$. Therefore, for all models $M$ of $\I$
such that $A \subseteq M$ we also have $\I(A) \subseteq \I(M) = M$ (because
closed sets are models) and $B \subseteq M$ by extension. Therefore for all
models $M$ of $\I$, $A \nsubseteq M$ or $B \subseteq M$ holds. That is 
$\I \models A \imp B$.

\vspace{1.2em}

\textit{only if part}. Suppose $\I \models A \imp B$. By definition, all models
$M$ of $\I$ satisfy $A \nsubseteq M$ or $B \subseteq M$. In particular, it must
hold for the smallest model (inclusion wise) containing $A$ being $\I(A)$ which
yields $A \nsubseteq \I(A) \lor B \subseteq \I(A)$. Because this formula holds 
also for $\I(A)$ (as it is a model) and A $\nsubseteq \I(A)$ being a 
contradiction with respect to the definition of a closure operator, we conclude 
that necessarily $B \subseteq \I(A)$ must be true.

\end{proof}


\vspace{1.2em}

To conclude this short introduction on closure and implications terminology, we 
would like to say that these objects model an intuitive way of representing
knowledge and inference (see music example). More than this, they are mainly 
used because they can be checked easily, namely in linear time in the size of
our implication basis (see algorithm linclosure,  
\cite{b._ganter_conceptual_2016, david_minimum_1980,maier_theory_1983}).

	In this section we went over general definitions of closure systems, and 
implication basis. The reader may find a more exhaustive presentation in 
\cite{b._ganter_conceptual_2016, davey_introduction_2002}. Note that these are 
not all the definitions from closure terminology we shall use. Nevertheless, we 
prefer to give them when required as they are more closely related to specific 
problems, while the aim of this section is to be general. The next part is 
dedicated to a review of propositional logic in accordance to our needs.

% Personal: find the algorithm of bernstein. 








