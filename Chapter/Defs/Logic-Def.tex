\section{Propositional logic and Horn formulas}

This section is dedicated to the introduction of some propositional logic 
notations and notions. Again, we assume the reader has some background in
propositional logic anyway (we are not going to cover the meaning of 
disjunction, conjunction and so forth). The reader can refer to 
\cite{cori_mathematical_2000} for an introduction out of our scope.

\vspace{1.2em}

Before going into definitions, we set up some notations. Let us denote by
$\Sg$ the set of propositional variables. Disjunction is written with $\lor$,
conjunction with $\land$, and negation $\lnot$. Truth values are 0 (resp. 
$\bot$) and 1 (resp. $\top$). 

Our aim is to build so called Horn formulas. To this purpose we must first 
introduce what we call clauses, and Horn clauses. 


\begin{definition}[\midn{clause}] Let $x_1, \ dots, \ x_n$, $n \in \Ens{N}$
be variables of $\Sg$. A \belemp{clause} $\mathcal{C}$ over $x_i$'s is a 
\belemp{disjunction} of literals $p_i$:

	\[ \mathcal{C} = \bigvee_{i = 1}^{n} p_i \]

\noindent where $p_i \in \{x_i, \lnot x_i \}$.
\end{definition}

\begin{definition}[\midn{(pure) Horn clause}] A clause $C$ over variables of 
$\Sg$
is said \belemp{Horn} (resp. \belemp{pure Horn}) if it contains at most (resp. 
exactly one) positive literal.

\end{definition}

\paragraph{Example} To clarify our idea let us give a simple example. Let $x_1,
x_2, \dots, x_n$ be boolean variables. We have the following:
\begin{itemize}
	\item $\emptyset$ is a clause (true),
	\item $(x_1 \lor x_2 \lor \lnot x_3 \lor \lnot x_4)$ is a clause,
	\item $(\lnot x_1 \lor \lnot x_2)$ is a Horn clause,
	\item $(\lnot x_1 \lor \lnot x_2 \lor x_3)$ is pure Horn.
\end{itemize}


\vspace{1.2em}

To begin to draw a link with the previous section, one can note that we can 
write Horn clauses with disjunction, or with logical implication $\imp$. Indeed
remind that ($x_1 \lor \lnot x_2 $) can be equivalently rewritten as $x2 \imp 
x_1$. Hence, the Horn clauses defined in the previous examples become:
\begin{itemize}
	\item $(x_1 \land x_2) \imp \bot$,
	\item $(x_1 \land x_2) \imp x_3$.
\end{itemize}
\noindent We can use terms body, head the same way as with implicational basis. 
Before going any further, remind that any boolean formula $h$ can be 
rewritten in terms of \belemp{Conjunctive Normal Form} (or CNF). 

\begin{definition}[\midn{(pure) Horn CNF}] A \belemp{Horn CNF} (resp. 
\belemp{pure 
Horn CNF}) $\I$ over $\Sg$ is a conjunction of Horn clauses (resp. pure Horn
clauses) $\mathcal{C}_i$, $i \in \Ens{N}$:

	\[ \I = \bigwedge_{i} \mathcal{C}_i\]
	
\end{definition}

\begin{definition}[\midn{(pure) Horn formula}] A boolean function $h$ over 
$\Sg$
is a \belemp{Horn formula} (resp. \belemp{pure Horn formula}) if it can be 
represented with a Horn CNF $\I$ (resp. pure Horn CNF).
	
\end{definition}

Here is the link with our implication basis. A Horn CNF $\I$ can be seen as an
implicational basis. This relies on some notes:
\begin{itemize}
	\item for $P, Q, R$ propositional formulas, $(P \imp Q) \land (P \imp R)$ 
	is equivalent to $P \imp (Q \land R)$,
	\item representing sets with their \belemp{characteristic vectors} (or 
	bitmaps), translating $\cup$ by $\lor$ and $\cap$ by $\land$ give almost
	a one-to-one correspondence between sets and boolean formulas and their 
	models.
\end{itemize}

\paragraph{Example} To clarify let's think of short example taken from the 
previous section: $\Sg = \{ a, b, c, d, e \}$ and 

\[ \I = \{ ab \imp de, \ a \imp c, \ ce \imp b  \} \]

\noindent If we think of \textit{a, b, c, d, e} as boolean variables indicating
their presence into a set, we can derive the following translation of $\I$:

	\[ ((a \land b) \imp (d \land e)) \land (a \imp c) \land
		 ((c \land e) \imp b) \]

\noindent which can be represented by a (pure) Horn CNF, say $\I_h$:

\begin{align*}
	\I_h & = ((a \land b) \imp d) \land ((a \land b) \imp e) \land (a \imp c) 
	\land ((c \land e) \imp b) \\
	& \equiv (d \lor \lnot a \lor \lnot b) \land (e \lor \lnot a \lor \lnot b) 
	\land 
	(c \lor \lnot a) \land (b \lor \lnot e \lor \lnot c) 
\end{align*}

\noindent We could also have gone from $\I_h$ to $\I$.

\vspace{1.2em}

For this reason, we can also represent a Horn CNF as a set of clauses and 
clauses as pairs (body, head).

\begin{definition}[\midn{semantic entailment}] A formula $Q$ 
\belemp{semantically follows} from a formula $F$, denoted $F \models Q$ if 
every model of $F$ satisfies $Q$.
	
\end{definition}

Since the meaning of semantic entailment is the same as encountered with sets,
we can use the $\models$ the same way for both. This concludes our overview of
logical needs. We have seen the strong link between sets and logic which will 
allow us to go from one representation to another without any requirements. The
next section is dedicated to a short presentation of hypergraphs.


















