\chapter{Implementation}



\section{Test set up}

\subsection{Tools}

Here, we understand by tools the general implementation ground we rely on:
the languages, some classes, file formatting and so forth. Let us begin by
a very large point of view. We use two languages: C++ (11) for implementing
the algorithms and testing. For data visualization we use python with libraries \lil{matplotlib}, \lil{numpy}, \lil{csv}. To be more precise about C++, 
we use MinGW-64 compiler and \lil{-O3} optimization flag. Our tests are timed
with release builds. We recorded CPU-time, not wall-clock time which consider also the time spent outside the program. CPU is an Intel I5, 2.4 GHz.

\vspace{1.2em}

We used the code from \url{https://github.com/yazevnul/fcai}, used in 
\cite{bazhanov_optimizations_2014} for experience on \textsc{LinClosure}. It has been made under FCA context. In this framework, sets are represented as \lil{boost::dynamic_bitset} instances (named \lil{BitSet}). Because \lil{boost}
was already present in the project and has various built-in tools, we used it
for timing and recording purposes with \lil{boost::timer} and \lil{boost::accumulators} respectively.

\vspace{1.2em}

On top of implementation of the algorithms we wrote few tools to ease I/O and
testing. Even though we will not go into details, let us introduce quickly
main use of those tools, especially the file format we chose. Because we may need to load or save some implication theories, we have to think of a file format able to represent such basis easily. For this aim, we made \lil{.imp}
files, structured as follows:
\begin{itemize}
	\item[-] the first line contains the number of attributes and the number of
	implications, separated by a space,
	\item[-] because of the \lil{Bitset} representation, an implication is 
	written as lists of indices, premise and conclusion, separated by ">". Indices are also space separated.
\end{itemize}
For instance, consider our running example from the previous chapter:
$\Sg = \{ a, \ b, \  c, \  d, \  e, \ f \}$ and $\I = \{ ab \imp cde, \ cd \imp f, \ c \imp a, \ d \imp b, \  abcd \imp  ef \}$. Considering $a$ indexed as 0 and $f$ indexed as 5, the corresponding \lil{.imp} file will be:

\begin{lstlisting}[language = inline, style = Light]
6 5
0 1 > 2 3 4
2 3 > 5
2 > 0
3 > 1
0 1 2 3 > 4 5
\end{lstlisting}

\noindent Using functions \lil{ReadFile} and \lil{WriteFile} in the namespace \lil{ImplicationTools}, one can easily read or write implication basis:

\begin{lstlisting}[language = CoreCpp, style = Light]
theory L, Lc;  // alias for std::vector<FCA::ImplicationInd>
ImplicationTools::ReadFile("input_file.imp", L);
Minimize(L, Lc);  // result of some minimization into Lc
ImplicationTools::WriteFile("output_file.imp", Lc);
\end{lstlisting}

\noindent Note that the meaning of attributes here is somehow lost. So far, this is not an issue and it may be fixed by keeping a trace of correspondence between indices and attributes names. To test minimization, we use a class
\lil{GridTester} writing CSV results.



\begin{itemize}
	\item csv exportation, bitfiles
	\item quickly describe the class GridTest
\end{itemize}

\subsection{Randomly generated data}

This paragraph is important for all following experiences.
We would like to put the emphasis on the way we randomly generate sets and 
implications. First, let us focus on set generation. We use \lil{boost::random} 
and \lil{time} libraries for generating pseudo-random numbers. In particular 
for a set $X$, we use discrete uniform random distribution on the interval 
$\lbrack 0 \ ; \ |\Sg| \rbrack$:
\begin{enumerate}
	\item determine the size $|X|$ of $X$ by drawing a number out of our 
	distribution,
	\item draw again $|X|$ numbers. Because of the interval, we are sure to
	obtain valid indices of elements to set in $X$.
\end{enumerate}
\noindent Note that an element can be drawn more than once, resulting in 
effective $|X|$ smaller than the one we got at the beginning. We do not consider
this as an issue. Generating theories then is as follows:
\begin{enumerate}
	\item generate a conclusion randomly,
	\item generate a premise. Because we want implications to be informative,
	we keep as a premise the difference \textit{premise $-$ conclusion}.
	\item because empty premise is likely to occur several times, resulting in
	$\emptyset \imp \Sg$, we allow for no more than one empty premise. 
	Nevertheless, in order not to loop forever with this condition, we fixed
	a maximal amount of re-roll with a variable \lil{MAX_ITER}. Passed this 
	number of failure, we accept an empty premise anyway.
\end{enumerate}
\noindent Note that this method can be discussed and probably improved. For us,
it seems like this method is sufficient to provide theories with informative
implications, and uniformly distributed sizes of premises and conclusions thus
a "good" representation of the theories space. We did not investigate further 
ways of generations since the main task was to test algorithms and not to study
implication structures in depth. This anyway is an interesting question for
further work.

\subsection{Real data}

In this part we will be interested in application to real datasets. The application we used is FCA since the framework we use is dedicated to FCA
testing. We will first present briefly FCA and its correlation with our
minimization issue, before describing some real datasets we have been using
and their characteristics.

\subsubsection{Introduction to FCA}

Formal Concept Analysis is a technique relying on array-like data and lattices
to describe hierarchies in data. It can be used in data mining, text mining or
chemistry for instance.

\subsubsection{Some real datasets}

\section{Pruning the algorithms}

Because we can plug-in various closure procedures (\textsc{Closure}, \textsc{LinCLosure}), we are interested in knowing which configuration is
the most efficient for each minimization procedure. However, because we may
have several possible configurations and tests can be highly time consuming,
we will rely on the next assumption based on the result of \cite{bazhanov_optimizations_2014}. Actually, it is exhibited in this paper
that \textsc{LinClosure} performs worst in general than \textsc{Closure}. However, as we will see there are some possible optimizations for \textsc{LinClosure} dealing with initialization steps. Consequently, 
we will first give priority to \textsc{Closure}, and try to replace it 
by \textsc{LinClosure} whenever we find a possible optimized use of 
for this closure method. We will keep as the "best version" the most efficient
in the tests we will run.

\subsection{\textsc{MinCover}}

Recall that \textsc{MinCover} is a two-steps algorithm: right-closure and redundancy elimination. et $\I$ over $\Sg$ be the basis we are trying to minimize. In the first step, we do not remove or add any implications from
$\I$. Furthermore, we do not alter its premises. Therefore, when using
\textsc{LinClosure}, we may need to initialize counters and list only one
time. However, in the second loop, it is question of first removing an implication from $\I$. From our point of view, removing only one implication
$A \imp B$ in counters may be as complex as \textsc{LinClosure} itself:
\begin{itemize}
	\item[-] if the data structure used for \textit{list} in \textsc{LinClosure}
	is a chained list, then removing $A \imp B$ from some \textit{list}[$a$], $a \in A$ is $O(|\B|)$. Because this has to be done for all $a \in A$, the 
	overhaul operation should be $O(|\I|)$, like \textsc{LinClosure} and in particular, the initialization step
	\item[-] if the data structure is an array, there are again two possibilities. Either we store in \textit{list}[$a$] directly implications or indices of implications in $\I$, but then finding and removing an implication will be $O(|\I|)$. Or we can use marking procedure to store a boolean value at some index $i$ representing the $i-th$ implication to know
	whether or not the $i$-th implication should belong to \textit{list}[$a$]. Removing an implication then may be $O(|\Sg|)$. However, we should take care
	of updating all the boolean values  if we remove an implication from $\I$, because the $i$-th index may not correspond to the implication $i$ anymore. To avoid this update, we can in fact do not remove implications in $\I$ at
	all and put all redundant ones in some trash-marked state. This would require some more conditional statements in the nested for loop of the second step of \textsc{LinClosure}, but it may offer indeed a slight optimization.
\end{itemize}
\noindent We did not test the last possible optimizations due to lack of time,
but also because of the results we shall exhibit hereafter about \textsc{MinCover}, stressing on the bad behaviour of \textsc{LinClosure} in practice. Therefore, the two pseudo-codes we compared for \textsc{MinCover} are
algorithms \ref{alg:MinCoverClo}, \ref{alg:MinCoverLin}.

\vspace{1.2em}

\begin{minipage}[t]{0.4\textwidth}
	\begin{algorithm}[H]
		\TitleOfAlgo{\textsc{MinCoverClo}}
		\KwIn{$\I$: an implication base}
		\KwOut{the canonical base of $\I$}
		
		\BlankLine
		\BlankLine
		
		\ForEach{$A \imp B \in \I$}{
			$\I := \I - \{ A \imp B \}$ \;
			$B  := \textsc{Closure}(\I, A \cup B)$ \;
			$\I := \I \cup \{ A \imp B \}$ \;
		}
		
		\BlankLine
		
		\ForEach{$A \imp B \in \I$}{
			$\I := \I - \{ A \imp B \}$ \;
			$A  := \textsc{Closure}(\I, A)$ \;
			\If{$A \neq B$}{
				$\I := \I \cup \{ A \imp B \}$ \;	
			}
		}
	
		
		\label{alg:MinCoverClo}
	\end{algorithm}
\end{minipage}
~
\begin{minipage}[t]{0.4\textwidth}
	\begin{algorithm}[H]
		\TitleOfAlgo{\textsc{MinCoverLin}}
		\KwIn{$\I$: an implication base}
		\KwOut{the canonical base of $\I$}
		
		\BlankLine
		\BlankLine
		
		\textsc{LinClosureInit}$(\I)$ \;
		
		\ForEach{$A \imp B \in \I$}{
			$\I := \I - \{ A \imp B \}$ \;
			$B  := \textsc{LinClosure}(\I, A \cup B)$ \;
			$\I := \I \cup \{ A \imp B \}$ \;
		}
		
		\BlankLine
		
		\ForEach{$A \imp B \in \I$}{
			$\I := \I - \{ A \imp B \}$ \;
			$A  := \textsc{Closure}(\I, A)$ \;
			\If{$A \neq B$}{
				$\I := \I \cup \{ A \imp B \}$ \;	
			}
		}
	
		
		\label{alg:MinCoverLin}
	\end{algorithm}
\end{minipage}

\vspace{1.2em}

In \textsc{MinCoverLin} we used a function called \textsc{LinClosureInit}. In fact, this function is the initialization step of \textsc{LinClosure}. It sets up the containers \textit{list} and \textit{count}. Then, when we call \textsc{LinClosure}, we juste consider pure computation of the closure for a given set. In both versions, redundancy elimination is done the same way.

\begin{itemize}
	\item results of tests and so forth
\end{itemize}

\subsection{\textsc{BercziMinimization}}

The algorithm issued by Berczi and al in \cite{berczi_directed_2017} (algoritm \ref{alg:Berczi-min}) can be fine-tuned at first sight without considering closure operators. Indeed, recall that we compute only closure of premises under
$\I$ being our input basis, and $\I_c$ our output one. Furthermore, the algorithm suggests to compute the closure of some premises under $\I$ several 
times, which is extensive and redundant. Furthermore, because we build $\I_c$ only by adding implications, all the closure previously computed can only grow.
Therefore we can improve \textsc{BercziMinimization} by the next means:
\begin{itemize}
	\item[-] compute all $\I(A), A \in \B(\I)$, and store them in a list: $C_{\I}$,
	\item[-] keep a list of growing $\I_c(A), A \in \B(\I)$: $C_{\I_c}$.
\end{itemize}
\noindent To illustrate, we can observe the pseudo-code \textsc{BercziImp}. This algorithm does not strictly reflect its implementation of course, but it
presents the two ideas we were talking about previously. Observe that the closures under $\I$ are the most complicated to compute (because $|\B(\I_c)| \leq |\B(\I)|$ even though we repeatedly add implications to $\I_c$) whence
the interest of avoiding useless computations especially for $\I$.

\begin{algorithm}
	\KwIn{$\I$: an implication theory}
	\KwOut{$\I_c$: the DQ-basis of $\I$}
	
	\BlankLine
	\BlankLine
	
	$\I_c := \emptyset$ \;
	$C_{\I} := \emptyset, \; C_{\I_c} := \emptyset $ \;
	
	\ForEach{$A \imp B \in \I$}{
		$C_{\I}[A] = \I(A)$ \;
		$C_{\I_c}[A] = A$ \;
	}

	\BlankLine
	
	\While{$\exists A \in \B(\I)$ s.t $C_{\I_c}[A] \neq C_{\I}[A]$}{
		\ForEach{$A \imp B \in \I$}{
			\If{$C_{\I}[A] \neq C_{\I_c}[A]$}{
				$C_{\I_c}[A] = \I_c(C_{\I_c}[A])$ \;
			}
		}
	
		\BlankLine
	
		$A_P, P := \min\{A, C_{\I_c}[A]: \;  C_{\I_c}[A] \neq C_{\I}[A] \}$ \;
		$\I_c := \I_c \cup \{P \imp C_{\I}[A_P] \}$ \
		
	}

	\BlankLine
	
	return $\I_c$ \;
	
	\caption{\textsc{BercziImp}}
	\label{alg:Berczi-imp}
\end{algorithm}

When getting the minimum, note that we get both the minimum $\I_c$-closed set not closed in $\I$ and its associated premise (or equivalently, the associated $\I$-closure). Because notations may be a bit heavy, let us illustrate the meaning of $C_{\I}$ and $C_{\I_c}$ in an example.


\paragraph{Example} As usual, let us retake our small example:
\begin{itemize}
	\item $\Sg = \{a, \ b, \ c, \ d, \ e, \ f \}$,
	\item $\I =$ \{\textit{ab $\imp$ cde, cd $\imp$ f, c $\imp$ a, d $\imp$ b, 
		abcd $\imp$ ef} \} 
\end{itemize}
\noindent Table \ref{tab:berczi-tra} is a trace of the algorithm, using $C_{\I}$ and $C_{\I_c}$. The first column contains premises of $\I$, the second one elements of $C_{\I}$ (that is closures of premises of $\I$ under $\I$) and
the third one, $C_{\I_c}$ (closures of $\B(\I)$ under $\I_c$). The last column
says whether $C_{\I}[A] = C_{\I_c}[A]$.

\begin{table}[ht]
\centering
\subfloat[after initialization step and first loop]{
	\begin{tabular}{| c | c | c | c |}
		\hline \rowcolor{clouds}
		$\B(\I)$ & $C_{\I}$ & $C_{\I_c}$ & $=$ \\ \hline
		$ab$     & $abcdef$ & $ab$ & $\times$ \\ \hline
		$cd$     & $abcdef$ & $cd$ & $\times$  \\ \hline \rowcolor{emerald!40!white}
		$c$      & $ca$     & $c$ & $\times$ \\ \hline 
		$d$      & $db$     & $d$ & $ \times$ \\ \hline
		$abcd$   & $abcdef$ & $abcd$ & $\times$ \\ \hline
	\end{tabular}
	\begin{tabular}{c}
		$\I_c$ before \\ 
		$\emptyset$   \\ \\
		$\I_c$ after \\
		$c \imp ca$
	\end{tabular}	
}\quad
\subfloat[second loop]{
	\begin{tabular}{| c | c | c | c |}
		\hline \rowcolor{clouds}
		$\B(\I)$ & $C_{\I}$ & $C_{\I_c}$ & $=$ \\ \hline
		$ab$     & $abcdef$ & $ab$ & $\times$ \\ \hline
		$cd$     & $abcdef$ & $acd$ & $\times$ \\ \hline
		$c$      & $ca$     & $ca$ & $\lor$ \\ \hline \rowcolor{emerald!40!white}
		$d$      & $db$     & $d$ & $\times$ \\ \hline
		$abcd$   & $abcdef$ & $abcd$ & $\times$ \\ \hline
	\end{tabular}
	\begin{tabular}{c}
		$\I_c$ before \\ 
		$c \imp ca$   \\ \\
		$\I_c$ after \\
		$c \imp ca, \ d \imp bd$
	\end{tabular}	
}

\subfloat[third loop]{
	\begin{tabular}{| c | c | c | c |}
		\hline \rowcolor{clouds}
		$\B(\I)$ & $C_{\I}$ & $C_{\I_c}$ & $=$ \\ \hline \rowcolor{emerald!40!white}
		$ab$     & $abcdef$ & $ab$  & $\times$ \\ \hline
		$cd$     & $abcdef$ & $abcd$ & $\times$ \\ \hline
		$c$      & $ca$     & $ca$ & $\lor$ \\ \hline 
		$d$      & $db$     & $db$  & $\lor$ \\ \hline
		$abcd$   & $abcdef$ & $abcd$ & $\times$\\ \hline
	\end{tabular}
	\begin{tabular}{c}
		$\I_c$ before \\ 
		$c \imp ca, \ d \imp bd$   \\ \\
		$\I_c$ after \\
		$c \imp ca, \ d \imp bd, \ ab \imp abcdef$
	\end{tabular}	
}

\subfloat[final loop]{
	\begin{tabular}{| c | c | c | c |}
		\hline \rowcolor{clouds}
		$\B(\I)$ & $C_{\I}$ & $C_{\I_c}$ & $=$ \\ \hline 
		$ab$     & $abcdef$ & $abcdef$ & $\lor$ \\ \hline 
		$cd$     & $abcdef$ & $abcdef$ & $\lor$ \\ \hline 
		$c$      & $ca$     & $ca$ & $\lor$ \\ \hline 
		$d$      & $db$     & $db$ & $\lor$ \\ \hline
		$abcd$   & $abcdef$ & $abcdef$ & $\lor$ \\ \hline
	\end{tabular}
	\begin{tabular}{c}
		Resulting $\I_c$  \\ 
		$c \imp ca, \ d \imp bd, \ ab \imp abcdef$
	\end{tabular}	
}
\caption{Example of execution of \textsc{BercziImp} using $C_{\I}$ and $C_{\I_c}$}
\label{tab:berczi-tra}
\end{table}

At each step, the highlighted row is the set of premises and closure satisfying $\min \{ C_{\I_c}[A]: \ C_{\I_c}[A] \neq C_{\I}[A] \}$, or this row contains the
minimal (inclusion-wise) $\I_c$-closed set not closed in $\I$. As wished, the
only some closures of $\I_c$ are updated at each step, instead of re-computing all of them every time.

\vspace{1.2em}

 More than fine-tuning the principle, we can optimize the use of \textsc{LinClosure} on two aspects:
\begin{itemize}
	\item[(i)] in the loop where we initialize lists, we can use the same
	pruning as in the right-closing step in \textsc{MinCover},
	\item[(ii)] when computing closures of $\I_c$, because $\I_c$ is only
	increasing in implications, updating list and counters can be done in $O(|\Sg|)$,	hence better than the initialization step of $O(|\I|)$.
\end{itemize}
\noindent Consequently, we will again compare for this algorithm two versions, given as \textsc{BercziImpClo} and \textsc{BercziImpLin} (\ref{alg:BercziImpClo}, \ref{alg:BercziImpLin}).

\vspace{1.2em}

\begin{algorithm}[H]
	\TitleOfAlgo{\textsc{BercziImpClo}}
	\KwIn{$\I$: an implication theory}
	\KwOut{$\I_c$: the DQ-basis of $\I$}
	
	\BlankLine
	\BlankLine
	
	$\I_c := \emptyset$ \;
	$C_{\I} := \emptyset, \; C_{\I_c} := \emptyset $ \;
	
	\ForEach{$A \imp B \in \I$}{
		$C_{\I}[A] = \textsc{Closure}(\I, A)$ \;
		$C_{\I_c}[A] = A$ \;
	}
	
	\BlankLine
	
	\While{$\exists A \in \B(\I)$ s.t $C_{\I_c}[A] \neq C_{\I}[A]$}{
		\ForEach{$A \imp B \in \I$}{
			\If{$C_{\I}[A] \neq C_{\I_c}[A]$}{
				$C_{\I_c}[A] = \textsc{Closure}(\I_c, C_{\I_c}[A])$ \;
			}
		}
		
		\BlankLine
		
		$A_P, P := \min\{A, C_{\I_c}[A]: \;  C_{\I_c}[A] \neq C_{\I}[A] \}$ \;
		$\I_c := \I_c \cup \{P \imp C_{\I}[A_P] \}$ \;
		
	}
	
	\BlankLine
	
	return $\I_c$ \;
	

	\label{alg:BercziImpClo}
\end{algorithm}

\begin{algorithm}[H]
	\TitleOfAlgo{\textsc{BercziImpLin}}
	\KwIn{$\I$: an implication theory}
	\KwOut{$\I_c$: the DQ-basis of $\I$}
	
	\BlankLine
	\BlankLine
	
	$\I_c := \emptyset$ \;
	$C_{\I} := \emptyset, \; C_{\I_c} := \emptyset $ \;
	
	\textsc{LinClosureInit}($\I$) \;
	
	\ForEach{$A \imp B \in \I$}{
		$C_{\I}[A] = \textsc{LinClosure}(\I, A)$ \;
		$C_{\I_c}[A] = A$ \;
	}
	
	\BlankLine
	
	\While{$\exists A \in \B(\I)$ s.t $C_{\I_c}[A] \neq C_{\I}[A]$}{
		\ForEach{$A \imp B \in \I$}{
			\If{$C_{\I}[A] \neq C_{\I_c}[A]$}{
				$C_{\I_c}[A] = \textsc{LinClosure}(\I_c, C_{\I_c}[A])$ \;
			}
		}
		
		\BlankLine
		
		$A_P, P := \min\{A, C_{\I_c}[A]: \; C_{\I_c}[A] \neq C_{\I}[A] \}$ \;
		$\I_c := \I_c \cup \{P \imp C_{\I}[A_P] \}$ \;
		\textsc{LinClosureAddImp}($\I_c, P \imp C_{\I}[A_P]$)\;
		
	}
	
	\BlankLine
	
	return $\I_c$ \;
	
	\label{alg:BercziImpLin}
\end{algorithm}

\vspace{1.2em}

In fact, \textsc{BercziImpClo} does differ from \textsc{BercziImp} only by the closure notations. However, when dealing with \textsc{LinClosure} in \textsc{BercziImpLin}, we use first \textsc{LinClosureInit} to instanciate 
\textit{list} and \textit{count} for \textsc{LinClosure} in $\I$. Hence a call
to this procedure performs only the effective closure computations. Then, 
\textsc{LinClosureAddImp} just updates \textit{list} and \textit{count} for $\I_c$ by adding a new implication in them. Note that before the first call to
this subroutine, \textit{list} and \textit{count} for $I_c$ are empty, hence
\textsc{LinClosure}($\I_c$, $X$) returns $X$.

\section{Joint comparison}