\centering
\subfloat[][FD-Graph of $1 \imp 23$]{
\begin{tikzpicture}
\node[Vertex, label=left:{1}] (1) at (-0.5, 0) {};
\node[Vertex, label=right:{2}] (2) at (0.5, 0.5) {};
\node[Vertex, label=right:{3}] (3) at (0.5, -0.5) {};

\draw[->] (1) -- (2);
\draw[->] (1) -- (3);
\end{tikzpicture}
}\qquad
\subfloat[][FD-Graph of $23 \imp 1$]{
\begin{tikzpicture}
\node[Vertex, label=below:{23}] (23) at (0, 0) {};
\node[Vertex, label=right:{1}] (1) at (1, 0) {};
\node[Vertex, label=left:{2}] (2) at (-1, 0.5) {};
\node[Vertex, label=left:{3}] (3) at (-1, -0.5) {};

\draw[->, dotted] (23) -- (2);
\draw[->, dotted] (23) -- (3);
\draw[->] (23) -- (1);
\end{tikzpicture}		
}

\subfloat[][FD-Graph of $12 \imp 34$]{
\begin{tikzpicture}
\node[Vertex, label=below:{12}] (12) at (0, 0) {};
\node[Vertex, label=left:{1}] (1) at (-1, 0.5) {};
\node[Vertex, label=left:{2}] (2) at (-1, -0.5) {};
\node[Vertex, label=right:{3}] (3) at (1, -0.5) {};
\node[Vertex, label=right:{4}] (4) at (1, 0.5) {};

\draw[->, dotted] (12) -- (2);
\draw[->, dotted] (12) -- (1);
\draw[->] (12) -- (3);
\draw[->] (12) -- (4);
\end{tikzpicture}	
}

\caption{Representation of some FD-graph}
\label{fig:FD-Graph-1}