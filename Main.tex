\documentclass[a4paper]{report}

\input{../Subfiles/Packages.tex}

\subfloat[Average time (in $s$), $|\Sg| = 100$]{
	\hspace{-3em}
	\scalebox{0.44}{%% Creator: Matplotlib, PGF backend
%%
%% To include the figure in your LaTeX document, write
%%   \input{<filename>.pgf}
%%
%% Make sure the required packages are loaded in your preamble
%%   \usepackage{pgf}
%%
%% Figures using additional raster images can only be included by \input if
%% they are in the same directory as the main LaTeX file. For loading figures
%% from other directories you can use the `import` package
%%   \usepackage{import}
%% and then include the figures with
%%   \import{<path to file>}{<filename>.pgf}
%%
%% Matplotlib used the following preamble
%%   \usepackage{fontspec}
%%   \setmainfont{DejaVu Serif}
%%   \setsansfont{DejaVu Sans}
%%   \setmonofont{DejaVu Sans Mono}
%%
\begingroup%
\makeatletter%
\begin{pgfpicture}%
\pgfpathrectangle{\pgfpointorigin}{\pgfqpoint{8.100000in}{6.600000in}}%
\pgfusepath{use as bounding box, clip}%
\begin{pgfscope}%
\pgfsetbuttcap%
\pgfsetmiterjoin%
\definecolor{currentfill}{rgb}{1.000000,1.000000,1.000000}%
\pgfsetfillcolor{currentfill}%
\pgfsetlinewidth{0.000000pt}%
\definecolor{currentstroke}{rgb}{1.000000,1.000000,1.000000}%
\pgfsetstrokecolor{currentstroke}%
\pgfsetdash{}{0pt}%
\pgfpathmoveto{\pgfqpoint{0.000000in}{0.000000in}}%
\pgfpathlineto{\pgfqpoint{8.100000in}{0.000000in}}%
\pgfpathlineto{\pgfqpoint{8.100000in}{6.600000in}}%
\pgfpathlineto{\pgfqpoint{0.000000in}{6.600000in}}%
\pgfpathclose%
\pgfusepath{fill}%
\end{pgfscope}%
\begin{pgfscope}%
\pgfsetbuttcap%
\pgfsetmiterjoin%
\definecolor{currentfill}{rgb}{1.000000,1.000000,1.000000}%
\pgfsetfillcolor{currentfill}%
\pgfsetlinewidth{0.000000pt}%
\definecolor{currentstroke}{rgb}{0.000000,0.000000,0.000000}%
\pgfsetstrokecolor{currentstroke}%
\pgfsetstrokeopacity{0.000000}%
\pgfsetdash{}{0pt}%
\pgfpathmoveto{\pgfqpoint{1.012500in}{0.726000in}}%
\pgfpathlineto{\pgfqpoint{7.290000in}{0.726000in}}%
\pgfpathlineto{\pgfqpoint{7.290000in}{5.808000in}}%
\pgfpathlineto{\pgfqpoint{1.012500in}{5.808000in}}%
\pgfpathclose%
\pgfusepath{fill}%
\end{pgfscope}%
\begin{pgfscope}%
\pgfsetbuttcap%
\pgfsetroundjoin%
\definecolor{currentfill}{rgb}{0.000000,0.000000,0.000000}%
\pgfsetfillcolor{currentfill}%
\pgfsetlinewidth{0.803000pt}%
\definecolor{currentstroke}{rgb}{0.000000,0.000000,0.000000}%
\pgfsetstrokecolor{currentstroke}%
\pgfsetdash{}{0pt}%
\pgfsys@defobject{currentmarker}{\pgfqpoint{0.000000in}{-0.048611in}}{\pgfqpoint{0.000000in}{0.000000in}}{%
\pgfpathmoveto{\pgfqpoint{0.000000in}{0.000000in}}%
\pgfpathlineto{\pgfqpoint{0.000000in}{-0.048611in}}%
\pgfusepath{stroke,fill}%
}%
\begin{pgfscope}%
\pgfsys@transformshift{1.012500in}{0.726000in}%
\pgfsys@useobject{currentmarker}{}%
\end{pgfscope}%
\end{pgfscope}%
\begin{pgfscope}%
\pgftext[x=1.012500in,y=0.628778in,,top]{\sffamily\fontsize{10.000000}{12.000000}\selectfont \(\displaystyle 0\)}%
\end{pgfscope}%
\begin{pgfscope}%
\pgfsetbuttcap%
\pgfsetroundjoin%
\definecolor{currentfill}{rgb}{0.000000,0.000000,0.000000}%
\pgfsetfillcolor{currentfill}%
\pgfsetlinewidth{0.803000pt}%
\definecolor{currentstroke}{rgb}{0.000000,0.000000,0.000000}%
\pgfsetstrokecolor{currentstroke}%
\pgfsetdash{}{0pt}%
\pgfsys@defobject{currentmarker}{\pgfqpoint{0.000000in}{-0.048611in}}{\pgfqpoint{0.000000in}{0.000000in}}{%
\pgfpathmoveto{\pgfqpoint{0.000000in}{0.000000in}}%
\pgfpathlineto{\pgfqpoint{0.000000in}{-0.048611in}}%
\pgfusepath{stroke,fill}%
}%
\begin{pgfscope}%
\pgfsys@transformshift{1.611784in}{0.726000in}%
\pgfsys@useobject{currentmarker}{}%
\end{pgfscope}%
\end{pgfscope}%
\begin{pgfscope}%
\pgftext[x=1.611784in,y=0.628778in,,top]{\sffamily\fontsize{10.000000}{12.000000}\selectfont \(\displaystyle 2000\)}%
\end{pgfscope}%
\begin{pgfscope}%
\pgfsetbuttcap%
\pgfsetroundjoin%
\definecolor{currentfill}{rgb}{0.000000,0.000000,0.000000}%
\pgfsetfillcolor{currentfill}%
\pgfsetlinewidth{0.803000pt}%
\definecolor{currentstroke}{rgb}{0.000000,0.000000,0.000000}%
\pgfsetstrokecolor{currentstroke}%
\pgfsetdash{}{0pt}%
\pgfsys@defobject{currentmarker}{\pgfqpoint{0.000000in}{-0.048611in}}{\pgfqpoint{0.000000in}{0.000000in}}{%
\pgfpathmoveto{\pgfqpoint{0.000000in}{0.000000in}}%
\pgfpathlineto{\pgfqpoint{0.000000in}{-0.048611in}}%
\pgfusepath{stroke,fill}%
}%
\begin{pgfscope}%
\pgfsys@transformshift{2.211068in}{0.726000in}%
\pgfsys@useobject{currentmarker}{}%
\end{pgfscope}%
\end{pgfscope}%
\begin{pgfscope}%
\pgftext[x=2.211068in,y=0.628778in,,top]{\sffamily\fontsize{10.000000}{12.000000}\selectfont \(\displaystyle 4000\)}%
\end{pgfscope}%
\begin{pgfscope}%
\pgfsetbuttcap%
\pgfsetroundjoin%
\definecolor{currentfill}{rgb}{0.000000,0.000000,0.000000}%
\pgfsetfillcolor{currentfill}%
\pgfsetlinewidth{0.803000pt}%
\definecolor{currentstroke}{rgb}{0.000000,0.000000,0.000000}%
\pgfsetstrokecolor{currentstroke}%
\pgfsetdash{}{0pt}%
\pgfsys@defobject{currentmarker}{\pgfqpoint{0.000000in}{-0.048611in}}{\pgfqpoint{0.000000in}{0.000000in}}{%
\pgfpathmoveto{\pgfqpoint{0.000000in}{0.000000in}}%
\pgfpathlineto{\pgfqpoint{0.000000in}{-0.048611in}}%
\pgfusepath{stroke,fill}%
}%
\begin{pgfscope}%
\pgfsys@transformshift{2.810352in}{0.726000in}%
\pgfsys@useobject{currentmarker}{}%
\end{pgfscope}%
\end{pgfscope}%
\begin{pgfscope}%
\pgftext[x=2.810352in,y=0.628778in,,top]{\sffamily\fontsize{10.000000}{12.000000}\selectfont \(\displaystyle 6000\)}%
\end{pgfscope}%
\begin{pgfscope}%
\pgfsetbuttcap%
\pgfsetroundjoin%
\definecolor{currentfill}{rgb}{0.000000,0.000000,0.000000}%
\pgfsetfillcolor{currentfill}%
\pgfsetlinewidth{0.803000pt}%
\definecolor{currentstroke}{rgb}{0.000000,0.000000,0.000000}%
\pgfsetstrokecolor{currentstroke}%
\pgfsetdash{}{0pt}%
\pgfsys@defobject{currentmarker}{\pgfqpoint{0.000000in}{-0.048611in}}{\pgfqpoint{0.000000in}{0.000000in}}{%
\pgfpathmoveto{\pgfqpoint{0.000000in}{0.000000in}}%
\pgfpathlineto{\pgfqpoint{0.000000in}{-0.048611in}}%
\pgfusepath{stroke,fill}%
}%
\begin{pgfscope}%
\pgfsys@transformshift{3.409636in}{0.726000in}%
\pgfsys@useobject{currentmarker}{}%
\end{pgfscope}%
\end{pgfscope}%
\begin{pgfscope}%
\pgftext[x=3.409636in,y=0.628778in,,top]{\sffamily\fontsize{10.000000}{12.000000}\selectfont \(\displaystyle 8000\)}%
\end{pgfscope}%
\begin{pgfscope}%
\pgfsetbuttcap%
\pgfsetroundjoin%
\definecolor{currentfill}{rgb}{0.000000,0.000000,0.000000}%
\pgfsetfillcolor{currentfill}%
\pgfsetlinewidth{0.803000pt}%
\definecolor{currentstroke}{rgb}{0.000000,0.000000,0.000000}%
\pgfsetstrokecolor{currentstroke}%
\pgfsetdash{}{0pt}%
\pgfsys@defobject{currentmarker}{\pgfqpoint{0.000000in}{-0.048611in}}{\pgfqpoint{0.000000in}{0.000000in}}{%
\pgfpathmoveto{\pgfqpoint{0.000000in}{0.000000in}}%
\pgfpathlineto{\pgfqpoint{0.000000in}{-0.048611in}}%
\pgfusepath{stroke,fill}%
}%
\begin{pgfscope}%
\pgfsys@transformshift{4.008920in}{0.726000in}%
\pgfsys@useobject{currentmarker}{}%
\end{pgfscope}%
\end{pgfscope}%
\begin{pgfscope}%
\pgftext[x=4.008920in,y=0.628778in,,top]{\sffamily\fontsize{10.000000}{12.000000}\selectfont \(\displaystyle 10000\)}%
\end{pgfscope}%
\begin{pgfscope}%
\pgfsetbuttcap%
\pgfsetroundjoin%
\definecolor{currentfill}{rgb}{0.000000,0.000000,0.000000}%
\pgfsetfillcolor{currentfill}%
\pgfsetlinewidth{0.803000pt}%
\definecolor{currentstroke}{rgb}{0.000000,0.000000,0.000000}%
\pgfsetstrokecolor{currentstroke}%
\pgfsetdash{}{0pt}%
\pgfsys@defobject{currentmarker}{\pgfqpoint{0.000000in}{-0.048611in}}{\pgfqpoint{0.000000in}{0.000000in}}{%
\pgfpathmoveto{\pgfqpoint{0.000000in}{0.000000in}}%
\pgfpathlineto{\pgfqpoint{0.000000in}{-0.048611in}}%
\pgfusepath{stroke,fill}%
}%
\begin{pgfscope}%
\pgfsys@transformshift{4.608204in}{0.726000in}%
\pgfsys@useobject{currentmarker}{}%
\end{pgfscope}%
\end{pgfscope}%
\begin{pgfscope}%
\pgftext[x=4.608204in,y=0.628778in,,top]{\sffamily\fontsize{10.000000}{12.000000}\selectfont \(\displaystyle 12000\)}%
\end{pgfscope}%
\begin{pgfscope}%
\pgfsetbuttcap%
\pgfsetroundjoin%
\definecolor{currentfill}{rgb}{0.000000,0.000000,0.000000}%
\pgfsetfillcolor{currentfill}%
\pgfsetlinewidth{0.803000pt}%
\definecolor{currentstroke}{rgb}{0.000000,0.000000,0.000000}%
\pgfsetstrokecolor{currentstroke}%
\pgfsetdash{}{0pt}%
\pgfsys@defobject{currentmarker}{\pgfqpoint{0.000000in}{-0.048611in}}{\pgfqpoint{0.000000in}{0.000000in}}{%
\pgfpathmoveto{\pgfqpoint{0.000000in}{0.000000in}}%
\pgfpathlineto{\pgfqpoint{0.000000in}{-0.048611in}}%
\pgfusepath{stroke,fill}%
}%
\begin{pgfscope}%
\pgfsys@transformshift{5.207488in}{0.726000in}%
\pgfsys@useobject{currentmarker}{}%
\end{pgfscope}%
\end{pgfscope}%
\begin{pgfscope}%
\pgftext[x=5.207488in,y=0.628778in,,top]{\sffamily\fontsize{10.000000}{12.000000}\selectfont \(\displaystyle 14000\)}%
\end{pgfscope}%
\begin{pgfscope}%
\pgfsetbuttcap%
\pgfsetroundjoin%
\definecolor{currentfill}{rgb}{0.000000,0.000000,0.000000}%
\pgfsetfillcolor{currentfill}%
\pgfsetlinewidth{0.803000pt}%
\definecolor{currentstroke}{rgb}{0.000000,0.000000,0.000000}%
\pgfsetstrokecolor{currentstroke}%
\pgfsetdash{}{0pt}%
\pgfsys@defobject{currentmarker}{\pgfqpoint{0.000000in}{-0.048611in}}{\pgfqpoint{0.000000in}{0.000000in}}{%
\pgfpathmoveto{\pgfqpoint{0.000000in}{0.000000in}}%
\pgfpathlineto{\pgfqpoint{0.000000in}{-0.048611in}}%
\pgfusepath{stroke,fill}%
}%
\begin{pgfscope}%
\pgfsys@transformshift{5.806772in}{0.726000in}%
\pgfsys@useobject{currentmarker}{}%
\end{pgfscope}%
\end{pgfscope}%
\begin{pgfscope}%
\pgftext[x=5.806772in,y=0.628778in,,top]{\sffamily\fontsize{10.000000}{12.000000}\selectfont \(\displaystyle 16000\)}%
\end{pgfscope}%
\begin{pgfscope}%
\pgfsetbuttcap%
\pgfsetroundjoin%
\definecolor{currentfill}{rgb}{0.000000,0.000000,0.000000}%
\pgfsetfillcolor{currentfill}%
\pgfsetlinewidth{0.803000pt}%
\definecolor{currentstroke}{rgb}{0.000000,0.000000,0.000000}%
\pgfsetstrokecolor{currentstroke}%
\pgfsetdash{}{0pt}%
\pgfsys@defobject{currentmarker}{\pgfqpoint{0.000000in}{-0.048611in}}{\pgfqpoint{0.000000in}{0.000000in}}{%
\pgfpathmoveto{\pgfqpoint{0.000000in}{0.000000in}}%
\pgfpathlineto{\pgfqpoint{0.000000in}{-0.048611in}}%
\pgfusepath{stroke,fill}%
}%
\begin{pgfscope}%
\pgfsys@transformshift{6.406056in}{0.726000in}%
\pgfsys@useobject{currentmarker}{}%
\end{pgfscope}%
\end{pgfscope}%
\begin{pgfscope}%
\pgftext[x=6.406056in,y=0.628778in,,top]{\sffamily\fontsize{10.000000}{12.000000}\selectfont \(\displaystyle 18000\)}%
\end{pgfscope}%
\begin{pgfscope}%
\pgfsetbuttcap%
\pgfsetroundjoin%
\definecolor{currentfill}{rgb}{0.000000,0.000000,0.000000}%
\pgfsetfillcolor{currentfill}%
\pgfsetlinewidth{0.803000pt}%
\definecolor{currentstroke}{rgb}{0.000000,0.000000,0.000000}%
\pgfsetstrokecolor{currentstroke}%
\pgfsetdash{}{0pt}%
\pgfsys@defobject{currentmarker}{\pgfqpoint{0.000000in}{-0.048611in}}{\pgfqpoint{0.000000in}{0.000000in}}{%
\pgfpathmoveto{\pgfqpoint{0.000000in}{0.000000in}}%
\pgfpathlineto{\pgfqpoint{0.000000in}{-0.048611in}}%
\pgfusepath{stroke,fill}%
}%
\begin{pgfscope}%
\pgfsys@transformshift{7.005340in}{0.726000in}%
\pgfsys@useobject{currentmarker}{}%
\end{pgfscope}%
\end{pgfscope}%
\begin{pgfscope}%
\pgftext[x=7.005340in,y=0.628778in,,top]{\sffamily\fontsize{10.000000}{12.000000}\selectfont \(\displaystyle 20000\)}%
\end{pgfscope}%
\begin{pgfscope}%
\pgftext[x=4.151250in,y=0.438809in,,top]{\sffamily\fontsize{12.000000}{14.400000}\selectfont \(\displaystyle |\mathcal{B}|\)}%
\end{pgfscope}%
\begin{pgfscope}%
\pgfsetbuttcap%
\pgfsetroundjoin%
\definecolor{currentfill}{rgb}{0.000000,0.000000,0.000000}%
\pgfsetfillcolor{currentfill}%
\pgfsetlinewidth{0.803000pt}%
\definecolor{currentstroke}{rgb}{0.000000,0.000000,0.000000}%
\pgfsetstrokecolor{currentstroke}%
\pgfsetdash{}{0pt}%
\pgfsys@defobject{currentmarker}{\pgfqpoint{-0.048611in}{0.000000in}}{\pgfqpoint{0.000000in}{0.000000in}}{%
\pgfpathmoveto{\pgfqpoint{0.000000in}{0.000000in}}%
\pgfpathlineto{\pgfqpoint{-0.048611in}{0.000000in}}%
\pgfusepath{stroke,fill}%
}%
\begin{pgfscope}%
\pgfsys@transformshift{1.012500in}{0.956312in}%
\pgfsys@useobject{currentmarker}{}%
\end{pgfscope}%
\end{pgfscope}%
\begin{pgfscope}%
\pgftext[x=0.845833in,y=0.903550in,left,base]{\sffamily\fontsize{10.000000}{12.000000}\selectfont \(\displaystyle 0\)}%
\end{pgfscope}%
\begin{pgfscope}%
\pgfsetbuttcap%
\pgfsetroundjoin%
\definecolor{currentfill}{rgb}{0.000000,0.000000,0.000000}%
\pgfsetfillcolor{currentfill}%
\pgfsetlinewidth{0.803000pt}%
\definecolor{currentstroke}{rgb}{0.000000,0.000000,0.000000}%
\pgfsetstrokecolor{currentstroke}%
\pgfsetdash{}{0pt}%
\pgfsys@defobject{currentmarker}{\pgfqpoint{-0.048611in}{0.000000in}}{\pgfqpoint{0.000000in}{0.000000in}}{%
\pgfpathmoveto{\pgfqpoint{0.000000in}{0.000000in}}%
\pgfpathlineto{\pgfqpoint{-0.048611in}{0.000000in}}%
\pgfusepath{stroke,fill}%
}%
\begin{pgfscope}%
\pgfsys@transformshift{1.012500in}{1.163097in}%
\pgfsys@useobject{currentmarker}{}%
\end{pgfscope}%
\end{pgfscope}%
\begin{pgfscope}%
\pgftext[x=0.776388in,y=1.110336in,left,base]{\sffamily\fontsize{10.000000}{12.000000}\selectfont \(\displaystyle 10\)}%
\end{pgfscope}%
\begin{pgfscope}%
\pgfsetbuttcap%
\pgfsetroundjoin%
\definecolor{currentfill}{rgb}{0.000000,0.000000,0.000000}%
\pgfsetfillcolor{currentfill}%
\pgfsetlinewidth{0.803000pt}%
\definecolor{currentstroke}{rgb}{0.000000,0.000000,0.000000}%
\pgfsetstrokecolor{currentstroke}%
\pgfsetdash{}{0pt}%
\pgfsys@defobject{currentmarker}{\pgfqpoint{-0.048611in}{0.000000in}}{\pgfqpoint{0.000000in}{0.000000in}}{%
\pgfpathmoveto{\pgfqpoint{0.000000in}{0.000000in}}%
\pgfpathlineto{\pgfqpoint{-0.048611in}{0.000000in}}%
\pgfusepath{stroke,fill}%
}%
\begin{pgfscope}%
\pgfsys@transformshift{1.012500in}{1.369883in}%
\pgfsys@useobject{currentmarker}{}%
\end{pgfscope}%
\end{pgfscope}%
\begin{pgfscope}%
\pgftext[x=0.776388in,y=1.317122in,left,base]{\sffamily\fontsize{10.000000}{12.000000}\selectfont \(\displaystyle 20\)}%
\end{pgfscope}%
\begin{pgfscope}%
\pgfsetbuttcap%
\pgfsetroundjoin%
\definecolor{currentfill}{rgb}{0.000000,0.000000,0.000000}%
\pgfsetfillcolor{currentfill}%
\pgfsetlinewidth{0.803000pt}%
\definecolor{currentstroke}{rgb}{0.000000,0.000000,0.000000}%
\pgfsetstrokecolor{currentstroke}%
\pgfsetdash{}{0pt}%
\pgfsys@defobject{currentmarker}{\pgfqpoint{-0.048611in}{0.000000in}}{\pgfqpoint{0.000000in}{0.000000in}}{%
\pgfpathmoveto{\pgfqpoint{0.000000in}{0.000000in}}%
\pgfpathlineto{\pgfqpoint{-0.048611in}{0.000000in}}%
\pgfusepath{stroke,fill}%
}%
\begin{pgfscope}%
\pgfsys@transformshift{1.012500in}{1.576669in}%
\pgfsys@useobject{currentmarker}{}%
\end{pgfscope}%
\end{pgfscope}%
\begin{pgfscope}%
\pgftext[x=0.776388in,y=1.523907in,left,base]{\sffamily\fontsize{10.000000}{12.000000}\selectfont \(\displaystyle 30\)}%
\end{pgfscope}%
\begin{pgfscope}%
\pgfsetbuttcap%
\pgfsetroundjoin%
\definecolor{currentfill}{rgb}{0.000000,0.000000,0.000000}%
\pgfsetfillcolor{currentfill}%
\pgfsetlinewidth{0.803000pt}%
\definecolor{currentstroke}{rgb}{0.000000,0.000000,0.000000}%
\pgfsetstrokecolor{currentstroke}%
\pgfsetdash{}{0pt}%
\pgfsys@defobject{currentmarker}{\pgfqpoint{-0.048611in}{0.000000in}}{\pgfqpoint{0.000000in}{0.000000in}}{%
\pgfpathmoveto{\pgfqpoint{0.000000in}{0.000000in}}%
\pgfpathlineto{\pgfqpoint{-0.048611in}{0.000000in}}%
\pgfusepath{stroke,fill}%
}%
\begin{pgfscope}%
\pgfsys@transformshift{1.012500in}{1.783455in}%
\pgfsys@useobject{currentmarker}{}%
\end{pgfscope}%
\end{pgfscope}%
\begin{pgfscope}%
\pgftext[x=0.776388in,y=1.730693in,left,base]{\sffamily\fontsize{10.000000}{12.000000}\selectfont \(\displaystyle 40\)}%
\end{pgfscope}%
\begin{pgfscope}%
\pgfsetbuttcap%
\pgfsetroundjoin%
\definecolor{currentfill}{rgb}{0.000000,0.000000,0.000000}%
\pgfsetfillcolor{currentfill}%
\pgfsetlinewidth{0.803000pt}%
\definecolor{currentstroke}{rgb}{0.000000,0.000000,0.000000}%
\pgfsetstrokecolor{currentstroke}%
\pgfsetdash{}{0pt}%
\pgfsys@defobject{currentmarker}{\pgfqpoint{-0.048611in}{0.000000in}}{\pgfqpoint{0.000000in}{0.000000in}}{%
\pgfpathmoveto{\pgfqpoint{0.000000in}{0.000000in}}%
\pgfpathlineto{\pgfqpoint{-0.048611in}{0.000000in}}%
\pgfusepath{stroke,fill}%
}%
\begin{pgfscope}%
\pgfsys@transformshift{1.012500in}{1.990240in}%
\pgfsys@useobject{currentmarker}{}%
\end{pgfscope}%
\end{pgfscope}%
\begin{pgfscope}%
\pgftext[x=0.776388in,y=1.937479in,left,base]{\sffamily\fontsize{10.000000}{12.000000}\selectfont \(\displaystyle 50\)}%
\end{pgfscope}%
\begin{pgfscope}%
\pgfsetbuttcap%
\pgfsetroundjoin%
\definecolor{currentfill}{rgb}{0.000000,0.000000,0.000000}%
\pgfsetfillcolor{currentfill}%
\pgfsetlinewidth{0.803000pt}%
\definecolor{currentstroke}{rgb}{0.000000,0.000000,0.000000}%
\pgfsetstrokecolor{currentstroke}%
\pgfsetdash{}{0pt}%
\pgfsys@defobject{currentmarker}{\pgfqpoint{-0.048611in}{0.000000in}}{\pgfqpoint{0.000000in}{0.000000in}}{%
\pgfpathmoveto{\pgfqpoint{0.000000in}{0.000000in}}%
\pgfpathlineto{\pgfqpoint{-0.048611in}{0.000000in}}%
\pgfusepath{stroke,fill}%
}%
\begin{pgfscope}%
\pgfsys@transformshift{1.012500in}{2.197026in}%
\pgfsys@useobject{currentmarker}{}%
\end{pgfscope}%
\end{pgfscope}%
\begin{pgfscope}%
\pgftext[x=0.776388in,y=2.144264in,left,base]{\sffamily\fontsize{10.000000}{12.000000}\selectfont \(\displaystyle 60\)}%
\end{pgfscope}%
\begin{pgfscope}%
\pgfsetbuttcap%
\pgfsetroundjoin%
\definecolor{currentfill}{rgb}{0.000000,0.000000,0.000000}%
\pgfsetfillcolor{currentfill}%
\pgfsetlinewidth{0.803000pt}%
\definecolor{currentstroke}{rgb}{0.000000,0.000000,0.000000}%
\pgfsetstrokecolor{currentstroke}%
\pgfsetdash{}{0pt}%
\pgfsys@defobject{currentmarker}{\pgfqpoint{-0.048611in}{0.000000in}}{\pgfqpoint{0.000000in}{0.000000in}}{%
\pgfpathmoveto{\pgfqpoint{0.000000in}{0.000000in}}%
\pgfpathlineto{\pgfqpoint{-0.048611in}{0.000000in}}%
\pgfusepath{stroke,fill}%
}%
\begin{pgfscope}%
\pgfsys@transformshift{1.012500in}{2.403812in}%
\pgfsys@useobject{currentmarker}{}%
\end{pgfscope}%
\end{pgfscope}%
\begin{pgfscope}%
\pgftext[x=0.776388in,y=2.351050in,left,base]{\sffamily\fontsize{10.000000}{12.000000}\selectfont \(\displaystyle 70\)}%
\end{pgfscope}%
\begin{pgfscope}%
\pgfsetbuttcap%
\pgfsetroundjoin%
\definecolor{currentfill}{rgb}{0.000000,0.000000,0.000000}%
\pgfsetfillcolor{currentfill}%
\pgfsetlinewidth{0.803000pt}%
\definecolor{currentstroke}{rgb}{0.000000,0.000000,0.000000}%
\pgfsetstrokecolor{currentstroke}%
\pgfsetdash{}{0pt}%
\pgfsys@defobject{currentmarker}{\pgfqpoint{-0.048611in}{0.000000in}}{\pgfqpoint{0.000000in}{0.000000in}}{%
\pgfpathmoveto{\pgfqpoint{0.000000in}{0.000000in}}%
\pgfpathlineto{\pgfqpoint{-0.048611in}{0.000000in}}%
\pgfusepath{stroke,fill}%
}%
\begin{pgfscope}%
\pgfsys@transformshift{1.012500in}{2.610597in}%
\pgfsys@useobject{currentmarker}{}%
\end{pgfscope}%
\end{pgfscope}%
\begin{pgfscope}%
\pgftext[x=0.776388in,y=2.557836in,left,base]{\sffamily\fontsize{10.000000}{12.000000}\selectfont \(\displaystyle 80\)}%
\end{pgfscope}%
\begin{pgfscope}%
\pgfsetbuttcap%
\pgfsetroundjoin%
\definecolor{currentfill}{rgb}{0.000000,0.000000,0.000000}%
\pgfsetfillcolor{currentfill}%
\pgfsetlinewidth{0.803000pt}%
\definecolor{currentstroke}{rgb}{0.000000,0.000000,0.000000}%
\pgfsetstrokecolor{currentstroke}%
\pgfsetdash{}{0pt}%
\pgfsys@defobject{currentmarker}{\pgfqpoint{-0.048611in}{0.000000in}}{\pgfqpoint{0.000000in}{0.000000in}}{%
\pgfpathmoveto{\pgfqpoint{0.000000in}{0.000000in}}%
\pgfpathlineto{\pgfqpoint{-0.048611in}{0.000000in}}%
\pgfusepath{stroke,fill}%
}%
\begin{pgfscope}%
\pgfsys@transformshift{1.012500in}{2.817383in}%
\pgfsys@useobject{currentmarker}{}%
\end{pgfscope}%
\end{pgfscope}%
\begin{pgfscope}%
\pgftext[x=0.776388in,y=2.764621in,left,base]{\sffamily\fontsize{10.000000}{12.000000}\selectfont \(\displaystyle 90\)}%
\end{pgfscope}%
\begin{pgfscope}%
\pgfsetbuttcap%
\pgfsetroundjoin%
\definecolor{currentfill}{rgb}{0.000000,0.000000,0.000000}%
\pgfsetfillcolor{currentfill}%
\pgfsetlinewidth{0.803000pt}%
\definecolor{currentstroke}{rgb}{0.000000,0.000000,0.000000}%
\pgfsetstrokecolor{currentstroke}%
\pgfsetdash{}{0pt}%
\pgfsys@defobject{currentmarker}{\pgfqpoint{-0.048611in}{0.000000in}}{\pgfqpoint{0.000000in}{0.000000in}}{%
\pgfpathmoveto{\pgfqpoint{0.000000in}{0.000000in}}%
\pgfpathlineto{\pgfqpoint{-0.048611in}{0.000000in}}%
\pgfusepath{stroke,fill}%
}%
\begin{pgfscope}%
\pgfsys@transformshift{1.012500in}{3.024169in}%
\pgfsys@useobject{currentmarker}{}%
\end{pgfscope}%
\end{pgfscope}%
\begin{pgfscope}%
\pgftext[x=0.706944in,y=2.971407in,left,base]{\sffamily\fontsize{10.000000}{12.000000}\selectfont \(\displaystyle 100\)}%
\end{pgfscope}%
\begin{pgfscope}%
\pgfsetbuttcap%
\pgfsetroundjoin%
\definecolor{currentfill}{rgb}{0.000000,0.000000,0.000000}%
\pgfsetfillcolor{currentfill}%
\pgfsetlinewidth{0.803000pt}%
\definecolor{currentstroke}{rgb}{0.000000,0.000000,0.000000}%
\pgfsetstrokecolor{currentstroke}%
\pgfsetdash{}{0pt}%
\pgfsys@defobject{currentmarker}{\pgfqpoint{-0.048611in}{0.000000in}}{\pgfqpoint{0.000000in}{0.000000in}}{%
\pgfpathmoveto{\pgfqpoint{0.000000in}{0.000000in}}%
\pgfpathlineto{\pgfqpoint{-0.048611in}{0.000000in}}%
\pgfusepath{stroke,fill}%
}%
\begin{pgfscope}%
\pgfsys@transformshift{1.012500in}{3.230954in}%
\pgfsys@useobject{currentmarker}{}%
\end{pgfscope}%
\end{pgfscope}%
\begin{pgfscope}%
\pgftext[x=0.706944in,y=3.178193in,left,base]{\sffamily\fontsize{10.000000}{12.000000}\selectfont \(\displaystyle 110\)}%
\end{pgfscope}%
\begin{pgfscope}%
\pgfsetbuttcap%
\pgfsetroundjoin%
\definecolor{currentfill}{rgb}{0.000000,0.000000,0.000000}%
\pgfsetfillcolor{currentfill}%
\pgfsetlinewidth{0.803000pt}%
\definecolor{currentstroke}{rgb}{0.000000,0.000000,0.000000}%
\pgfsetstrokecolor{currentstroke}%
\pgfsetdash{}{0pt}%
\pgfsys@defobject{currentmarker}{\pgfqpoint{-0.048611in}{0.000000in}}{\pgfqpoint{0.000000in}{0.000000in}}{%
\pgfpathmoveto{\pgfqpoint{0.000000in}{0.000000in}}%
\pgfpathlineto{\pgfqpoint{-0.048611in}{0.000000in}}%
\pgfusepath{stroke,fill}%
}%
\begin{pgfscope}%
\pgfsys@transformshift{1.012500in}{3.437740in}%
\pgfsys@useobject{currentmarker}{}%
\end{pgfscope}%
\end{pgfscope}%
\begin{pgfscope}%
\pgftext[x=0.706944in,y=3.384979in,left,base]{\sffamily\fontsize{10.000000}{12.000000}\selectfont \(\displaystyle 120\)}%
\end{pgfscope}%
\begin{pgfscope}%
\pgfsetbuttcap%
\pgfsetroundjoin%
\definecolor{currentfill}{rgb}{0.000000,0.000000,0.000000}%
\pgfsetfillcolor{currentfill}%
\pgfsetlinewidth{0.803000pt}%
\definecolor{currentstroke}{rgb}{0.000000,0.000000,0.000000}%
\pgfsetstrokecolor{currentstroke}%
\pgfsetdash{}{0pt}%
\pgfsys@defobject{currentmarker}{\pgfqpoint{-0.048611in}{0.000000in}}{\pgfqpoint{0.000000in}{0.000000in}}{%
\pgfpathmoveto{\pgfqpoint{0.000000in}{0.000000in}}%
\pgfpathlineto{\pgfqpoint{-0.048611in}{0.000000in}}%
\pgfusepath{stroke,fill}%
}%
\begin{pgfscope}%
\pgfsys@transformshift{1.012500in}{3.644526in}%
\pgfsys@useobject{currentmarker}{}%
\end{pgfscope}%
\end{pgfscope}%
\begin{pgfscope}%
\pgftext[x=0.706944in,y=3.591764in,left,base]{\sffamily\fontsize{10.000000}{12.000000}\selectfont \(\displaystyle 130\)}%
\end{pgfscope}%
\begin{pgfscope}%
\pgfsetbuttcap%
\pgfsetroundjoin%
\definecolor{currentfill}{rgb}{0.000000,0.000000,0.000000}%
\pgfsetfillcolor{currentfill}%
\pgfsetlinewidth{0.803000pt}%
\definecolor{currentstroke}{rgb}{0.000000,0.000000,0.000000}%
\pgfsetstrokecolor{currentstroke}%
\pgfsetdash{}{0pt}%
\pgfsys@defobject{currentmarker}{\pgfqpoint{-0.048611in}{0.000000in}}{\pgfqpoint{0.000000in}{0.000000in}}{%
\pgfpathmoveto{\pgfqpoint{0.000000in}{0.000000in}}%
\pgfpathlineto{\pgfqpoint{-0.048611in}{0.000000in}}%
\pgfusepath{stroke,fill}%
}%
\begin{pgfscope}%
\pgfsys@transformshift{1.012500in}{3.851311in}%
\pgfsys@useobject{currentmarker}{}%
\end{pgfscope}%
\end{pgfscope}%
\begin{pgfscope}%
\pgftext[x=0.706944in,y=3.798550in,left,base]{\sffamily\fontsize{10.000000}{12.000000}\selectfont \(\displaystyle 140\)}%
\end{pgfscope}%
\begin{pgfscope}%
\pgfsetbuttcap%
\pgfsetroundjoin%
\definecolor{currentfill}{rgb}{0.000000,0.000000,0.000000}%
\pgfsetfillcolor{currentfill}%
\pgfsetlinewidth{0.803000pt}%
\definecolor{currentstroke}{rgb}{0.000000,0.000000,0.000000}%
\pgfsetstrokecolor{currentstroke}%
\pgfsetdash{}{0pt}%
\pgfsys@defobject{currentmarker}{\pgfqpoint{-0.048611in}{0.000000in}}{\pgfqpoint{0.000000in}{0.000000in}}{%
\pgfpathmoveto{\pgfqpoint{0.000000in}{0.000000in}}%
\pgfpathlineto{\pgfqpoint{-0.048611in}{0.000000in}}%
\pgfusepath{stroke,fill}%
}%
\begin{pgfscope}%
\pgfsys@transformshift{1.012500in}{4.058097in}%
\pgfsys@useobject{currentmarker}{}%
\end{pgfscope}%
\end{pgfscope}%
\begin{pgfscope}%
\pgftext[x=0.706944in,y=4.005336in,left,base]{\sffamily\fontsize{10.000000}{12.000000}\selectfont \(\displaystyle 150\)}%
\end{pgfscope}%
\begin{pgfscope}%
\pgfsetbuttcap%
\pgfsetroundjoin%
\definecolor{currentfill}{rgb}{0.000000,0.000000,0.000000}%
\pgfsetfillcolor{currentfill}%
\pgfsetlinewidth{0.803000pt}%
\definecolor{currentstroke}{rgb}{0.000000,0.000000,0.000000}%
\pgfsetstrokecolor{currentstroke}%
\pgfsetdash{}{0pt}%
\pgfsys@defobject{currentmarker}{\pgfqpoint{-0.048611in}{0.000000in}}{\pgfqpoint{0.000000in}{0.000000in}}{%
\pgfpathmoveto{\pgfqpoint{0.000000in}{0.000000in}}%
\pgfpathlineto{\pgfqpoint{-0.048611in}{0.000000in}}%
\pgfusepath{stroke,fill}%
}%
\begin{pgfscope}%
\pgfsys@transformshift{1.012500in}{4.264883in}%
\pgfsys@useobject{currentmarker}{}%
\end{pgfscope}%
\end{pgfscope}%
\begin{pgfscope}%
\pgftext[x=0.706944in,y=4.212121in,left,base]{\sffamily\fontsize{10.000000}{12.000000}\selectfont \(\displaystyle 160\)}%
\end{pgfscope}%
\begin{pgfscope}%
\pgfsetbuttcap%
\pgfsetroundjoin%
\definecolor{currentfill}{rgb}{0.000000,0.000000,0.000000}%
\pgfsetfillcolor{currentfill}%
\pgfsetlinewidth{0.803000pt}%
\definecolor{currentstroke}{rgb}{0.000000,0.000000,0.000000}%
\pgfsetstrokecolor{currentstroke}%
\pgfsetdash{}{0pt}%
\pgfsys@defobject{currentmarker}{\pgfqpoint{-0.048611in}{0.000000in}}{\pgfqpoint{0.000000in}{0.000000in}}{%
\pgfpathmoveto{\pgfqpoint{0.000000in}{0.000000in}}%
\pgfpathlineto{\pgfqpoint{-0.048611in}{0.000000in}}%
\pgfusepath{stroke,fill}%
}%
\begin{pgfscope}%
\pgfsys@transformshift{1.012500in}{4.471668in}%
\pgfsys@useobject{currentmarker}{}%
\end{pgfscope}%
\end{pgfscope}%
\begin{pgfscope}%
\pgftext[x=0.706944in,y=4.418907in,left,base]{\sffamily\fontsize{10.000000}{12.000000}\selectfont \(\displaystyle 170\)}%
\end{pgfscope}%
\begin{pgfscope}%
\pgfsetbuttcap%
\pgfsetroundjoin%
\definecolor{currentfill}{rgb}{0.000000,0.000000,0.000000}%
\pgfsetfillcolor{currentfill}%
\pgfsetlinewidth{0.803000pt}%
\definecolor{currentstroke}{rgb}{0.000000,0.000000,0.000000}%
\pgfsetstrokecolor{currentstroke}%
\pgfsetdash{}{0pt}%
\pgfsys@defobject{currentmarker}{\pgfqpoint{-0.048611in}{0.000000in}}{\pgfqpoint{0.000000in}{0.000000in}}{%
\pgfpathmoveto{\pgfqpoint{0.000000in}{0.000000in}}%
\pgfpathlineto{\pgfqpoint{-0.048611in}{0.000000in}}%
\pgfusepath{stroke,fill}%
}%
\begin{pgfscope}%
\pgfsys@transformshift{1.012500in}{4.678454in}%
\pgfsys@useobject{currentmarker}{}%
\end{pgfscope}%
\end{pgfscope}%
\begin{pgfscope}%
\pgftext[x=0.706944in,y=4.625693in,left,base]{\sffamily\fontsize{10.000000}{12.000000}\selectfont \(\displaystyle 180\)}%
\end{pgfscope}%
\begin{pgfscope}%
\pgfsetbuttcap%
\pgfsetroundjoin%
\definecolor{currentfill}{rgb}{0.000000,0.000000,0.000000}%
\pgfsetfillcolor{currentfill}%
\pgfsetlinewidth{0.803000pt}%
\definecolor{currentstroke}{rgb}{0.000000,0.000000,0.000000}%
\pgfsetstrokecolor{currentstroke}%
\pgfsetdash{}{0pt}%
\pgfsys@defobject{currentmarker}{\pgfqpoint{-0.048611in}{0.000000in}}{\pgfqpoint{0.000000in}{0.000000in}}{%
\pgfpathmoveto{\pgfqpoint{0.000000in}{0.000000in}}%
\pgfpathlineto{\pgfqpoint{-0.048611in}{0.000000in}}%
\pgfusepath{stroke,fill}%
}%
\begin{pgfscope}%
\pgfsys@transformshift{1.012500in}{4.885240in}%
\pgfsys@useobject{currentmarker}{}%
\end{pgfscope}%
\end{pgfscope}%
\begin{pgfscope}%
\pgftext[x=0.706944in,y=4.832478in,left,base]{\sffamily\fontsize{10.000000}{12.000000}\selectfont \(\displaystyle 190\)}%
\end{pgfscope}%
\begin{pgfscope}%
\pgfsetbuttcap%
\pgfsetroundjoin%
\definecolor{currentfill}{rgb}{0.000000,0.000000,0.000000}%
\pgfsetfillcolor{currentfill}%
\pgfsetlinewidth{0.803000pt}%
\definecolor{currentstroke}{rgb}{0.000000,0.000000,0.000000}%
\pgfsetstrokecolor{currentstroke}%
\pgfsetdash{}{0pt}%
\pgfsys@defobject{currentmarker}{\pgfqpoint{-0.048611in}{0.000000in}}{\pgfqpoint{0.000000in}{0.000000in}}{%
\pgfpathmoveto{\pgfqpoint{0.000000in}{0.000000in}}%
\pgfpathlineto{\pgfqpoint{-0.048611in}{0.000000in}}%
\pgfusepath{stroke,fill}%
}%
\begin{pgfscope}%
\pgfsys@transformshift{1.012500in}{5.092026in}%
\pgfsys@useobject{currentmarker}{}%
\end{pgfscope}%
\end{pgfscope}%
\begin{pgfscope}%
\pgftext[x=0.706944in,y=5.039264in,left,base]{\sffamily\fontsize{10.000000}{12.000000}\selectfont \(\displaystyle 200\)}%
\end{pgfscope}%
\begin{pgfscope}%
\pgfsetbuttcap%
\pgfsetroundjoin%
\definecolor{currentfill}{rgb}{0.000000,0.000000,0.000000}%
\pgfsetfillcolor{currentfill}%
\pgfsetlinewidth{0.803000pt}%
\definecolor{currentstroke}{rgb}{0.000000,0.000000,0.000000}%
\pgfsetstrokecolor{currentstroke}%
\pgfsetdash{}{0pt}%
\pgfsys@defobject{currentmarker}{\pgfqpoint{-0.048611in}{0.000000in}}{\pgfqpoint{0.000000in}{0.000000in}}{%
\pgfpathmoveto{\pgfqpoint{0.000000in}{0.000000in}}%
\pgfpathlineto{\pgfqpoint{-0.048611in}{0.000000in}}%
\pgfusepath{stroke,fill}%
}%
\begin{pgfscope}%
\pgfsys@transformshift{1.012500in}{5.298811in}%
\pgfsys@useobject{currentmarker}{}%
\end{pgfscope}%
\end{pgfscope}%
\begin{pgfscope}%
\pgftext[x=0.706944in,y=5.246050in,left,base]{\sffamily\fontsize{10.000000}{12.000000}\selectfont \(\displaystyle 210\)}%
\end{pgfscope}%
\begin{pgfscope}%
\pgfsetbuttcap%
\pgfsetroundjoin%
\definecolor{currentfill}{rgb}{0.000000,0.000000,0.000000}%
\pgfsetfillcolor{currentfill}%
\pgfsetlinewidth{0.803000pt}%
\definecolor{currentstroke}{rgb}{0.000000,0.000000,0.000000}%
\pgfsetstrokecolor{currentstroke}%
\pgfsetdash{}{0pt}%
\pgfsys@defobject{currentmarker}{\pgfqpoint{-0.048611in}{0.000000in}}{\pgfqpoint{0.000000in}{0.000000in}}{%
\pgfpathmoveto{\pgfqpoint{0.000000in}{0.000000in}}%
\pgfpathlineto{\pgfqpoint{-0.048611in}{0.000000in}}%
\pgfusepath{stroke,fill}%
}%
\begin{pgfscope}%
\pgfsys@transformshift{1.012500in}{5.505597in}%
\pgfsys@useobject{currentmarker}{}%
\end{pgfscope}%
\end{pgfscope}%
\begin{pgfscope}%
\pgftext[x=0.706944in,y=5.452835in,left,base]{\sffamily\fontsize{10.000000}{12.000000}\selectfont \(\displaystyle 220\)}%
\end{pgfscope}%
\begin{pgfscope}%
\pgfsetbuttcap%
\pgfsetroundjoin%
\definecolor{currentfill}{rgb}{0.000000,0.000000,0.000000}%
\pgfsetfillcolor{currentfill}%
\pgfsetlinewidth{0.803000pt}%
\definecolor{currentstroke}{rgb}{0.000000,0.000000,0.000000}%
\pgfsetstrokecolor{currentstroke}%
\pgfsetdash{}{0pt}%
\pgfsys@defobject{currentmarker}{\pgfqpoint{-0.048611in}{0.000000in}}{\pgfqpoint{0.000000in}{0.000000in}}{%
\pgfpathmoveto{\pgfqpoint{0.000000in}{0.000000in}}%
\pgfpathlineto{\pgfqpoint{-0.048611in}{0.000000in}}%
\pgfusepath{stroke,fill}%
}%
\begin{pgfscope}%
\pgfsys@transformshift{1.012500in}{5.712383in}%
\pgfsys@useobject{currentmarker}{}%
\end{pgfscope}%
\end{pgfscope}%
\begin{pgfscope}%
\pgftext[x=0.706944in,y=5.659621in,left,base]{\sffamily\fontsize{10.000000}{12.000000}\selectfont \(\displaystyle 230\)}%
\end{pgfscope}%
\begin{pgfscope}%
\pgftext[x=0.651388in,y=3.267000in,,bottom,rotate=90.000000]{\sffamily\fontsize{12.000000}{14.400000}\selectfont seconds}%
\end{pgfscope}%
\begin{pgfscope}%
\pgfpathrectangle{\pgfqpoint{1.012500in}{0.726000in}}{\pgfqpoint{6.277500in}{5.082000in}}%
\pgfusepath{clip}%
\pgfsetrectcap%
\pgfsetroundjoin%
\pgfsetlinewidth{1.505625pt}%
\definecolor{currentstroke}{rgb}{0.172549,0.243137,0.313725}%
\pgfsetstrokecolor{currentstroke}%
\pgfsetdash{}{0pt}%
\pgfpathmoveto{\pgfqpoint{1.312142in}{0.970076in}}%
\pgfpathlineto{\pgfqpoint{1.611784in}{1.009013in}}%
\pgfpathlineto{\pgfqpoint{1.911426in}{1.080293in}}%
\pgfpathlineto{\pgfqpoint{2.211068in}{1.184390in}}%
\pgfpathlineto{\pgfqpoint{2.510710in}{1.283974in}}%
\pgfpathlineto{\pgfqpoint{2.810352in}{1.427928in}}%
\pgfpathlineto{\pgfqpoint{3.109994in}{1.610385in}}%
\pgfpathlineto{\pgfqpoint{3.409636in}{1.797896in}}%
\pgfpathlineto{\pgfqpoint{3.709278in}{2.054201in}}%
\pgfpathlineto{\pgfqpoint{4.008920in}{2.319644in}}%
\pgfpathlineto{\pgfqpoint{4.308562in}{2.294087in}}%
\pgfpathlineto{\pgfqpoint{4.608204in}{2.474652in}}%
\pgfpathlineto{\pgfqpoint{4.907846in}{2.582923in}}%
\pgfpathlineto{\pgfqpoint{5.207488in}{2.934833in}}%
\pgfpathlineto{\pgfqpoint{5.507130in}{3.347313in}}%
\pgfpathlineto{\pgfqpoint{5.806772in}{3.850360in}}%
\pgfpathlineto{\pgfqpoint{6.106414in}{4.259382in}}%
\pgfpathlineto{\pgfqpoint{6.406056in}{4.379649in}}%
\pgfpathlineto{\pgfqpoint{6.705698in}{4.993616in}}%
\pgfpathlineto{\pgfqpoint{7.005340in}{5.577000in}}%
\pgfusepath{stroke}%
\end{pgfscope}%
\begin{pgfscope}%
\pgfpathrectangle{\pgfqpoint{1.012500in}{0.726000in}}{\pgfqpoint{6.277500in}{5.082000in}}%
\pgfusepath{clip}%
\pgfsetrectcap%
\pgfsetroundjoin%
\pgfsetlinewidth{1.505625pt}%
\definecolor{currentstroke}{rgb}{0.086275,0.627451,0.521569}%
\pgfsetstrokecolor{currentstroke}%
\pgfsetdash{}{0pt}%
\pgfpathmoveto{\pgfqpoint{1.312142in}{0.957000in}}%
\pgfpathlineto{\pgfqpoint{1.611784in}{0.958645in}}%
\pgfpathlineto{\pgfqpoint{1.911426in}{0.961737in}}%
\pgfpathlineto{\pgfqpoint{2.211068in}{0.966354in}}%
\pgfpathlineto{\pgfqpoint{2.510710in}{0.970532in}}%
\pgfpathlineto{\pgfqpoint{2.810352in}{0.976822in}}%
\pgfpathlineto{\pgfqpoint{3.109994in}{0.984729in}}%
\pgfpathlineto{\pgfqpoint{3.409636in}{0.992454in}}%
\pgfpathlineto{\pgfqpoint{3.709278in}{1.004108in}}%
\pgfpathlineto{\pgfqpoint{4.008920in}{1.016118in}}%
\pgfpathlineto{\pgfqpoint{4.308562in}{1.012648in}}%
\pgfpathlineto{\pgfqpoint{4.608204in}{1.019145in}}%
\pgfpathlineto{\pgfqpoint{4.907846in}{1.024302in}}%
\pgfpathlineto{\pgfqpoint{5.207488in}{1.037882in}}%
\pgfpathlineto{\pgfqpoint{5.507130in}{1.055243in}}%
\pgfpathlineto{\pgfqpoint{5.806772in}{1.078164in}}%
\pgfpathlineto{\pgfqpoint{6.106414in}{1.096758in}}%
\pgfpathlineto{\pgfqpoint{6.406056in}{1.100500in}}%
\pgfpathlineto{\pgfqpoint{6.705698in}{1.127999in}}%
\pgfpathlineto{\pgfqpoint{7.005340in}{1.157340in}}%
\pgfusepath{stroke}%
\end{pgfscope}%
\begin{pgfscope}%
\pgfsetrectcap%
\pgfsetmiterjoin%
\pgfsetlinewidth{0.803000pt}%
\definecolor{currentstroke}{rgb}{0.000000,0.000000,0.000000}%
\pgfsetstrokecolor{currentstroke}%
\pgfsetdash{}{0pt}%
\pgfpathmoveto{\pgfqpoint{1.012500in}{0.726000in}}%
\pgfpathlineto{\pgfqpoint{1.012500in}{5.808000in}}%
\pgfusepath{stroke}%
\end{pgfscope}%
\begin{pgfscope}%
\pgfsetrectcap%
\pgfsetmiterjoin%
\pgfsetlinewidth{0.803000pt}%
\definecolor{currentstroke}{rgb}{0.000000,0.000000,0.000000}%
\pgfsetstrokecolor{currentstroke}%
\pgfsetdash{}{0pt}%
\pgfpathmoveto{\pgfqpoint{7.290000in}{0.726000in}}%
\pgfpathlineto{\pgfqpoint{7.290000in}{5.808000in}}%
\pgfusepath{stroke}%
\end{pgfscope}%
\begin{pgfscope}%
\pgfsetrectcap%
\pgfsetmiterjoin%
\pgfsetlinewidth{0.803000pt}%
\definecolor{currentstroke}{rgb}{0.000000,0.000000,0.000000}%
\pgfsetstrokecolor{currentstroke}%
\pgfsetdash{}{0pt}%
\pgfpathmoveto{\pgfqpoint{1.012500in}{0.726000in}}%
\pgfpathlineto{\pgfqpoint{7.290000in}{0.726000in}}%
\pgfusepath{stroke}%
\end{pgfscope}%
\begin{pgfscope}%
\pgfsetrectcap%
\pgfsetmiterjoin%
\pgfsetlinewidth{0.803000pt}%
\definecolor{currentstroke}{rgb}{0.000000,0.000000,0.000000}%
\pgfsetstrokecolor{currentstroke}%
\pgfsetdash{}{0pt}%
\pgfpathmoveto{\pgfqpoint{1.012500in}{5.808000in}}%
\pgfpathlineto{\pgfqpoint{7.290000in}{5.808000in}}%
\pgfusepath{stroke}%
\end{pgfscope}%
\begin{pgfscope}%
\pgfsetbuttcap%
\pgfsetmiterjoin%
\definecolor{currentfill}{rgb}{1.000000,1.000000,1.000000}%
\pgfsetfillcolor{currentfill}%
\pgfsetfillopacity{0.800000}%
\pgfsetlinewidth{1.003750pt}%
\definecolor{currentstroke}{rgb}{0.800000,0.800000,0.800000}%
\pgfsetstrokecolor{currentstroke}%
\pgfsetstrokeopacity{0.800000}%
\pgfsetdash{}{0pt}%
\pgfpathmoveto{\pgfqpoint{1.109722in}{5.289174in}}%
\pgfpathlineto{\pgfqpoint{2.257970in}{5.289174in}}%
\pgfpathquadraticcurveto{\pgfqpoint{2.285748in}{5.289174in}}{\pgfqpoint{2.285748in}{5.316952in}}%
\pgfpathlineto{\pgfqpoint{2.285748in}{5.710778in}}%
\pgfpathquadraticcurveto{\pgfqpoint{2.285748in}{5.738556in}}{\pgfqpoint{2.257970in}{5.738556in}}%
\pgfpathlineto{\pgfqpoint{1.109722in}{5.738556in}}%
\pgfpathquadraticcurveto{\pgfqpoint{1.081944in}{5.738556in}}{\pgfqpoint{1.081944in}{5.710778in}}%
\pgfpathlineto{\pgfqpoint{1.081944in}{5.316952in}}%
\pgfpathquadraticcurveto{\pgfqpoint{1.081944in}{5.289174in}}{\pgfqpoint{1.109722in}{5.289174in}}%
\pgfpathclose%
\pgfusepath{stroke,fill}%
\end{pgfscope}%
\begin{pgfscope}%
\pgfsetrectcap%
\pgfsetroundjoin%
\pgfsetlinewidth{1.505625pt}%
\definecolor{currentstroke}{rgb}{0.172549,0.243137,0.313725}%
\pgfsetstrokecolor{currentstroke}%
\pgfsetdash{}{0pt}%
\pgfpathmoveto{\pgfqpoint{1.137500in}{5.626088in}}%
\pgfpathlineto{\pgfqpoint{1.415278in}{5.626088in}}%
\pgfusepath{stroke}%
\end{pgfscope}%
\begin{pgfscope}%
\pgftext[x=1.526389in,y=5.577477in,left,base]{\sffamily\fontsize{10.000000}{12.000000}\selectfont \textsc{Linclosure}}%
\end{pgfscope}%
\begin{pgfscope}%
\pgfsetrectcap%
\pgfsetroundjoin%
\pgfsetlinewidth{1.505625pt}%
\definecolor{currentstroke}{rgb}{0.086275,0.627451,0.521569}%
\pgfsetstrokecolor{currentstroke}%
\pgfsetdash{}{0pt}%
\pgfpathmoveto{\pgfqpoint{1.137500in}{5.422231in}}%
\pgfpathlineto{\pgfqpoint{1.415278in}{5.422231in}}%
\pgfusepath{stroke}%
\end{pgfscope}%
\begin{pgfscope}%
\pgftext[x=1.526389in,y=5.373620in,left,base]{\sffamily\fontsize{10.000000}{12.000000}\selectfont \textsc{Closure}}%
\end{pgfscope}%
\end{pgfpicture}%
\makeatother%
\endgroup%
}
}
\subfloat[Average time (in $s$), $|\B| = 1000$]{
	\scalebox{0.5}{%% Creator: Matplotlib, PGF backend
%%
%% To include the figure in your LaTeX document, write
%%   \input{<filename>.pgf}
%%
%% Make sure the required packages are loaded in your preamble
%%   \usepackage{pgf}
%%
%% Figures using additional raster images can only be included by \input if
%% they are in the same directory as the main LaTeX file. For loading figures
%% from other directories you can use the `import` package
%%   \usepackage{import}
%% and then include the figures with
%%   \import{<path to file>}{<filename>.pgf}
%%
%% Matplotlib used the following preamble
%%   \usepackage{fontspec}
%%   \setmainfont{DejaVu Serif}
%%   \setsansfont{DejaVu Sans}
%%   \setmonofont{DejaVu Sans Mono}
%%
\begingroup%
\makeatletter%
\begin{pgfpicture}%
\pgfpathrectangle{\pgfpointorigin}{\pgfqpoint{7.140000in}{5.340000in}}%
\pgfusepath{use as bounding box, clip}%
\begin{pgfscope}%
\pgfsetbuttcap%
\pgfsetmiterjoin%
\definecolor{currentfill}{rgb}{1.000000,1.000000,1.000000}%
\pgfsetfillcolor{currentfill}%
\pgfsetlinewidth{0.000000pt}%
\definecolor{currentstroke}{rgb}{1.000000,1.000000,1.000000}%
\pgfsetstrokecolor{currentstroke}%
\pgfsetdash{}{0pt}%
\pgfpathmoveto{\pgfqpoint{0.000000in}{0.000000in}}%
\pgfpathlineto{\pgfqpoint{7.140000in}{0.000000in}}%
\pgfpathlineto{\pgfqpoint{7.140000in}{5.340000in}}%
\pgfpathlineto{\pgfqpoint{0.000000in}{5.340000in}}%
\pgfpathclose%
\pgfusepath{fill}%
\end{pgfscope}%
\begin{pgfscope}%
\pgfsetbuttcap%
\pgfsetmiterjoin%
\definecolor{currentfill}{rgb}{1.000000,1.000000,1.000000}%
\pgfsetfillcolor{currentfill}%
\pgfsetlinewidth{0.000000pt}%
\definecolor{currentstroke}{rgb}{0.000000,0.000000,0.000000}%
\pgfsetstrokecolor{currentstroke}%
\pgfsetstrokeopacity{0.000000}%
\pgfsetdash{}{0pt}%
\pgfpathmoveto{\pgfqpoint{0.892500in}{0.587400in}}%
\pgfpathlineto{\pgfqpoint{6.426000in}{0.587400in}}%
\pgfpathlineto{\pgfqpoint{6.426000in}{4.699200in}}%
\pgfpathlineto{\pgfqpoint{0.892500in}{4.699200in}}%
\pgfpathclose%
\pgfusepath{fill}%
\end{pgfscope}%
\begin{pgfscope}%
\pgfsetbuttcap%
\pgfsetroundjoin%
\definecolor{currentfill}{rgb}{0.000000,0.000000,0.000000}%
\pgfsetfillcolor{currentfill}%
\pgfsetlinewidth{0.803000pt}%
\definecolor{currentstroke}{rgb}{0.000000,0.000000,0.000000}%
\pgfsetstrokecolor{currentstroke}%
\pgfsetdash{}{0pt}%
\pgfsys@defobject{currentmarker}{\pgfqpoint{0.000000in}{-0.048611in}}{\pgfqpoint{0.000000in}{0.000000in}}{%
\pgfpathmoveto{\pgfqpoint{0.000000in}{0.000000in}}%
\pgfpathlineto{\pgfqpoint{0.000000in}{-0.048611in}}%
\pgfusepath{stroke,fill}%
}%
\begin{pgfscope}%
\pgfsys@transformshift{1.041360in}{0.587400in}%
\pgfsys@useobject{currentmarker}{}%
\end{pgfscope}%
\end{pgfscope}%
\begin{pgfscope}%
\pgftext[x=1.041360in,y=0.490178in,,top]{\sffamily\fontsize{10.000000}{12.000000}\selectfont \(\displaystyle 0\)}%
\end{pgfscope}%
\begin{pgfscope}%
\pgfsetbuttcap%
\pgfsetroundjoin%
\definecolor{currentfill}{rgb}{0.000000,0.000000,0.000000}%
\pgfsetfillcolor{currentfill}%
\pgfsetlinewidth{0.803000pt}%
\definecolor{currentstroke}{rgb}{0.000000,0.000000,0.000000}%
\pgfsetstrokecolor{currentstroke}%
\pgfsetdash{}{0pt}%
\pgfsys@defobject{currentmarker}{\pgfqpoint{0.000000in}{-0.048611in}}{\pgfqpoint{0.000000in}{0.000000in}}{%
\pgfpathmoveto{\pgfqpoint{0.000000in}{0.000000in}}%
\pgfpathlineto{\pgfqpoint{0.000000in}{-0.048611in}}%
\pgfusepath{stroke,fill}%
}%
\begin{pgfscope}%
\pgfsys@transformshift{1.554672in}{0.587400in}%
\pgfsys@useobject{currentmarker}{}%
\end{pgfscope}%
\end{pgfscope}%
\begin{pgfscope}%
\pgftext[x=1.554672in,y=0.490178in,,top]{\sffamily\fontsize{10.000000}{12.000000}\selectfont \(\displaystyle 10000\)}%
\end{pgfscope}%
\begin{pgfscope}%
\pgfsetbuttcap%
\pgfsetroundjoin%
\definecolor{currentfill}{rgb}{0.000000,0.000000,0.000000}%
\pgfsetfillcolor{currentfill}%
\pgfsetlinewidth{0.803000pt}%
\definecolor{currentstroke}{rgb}{0.000000,0.000000,0.000000}%
\pgfsetstrokecolor{currentstroke}%
\pgfsetdash{}{0pt}%
\pgfsys@defobject{currentmarker}{\pgfqpoint{0.000000in}{-0.048611in}}{\pgfqpoint{0.000000in}{0.000000in}}{%
\pgfpathmoveto{\pgfqpoint{0.000000in}{0.000000in}}%
\pgfpathlineto{\pgfqpoint{0.000000in}{-0.048611in}}%
\pgfusepath{stroke,fill}%
}%
\begin{pgfscope}%
\pgfsys@transformshift{2.067984in}{0.587400in}%
\pgfsys@useobject{currentmarker}{}%
\end{pgfscope}%
\end{pgfscope}%
\begin{pgfscope}%
\pgftext[x=2.067984in,y=0.490178in,,top]{\sffamily\fontsize{10.000000}{12.000000}\selectfont \(\displaystyle 20000\)}%
\end{pgfscope}%
\begin{pgfscope}%
\pgfsetbuttcap%
\pgfsetroundjoin%
\definecolor{currentfill}{rgb}{0.000000,0.000000,0.000000}%
\pgfsetfillcolor{currentfill}%
\pgfsetlinewidth{0.803000pt}%
\definecolor{currentstroke}{rgb}{0.000000,0.000000,0.000000}%
\pgfsetstrokecolor{currentstroke}%
\pgfsetdash{}{0pt}%
\pgfsys@defobject{currentmarker}{\pgfqpoint{0.000000in}{-0.048611in}}{\pgfqpoint{0.000000in}{0.000000in}}{%
\pgfpathmoveto{\pgfqpoint{0.000000in}{0.000000in}}%
\pgfpathlineto{\pgfqpoint{0.000000in}{-0.048611in}}%
\pgfusepath{stroke,fill}%
}%
\begin{pgfscope}%
\pgfsys@transformshift{2.581295in}{0.587400in}%
\pgfsys@useobject{currentmarker}{}%
\end{pgfscope}%
\end{pgfscope}%
\begin{pgfscope}%
\pgftext[x=2.581295in,y=0.490178in,,top]{\sffamily\fontsize{10.000000}{12.000000}\selectfont \(\displaystyle 30000\)}%
\end{pgfscope}%
\begin{pgfscope}%
\pgfsetbuttcap%
\pgfsetroundjoin%
\definecolor{currentfill}{rgb}{0.000000,0.000000,0.000000}%
\pgfsetfillcolor{currentfill}%
\pgfsetlinewidth{0.803000pt}%
\definecolor{currentstroke}{rgb}{0.000000,0.000000,0.000000}%
\pgfsetstrokecolor{currentstroke}%
\pgfsetdash{}{0pt}%
\pgfsys@defobject{currentmarker}{\pgfqpoint{0.000000in}{-0.048611in}}{\pgfqpoint{0.000000in}{0.000000in}}{%
\pgfpathmoveto{\pgfqpoint{0.000000in}{0.000000in}}%
\pgfpathlineto{\pgfqpoint{0.000000in}{-0.048611in}}%
\pgfusepath{stroke,fill}%
}%
\begin{pgfscope}%
\pgfsys@transformshift{3.094607in}{0.587400in}%
\pgfsys@useobject{currentmarker}{}%
\end{pgfscope}%
\end{pgfscope}%
\begin{pgfscope}%
\pgftext[x=3.094607in,y=0.490178in,,top]{\sffamily\fontsize{10.000000}{12.000000}\selectfont \(\displaystyle 40000\)}%
\end{pgfscope}%
\begin{pgfscope}%
\pgfsetbuttcap%
\pgfsetroundjoin%
\definecolor{currentfill}{rgb}{0.000000,0.000000,0.000000}%
\pgfsetfillcolor{currentfill}%
\pgfsetlinewidth{0.803000pt}%
\definecolor{currentstroke}{rgb}{0.000000,0.000000,0.000000}%
\pgfsetstrokecolor{currentstroke}%
\pgfsetdash{}{0pt}%
\pgfsys@defobject{currentmarker}{\pgfqpoint{0.000000in}{-0.048611in}}{\pgfqpoint{0.000000in}{0.000000in}}{%
\pgfpathmoveto{\pgfqpoint{0.000000in}{0.000000in}}%
\pgfpathlineto{\pgfqpoint{0.000000in}{-0.048611in}}%
\pgfusepath{stroke,fill}%
}%
\begin{pgfscope}%
\pgfsys@transformshift{3.607919in}{0.587400in}%
\pgfsys@useobject{currentmarker}{}%
\end{pgfscope}%
\end{pgfscope}%
\begin{pgfscope}%
\pgftext[x=3.607919in,y=0.490178in,,top]{\sffamily\fontsize{10.000000}{12.000000}\selectfont \(\displaystyle 50000\)}%
\end{pgfscope}%
\begin{pgfscope}%
\pgfsetbuttcap%
\pgfsetroundjoin%
\definecolor{currentfill}{rgb}{0.000000,0.000000,0.000000}%
\pgfsetfillcolor{currentfill}%
\pgfsetlinewidth{0.803000pt}%
\definecolor{currentstroke}{rgb}{0.000000,0.000000,0.000000}%
\pgfsetstrokecolor{currentstroke}%
\pgfsetdash{}{0pt}%
\pgfsys@defobject{currentmarker}{\pgfqpoint{0.000000in}{-0.048611in}}{\pgfqpoint{0.000000in}{0.000000in}}{%
\pgfpathmoveto{\pgfqpoint{0.000000in}{0.000000in}}%
\pgfpathlineto{\pgfqpoint{0.000000in}{-0.048611in}}%
\pgfusepath{stroke,fill}%
}%
\begin{pgfscope}%
\pgfsys@transformshift{4.121231in}{0.587400in}%
\pgfsys@useobject{currentmarker}{}%
\end{pgfscope}%
\end{pgfscope}%
\begin{pgfscope}%
\pgftext[x=4.121231in,y=0.490178in,,top]{\sffamily\fontsize{10.000000}{12.000000}\selectfont \(\displaystyle 60000\)}%
\end{pgfscope}%
\begin{pgfscope}%
\pgfsetbuttcap%
\pgfsetroundjoin%
\definecolor{currentfill}{rgb}{0.000000,0.000000,0.000000}%
\pgfsetfillcolor{currentfill}%
\pgfsetlinewidth{0.803000pt}%
\definecolor{currentstroke}{rgb}{0.000000,0.000000,0.000000}%
\pgfsetstrokecolor{currentstroke}%
\pgfsetdash{}{0pt}%
\pgfsys@defobject{currentmarker}{\pgfqpoint{0.000000in}{-0.048611in}}{\pgfqpoint{0.000000in}{0.000000in}}{%
\pgfpathmoveto{\pgfqpoint{0.000000in}{0.000000in}}%
\pgfpathlineto{\pgfqpoint{0.000000in}{-0.048611in}}%
\pgfusepath{stroke,fill}%
}%
\begin{pgfscope}%
\pgfsys@transformshift{4.634542in}{0.587400in}%
\pgfsys@useobject{currentmarker}{}%
\end{pgfscope}%
\end{pgfscope}%
\begin{pgfscope}%
\pgftext[x=4.634542in,y=0.490178in,,top]{\sffamily\fontsize{10.000000}{12.000000}\selectfont \(\displaystyle 70000\)}%
\end{pgfscope}%
\begin{pgfscope}%
\pgfsetbuttcap%
\pgfsetroundjoin%
\definecolor{currentfill}{rgb}{0.000000,0.000000,0.000000}%
\pgfsetfillcolor{currentfill}%
\pgfsetlinewidth{0.803000pt}%
\definecolor{currentstroke}{rgb}{0.000000,0.000000,0.000000}%
\pgfsetstrokecolor{currentstroke}%
\pgfsetdash{}{0pt}%
\pgfsys@defobject{currentmarker}{\pgfqpoint{0.000000in}{-0.048611in}}{\pgfqpoint{0.000000in}{0.000000in}}{%
\pgfpathmoveto{\pgfqpoint{0.000000in}{0.000000in}}%
\pgfpathlineto{\pgfqpoint{0.000000in}{-0.048611in}}%
\pgfusepath{stroke,fill}%
}%
\begin{pgfscope}%
\pgfsys@transformshift{5.147854in}{0.587400in}%
\pgfsys@useobject{currentmarker}{}%
\end{pgfscope}%
\end{pgfscope}%
\begin{pgfscope}%
\pgftext[x=5.147854in,y=0.490178in,,top]{\sffamily\fontsize{10.000000}{12.000000}\selectfont \(\displaystyle 80000\)}%
\end{pgfscope}%
\begin{pgfscope}%
\pgfsetbuttcap%
\pgfsetroundjoin%
\definecolor{currentfill}{rgb}{0.000000,0.000000,0.000000}%
\pgfsetfillcolor{currentfill}%
\pgfsetlinewidth{0.803000pt}%
\definecolor{currentstroke}{rgb}{0.000000,0.000000,0.000000}%
\pgfsetstrokecolor{currentstroke}%
\pgfsetdash{}{0pt}%
\pgfsys@defobject{currentmarker}{\pgfqpoint{0.000000in}{-0.048611in}}{\pgfqpoint{0.000000in}{0.000000in}}{%
\pgfpathmoveto{\pgfqpoint{0.000000in}{0.000000in}}%
\pgfpathlineto{\pgfqpoint{0.000000in}{-0.048611in}}%
\pgfusepath{stroke,fill}%
}%
\begin{pgfscope}%
\pgfsys@transformshift{5.661166in}{0.587400in}%
\pgfsys@useobject{currentmarker}{}%
\end{pgfscope}%
\end{pgfscope}%
\begin{pgfscope}%
\pgftext[x=5.661166in,y=0.490178in,,top]{\sffamily\fontsize{10.000000}{12.000000}\selectfont \(\displaystyle 90000\)}%
\end{pgfscope}%
\begin{pgfscope}%
\pgfsetbuttcap%
\pgfsetroundjoin%
\definecolor{currentfill}{rgb}{0.000000,0.000000,0.000000}%
\pgfsetfillcolor{currentfill}%
\pgfsetlinewidth{0.803000pt}%
\definecolor{currentstroke}{rgb}{0.000000,0.000000,0.000000}%
\pgfsetstrokecolor{currentstroke}%
\pgfsetdash{}{0pt}%
\pgfsys@defobject{currentmarker}{\pgfqpoint{0.000000in}{-0.048611in}}{\pgfqpoint{0.000000in}{0.000000in}}{%
\pgfpathmoveto{\pgfqpoint{0.000000in}{0.000000in}}%
\pgfpathlineto{\pgfqpoint{0.000000in}{-0.048611in}}%
\pgfusepath{stroke,fill}%
}%
\begin{pgfscope}%
\pgfsys@transformshift{6.174477in}{0.587400in}%
\pgfsys@useobject{currentmarker}{}%
\end{pgfscope}%
\end{pgfscope}%
\begin{pgfscope}%
\pgftext[x=6.174477in,y=0.490178in,,top]{\sffamily\fontsize{10.000000}{12.000000}\selectfont \(\displaystyle 100000\)}%
\end{pgfscope}%
\begin{pgfscope}%
\pgftext[x=3.659250in,y=0.300209in,,top]{\sffamily\fontsize{12.000000}{14.400000}\selectfont \(\displaystyle |\Sigma|\)}%
\end{pgfscope}%
\begin{pgfscope}%
\pgfsetbuttcap%
\pgfsetroundjoin%
\definecolor{currentfill}{rgb}{0.000000,0.000000,0.000000}%
\pgfsetfillcolor{currentfill}%
\pgfsetlinewidth{0.803000pt}%
\definecolor{currentstroke}{rgb}{0.000000,0.000000,0.000000}%
\pgfsetstrokecolor{currentstroke}%
\pgfsetdash{}{0pt}%
\pgfsys@defobject{currentmarker}{\pgfqpoint{-0.048611in}{0.000000in}}{\pgfqpoint{0.000000in}{0.000000in}}{%
\pgfpathmoveto{\pgfqpoint{0.000000in}{0.000000in}}%
\pgfpathlineto{\pgfqpoint{-0.048611in}{0.000000in}}%
\pgfusepath{stroke,fill}%
}%
\begin{pgfscope}%
\pgfsys@transformshift{0.892500in}{0.769417in}%
\pgfsys@useobject{currentmarker}{}%
\end{pgfscope}%
\end{pgfscope}%
\begin{pgfscope}%
\pgftext[x=0.725833in,y=0.716655in,left,base]{\sffamily\fontsize{10.000000}{12.000000}\selectfont \(\displaystyle 0\)}%
\end{pgfscope}%
\begin{pgfscope}%
\pgfsetbuttcap%
\pgfsetroundjoin%
\definecolor{currentfill}{rgb}{0.000000,0.000000,0.000000}%
\pgfsetfillcolor{currentfill}%
\pgfsetlinewidth{0.803000pt}%
\definecolor{currentstroke}{rgb}{0.000000,0.000000,0.000000}%
\pgfsetstrokecolor{currentstroke}%
\pgfsetdash{}{0pt}%
\pgfsys@defobject{currentmarker}{\pgfqpoint{-0.048611in}{0.000000in}}{\pgfqpoint{0.000000in}{0.000000in}}{%
\pgfpathmoveto{\pgfqpoint{0.000000in}{0.000000in}}%
\pgfpathlineto{\pgfqpoint{-0.048611in}{0.000000in}}%
\pgfusepath{stroke,fill}%
}%
\begin{pgfscope}%
\pgfsys@transformshift{0.892500in}{0.923365in}%
\pgfsys@useobject{currentmarker}{}%
\end{pgfscope}%
\end{pgfscope}%
\begin{pgfscope}%
\pgftext[x=0.725833in,y=0.870603in,left,base]{\sffamily\fontsize{10.000000}{12.000000}\selectfont \(\displaystyle 1\)}%
\end{pgfscope}%
\begin{pgfscope}%
\pgfsetbuttcap%
\pgfsetroundjoin%
\definecolor{currentfill}{rgb}{0.000000,0.000000,0.000000}%
\pgfsetfillcolor{currentfill}%
\pgfsetlinewidth{0.803000pt}%
\definecolor{currentstroke}{rgb}{0.000000,0.000000,0.000000}%
\pgfsetstrokecolor{currentstroke}%
\pgfsetdash{}{0pt}%
\pgfsys@defobject{currentmarker}{\pgfqpoint{-0.048611in}{0.000000in}}{\pgfqpoint{0.000000in}{0.000000in}}{%
\pgfpathmoveto{\pgfqpoint{0.000000in}{0.000000in}}%
\pgfpathlineto{\pgfqpoint{-0.048611in}{0.000000in}}%
\pgfusepath{stroke,fill}%
}%
\begin{pgfscope}%
\pgfsys@transformshift{0.892500in}{1.077312in}%
\pgfsys@useobject{currentmarker}{}%
\end{pgfscope}%
\end{pgfscope}%
\begin{pgfscope}%
\pgftext[x=0.725833in,y=1.024551in,left,base]{\sffamily\fontsize{10.000000}{12.000000}\selectfont \(\displaystyle 2\)}%
\end{pgfscope}%
\begin{pgfscope}%
\pgfsetbuttcap%
\pgfsetroundjoin%
\definecolor{currentfill}{rgb}{0.000000,0.000000,0.000000}%
\pgfsetfillcolor{currentfill}%
\pgfsetlinewidth{0.803000pt}%
\definecolor{currentstroke}{rgb}{0.000000,0.000000,0.000000}%
\pgfsetstrokecolor{currentstroke}%
\pgfsetdash{}{0pt}%
\pgfsys@defobject{currentmarker}{\pgfqpoint{-0.048611in}{0.000000in}}{\pgfqpoint{0.000000in}{0.000000in}}{%
\pgfpathmoveto{\pgfqpoint{0.000000in}{0.000000in}}%
\pgfpathlineto{\pgfqpoint{-0.048611in}{0.000000in}}%
\pgfusepath{stroke,fill}%
}%
\begin{pgfscope}%
\pgfsys@transformshift{0.892500in}{1.231260in}%
\pgfsys@useobject{currentmarker}{}%
\end{pgfscope}%
\end{pgfscope}%
\begin{pgfscope}%
\pgftext[x=0.725833in,y=1.178498in,left,base]{\sffamily\fontsize{10.000000}{12.000000}\selectfont \(\displaystyle 3\)}%
\end{pgfscope}%
\begin{pgfscope}%
\pgfsetbuttcap%
\pgfsetroundjoin%
\definecolor{currentfill}{rgb}{0.000000,0.000000,0.000000}%
\pgfsetfillcolor{currentfill}%
\pgfsetlinewidth{0.803000pt}%
\definecolor{currentstroke}{rgb}{0.000000,0.000000,0.000000}%
\pgfsetstrokecolor{currentstroke}%
\pgfsetdash{}{0pt}%
\pgfsys@defobject{currentmarker}{\pgfqpoint{-0.048611in}{0.000000in}}{\pgfqpoint{0.000000in}{0.000000in}}{%
\pgfpathmoveto{\pgfqpoint{0.000000in}{0.000000in}}%
\pgfpathlineto{\pgfqpoint{-0.048611in}{0.000000in}}%
\pgfusepath{stroke,fill}%
}%
\begin{pgfscope}%
\pgfsys@transformshift{0.892500in}{1.385208in}%
\pgfsys@useobject{currentmarker}{}%
\end{pgfscope}%
\end{pgfscope}%
\begin{pgfscope}%
\pgftext[x=0.725833in,y=1.332446in,left,base]{\sffamily\fontsize{10.000000}{12.000000}\selectfont \(\displaystyle 4\)}%
\end{pgfscope}%
\begin{pgfscope}%
\pgfsetbuttcap%
\pgfsetroundjoin%
\definecolor{currentfill}{rgb}{0.000000,0.000000,0.000000}%
\pgfsetfillcolor{currentfill}%
\pgfsetlinewidth{0.803000pt}%
\definecolor{currentstroke}{rgb}{0.000000,0.000000,0.000000}%
\pgfsetstrokecolor{currentstroke}%
\pgfsetdash{}{0pt}%
\pgfsys@defobject{currentmarker}{\pgfqpoint{-0.048611in}{0.000000in}}{\pgfqpoint{0.000000in}{0.000000in}}{%
\pgfpathmoveto{\pgfqpoint{0.000000in}{0.000000in}}%
\pgfpathlineto{\pgfqpoint{-0.048611in}{0.000000in}}%
\pgfusepath{stroke,fill}%
}%
\begin{pgfscope}%
\pgfsys@transformshift{0.892500in}{1.539155in}%
\pgfsys@useobject{currentmarker}{}%
\end{pgfscope}%
\end{pgfscope}%
\begin{pgfscope}%
\pgftext[x=0.725833in,y=1.486394in,left,base]{\sffamily\fontsize{10.000000}{12.000000}\selectfont \(\displaystyle 5\)}%
\end{pgfscope}%
\begin{pgfscope}%
\pgfsetbuttcap%
\pgfsetroundjoin%
\definecolor{currentfill}{rgb}{0.000000,0.000000,0.000000}%
\pgfsetfillcolor{currentfill}%
\pgfsetlinewidth{0.803000pt}%
\definecolor{currentstroke}{rgb}{0.000000,0.000000,0.000000}%
\pgfsetstrokecolor{currentstroke}%
\pgfsetdash{}{0pt}%
\pgfsys@defobject{currentmarker}{\pgfqpoint{-0.048611in}{0.000000in}}{\pgfqpoint{0.000000in}{0.000000in}}{%
\pgfpathmoveto{\pgfqpoint{0.000000in}{0.000000in}}%
\pgfpathlineto{\pgfqpoint{-0.048611in}{0.000000in}}%
\pgfusepath{stroke,fill}%
}%
\begin{pgfscope}%
\pgfsys@transformshift{0.892500in}{1.693103in}%
\pgfsys@useobject{currentmarker}{}%
\end{pgfscope}%
\end{pgfscope}%
\begin{pgfscope}%
\pgftext[x=0.725833in,y=1.640341in,left,base]{\sffamily\fontsize{10.000000}{12.000000}\selectfont \(\displaystyle 6\)}%
\end{pgfscope}%
\begin{pgfscope}%
\pgfsetbuttcap%
\pgfsetroundjoin%
\definecolor{currentfill}{rgb}{0.000000,0.000000,0.000000}%
\pgfsetfillcolor{currentfill}%
\pgfsetlinewidth{0.803000pt}%
\definecolor{currentstroke}{rgb}{0.000000,0.000000,0.000000}%
\pgfsetstrokecolor{currentstroke}%
\pgfsetdash{}{0pt}%
\pgfsys@defobject{currentmarker}{\pgfqpoint{-0.048611in}{0.000000in}}{\pgfqpoint{0.000000in}{0.000000in}}{%
\pgfpathmoveto{\pgfqpoint{0.000000in}{0.000000in}}%
\pgfpathlineto{\pgfqpoint{-0.048611in}{0.000000in}}%
\pgfusepath{stroke,fill}%
}%
\begin{pgfscope}%
\pgfsys@transformshift{0.892500in}{1.847051in}%
\pgfsys@useobject{currentmarker}{}%
\end{pgfscope}%
\end{pgfscope}%
\begin{pgfscope}%
\pgftext[x=0.725833in,y=1.794289in,left,base]{\sffamily\fontsize{10.000000}{12.000000}\selectfont \(\displaystyle 7\)}%
\end{pgfscope}%
\begin{pgfscope}%
\pgfsetbuttcap%
\pgfsetroundjoin%
\definecolor{currentfill}{rgb}{0.000000,0.000000,0.000000}%
\pgfsetfillcolor{currentfill}%
\pgfsetlinewidth{0.803000pt}%
\definecolor{currentstroke}{rgb}{0.000000,0.000000,0.000000}%
\pgfsetstrokecolor{currentstroke}%
\pgfsetdash{}{0pt}%
\pgfsys@defobject{currentmarker}{\pgfqpoint{-0.048611in}{0.000000in}}{\pgfqpoint{0.000000in}{0.000000in}}{%
\pgfpathmoveto{\pgfqpoint{0.000000in}{0.000000in}}%
\pgfpathlineto{\pgfqpoint{-0.048611in}{0.000000in}}%
\pgfusepath{stroke,fill}%
}%
\begin{pgfscope}%
\pgfsys@transformshift{0.892500in}{2.000998in}%
\pgfsys@useobject{currentmarker}{}%
\end{pgfscope}%
\end{pgfscope}%
\begin{pgfscope}%
\pgftext[x=0.725833in,y=1.948237in,left,base]{\sffamily\fontsize{10.000000}{12.000000}\selectfont \(\displaystyle 8\)}%
\end{pgfscope}%
\begin{pgfscope}%
\pgfsetbuttcap%
\pgfsetroundjoin%
\definecolor{currentfill}{rgb}{0.000000,0.000000,0.000000}%
\pgfsetfillcolor{currentfill}%
\pgfsetlinewidth{0.803000pt}%
\definecolor{currentstroke}{rgb}{0.000000,0.000000,0.000000}%
\pgfsetstrokecolor{currentstroke}%
\pgfsetdash{}{0pt}%
\pgfsys@defobject{currentmarker}{\pgfqpoint{-0.048611in}{0.000000in}}{\pgfqpoint{0.000000in}{0.000000in}}{%
\pgfpathmoveto{\pgfqpoint{0.000000in}{0.000000in}}%
\pgfpathlineto{\pgfqpoint{-0.048611in}{0.000000in}}%
\pgfusepath{stroke,fill}%
}%
\begin{pgfscope}%
\pgfsys@transformshift{0.892500in}{2.154946in}%
\pgfsys@useobject{currentmarker}{}%
\end{pgfscope}%
\end{pgfscope}%
\begin{pgfscope}%
\pgftext[x=0.725833in,y=2.102184in,left,base]{\sffamily\fontsize{10.000000}{12.000000}\selectfont \(\displaystyle 9\)}%
\end{pgfscope}%
\begin{pgfscope}%
\pgfsetbuttcap%
\pgfsetroundjoin%
\definecolor{currentfill}{rgb}{0.000000,0.000000,0.000000}%
\pgfsetfillcolor{currentfill}%
\pgfsetlinewidth{0.803000pt}%
\definecolor{currentstroke}{rgb}{0.000000,0.000000,0.000000}%
\pgfsetstrokecolor{currentstroke}%
\pgfsetdash{}{0pt}%
\pgfsys@defobject{currentmarker}{\pgfqpoint{-0.048611in}{0.000000in}}{\pgfqpoint{0.000000in}{0.000000in}}{%
\pgfpathmoveto{\pgfqpoint{0.000000in}{0.000000in}}%
\pgfpathlineto{\pgfqpoint{-0.048611in}{0.000000in}}%
\pgfusepath{stroke,fill}%
}%
\begin{pgfscope}%
\pgfsys@transformshift{0.892500in}{2.308893in}%
\pgfsys@useobject{currentmarker}{}%
\end{pgfscope}%
\end{pgfscope}%
\begin{pgfscope}%
\pgftext[x=0.656388in,y=2.256132in,left,base]{\sffamily\fontsize{10.000000}{12.000000}\selectfont \(\displaystyle 10\)}%
\end{pgfscope}%
\begin{pgfscope}%
\pgfsetbuttcap%
\pgfsetroundjoin%
\definecolor{currentfill}{rgb}{0.000000,0.000000,0.000000}%
\pgfsetfillcolor{currentfill}%
\pgfsetlinewidth{0.803000pt}%
\definecolor{currentstroke}{rgb}{0.000000,0.000000,0.000000}%
\pgfsetstrokecolor{currentstroke}%
\pgfsetdash{}{0pt}%
\pgfsys@defobject{currentmarker}{\pgfqpoint{-0.048611in}{0.000000in}}{\pgfqpoint{0.000000in}{0.000000in}}{%
\pgfpathmoveto{\pgfqpoint{0.000000in}{0.000000in}}%
\pgfpathlineto{\pgfqpoint{-0.048611in}{0.000000in}}%
\pgfusepath{stroke,fill}%
}%
\begin{pgfscope}%
\pgfsys@transformshift{0.892500in}{2.462841in}%
\pgfsys@useobject{currentmarker}{}%
\end{pgfscope}%
\end{pgfscope}%
\begin{pgfscope}%
\pgftext[x=0.656388in,y=2.410080in,left,base]{\sffamily\fontsize{10.000000}{12.000000}\selectfont \(\displaystyle 11\)}%
\end{pgfscope}%
\begin{pgfscope}%
\pgfsetbuttcap%
\pgfsetroundjoin%
\definecolor{currentfill}{rgb}{0.000000,0.000000,0.000000}%
\pgfsetfillcolor{currentfill}%
\pgfsetlinewidth{0.803000pt}%
\definecolor{currentstroke}{rgb}{0.000000,0.000000,0.000000}%
\pgfsetstrokecolor{currentstroke}%
\pgfsetdash{}{0pt}%
\pgfsys@defobject{currentmarker}{\pgfqpoint{-0.048611in}{0.000000in}}{\pgfqpoint{0.000000in}{0.000000in}}{%
\pgfpathmoveto{\pgfqpoint{0.000000in}{0.000000in}}%
\pgfpathlineto{\pgfqpoint{-0.048611in}{0.000000in}}%
\pgfusepath{stroke,fill}%
}%
\begin{pgfscope}%
\pgfsys@transformshift{0.892500in}{2.616789in}%
\pgfsys@useobject{currentmarker}{}%
\end{pgfscope}%
\end{pgfscope}%
\begin{pgfscope}%
\pgftext[x=0.656388in,y=2.564027in,left,base]{\sffamily\fontsize{10.000000}{12.000000}\selectfont \(\displaystyle 12\)}%
\end{pgfscope}%
\begin{pgfscope}%
\pgfsetbuttcap%
\pgfsetroundjoin%
\definecolor{currentfill}{rgb}{0.000000,0.000000,0.000000}%
\pgfsetfillcolor{currentfill}%
\pgfsetlinewidth{0.803000pt}%
\definecolor{currentstroke}{rgb}{0.000000,0.000000,0.000000}%
\pgfsetstrokecolor{currentstroke}%
\pgfsetdash{}{0pt}%
\pgfsys@defobject{currentmarker}{\pgfqpoint{-0.048611in}{0.000000in}}{\pgfqpoint{0.000000in}{0.000000in}}{%
\pgfpathmoveto{\pgfqpoint{0.000000in}{0.000000in}}%
\pgfpathlineto{\pgfqpoint{-0.048611in}{0.000000in}}%
\pgfusepath{stroke,fill}%
}%
\begin{pgfscope}%
\pgfsys@transformshift{0.892500in}{2.770736in}%
\pgfsys@useobject{currentmarker}{}%
\end{pgfscope}%
\end{pgfscope}%
\begin{pgfscope}%
\pgftext[x=0.656388in,y=2.717975in,left,base]{\sffamily\fontsize{10.000000}{12.000000}\selectfont \(\displaystyle 13\)}%
\end{pgfscope}%
\begin{pgfscope}%
\pgfsetbuttcap%
\pgfsetroundjoin%
\definecolor{currentfill}{rgb}{0.000000,0.000000,0.000000}%
\pgfsetfillcolor{currentfill}%
\pgfsetlinewidth{0.803000pt}%
\definecolor{currentstroke}{rgb}{0.000000,0.000000,0.000000}%
\pgfsetstrokecolor{currentstroke}%
\pgfsetdash{}{0pt}%
\pgfsys@defobject{currentmarker}{\pgfqpoint{-0.048611in}{0.000000in}}{\pgfqpoint{0.000000in}{0.000000in}}{%
\pgfpathmoveto{\pgfqpoint{0.000000in}{0.000000in}}%
\pgfpathlineto{\pgfqpoint{-0.048611in}{0.000000in}}%
\pgfusepath{stroke,fill}%
}%
\begin{pgfscope}%
\pgfsys@transformshift{0.892500in}{2.924684in}%
\pgfsys@useobject{currentmarker}{}%
\end{pgfscope}%
\end{pgfscope}%
\begin{pgfscope}%
\pgftext[x=0.656388in,y=2.871923in,left,base]{\sffamily\fontsize{10.000000}{12.000000}\selectfont \(\displaystyle 14\)}%
\end{pgfscope}%
\begin{pgfscope}%
\pgfsetbuttcap%
\pgfsetroundjoin%
\definecolor{currentfill}{rgb}{0.000000,0.000000,0.000000}%
\pgfsetfillcolor{currentfill}%
\pgfsetlinewidth{0.803000pt}%
\definecolor{currentstroke}{rgb}{0.000000,0.000000,0.000000}%
\pgfsetstrokecolor{currentstroke}%
\pgfsetdash{}{0pt}%
\pgfsys@defobject{currentmarker}{\pgfqpoint{-0.048611in}{0.000000in}}{\pgfqpoint{0.000000in}{0.000000in}}{%
\pgfpathmoveto{\pgfqpoint{0.000000in}{0.000000in}}%
\pgfpathlineto{\pgfqpoint{-0.048611in}{0.000000in}}%
\pgfusepath{stroke,fill}%
}%
\begin{pgfscope}%
\pgfsys@transformshift{0.892500in}{3.078632in}%
\pgfsys@useobject{currentmarker}{}%
\end{pgfscope}%
\end{pgfscope}%
\begin{pgfscope}%
\pgftext[x=0.656388in,y=3.025870in,left,base]{\sffamily\fontsize{10.000000}{12.000000}\selectfont \(\displaystyle 15\)}%
\end{pgfscope}%
\begin{pgfscope}%
\pgfsetbuttcap%
\pgfsetroundjoin%
\definecolor{currentfill}{rgb}{0.000000,0.000000,0.000000}%
\pgfsetfillcolor{currentfill}%
\pgfsetlinewidth{0.803000pt}%
\definecolor{currentstroke}{rgb}{0.000000,0.000000,0.000000}%
\pgfsetstrokecolor{currentstroke}%
\pgfsetdash{}{0pt}%
\pgfsys@defobject{currentmarker}{\pgfqpoint{-0.048611in}{0.000000in}}{\pgfqpoint{0.000000in}{0.000000in}}{%
\pgfpathmoveto{\pgfqpoint{0.000000in}{0.000000in}}%
\pgfpathlineto{\pgfqpoint{-0.048611in}{0.000000in}}%
\pgfusepath{stroke,fill}%
}%
\begin{pgfscope}%
\pgfsys@transformshift{0.892500in}{3.232579in}%
\pgfsys@useobject{currentmarker}{}%
\end{pgfscope}%
\end{pgfscope}%
\begin{pgfscope}%
\pgftext[x=0.656388in,y=3.179818in,left,base]{\sffamily\fontsize{10.000000}{12.000000}\selectfont \(\displaystyle 16\)}%
\end{pgfscope}%
\begin{pgfscope}%
\pgfsetbuttcap%
\pgfsetroundjoin%
\definecolor{currentfill}{rgb}{0.000000,0.000000,0.000000}%
\pgfsetfillcolor{currentfill}%
\pgfsetlinewidth{0.803000pt}%
\definecolor{currentstroke}{rgb}{0.000000,0.000000,0.000000}%
\pgfsetstrokecolor{currentstroke}%
\pgfsetdash{}{0pt}%
\pgfsys@defobject{currentmarker}{\pgfqpoint{-0.048611in}{0.000000in}}{\pgfqpoint{0.000000in}{0.000000in}}{%
\pgfpathmoveto{\pgfqpoint{0.000000in}{0.000000in}}%
\pgfpathlineto{\pgfqpoint{-0.048611in}{0.000000in}}%
\pgfusepath{stroke,fill}%
}%
\begin{pgfscope}%
\pgfsys@transformshift{0.892500in}{3.386527in}%
\pgfsys@useobject{currentmarker}{}%
\end{pgfscope}%
\end{pgfscope}%
\begin{pgfscope}%
\pgftext[x=0.656388in,y=3.333765in,left,base]{\sffamily\fontsize{10.000000}{12.000000}\selectfont \(\displaystyle 17\)}%
\end{pgfscope}%
\begin{pgfscope}%
\pgfsetbuttcap%
\pgfsetroundjoin%
\definecolor{currentfill}{rgb}{0.000000,0.000000,0.000000}%
\pgfsetfillcolor{currentfill}%
\pgfsetlinewidth{0.803000pt}%
\definecolor{currentstroke}{rgb}{0.000000,0.000000,0.000000}%
\pgfsetstrokecolor{currentstroke}%
\pgfsetdash{}{0pt}%
\pgfsys@defobject{currentmarker}{\pgfqpoint{-0.048611in}{0.000000in}}{\pgfqpoint{0.000000in}{0.000000in}}{%
\pgfpathmoveto{\pgfqpoint{0.000000in}{0.000000in}}%
\pgfpathlineto{\pgfqpoint{-0.048611in}{0.000000in}}%
\pgfusepath{stroke,fill}%
}%
\begin{pgfscope}%
\pgfsys@transformshift{0.892500in}{3.540475in}%
\pgfsys@useobject{currentmarker}{}%
\end{pgfscope}%
\end{pgfscope}%
\begin{pgfscope}%
\pgftext[x=0.656388in,y=3.487713in,left,base]{\sffamily\fontsize{10.000000}{12.000000}\selectfont \(\displaystyle 18\)}%
\end{pgfscope}%
\begin{pgfscope}%
\pgfsetbuttcap%
\pgfsetroundjoin%
\definecolor{currentfill}{rgb}{0.000000,0.000000,0.000000}%
\pgfsetfillcolor{currentfill}%
\pgfsetlinewidth{0.803000pt}%
\definecolor{currentstroke}{rgb}{0.000000,0.000000,0.000000}%
\pgfsetstrokecolor{currentstroke}%
\pgfsetdash{}{0pt}%
\pgfsys@defobject{currentmarker}{\pgfqpoint{-0.048611in}{0.000000in}}{\pgfqpoint{0.000000in}{0.000000in}}{%
\pgfpathmoveto{\pgfqpoint{0.000000in}{0.000000in}}%
\pgfpathlineto{\pgfqpoint{-0.048611in}{0.000000in}}%
\pgfusepath{stroke,fill}%
}%
\begin{pgfscope}%
\pgfsys@transformshift{0.892500in}{3.694422in}%
\pgfsys@useobject{currentmarker}{}%
\end{pgfscope}%
\end{pgfscope}%
\begin{pgfscope}%
\pgftext[x=0.656388in,y=3.641661in,left,base]{\sffamily\fontsize{10.000000}{12.000000}\selectfont \(\displaystyle 19\)}%
\end{pgfscope}%
\begin{pgfscope}%
\pgfsetbuttcap%
\pgfsetroundjoin%
\definecolor{currentfill}{rgb}{0.000000,0.000000,0.000000}%
\pgfsetfillcolor{currentfill}%
\pgfsetlinewidth{0.803000pt}%
\definecolor{currentstroke}{rgb}{0.000000,0.000000,0.000000}%
\pgfsetstrokecolor{currentstroke}%
\pgfsetdash{}{0pt}%
\pgfsys@defobject{currentmarker}{\pgfqpoint{-0.048611in}{0.000000in}}{\pgfqpoint{0.000000in}{0.000000in}}{%
\pgfpathmoveto{\pgfqpoint{0.000000in}{0.000000in}}%
\pgfpathlineto{\pgfqpoint{-0.048611in}{0.000000in}}%
\pgfusepath{stroke,fill}%
}%
\begin{pgfscope}%
\pgfsys@transformshift{0.892500in}{3.848370in}%
\pgfsys@useobject{currentmarker}{}%
\end{pgfscope}%
\end{pgfscope}%
\begin{pgfscope}%
\pgftext[x=0.656388in,y=3.795608in,left,base]{\sffamily\fontsize{10.000000}{12.000000}\selectfont \(\displaystyle 20\)}%
\end{pgfscope}%
\begin{pgfscope}%
\pgfsetbuttcap%
\pgfsetroundjoin%
\definecolor{currentfill}{rgb}{0.000000,0.000000,0.000000}%
\pgfsetfillcolor{currentfill}%
\pgfsetlinewidth{0.803000pt}%
\definecolor{currentstroke}{rgb}{0.000000,0.000000,0.000000}%
\pgfsetstrokecolor{currentstroke}%
\pgfsetdash{}{0pt}%
\pgfsys@defobject{currentmarker}{\pgfqpoint{-0.048611in}{0.000000in}}{\pgfqpoint{0.000000in}{0.000000in}}{%
\pgfpathmoveto{\pgfqpoint{0.000000in}{0.000000in}}%
\pgfpathlineto{\pgfqpoint{-0.048611in}{0.000000in}}%
\pgfusepath{stroke,fill}%
}%
\begin{pgfscope}%
\pgfsys@transformshift{0.892500in}{4.002318in}%
\pgfsys@useobject{currentmarker}{}%
\end{pgfscope}%
\end{pgfscope}%
\begin{pgfscope}%
\pgftext[x=0.656388in,y=3.949556in,left,base]{\sffamily\fontsize{10.000000}{12.000000}\selectfont \(\displaystyle 21\)}%
\end{pgfscope}%
\begin{pgfscope}%
\pgfsetbuttcap%
\pgfsetroundjoin%
\definecolor{currentfill}{rgb}{0.000000,0.000000,0.000000}%
\pgfsetfillcolor{currentfill}%
\pgfsetlinewidth{0.803000pt}%
\definecolor{currentstroke}{rgb}{0.000000,0.000000,0.000000}%
\pgfsetstrokecolor{currentstroke}%
\pgfsetdash{}{0pt}%
\pgfsys@defobject{currentmarker}{\pgfqpoint{-0.048611in}{0.000000in}}{\pgfqpoint{0.000000in}{0.000000in}}{%
\pgfpathmoveto{\pgfqpoint{0.000000in}{0.000000in}}%
\pgfpathlineto{\pgfqpoint{-0.048611in}{0.000000in}}%
\pgfusepath{stroke,fill}%
}%
\begin{pgfscope}%
\pgfsys@transformshift{0.892500in}{4.156265in}%
\pgfsys@useobject{currentmarker}{}%
\end{pgfscope}%
\end{pgfscope}%
\begin{pgfscope}%
\pgftext[x=0.656388in,y=4.103504in,left,base]{\sffamily\fontsize{10.000000}{12.000000}\selectfont \(\displaystyle 22\)}%
\end{pgfscope}%
\begin{pgfscope}%
\pgfsetbuttcap%
\pgfsetroundjoin%
\definecolor{currentfill}{rgb}{0.000000,0.000000,0.000000}%
\pgfsetfillcolor{currentfill}%
\pgfsetlinewidth{0.803000pt}%
\definecolor{currentstroke}{rgb}{0.000000,0.000000,0.000000}%
\pgfsetstrokecolor{currentstroke}%
\pgfsetdash{}{0pt}%
\pgfsys@defobject{currentmarker}{\pgfqpoint{-0.048611in}{0.000000in}}{\pgfqpoint{0.000000in}{0.000000in}}{%
\pgfpathmoveto{\pgfqpoint{0.000000in}{0.000000in}}%
\pgfpathlineto{\pgfqpoint{-0.048611in}{0.000000in}}%
\pgfusepath{stroke,fill}%
}%
\begin{pgfscope}%
\pgfsys@transformshift{0.892500in}{4.310213in}%
\pgfsys@useobject{currentmarker}{}%
\end{pgfscope}%
\end{pgfscope}%
\begin{pgfscope}%
\pgftext[x=0.656388in,y=4.257451in,left,base]{\sffamily\fontsize{10.000000}{12.000000}\selectfont \(\displaystyle 23\)}%
\end{pgfscope}%
\begin{pgfscope}%
\pgfsetbuttcap%
\pgfsetroundjoin%
\definecolor{currentfill}{rgb}{0.000000,0.000000,0.000000}%
\pgfsetfillcolor{currentfill}%
\pgfsetlinewidth{0.803000pt}%
\definecolor{currentstroke}{rgb}{0.000000,0.000000,0.000000}%
\pgfsetstrokecolor{currentstroke}%
\pgfsetdash{}{0pt}%
\pgfsys@defobject{currentmarker}{\pgfqpoint{-0.048611in}{0.000000in}}{\pgfqpoint{0.000000in}{0.000000in}}{%
\pgfpathmoveto{\pgfqpoint{0.000000in}{0.000000in}}%
\pgfpathlineto{\pgfqpoint{-0.048611in}{0.000000in}}%
\pgfusepath{stroke,fill}%
}%
\begin{pgfscope}%
\pgfsys@transformshift{0.892500in}{4.464161in}%
\pgfsys@useobject{currentmarker}{}%
\end{pgfscope}%
\end{pgfscope}%
\begin{pgfscope}%
\pgftext[x=0.656388in,y=4.411399in,left,base]{\sffamily\fontsize{10.000000}{12.000000}\selectfont \(\displaystyle 24\)}%
\end{pgfscope}%
\begin{pgfscope}%
\pgfsetbuttcap%
\pgfsetroundjoin%
\definecolor{currentfill}{rgb}{0.000000,0.000000,0.000000}%
\pgfsetfillcolor{currentfill}%
\pgfsetlinewidth{0.803000pt}%
\definecolor{currentstroke}{rgb}{0.000000,0.000000,0.000000}%
\pgfsetstrokecolor{currentstroke}%
\pgfsetdash{}{0pt}%
\pgfsys@defobject{currentmarker}{\pgfqpoint{-0.048611in}{0.000000in}}{\pgfqpoint{0.000000in}{0.000000in}}{%
\pgfpathmoveto{\pgfqpoint{0.000000in}{0.000000in}}%
\pgfpathlineto{\pgfqpoint{-0.048611in}{0.000000in}}%
\pgfusepath{stroke,fill}%
}%
\begin{pgfscope}%
\pgfsys@transformshift{0.892500in}{4.618108in}%
\pgfsys@useobject{currentmarker}{}%
\end{pgfscope}%
\end{pgfscope}%
\begin{pgfscope}%
\pgftext[x=0.656388in,y=4.565347in,left,base]{\sffamily\fontsize{10.000000}{12.000000}\selectfont \(\displaystyle 25\)}%
\end{pgfscope}%
\begin{pgfscope}%
\pgftext[x=0.600833in,y=2.643300in,,bottom,rotate=90.000000]{\sffamily\fontsize{12.000000}{14.400000}\selectfont seconds}%
\end{pgfscope}%
\begin{pgfscope}%
\pgfpathrectangle{\pgfqpoint{0.892500in}{0.587400in}}{\pgfqpoint{5.533500in}{4.111800in}}%
\pgfusepath{clip}%
\pgfsetrectcap%
\pgfsetroundjoin%
\pgfsetlinewidth{1.505625pt}%
\definecolor{currentstroke}{rgb}{0.172549,0.243137,0.313725}%
\pgfsetstrokecolor{currentstroke}%
\pgfsetdash{}{0pt}%
\pgfpathmoveto{\pgfqpoint{1.144023in}{1.473390in}}%
\pgfpathlineto{\pgfqpoint{1.246685in}{1.509978in}}%
\pgfpathlineto{\pgfqpoint{1.349347in}{1.552530in}}%
\pgfpathlineto{\pgfqpoint{1.452010in}{1.490252in}}%
\pgfpathlineto{\pgfqpoint{1.554672in}{1.519575in}}%
\pgfpathlineto{\pgfqpoint{1.657334in}{1.651224in}}%
\pgfpathlineto{\pgfqpoint{1.759997in}{1.685646in}}%
\pgfpathlineto{\pgfqpoint{1.862659in}{1.714343in}}%
\pgfpathlineto{\pgfqpoint{1.965321in}{1.757064in}}%
\pgfpathlineto{\pgfqpoint{2.067984in}{1.875940in}}%
\pgfpathlineto{\pgfqpoint{2.170646in}{1.885128in}}%
\pgfpathlineto{\pgfqpoint{2.273308in}{1.966913in}}%
\pgfpathlineto{\pgfqpoint{2.375971in}{2.045498in}}%
\pgfpathlineto{\pgfqpoint{2.478633in}{2.175416in}}%
\pgfpathlineto{\pgfqpoint{2.581295in}{2.221504in}}%
\pgfpathlineto{\pgfqpoint{2.683958in}{2.155908in}}%
\pgfpathlineto{\pgfqpoint{2.786620in}{2.185255in}}%
\pgfpathlineto{\pgfqpoint{2.889282in}{2.244668in}}%
\pgfpathlineto{\pgfqpoint{2.991945in}{2.365084in}}%
\pgfpathlineto{\pgfqpoint{3.094607in}{2.441596in}}%
\pgfpathlineto{\pgfqpoint{3.197269in}{2.544895in}}%
\pgfpathlineto{\pgfqpoint{3.299932in}{2.563677in}}%
\pgfpathlineto{\pgfqpoint{3.402594in}{2.666853in}}%
\pgfpathlineto{\pgfqpoint{3.505256in}{2.783129in}}%
\pgfpathlineto{\pgfqpoint{3.607919in}{2.758852in}}%
\pgfpathlineto{\pgfqpoint{3.710581in}{2.962540in}}%
\pgfpathlineto{\pgfqpoint{3.813244in}{2.945005in}}%
\pgfpathlineto{\pgfqpoint{3.915906in}{3.122922in}}%
\pgfpathlineto{\pgfqpoint{4.018568in}{3.204900in}}%
\pgfpathlineto{\pgfqpoint{4.121231in}{3.343591in}}%
\pgfpathlineto{\pgfqpoint{4.223893in}{3.351073in}}%
\pgfpathlineto{\pgfqpoint{4.326555in}{3.163226in}}%
\pgfpathlineto{\pgfqpoint{4.429218in}{3.238275in}}%
\pgfpathlineto{\pgfqpoint{4.531880in}{3.321392in}}%
\pgfpathlineto{\pgfqpoint{4.634542in}{3.390083in}}%
\pgfpathlineto{\pgfqpoint{4.737205in}{3.465471in}}%
\pgfpathlineto{\pgfqpoint{4.839867in}{3.508577in}}%
\pgfpathlineto{\pgfqpoint{4.942529in}{3.609536in}}%
\pgfpathlineto{\pgfqpoint{5.045192in}{3.676903in}}%
\pgfpathlineto{\pgfqpoint{5.147854in}{3.746287in}}%
\pgfpathlineto{\pgfqpoint{5.250516in}{3.864534in}}%
\pgfpathlineto{\pgfqpoint{5.353179in}{3.918339in}}%
\pgfpathlineto{\pgfqpoint{5.455841in}{3.999839in}}%
\pgfpathlineto{\pgfqpoint{5.558503in}{4.075781in}}%
\pgfpathlineto{\pgfqpoint{5.661166in}{4.158236in}}%
\pgfpathlineto{\pgfqpoint{5.763828in}{4.214504in}}%
\pgfpathlineto{\pgfqpoint{5.866490in}{4.300253in}}%
\pgfpathlineto{\pgfqpoint{5.969153in}{4.401134in}}%
\pgfpathlineto{\pgfqpoint{6.071815in}{4.470796in}}%
\pgfpathlineto{\pgfqpoint{6.174477in}{4.512300in}}%
\pgfusepath{stroke}%
\end{pgfscope}%
\begin{pgfscope}%
\pgfpathrectangle{\pgfqpoint{0.892500in}{0.587400in}}{\pgfqpoint{5.533500in}{4.111800in}}%
\pgfusepath{clip}%
\pgfsetrectcap%
\pgfsetroundjoin%
\pgfsetlinewidth{1.505625pt}%
\definecolor{currentstroke}{rgb}{0.086275,0.627451,0.521569}%
\pgfsetstrokecolor{currentstroke}%
\pgfsetdash{}{0pt}%
\pgfpathmoveto{\pgfqpoint{1.144023in}{0.782503in}}%
\pgfpathlineto{\pgfqpoint{1.246685in}{0.781829in}}%
\pgfpathlineto{\pgfqpoint{1.349347in}{0.782286in}}%
\pgfpathlineto{\pgfqpoint{1.452010in}{0.779279in}}%
\pgfpathlineto{\pgfqpoint{1.554672in}{0.778437in}}%
\pgfpathlineto{\pgfqpoint{1.657334in}{0.778846in}}%
\pgfpathlineto{\pgfqpoint{1.759997in}{0.777716in}}%
\pgfpathlineto{\pgfqpoint{1.862659in}{0.776609in}}%
\pgfpathlineto{\pgfqpoint{1.965321in}{0.776104in}}%
\pgfpathlineto{\pgfqpoint{2.067984in}{0.776393in}}%
\pgfpathlineto{\pgfqpoint{2.170646in}{0.775022in}}%
\pgfpathlineto{\pgfqpoint{2.273308in}{0.774853in}}%
\pgfpathlineto{\pgfqpoint{2.375971in}{0.774853in}}%
\pgfpathlineto{\pgfqpoint{2.478633in}{0.775214in}}%
\pgfpathlineto{\pgfqpoint{2.581295in}{0.775286in}}%
\pgfpathlineto{\pgfqpoint{2.683958in}{0.774468in}}%
\pgfpathlineto{\pgfqpoint{2.786620in}{0.774468in}}%
\pgfpathlineto{\pgfqpoint{2.889282in}{0.774396in}}%
\pgfpathlineto{\pgfqpoint{2.991945in}{0.774324in}}%
\pgfpathlineto{\pgfqpoint{3.094607in}{0.774781in}}%
\pgfpathlineto{\pgfqpoint{3.197269in}{0.775407in}}%
\pgfpathlineto{\pgfqpoint{3.299932in}{0.774300in}}%
\pgfpathlineto{\pgfqpoint{3.402594in}{0.774613in}}%
\pgfpathlineto{\pgfqpoint{3.505256in}{0.775310in}}%
\pgfpathlineto{\pgfqpoint{3.607919in}{0.774492in}}%
\pgfpathlineto{\pgfqpoint{3.710581in}{0.774877in}}%
\pgfpathlineto{\pgfqpoint{3.813244in}{0.775046in}}%
\pgfpathlineto{\pgfqpoint{3.915906in}{0.775166in}}%
\pgfpathlineto{\pgfqpoint{4.018568in}{0.775214in}}%
\pgfpathlineto{\pgfqpoint{4.121231in}{0.775599in}}%
\pgfpathlineto{\pgfqpoint{4.223893in}{0.775070in}}%
\pgfpathlineto{\pgfqpoint{4.326555in}{0.774877in}}%
\pgfpathlineto{\pgfqpoint{4.429218in}{0.774709in}}%
\pgfpathlineto{\pgfqpoint{4.531880in}{0.774709in}}%
\pgfpathlineto{\pgfqpoint{4.634542in}{0.774998in}}%
\pgfpathlineto{\pgfqpoint{4.737205in}{0.775142in}}%
\pgfpathlineto{\pgfqpoint{4.839867in}{0.774926in}}%
\pgfpathlineto{\pgfqpoint{4.942529in}{0.775166in}}%
\pgfpathlineto{\pgfqpoint{5.045192in}{0.775311in}}%
\pgfpathlineto{\pgfqpoint{5.147854in}{0.775311in}}%
\pgfpathlineto{\pgfqpoint{5.250516in}{0.775623in}}%
\pgfpathlineto{\pgfqpoint{5.353179in}{0.775238in}}%
\pgfpathlineto{\pgfqpoint{5.455841in}{0.775527in}}%
\pgfpathlineto{\pgfqpoint{5.558503in}{0.775575in}}%
\pgfpathlineto{\pgfqpoint{5.661166in}{0.775575in}}%
\pgfpathlineto{\pgfqpoint{5.763828in}{0.775744in}}%
\pgfpathlineto{\pgfqpoint{5.866490in}{0.775912in}}%
\pgfpathlineto{\pgfqpoint{5.969153in}{0.776417in}}%
\pgfpathlineto{\pgfqpoint{6.071815in}{0.776321in}}%
\pgfpathlineto{\pgfqpoint{6.174477in}{0.775888in}}%
\pgfusepath{stroke}%
\end{pgfscope}%
\begin{pgfscope}%
\pgfsetrectcap%
\pgfsetmiterjoin%
\pgfsetlinewidth{0.803000pt}%
\definecolor{currentstroke}{rgb}{0.000000,0.000000,0.000000}%
\pgfsetstrokecolor{currentstroke}%
\pgfsetdash{}{0pt}%
\pgfpathmoveto{\pgfqpoint{0.892500in}{0.587400in}}%
\pgfpathlineto{\pgfqpoint{0.892500in}{4.699200in}}%
\pgfusepath{stroke}%
\end{pgfscope}%
\begin{pgfscope}%
\pgfsetrectcap%
\pgfsetmiterjoin%
\pgfsetlinewidth{0.803000pt}%
\definecolor{currentstroke}{rgb}{0.000000,0.000000,0.000000}%
\pgfsetstrokecolor{currentstroke}%
\pgfsetdash{}{0pt}%
\pgfpathmoveto{\pgfqpoint{6.426000in}{0.587400in}}%
\pgfpathlineto{\pgfqpoint{6.426000in}{4.699200in}}%
\pgfusepath{stroke}%
\end{pgfscope}%
\begin{pgfscope}%
\pgfsetrectcap%
\pgfsetmiterjoin%
\pgfsetlinewidth{0.803000pt}%
\definecolor{currentstroke}{rgb}{0.000000,0.000000,0.000000}%
\pgfsetstrokecolor{currentstroke}%
\pgfsetdash{}{0pt}%
\pgfpathmoveto{\pgfqpoint{0.892500in}{0.587400in}}%
\pgfpathlineto{\pgfqpoint{6.426000in}{0.587400in}}%
\pgfusepath{stroke}%
\end{pgfscope}%
\begin{pgfscope}%
\pgfsetrectcap%
\pgfsetmiterjoin%
\pgfsetlinewidth{0.803000pt}%
\definecolor{currentstroke}{rgb}{0.000000,0.000000,0.000000}%
\pgfsetstrokecolor{currentstroke}%
\pgfsetdash{}{0pt}%
\pgfpathmoveto{\pgfqpoint{0.892500in}{4.699200in}}%
\pgfpathlineto{\pgfqpoint{6.426000in}{4.699200in}}%
\pgfusepath{stroke}%
\end{pgfscope}%
\begin{pgfscope}%
\pgfsetbuttcap%
\pgfsetmiterjoin%
\definecolor{currentfill}{rgb}{1.000000,1.000000,1.000000}%
\pgfsetfillcolor{currentfill}%
\pgfsetfillopacity{0.800000}%
\pgfsetlinewidth{1.003750pt}%
\definecolor{currentstroke}{rgb}{0.800000,0.800000,0.800000}%
\pgfsetstrokecolor{currentstroke}%
\pgfsetstrokeopacity{0.800000}%
\pgfsetdash{}{0pt}%
\pgfpathmoveto{\pgfqpoint{0.989722in}{4.180374in}}%
\pgfpathlineto{\pgfqpoint{2.137970in}{4.180374in}}%
\pgfpathquadraticcurveto{\pgfqpoint{2.165748in}{4.180374in}}{\pgfqpoint{2.165748in}{4.208152in}}%
\pgfpathlineto{\pgfqpoint{2.165748in}{4.601978in}}%
\pgfpathquadraticcurveto{\pgfqpoint{2.165748in}{4.629756in}}{\pgfqpoint{2.137970in}{4.629756in}}%
\pgfpathlineto{\pgfqpoint{0.989722in}{4.629756in}}%
\pgfpathquadraticcurveto{\pgfqpoint{0.961944in}{4.629756in}}{\pgfqpoint{0.961944in}{4.601978in}}%
\pgfpathlineto{\pgfqpoint{0.961944in}{4.208152in}}%
\pgfpathquadraticcurveto{\pgfqpoint{0.961944in}{4.180374in}}{\pgfqpoint{0.989722in}{4.180374in}}%
\pgfpathclose%
\pgfusepath{stroke,fill}%
\end{pgfscope}%
\begin{pgfscope}%
\pgfsetrectcap%
\pgfsetroundjoin%
\pgfsetlinewidth{1.505625pt}%
\definecolor{currentstroke}{rgb}{0.172549,0.243137,0.313725}%
\pgfsetstrokecolor{currentstroke}%
\pgfsetdash{}{0pt}%
\pgfpathmoveto{\pgfqpoint{1.017500in}{4.517288in}}%
\pgfpathlineto{\pgfqpoint{1.295278in}{4.517288in}}%
\pgfusepath{stroke}%
\end{pgfscope}%
\begin{pgfscope}%
\pgftext[x=1.406389in,y=4.468677in,left,base]{\sffamily\fontsize{10.000000}{12.000000}\selectfont \textsc{Linclosure}}%
\end{pgfscope}%
\begin{pgfscope}%
\pgfsetrectcap%
\pgfsetroundjoin%
\pgfsetlinewidth{1.505625pt}%
\definecolor{currentstroke}{rgb}{0.086275,0.627451,0.521569}%
\pgfsetstrokecolor{currentstroke}%
\pgfsetdash{}{0pt}%
\pgfpathmoveto{\pgfqpoint{1.017500in}{4.313431in}}%
\pgfpathlineto{\pgfqpoint{1.295278in}{4.313431in}}%
\pgfusepath{stroke}%
\end{pgfscope}%
\begin{pgfscope}%
\pgftext[x=1.406389in,y=4.264820in,left,base]{\sffamily\fontsize{10.000000}{12.000000}\selectfont \textsc{Closure}}%
\end{pgfscope}%
\end{pgfpicture}%
\makeatother%
\endgroup%
}
}

\centering

\subfloat[Some landmarks times, $|\Sg| = 100$]{
\begin{tabular}{| c || c | c |}
	\hline \rowcolor{clouds}
	$|\B|$ & \textsc{LinClosure} & \textsc{Closure} \\ \hline
	5000  & 15.846  & 0.688 \\ \hline
	10000 & 65.930  & 2.724 \\ \hline
	15000 & 115.627 & 4.784 \\ \hline 
	20000 & 223.453 & 9.721 \\ \hline 
\end{tabular}
}\qquad
\subfloat[Some landmarks times]{
	\begin{tabular}{| c || c | c |}
		\hline \rowcolor{clouds}
		$|\Sigma|$ & \textsc{LinClosure} & \textsc{Closure} \\ \hline
		10000  & 4.873  & 0.058 \\ \hline
		20000  & 7.187  & 0.045 \\ \hline
		30000  & 9.432  & 0.038 \\ \hline 
		40000  & 10.862 & 0.034 \\ \hline 
		50000  & 12.923 & 0.032 \\ \hline
		60000  & 16.721 & 0.040 \\ \hline
		70000  & 17.023 & 0.036 \\ \hline
		80000  & 19.337 & 0.038 \\ \hline 
		90000  & 22.013 & 0.040 \\ \hline 
		100000 & 24.312 & 0.042 \\ \hline
	\end{tabular}
}


\caption{Comparison of closure operators for \textsc{MinCover}}
\label{fig:MinCover-Prune}
\input{../Subfiles/Languages.tex}


\title{Master Thesis: Horn Minimization}
\author{Simon VILMIN}
\date{year 2017 - 2018}

\begin{document}
\maketitle
\tableofcontents

\chapter{Introduction and definitions}

This report summarizes the work on the Horn-minimization subject. We first give
mathematical definitions we may need for all the report, the reader may refer 
to \cite{Lat&Ord}, \cite{CExp}. The aim of this chapter is to give various 
aspects of a same object, namely an implication basis. We first go through 
logical definitions, then we move to set theoretic aspect (with closure) 


% BD, Implicational Basis and stuff
\section{Introduction to implications through closure systems}

In this section, we will establish basic definitions and provide some examples
to have a sufficient (and strong enough) background to understand the topic of
Horn minimization. We assume some knowledge of set theory. For more complete
and detailed introduction to our topic, the reader may refer to 
\cite{b._ganter_conceptual_2016, davey_introduction_2002}. Some of the most 
practical applications of the following material 
dwell into artificial intelligence, with \belemp{Formal Concept Analysis} and
\belemp{Attribute Exploration}. We can find other applications within database
field (see \cite{maier_theory_1983}). For now, let us begin with an example to 
illustrate implications.

\vspace{1.2em}

\paragraph{Example} Let us imagine we are provided with a set of music 
genres: \textit{shoegaze, electronic, coldwave, pop, rock, dream-pop,
jazz, experimental, atmospheric, contemporary jazz}
 Assume we can attach various genre to a given music. Our aim is to 
draw inference of styles with other ones. In other words, say we have a music 
with tags \textit{rock, pop}, can we deduce other tag? Remind this example aims 
to 
illustrate, not to settle any musical knowledge. Let's try to give some ideas:
\begin{itemize}
	\item a music being \textit{jazz, experimental} can also be categorized
	as \textit{contemporary jazz},
	\item a music called \textit{coldwave, rock} is likely to be tagged 
	\textit{shoegaze},
	\item song with \textit{atmospheric} will probably be said to be 
	\textit{experimental, electronic},
	\item a last one, \textit{shoegaze, pop} will lead to \textit{dream-pop}.
\end{itemize}
\noindent Here we drew what we will call \belemp{implications} because we can
summarize our sentences by \textit{if a tag is present, then this one is too},
and denote them:
\begin{itemize}
	\item \textit{jazz, experimental} $\imp$ \textit{contemporary jazz},
	\item \textit{coldwave, rock}  $\imp$ \textit{shoegaze},
	\item \textit{atmospheric} $\imp$ \textit{experimental, electronic},
	\item \textit{shoegaze, pop} $\imp$ \textit{dream-pop}.
\end{itemize}
\noindent In a sense, those implications describe some possible knowledge of 
our genre set or \belemp{attributes} set.

\vspace{1.2em}

From our point of view, this example is sufficient to say that an 
\belemp{attribute set} is simply a set. We call its elements 
\belemp{attributes} to stick with the literature terminology. If not specified
in the context, we will call $\Sg$ such a set, and $a, \ b, \ c,\dots$ its 
elements. Subsets of $\Sg$ we will be denoted by capital letters $A, \ B, 
\dots$. With the example and $\Sg$, we can go over some more definitions

\begin{definition}[\midn{Implication basis}] Let $\Sg$ be an attribute set. An 
\belemp{implication basis} $\I$ over $\Sg$ is a set of \belemp{implications} 
where implications are pairs ($A$, $B$), denoted $A \imp B$, with $A, B 
\subseteq \Sg$.
	
\end{definition}

\noindent Usually, given an implication $A \imp B$, $A$ is called \belemp{body}
or \belemp{premise} while $B$ is called \belemp{head} or \belemp{consequence}.

\paragraph{Examples} Let $\Sg = \{a, \ b, \ c, \ d, \ e \}$. Some 
possible implication basis are (for the sake of readability, a subset of 
$\Sg$ we will be written as a concatenation of its elements):
\begin{itemize}
	\item $\{ab \imp de, \ a \imp c, ce \imp b \}$
	\item $\emptyset$
	\item $ \{ abc \imp ab, \ d \imp abcde \}$
\end{itemize}

Back to the musical example, the implication basis we described is of course 
$\{$ \textit{jazz, experimental} $\imp$ \textit{contemporary 
jazz}, \textit{coldwave, rock}  $\imp$ \textit{shoegaze}, \textit{atmospheric} 
$\imp$ \textit{experimental, electronic}, \textit{shoegaze, pop} $\imp$ 
\textit{dream-pop} $\}$.

\vspace{1.2em}

Each implication basis describe (possibly different!) knowledge from $\Sg$. Say
we have $\I$ over $\Sg$, at least two questions arise:
\begin{itemize}
	\item[(i)] is $A \subseteq \Sg$ \textit{"coherent"} with respect to $\I$?  
	\item[(ii)] What can we deduce from $A \subseteq \Sg$?	
\end{itemize}

\paragraph{Example} Consider again musical example. The first question would be 
"given our basis, can we imagine having a music with some tags only ?". For 
instance, consider a music with tags \textit{shoegaze, pop}. In our basis, we 
cannot have a music with these tags \textit{only} because the implication 
\textit{shoegaze, pop} $\imp$ \textit{dream-pop} states that if a music has 
\textit{shoegaze, pop} tags, it \textit{must} also have \textit{dream-pop}. On 
the other hand, taking only the tag \textit{shoegaze} is possible. 

The second question would be "given a set of tags, which tags will we obtain
in order to stay coherent with our implications?". Take \textit{coldwave, pop, 
rock}. Using our implications we will reach two other attributes: 
\textit{shoegaze, dream-pop}.

\vspace{1.2em}

In this last paragraph we described some of the \textit{most important} basic 
notions of our problem: \belemp{model} and \belemp{closure}.

\begin{definition}[\midn{Model of an implication}] Let $\Sg$ be an attribute 
set, and
$A \imp B$ an implication over $\Sg$. A subset $M \subseteq \Sg$ is a 
\belemp{model} of $A \imp B$, written $M \models A \imp B$ if $A \nsubseteq M$ 
or $B \subseteq M$.

\end{definition}

\begin{definition}[\midn{Model of an implication basis}] Let $\I$ be a basis 
over 
$\Sg$. A subset $M \subseteq \Sg$ is a \belemp{model} of $\I$, $M \models \I$, 
if $M \models A \imp B$ for each $A \imp B \in \I$.

\end{definition}

In other words, a subset $M$ of $\Sg$ will be a model of an implication if when 
it contains the body it contains also the head, or the body is not in $M$. An 
implication $A \in B$ \belemp{follows} from a basis $\I$, denoted $\I \models A 
\imp B$ if all models of $\I$ are models of $A \imp B$. In the example of 
musics we talked about, \textit{shoegaze} is a model when \textit{shoegaze, 
pop} is not. For completeness, we will define closure operators and closure 
systems in general before applying it to our context.


\begin{definition}[\midn{Closure operator, closure system}] Let
$\Sigma$ be a set, and define $\phi : \Sigma \longrightarrow \Sigma$ an
application. $\phi$ is a \belemp{closure operator} if it has the three following
properties for all $X, Y \subseteq \Sigma$:
\begin{itemize}
	\item[(i)] $X \subseteq \phi(X)$ (\midemp{extensive})
	\item[(ii)] $X \subseteq Y \longrightarrow \phi(X) \subseteq \phi(Y)$ 
		(\midemp{monotone})
	\item[(iii)] $\phi(\phi(X)) = \phi(X)$ (\midemp{idempotent})
\end{itemize}

\noindent The pair $(\Sigma, \phi)$ is called a \belemp{closure system}.
	
\end{definition}


\begin{definition}[\midn{Closed set}] Let $(\Sigma, \phi)$ be a closure 
system. A subset $X$ of $\Sigma$ is a \belemp{closed set} (with respect to 
$\phi$) if $\phi(X) = X$. We will denote by $\Sigma_{\phi}$ the set of all 
closed sets of $(\Sigma, \phi)$, that is:
	
	\[ \Sigma_{\phi} = \left\{ X \subseteq \Sigma \; | \; \phi(X) = X 
	\right\}  \]

\noindent and $\Sigma_{\phi}$ has the following properties:
\begin{itemize}
	\item[(i)] $\Sigma \in \Sigma_{\phi}$,
	\item[(ii)] if $X, Y \in \Sigma_{\phi}$, so does $X \cap Y$ 
		(\midemp{$\Sigma_{\phi}$ is closed under intersection}).
\end{itemize}
	
\end{definition}

\noindent Note that a closure system can be characterized either by its closure
operator, or by its set of closed sets. In other words, we can derive 
$\Sigma_{\phi}$ from $\phi$, as $\phi$ from $\Sigma_{\phi}$. The closed set 
associated to $X$ is the smallest closed set containing $X$, i.e:

	\[ \phi(X) = \bigcap \{Y \in \Sigma_\phi \; | \; X \subseteq Y \} \]

\noindent Since $\Sigma_{\phi}$ is closed under intersection, the resultant of
the intersection is also a closed set.


\paragraph{Example} Let $\Sigma = \llbracket 1 \; ; \; 4 \rrbracket$ and 
$\phi(X) = X \cup \{4 \}$. The pair $(\Sigma, \phi)$ is a closure system whose
closed sets are all the subsets containing 4. Another interesting definition of 
$\Sigma_{\phi}$ is:

	\[ \Sigma_{\phi} = \{ \phi(X) \; | \; X \subseteq \Sigma \} \]

\noindent Then, in our case:

	\[ \Sigma_{\phi} = \{ \{ 4\}, \{ 1, 4\}, \{ 2, 4\},
		\{ 3, 4\}, \{ 1, 2, 4\}, \{ 1, 3, 4\},\{ 2, 3, 4\}, 
		\{ 1, 2, 3, 4\} \}
	\]

\vspace{1.2em}

\noindent Now we have defined closure systems, let's get back to our 
implicational context. Let $\I$ be a basis over an attribute set $\Sg$. Let $
M \subseteq \Sg$ and define $\I(M)$ as the smallest model of $\I$ containing 
$M$. In this sens, $\I$ defines a \belemp{closure system} over $\Sg$ for 
which the closed sets are exactly the models of $\I$. The point is: how to 
define $\I(M)$ by computations? We rely on \cite{b._ganter_conceptual_2016} for 
this purpose. Let 

	\[ M^{\I} = M \cup \{ B \ | \ A \imp B \in \I, \ A \subset M \land
			B \nsubseteq M \} 
	\]

\noindent Then $\I(M)$ is obtained by repeated application of the operator 
$M^{\I}$, that is $\I(M) = M^{\I \I \dots \I}$ until the computed $M$ is 
unchanged. For readers with background in SAT-solving or graph theory, this is
equivalent to \textit{marking procedure, forward chaining}. In words, if we can 
find an implication $A \imp B$
with the body included in $M$, but not the head $B$, we add $B$ to $M$. Let us 
give an 
example to set things clear.

\paragraph{Example} As usual, let's stick to our musical example. For the 
recall, our set $\Sg$ is: \textit{shoegaze, electronic, coldwave, pop, rock, 
dream-pop, jazz, experimental, atmospheric, contemporary jazz} and the basis
we are working on $\I$ being:
\begin{itemize}
	\item[ ] \textit{jazz, experimental} $\imp$ \textit{contemporary jazz},
	\item[ ] \textit{coldwave, rock}  $\imp$ \textit{shoegaze},
	\item[ ] \textit{atmospheric} $\imp$ \textit{experimental, electronic},
	\item[ ] \textit{shoegaze, pop} $\imp$ \textit{dream-pop}.
\end{itemize}

\noindent In previous examples, we were talking about two subsets of $\Sg$:
\begin{itemize}
	\item $ A = $\textit{coldwave, pop, rock}
	\item $B = $\textit{shoegaze}
\end{itemize}

\noindent Let us try to compute their closure with respect to $\I$. $A$ is not
a model of $\I$ because of the implication \textit{coldwave, rock}  $\imp$ 
\textit{shoegaze}.  Indeed \textit{coldwave, rock} is included in $A$ but not
\textit{shoegaze} so we add it, thus $A = $\textit{coldwave, pop, 
rock, shoegaze}. The new $A$ is still not a model of $\I$, see the implication
\textit{shoegaze, pop} $\imp$ \textit{dream-pop}. We must also add 
\textit{dream-pop}. Finally, $\I(A) = $\textit{coldwave, pop, rock, 
shoegaze, dream-pop} will be the smallest model (closed set) containing $A$.

On the other hand, $B$ does respect all the implications so that it is already 
a model of $\I$, hence $\I(B) = B$. One interesting point about closure of a 
subset with respect to an implicational basis can be computed in linear time
(in the size of the basis). Of course there exists various algorithms for
computing the closure, but since it is not the object of our study we give
only the procedure we will use. To have more details on other algorithms
to compute the closure of a set with relation to an implicational basis, the
reader may refer to \cite{bazhanov_optimizations_2014, 
b._ganter_conceptual_2016}.

\begin{algorithm}
\KwIn{A basis $\I$, $X \subseteq \Sg$}
\KwOut{The closure $\I(X)$ of $X$ in $\I$}
	
\BlankLine
\BlankLine

\ForEach{$A \imp B \in \I$}{
	$count[A \imp B] := |A|$ \;
	
	\If{$|A| = 0$}{
		$X := X \cup B$ \;
	}

	\ForEach{$a \in A$}{
		$list[a] += A \imp B$ \;
	}
}

$update := X$ \;

\While{$update \neq \emptyset$}{
	choose $m \in update$ \;
	$update := update \ \{m\}$ \;
	
	\ForEach{$A \imp B \in list[m]$}{
		$count[A \imp B] -= 1$ \;
		\If{$count[A \imp B] = 0$}{
			$add := B \ X$ \;
			$X := X \cup add$ \;
			$update := update \cup add$ \;
		}
		
	}
}

return $X(\I)$ \;

	
\caption{LinClosure}
\label{alg:linclosure}
\end{algorithm}

\vspace{1.2em}

\begin{definition}[\midn{Closure of an implicational basis}] Given $\I$, the
closure $\I^+$ if $\I$ is the of all implications holding in $\I$.
	
\end{definition}

\begin{definition}[\midn{Equivalence of implicational basis}] Two implicational
basis are equivalent if they have the same closure.
	
\end{definition}

\noindent Also, two basis are equivalent if they have the same models. The 
question would be: how to determine the closure of an implicational basis? For
this purpose, one could use \belemp{Armstrong rules} (see 
\cite{b._ganter_conceptual_2016, maier_theory_1983}).

\vspace{1.2em}

Before concluding this section, we would like to introduce an useful result as
much as some remarks on the use of closure systems and implications. First 
a proposition (admitted here):

\begin{proposition} \label{prop:def.equiv_imp_clos} 
Let $\I$ be an implication basis over an attribute set $\Sg$. An implication $A 
\imp B$ follows from $\I$ if and only if $B \subseteq \I(A)$.
\end{proposition}

\begin{proof} \textit{if part}. Suppose $B \subseteq \I(A)$. $\I(A)$ is the
smallest model of $\I$ containing $A$. Therefore, for all models $M$ of $\I$
such that $A \subseteq M$ we also have $\I(A) \subseteq \I(M) = M$ (because
closed sets are models) and $B \subseteq M$ by extension. Therefore for all
models $M$ of $\I$, $A \nsubseteq M$ or $B \subseteq M$ holds. That is 
$\I \models A \imp B$.

\vspace{1.2em}

\textit{only if part}. Suppose $\I \models A \imp B$. By definition, all models
$M$ of $\I$ satisfy $A \nsubseteq M$ or $B \subseteq M$. In particular, it must
hold for the smallest model (inclusion wise) containing $A$ being $\I(A)$ which
yields $A \nsubseteq \I(A) \lor B \subseteq \I(A)$. Because this formula holds 
also for $\I(A)$ (as it is a model) and A $\nsubseteq \I(A)$ being a 
contradiction with respect to the definition of a closure operator, we conclude 
that necessarily $B \subseteq \I(A)$ must be true.

\end{proof}


\vspace{1.2em}

To conclude this short introduction on closure and implications terminology, we 
would like to say that these objects model an intuitive way of representing
knowledge and inference (see music example). More than this, they are mainly 
used because they can be checked easily, namely in linear time in the size of
our implication basis (see algorithm linclosure,  
\cite{b._ganter_conceptual_2016, david_minimum_1980,maier_theory_1983}).

	In this section we went over general definitions of closure systems, and 
implication basis. The reader may find a more exhaustive presentation in 
\cite{b._ganter_conceptual_2016, davey_introduction_2002}. Note that these are 
not all the definitions from closure terminology we shall use. Nevertheless, we 
prefer to give them when required as they are more closely related to specific 
problems, while the aim of this section is to be general. The next part is 
dedicated to a review of propositional logic in accordance to our needs.

% Personal: find the algorithm of bernstein. 











% Logical POV
\section{Logic pov}

Note that we consider the reader to have a few background in mathematical logic,
so that the following definitions just stand as a summary and refresher. The 
reminder is only about propositional logic up to our needs.

\begin{definition}[\belemp{clause}]
	Pouet et pouf sont sur un bateau.
\end{definition}

\begin{definition}[\belemp{Horn formula}]
	
\end{definition}

\begin{definition}[\belemp{semantic entailment}]
	
\end{definition}


% (Hyper)-graphs POV






% Chapter II: A review on algorithms for horn minimization
\chapter{Algorithms}

In the first chapter we introduced the aim of our study, our working procedure
and the main tools for us to understand the next explanations. In this part, we
will focus on reviewing the existing algorithms for the Horn minimization task. 


% History / soa of the problem.
% Start from explaining the applications in real life, come to implications, 
% explain the various existing problems, come to ours, explain that it is 
% polynomial, develop the sens of various algorithms in various terminologies.
% Explain (again) that our task is to compare those algorithms in both
% theoretical and practical aspects.


% Maier Algorithm
\subsection{Maier Algorithm: using equivalence classes}


This part relies mainly on \cite{david_minimum_1980}, \cite{maier_theory_1983} 
in which an algorithm for minimizing set of functional dependencies (FD). 
Because FDs behave as implications in implicational basis, we will adopt the 
latter terminology.

\subsubsection{Theoretical set up and definitions}

We are going to work in the FD framework, that is first notations we defined. 
Before diving into the algorithm, we may need some definitions. Indeed, the 
minimization procedure in this case relies on a criterion for minimality using
specific objects. 

\begin{definition}[\midn{Equivalence classes}] Let $X \subseteq \Sigma$. The set
	
	\[ E_{\I}(X) = \{ A \imp B | A \imp B \in \I, A \equiv X \} \]
	
\noindent is the \belemp{equivalence class} of $X$ under $\I$. In fact, it 
is the set of implications of $\I$ with body equivalent to $X$. The set of all 
non-empty such classes is denoted $\bar{E}_{\I}$.
	
\end{definition}

One should note in order to get $\bar{E}_{\I}$, there is no need to check all
the subsets of $\Sigma$. Because $\bar{E}_{\I}$ defines a partition of $\I$ 
based on implication bodies, it is sufficient to go over implications of $\I$.

\vspace{1.2em}

The second tool used by Maier in his algorithm is \belemp{direct determination}.
This notion is extensively discussed in \cite{david_minimum_1980, 
maier_theory_1983}. Thus, the definition we are going to define is more the 
result of a very useful necessary and sufficient condition than the definition
as given at the beginning. Our aim is not to (re)build all the material, but to 
explain the algorithm.

\begin{definition}[\midn{Direct Determination}] Given a basis $\I$, we say that
$A$ \belemp{directly determines} $B$ under $\I$, denoted $A \ddv B$, if $\I -
\bar{E}_{\I} \models A \imp B$. 
	
\end{definition}

More intuitively, we say that $A$ directly determines $B$ if we can reach $B$
without using any implications with body equivalent to $A$ (including $A$). 
Those definitions are sufficient to understand the algorithm. 

\subsubsection{Algorithm}

In this section we will investigate the Maier Algorithm to minimize a basis 
$\I$. The principle is short enough to describe it as follows:
\begin{enumerate}
	\item remove redundant implications
	\item remove direct determinations in this sens: if $A \ddv B$ with $A \imp 
	C$ and $B \imp D$ implications of $\I$, remove $A \imp C$ and replace by 
	$B \imp D$ by $B \imp D \cup C$. 
\end{enumerate}
\noindent Of course, proof of correctness has been given in cite 
\cite{david_minimum_1980, maier_theory_1983}. Nevertheless, we shall give later
some different elements for proving the algorithm ends with the right result. 
We do not give them here, because those elements also provide a proof for an
another algorithm (namely Ausiello and al. algorithm) through equivalence 
properties. For now, we send the reader to cited papers. The procedure is 
presented in algorithm \ref{alg:Maier-Algorithm}.


\begin{algorithm}
\KwIn{}
\KwOut{}

\BlankLine
\BlankLine

\emeemp{// redundancy elimination } \\
\ForEach{$A \imp B \in \I$}{
	$\I^{-} := \I - \{ A \imp B \}$ \;
	\If{$\I^{-}(A) = \I(A)$}{
		$\I := \I - \{ A \imp B \}$ \;
	}
}

$\bar{E}_{\I} := EquivClasses(\I)$ \;

\BlankLine

\emeemp{// direct determination elimination} \\

\ForEach{$E_{\I} \in \bar{E}_{\I}$}{
	\ForEach{$A \imp C \in E_{\I}$}{
		\If{$\exists B \imp D \in E_{\I}$ such that $A \ddv B$}{
			remove $A \imp B$ from $\I$ \;
			replace $B \imp D$ by $B \imp C \cup D$ \;	
		}
	}
}

\caption{Maier minimization algorithm}
\label{alg:Maier-Algorithm}
\end{algorithm}

\vspace{1.2em}

\paragraph{Observations} There are some observations and questions to answer
about procedure \ref{alg:Maier-Algorithm}:
\begin{itemize}
	\item How do we compute $\bar{E}_{\I}$?
	\item How do we check for direct determination?
	\item Does the operation of remove and replace alter an equivalence class 
	more than removing one of its implications?
\end{itemize}  
\noindent Let us take those questions in order. To find $\bar{E}_{\I}$, we 
will consider applying a variation of \textsc{LinClosure} to each implication
of $\I$. Thus for each $A \imp B$, instead of returning closure of $A$ under
$\I$, we will a bit vector of size $|\body{\I}|$ with the $i$-th entry being
$1$ if the body of this implication is in the closure of $A$, and $0$ otherwise.
the $i$-th bit of the vector is set if \textit{count}[$i$-th implication] = 0 in
the procedure. Running this modified version of \textsc{LinClosure} over all
implications produces a $|\body{\I}| \times |\body{\I}|$ matrix such that
the entry [$A \imp C, B \imp D$] is 1 if $\I \models A \imp B$. Consequently,
finding equivalence classes shall take no more than a run over this matrix to 
be done (after having computed closures).

\vspace{1.2em}

Direct determination can be considered as well with a modification of 
\textsc{LinClosure} on $\I - E_{\I}(A)$. In this version, for a given $A \imp 
C$, we also have to be provided with $E_{\I}(A)$. Then, direct determination 
will be exposed the first time we reach \textit{count}[$B \imp D$] = 0 in the 
computation, for $B \imp D$ different from $A \imp C$ and belonging to 
$E_{\I}(A)$. Because we stop to the first time this will ensure that we did not 
reach any other body of this equivalence class, otherwise we would have stopped 
before. Because equivalence classes partition the implications of $\I$ we will 
proceed to this operation at most once for each implication.

\vspace{1.2em}

Finally, we consider the question of whether equivalence classes are altered 
more than removal during computations. Of course we remove direct determination
is found. Does replacing $B \imp D$ by $B \imp C \cup D$ has any chances to 
add other implication to $E_{\I}(A)$? Fortunately the answer is no. Indeed, 
before removing $A \imp C$ from $E_{\I}(A)$, anything we could reach from $A$
could be reached from $B$ (because of their equivalence) and thus would have
implied $C$ from $A$ and $D$ from $B$. Thus anything reachable from $C \cup D$
would already be in $E_{\I}(A)$.

 
 
\vspace{1.2em}


\paragraph{Complexity} Again, the complexity of this algorithm has been studied
and discussed in \cite{david_minimum_1980, maier_theory_1983}. It has been 
challenged (somewhat) in \cite{ausiello_graph_1983, ausiello_minimal_1986}. We
provide a complexity analysis anyway to adapt previous material to our notations
(and ease comparison with other algorithms) and to have this report to be 
self-contained as much as possible. Let us recall few elements to analyse
complexity:
\begin{itemize}
	\item $\body{\I}$ is the number of distinct bodies in $\I$ (therefore the
	total number of implications for us), and $|\body{\I}|$ its size,
	\item $|\Sg|$ is the number of attributes in the universe $\Sg$,
	\item $|\I|$ is the size of $\I$ in memory, that is roughly $|\body{\I}| 
	\times |\Sg|$.
\end{itemize}
\noindent Let us proceed by steps. First, non-redundancy elimination. Because 
the inner \textbf{foreach} loop goes for all implications of $\I$ requiring 
$O(|\body{\I}|)$ times looping. Then, computing closure of $A$ under $\I$ and 
$\I^{-}$ requires $O(|\I|)$ operations thanks to \textsc{LinClosure}. Hence,
redundancy elimination can be done in $O(|\body{\I}| \times |\I|) = 
O(|\body{\I}|^2 \times |\Sg|)$ time. Secondly, finding equivalence classes as
mentioned previously can be done by using modified version of 
\textsc{LinClosure} performing same time as the original one. Since we have to
go over (at most) all implications to find all implications with bodies implied
by a given one, this result in $O(|\body{\I}| \times |\I|)$ complexity. Then,
having the $|\body{\I}| \times |\body{\I}|$ matrix, building equivalence classes
need a run over the matrix, that is $O(|\body{\I}|^2)$ operations. Finally,
removing direct determination can be done at most for all implications, and 
checking for this property is again made through \textsc{LinClosure}. Thus 
performing the last step of the algorithm is $O(|\body{\I}| \times |\I|) = 
O(|\body{\I}|^2 \times |\Sg|)$ time. As a conclusion, the whole algorithm works
in $O(|\body{\I}| \times |\I|)$.



% Ausiello Graph algorithm (FD-Graph)
\subsection{Ausiello Algorithm: minimality through directed graphs}

This section is dedicated to a minimization algorithm relying on hypergraphs. 
It has been set up by Ausiello and al. in \cite{ausiello_directed_2017, 
ausiello_graph_1983, ausiello_minimal_1986}. Starting from a directed hypergraph
representation, it builds a special kind of DAG, called \belemp{FD-Graph} with
which it reduces the initial hypergraph. In order, we are going to define what
is a FD-graph, provide the general idea for the algorithm as explained in 
\cite{ausiello_minimal_1986} and then go into further details and more precise
algorithms for such computations as exposed in \cite{ausiello_graph_1983}.

\subsubsection{FD-Graphs and minimum covers}

In this part, we assume that our basis $\I$ is represented through the framework
of hypergraphs (see chapter 1). Again, all the definitions we are going to state
here are exposed in \cite{ausiello_graph_1983, ausiello_minimal_1986}. First,
let us define the central object we will work on, FD-Graphs.



\begin{definition}[\midn{FD-Graph}] Given a hypergraph $\I = (\Sg, C)$ 
representing an implication basis, the directed graph $G_{\I} = (V, \  E)$ such 
that:
\begin{itemize}
	\item $V = V_0 \cup V_1$ is the set of nodes where:
		\begin{itemize}
			\item $V_0 = \Sg$ is the set of \belemp{simple} nodes (a node
			per attribute in $\Sg$),
			\item $V_1 = \{X | X \in \body{\I} \}$ is the set of 
			\belemp{compound} nodes (a node per distinct body in $\I$),
		\end{itemize}

	\item $E = E_0 \cup E_1$ is the set of arcs where:
		\begin{itemize}
			\item $E_0$ is the set of \belemp{full} arcs. We have a full arc 
			$(X, i)$ in	$E_0$ if $(X, i)$ is an hyperarc of $\I$,
			\item $E_1$ the set of \belemp{dotted} arcs. For each compound node 
			$X$ of $V^1$, we have a dotted arc $(X, \ i)$ to every attributes 
			$i$ of $X$,
		\end{itemize}

\end{itemize}
\noindent is the \belemp{Functionnal Dependency Graph} or \belemp{FD-Graph} 
associated to $\I$.
\end{definition}

In fact, hypergraph representation is not mandatory to represent a FD-Graph, 
since as we saw, hypergraphs in our context rely on "usual" implicational 
basis. To enlighten the definition of FD-Graph, simple examples are shown in 
\ref{fig:FD-Graph-1}.

\begin{center}
	\begin{figure}[ht]\centering
\subfloat[][FD-Graph of $1 \imp 23$]{
\begin{tikzpicture}
\node[Vertex, label=left:{1}] (1) at (-0.5, 0) {};
\node[Vertex, label=right:{2}] (2) at (0.5, 0.5) {};
\node[Vertex, label=right:{3}] (3) at (0.5, -0.5) {};

\draw[->] (1) -- (2);
\draw[->] (1) -- (3);
\end{tikzpicture}
}\qquad
\subfloat[][FD-Graph of $23 \imp 1$]{
\begin{tikzpicture}
\node[Vertex, label=below:{23}] (23) at (0, 0) {};
\node[Vertex, label=right:{1}] (1) at (1, 0) {};
\node[Vertex, label=left:{2}] (2) at (-1, 0.5) {};
\node[Vertex, label=left:{3}] (3) at (-1, -0.5) {};

\draw[->, dotted] (23) -- (2);
\draw[->, dotted] (23) -- (3);
\draw[->] (23) -- (1);
\end{tikzpicture}		
}

\subfloat[][FD-Graph of $12 \imp 34$]{
\begin{tikzpicture}
\node[Vertex, label=below:{12}] (12) at (0, 0) {};
\node[Vertex, label=left:{1}] (1) at (-1, 0.5) {};
\node[Vertex, label=left:{2}] (2) at (-1, -0.5) {};
\node[Vertex, label=right:{3}] (3) at (1, -0.5) {};
\node[Vertex, label=right:{4}] (4) at (1, 0.5) {};

\draw[->, dotted] (12) -- (2);
\draw[->, dotted] (12) -- (1);
\draw[->] (12) -- (3);
\draw[->] (12) -- (4);
\end{tikzpicture}	
}

\caption{Representation of some FD-graph}
\label{fig:FD-Graph-1}
\end{figure}
\end{center}

We can see in those example that drawing an FD-Graph goes this way:
\begin{itemize}
	\item For each $A \imp B$ of $\I$ we draw a \belemp{full} arc from the node
	$A$ to every attribute of $B$,
	\item For each compound node $A$, we draw a \belemp{dotted} arc from
	$A$ to all of its attribute.
\end{itemize}
\noindent which is indeed what we formally defined previously. Furthermore, for
this algorithm, we consider a basis $\I$ over an attribute set $\Sg$, such that:
\begin{itemize}
	\item there is no $A \imp B$, $A' \imp B'$ in $\I$ such that $A = A'$ when
	$B \neq B'$,
	\item for all $A \imp B$ of $\I$, $A \cap B = \emptyset$
\end{itemize}

\vspace{1.2em}

An important notion we may need for all the following is \belemp{FD-paths}.

\begin{definition}[\midn{FD-Path}] Given an FD-Graph $G_{\I} = (V, E)$, an
\belemp{FD-Path} $\langle i, \ j \rangle$ is a minimal subgraph $\bar{G}_{\I} = 
(\bar{V}, \bar{E})$ of $G_{\I}$ such that $i, j \in \bar{V}$ and either $(i, j)
\in \bar{E}$ or one of the following holds:

\begin{itemize}
	\item $j$ is a simple node and there exists $k \in \bar{V}$ such that $(k, 
	j) \in \bar{E}$ and there exists a FD-Path $\langle i, \ k \rangle$ included
	in $\bar{G}$,  
	\item $j = \bigcup_{k = 1}^n j_k$ is a compound node and there exists 
	FD-paths $\langle i, \ j_k \rangle$ included in $\bar{G}$, for all $k = 1, 
	\ \dots, \ n$.
\end{itemize}
	
\end{definition}

\noindent As is, the definition may not seems clear. Informally, a FD-path 
from a node $i$ to $j$ describes the implications we use to derive $i \imp j$
(with Armstrong rules, especially transitivity and union). Intuitively, 
directed paths are FD-paths. But there is also one case in which
we can go "backward" in the graph. For better understanding, see examples of
figure \ref{fig:FD-Graph-3} based on the basis described in figure 
\ref{fig:FD-Graph-2}.

\begin{center}
	\centering
\begin{tikzpicture}
\node[Vertex, label=left:{ab}] (ab) at (-2, 0) {};
\node[Vertex, label=left:{f}] (f) at (-2, 2) {};
\node[Vertex, label=right:{af}] (af) at (-1, 3) {};
\node[Vertex, label=above:{g}] (g) at (-1, 4) {};
\node[Vertex, label=below:{a}] (a) at (-1, 1) {};
\node[Vertex, label=below:{b}] (b) at (-1, -1) {};
\node[Vertex, label=below:{c}] (c) at (0, 1) {};
\node[Vertex, label=below:{d}] (d) at (0, -1) {};
\node[Vertex, label=above:{h}] (h) at (1, 2) {};
\node[Vertex, label=left:{cd}] (cd) at (1, 0) {};
\node[Vertex, label=below:{e}] (e) at (2, 0) {};

\draw[->, dotted] (af) -- (f);
\draw[->, dotted] (af) -- (a);
\draw[->, dotted] (ab) -- (a);
\draw[->, dotted] (ab) -- (b);
\draw[->, dotted] (cd) -- (c);
\draw[->, dotted] (cd) -- (d);
\draw[->] (af) -- (g);
\draw[->] (ab) -- (f);
\draw[->] (a) -- (c);
\draw[->] (b) -- (d);
\draw[->] (c) -- (h);
\draw[->] (cd) -- (e);
\end{tikzpicture}

\caption{FD-Graph of some implicational basis}
\label{fig:FD-Graph-2}

\end{center}

There are either \belemp{dotted} or \belemp{full} paths. A path $\langle i, j
\rangle$ is dotted if all arcs leaving $i$ are dotted, it is full otherwise.

\begin{center}
	\begin{figure}[ht]\centering
\subfloat[][dotted FD-path for $ab \imp e$]{
\begin{tikzpicture}
\node[Vertex, label=left:{ab}] (ab) at (-2, 0) {};
\node[Vertex, label=below:{a}] (a) at (-1, 1) {};
\node[Vertex, label=below:{b}] (b) at (-1, -1) {};
\node[Vertex, label=below:{c}] (c) at (0, 1) {};
\node[Vertex, label=below:{d}] (d) at (0, -1) {};
\node[Vertex, label=below:{cd}] (cd) at (1, 0) {};
\node[Vertex, label=right:{e}] (e) at (2, 0) {};

\draw[->, dotted] (ab) -- (a);
\draw[->, dotted] (ab) -- (b);
\draw[->, dotted] (cd) -- (c);
\draw[->, dotted] (cd) -- (d);
\draw[->] (a) -- (c);
\draw[->] (b) -- (d);
\draw[->] (cd) -- (e);

\end{tikzpicture}
}\qquad
\subfloat[][full FD-path for $ab \imp g$]{
\begin{tikzpicture}
\node[Vertex, label=below:{ab}] (ab) at (0, 0) {};
\node[Vertex, label=right:{a}] (a) at (1, 1) {};
\node[Vertex, label=left:{f}] (f) at (0, 2) {};
\node[Vertex, label=right:{af}] (af) at (1, 3) {};
\node[Vertex, label=right:{g}] (g) at (1, 4) {};

\draw[->, dotted] (ab) -- (a);
\draw[->, dotted] (af) -- (a);
\draw[->, dotted] (af) -- (f);
\draw[->] (ab) -- (f);
\draw[->] (af) -- (g);
\end{tikzpicture}		
}

\caption{Representation of some FD-paths}
\label{fig:FD-Graph-3}
\end{figure}
\end{center}

\vspace{1.2em}

Having explained FD-Graphs, we will now move to explanations of the algorithm
developed by Ausiello and al.

\subsubsection{Ausiello algorithm: general point of view}

The following procedure (\ref{alg:ausiello_86}) finds from a given basis its
\belemp{minimal cover}. 

\begin{algorithm}[H]
\KwIn{$\I$ an implication basis}
\KwOut{$\I_c$ a minimal cover for $\I$}

\BlankLine
\BlankLine

Find the \belemp{FD-Graph} of $\I$ \;
Remove \belemp{redundant} nodes \;
Remove \belemp{superfluous} nodes \;
Remove \belemp{redundant} arc \;
Derive $\I_c$ from the new graph \;

\caption{Ausiello algorithm (1986)}
\label{alg:ausiello_86}
\end{algorithm}

\vspace{1.2em}

Let us emphasize on the meaning of those removal steps.
The algorithm studied in the sources, aims to minimize an implicational basis 
in terms of bodies. As with the Duquenne-Guigues basis. It uses 3 steps. Two
to remove redundancy, the third one aims to lighten bodies and heads of 
remaining implications. We will try to express each of these steps in terms of
sets, closure, and so forth. 

\subsubsection{Removing redundant nodes}

The first step is about removing redundant implications (without right
maximization). In terms of FD-graphs, we remove \belemp{redundant} nodes. A
compound node (only) $i$ is said redundant if for each full arc $ij$ leaving $i$
there exists a dotted path $(i, j)$. We give an example in the figure
\ref{fig:FD-Graph-4}.

\begin{figure}[ht]\centering
\subfloat[][FD-Graph with redundant node ($ab$)]{
\begin{tikzpicture}
\node[Vertex, label=left:{ab}] (ab) at (0, 0) {};
\node[Vertex, label=below:{a}] (a) at (-1, 1) {};
\node[Vertex, label=below:{b}] (b) at (-1, -1) {};
\node[Vertex, label=below:{c}] (c) at (1, 1) {};
\node[Vertex, label=below:{d}] (d) at (1, -1) {};

\draw[->, dotted] (ab) -- (a);
\draw[->, dotted] (ab) -- (b);
\draw[->] (a) -- (c);
\draw[->] (b) -- (d);
\draw[->] (ab) -- (c);
\draw[->] (ab) -- (d);

\end{tikzpicture}
}\qquad
\subfloat[][FD-Graph with redundant node removed]{
\begin{tikzpicture}
\node[Vertex, label=below:{a}] (a) at (-1, 1) {};
\node[Vertex, label=below:{b}] (b) at (-1, -1) {};
\node[Vertex, label=below:{c}] (c) at (1, 1) {};
\node[Vertex, label=below:{d}] (d) at (1, -1) {};

\draw[->] (a) -- (c);
\draw[->] (b) -- (d);
\end{tikzpicture}		
}

\caption{Elimination of redundant nodes}
\label{fig:FD-Graph-4}
\end{figure}

In this example, the basis associated basis is $\I = { ab \imp cd \, ; \, 
	a \imp c \, ; \, b \imp d}$. Indeed, in this case, $ab \imp cd$ is 
	redundant 
because $\I - {ab \imp cd} \models ab \imp cd$. So removing a redundant node is
removing exactly one implication in $\I$ since $\I$ is reduced. In details, let
$A \imp B$ be an implication of $\I$ with $A = a_1 a_2 \dots a_n$ and $B = b_1
b_2 \dots b_m$. $A \imp B$ will be redundant in terms of FD-Graph if for each 
$b_i$ there exists $X_i \subset A$ such that $X_i \imp b_i$. That is:

\[ \bigcup_i X_i \subseteq A \imp B \]

\noindent Which may be rewritten in terms of $\I$ closure as 

\[ \bigcup_i \I(X_i) = \I(A) \]

\noindent Thinking of opposite direction, $A \imp B$ will be nonredundant if
there exists $b \in B$ such that $((b \in \I(X)) \land (X \subseteq A)) \imp 
(X = A)$.

\vspace{1.2em}

\noindent To sum up: \aliemp{the first step is about considering each $A \imp 
	B$ where $|A| > 1$, and removing it of $\I$ if $\I^{-}(A) = \I(A)$}. Note:
$\I^{-} = \I - {A \imp B}$.  


\subsubsection{Removing superfluous nodes}

Next, we remove from the nonredundant FD-Graph \belemp{superfluous} nodes. A 
node $i$ is \belemp{superfluous} if there is an equivalent node $j$ and a 
dotted path from $i$ to $j$. Two nodes $i, j$ are \belemp{equivalent} if there
are paths $(i, j)$ and $(j, i)$. 

\vspace{1.2em}

In terms of sets and closure, two attribute sets $A, B \subseteq \Sg$ are 
equivalent in $\I$ if $\I \models A \imp B, B \imp A$, that is, if $\I(A) = 
\I(B)$. 

\begin{figure}[ht]\centering
\subfloat[][FD-Graph with superfluous node ($ab$)]{
\begin{tikzpicture}
\node[Vertex, label=below:{ab}] (ab) at (0, 0) {};
\node[Vertex, label=above:{a}] (a) at (1, 1) {};
\node[Vertex, label=above:{b}] (b) at (1, -1) {};
\node[Vertex, label=below:{c}] (c) at (2, 1) {};
\node[Vertex, label=below:{d}] (d) at (2, -1) {};
\node[Vertex, label=right:{cd}] (cd) at (3, 0) {};
\node[Vertex, label=left:{e}] (e) at (-1, 0) {};

\draw[->, dotted] (ab) -- (a);
\draw[->, dotted] (ab) -- (b);
\draw[->, dotted] (cd) -- (c);
\draw[->, dotted] (cd) -- (d);
\draw[->] (a) -- (c);
\draw[->] (b) -- (d);
\draw[->] (cd) -- (a);
\draw[->] (cd) -- (b);
\draw[->] (ab) -- (e);

\end{tikzpicture}
}\qquad
\subfloat[][FD-Graph with superfluous node removed]{
\begin{tikzpicture}
\node[Vertex, label=above:{a}] (a) at (1, 1) {};
\node[Vertex, label=above:{b}] (b) at (1, -1) {};
\node[Vertex, label=below:{c}] (c) at (2, 1) {};
\node[Vertex, label=below:{d}] (d) at (2, -1) {};
\node[Vertex, label=right:{cd}] (cd) at (3, 0) {};
\node[Vertex, label=left:{e}] (e) at (-1, 0) {};

\draw[->, dotted] (cd) -- (c);
\draw[->, dotted] (cd) -- (d);
\draw[->] (a) -- (c);
\draw[->] (b) -- (d);
\draw[->] (cd) -- (a);
\draw[->] (cd) -- (b);
\draw[->] (cd) -- (e);
\end{tikzpicture}		
}

\caption{Elimination of superfluous node}
\label{fig:FD-Graph-5}
\end{figure}

\noindent The algorithm suggests the following:
\begin{itemize}
	\item find a superfluous node $i$, and an equivalent node $j$ with a dotted
	path from $i$ to $j$
	\item for each full arc $ik$, we add a full arc $jk$
	\item then we remove the node $i$ and all of its outgoing arcs from the 
	graph
	\item repeat until no more superfluous nodes exist
\end{itemize}
\noindent An example of this procedure is given in the figure
\ref{fig:FD-Graph-5}. In this example $\I = {ab \imp e \, ; \, a \imp c
	\, ; \, b \imp d \, ; \, cd \imp ab}$. The node $ab$ is superfluous. Since 
	our
basis are reduced, note that removing a superfluous node is removing exactly 
one implication in $\I$. In this case, the resulting $\I$ will be

\[ \I = {a \imp c, b \imp d, cd \imp abe} \] 

\noindent Now we may rewrite this operation in our terms. Let $A \imp B$ and 
for instance $C \imp D$ be part of $\I$ to be general. Then $A$ is superfluous
body if

\[ \I \models A \imp C, C \imp A \land \exists X \subset A \; s.t \;
\I \models X \imp C \]

\noindent In this case, we apply the following operations
\begin{itemize}
	\item $C \imp D$ becomes $C \imp (D \cup B)$
	\item we remove $A \imp B$ from $\I$
\end{itemize}

\noindent We exhibit some arguments to convince ourselves that this is a valid
operation. Let us call temporarily $\I^{-}$ the basis we obtain after the 
previous operations. We would like to show that $\I^{-} \models A \imp B$, i.e
that $\I^{-} \equiv \I$. We removed $A \imp B$ but we still have $\I^{-} \models
X \imp C$ and then $X \imp B$ because $C \imp D \cup B$. That is, $B \subset 
\I^{-}(X)$. Because $X \subset A$, we have $B \subset \I^{-}(A)$ which is what
we wanted. 


\subsubsection{Removing redundant arcs}

Finally, we remove from a minimum nonredundant FD-Graph, \belemp{redundant 
	arcs}:
\begin{itemize}
	\item dotted case: a dotted arc $ij$ is redundant if there is a dotted 
	path $(i, j)$ not using $ij$,
	\item full case: a full arc $ij$ is redundant if there is a dotted/full 
	path $(i, j)$ not using $ij$.
\end{itemize}

\noindent we can think of eliminating redundant arcs as explicit transitivity
elimination. If we remove a full arc in $A \imp B$ then we are minimizing $B$.
If we remove a dotted arc, we are minimizing $A$. We have examples in 
\ref{fig:FD-Graph-6}. 

\begin{figure}[ht]\centering
\subfloat[][FD-Graph with redundant arcs ($abc \imp a$, $f \imp d$)]{
\begin{tikzpicture}
\node[Vertex, label=above:{abc}] (abc) at (0, 1) {};
\node[Vertex, label=left:{a}] (a) at (-1, 0) {};
\node[Vertex, label=right:{b}] (b) at (0, 0) {};
\node[Vertex, label=right:{c}] (c) at (1, 0) {};
\node[Vertex, label=below:{d}] (d) at (0, -1) {};
\node[Vertex, label=below:{e}] (e) at (-1, -2) {};
\node[Vertex, label=below:{f}] (f) at (1, -2) {};

\draw[->, dotted] (abc) -- (a);
\draw[->, dotted] (abc) -- (b);
\draw[->, dotted] (abc) -- (c);
\draw[->] (b) -- (d);
\draw[->] (d) -- (a);
\draw[->] (c) -- (d);
\draw[->] (d) -- (e);
\draw[->] (e) -- (f);
\draw[->] (f) -- (d);

\end{tikzpicture}
}\qquad
\subfloat[][FD-Graph with redundant arcs removed]{
\begin{tikzpicture}
\node[Vertex, label=above:{bc}] (bc) at (0, 1) {};
\node[Vertex, label=left:{a}] (a) at (-1, 0) {};
\node[Vertex, label=right:{b}] (b) at (0, 0) {};
\node[Vertex, label=right:{c}] (c) at (1, 0) {};
\node[Vertex, label=below:{d}] (d) at (0, -1) {};
\node[Vertex, label=below:{e}] (e) at (-1, -2) {};
\node[Vertex, label=below:{f}] (f) at (1, -2) {};

\draw[->, dotted] (bc) -- (b);
\draw[->, dotted] (bc) -- (c);
\draw[->] (b) -- (d);
\draw[->] (d) -- (a);
\draw[->] (d) -- (e);
\draw[->] (e) -- (f);
\end{tikzpicture}		
}

\caption{Elimination of redundant arcs}
\label{fig:FD-Graph-6}
\end{figure}

In terms of sets, consider an implication $A \imp B$. Removing a dotted arc is
saying that given $a \in A$, and substituting $A \imp B$ by $A - a \imp B$ in
$\I$ preserves $\I \models A \imp a_i$. On the other side, removing a full arc
is, given $b \in B$ and replacing by $A \imp B - b$, we preserve $\I \models 
A \imp b$. It appears that this steps goes beyond our scope, because it aims
to minimize implication itself, not the number of implications. So, in our 
minimization context, we can stick to the first two operations to obtain a 
minimal representation in our terms. The algorithm we have to follow becomes
the procedure \ref{alg:ausiello_redux_86}

\vspace{1.2em}

\begin{algorithm}[H]
	\KwIn{$\I$ an implication basis}
	\KwOut{$\I_c$ a minimal cover for $\I$}
	
	\BlankLine
	\BlankLine
	
	Find the \belemp{FD-Graph} of $\I$ \;
	Remove \belemp{redundant} nodes \;
	Remove \belemp{superfluous} nodes \;
	Derive $\I_c$ from the new graph \;
	
	\caption{Ausiello algorithm (1986, reduced)}
	\label{alg:ausiello_redux_86}
\end{algorithm}

\vspace{1.2em}

In fact, this algorithm stands for a high-level principle of what has to be done
to minimize a given basis. The question we are going to investigate in the next
paragraphs is how to do such computations.


\subsubsection{Ausiello algorithm: effective procedure}

In this part we will focus on how the algorithm proposed by Ausiello in 
\cite{ausiello_minimal_1986} performs redundancy and superfluousness 
elimination. Actually, those operations are detailed in 
\cite{ausiello_graph_1983}. But in order to stick to our subject (reviewing 
the existing algorithms) we study it here. Also, note that we will consider
that the mapping step from $\I$ to $G_{\I}$ is already done. Also, as mentioned
in the related articles, going from one representation to another can be done
in linear time.

\vspace{1.2em}

The algorithm relies on closure of a FD-Graph. Of course the idea of closure is
similar to the one we encountered before. Nevertheless, Ausiello and al. 
introduce a notion of priority in the way they determine the closure. Because
we cannot represent to different arcs with same nodes (two arcs $(i, \ j)$), me 
must provide priority between dotted and full arcs. The authors decided to 
stress on dotted arcs since they are more likely to be a criterion for removal 
during minimization.

\paragraph{Closure of a FD-Graph} The closure is based on the following data 
structures:
\begin{itemize}
	\item $V_0$: set of \belemp{simple} nodes,
	\item $V_1$: set of \belemp{compound} nodes,
	\item $D_i$ ($\forall i \in V$): nodes from \belemp{incoming dotted} arcs
		$\{j \in V \ | \ (j, \  i) \text{ is a dotted arc} \}$,
	\item $L_{i}^0$ ($\forall i \in V$): nodes from \belemp{outgoing full} arcs
		$\{j \in V \ | \ (i, \  j) \text{ is a full arc} \}$,
	\item $L_{i}^1$ ($\forall i \in V$): nodes from \belemp{outgoing dotted} 	
		arcs $\{j \in V \ | \ (i, \  j) \text{ is a dotted arc} \}$,
	\item $L_{i}^{0+}, L_{i}^{1+}$ ($\forall i \in V$): the respective closures
		of $L_i^0, L_i^1$,
	\item $q_m$ ($\forall m \in V^1$): counter of nodes in $m$ belonging to 
	$L_i^{0+} \cup L_i^{1+}$ for some $i \in V$.
\end{itemize}

\noindent To make understanding easier, we first give pseudo-code closer from
principle than algorithms. From a general point of view, to determine the 
closure of a FD-graph, we must compute the closure of all its nodes. The 
closure of a node is described by its full and dotted outgoing arcs. Because we
put a priority on dotted possibilities, they will be computed before. Principle
are given in algorithmic/pseudo-code form so that identification between steps
of procedures and ideas of principle are easier to see.

\vspace{1.2em}

First, we introduce the procedure NodeClosure which computes the closure of a
node with respect to a type of arc. In other words, to compute the full closure
of a node, we must first apply NodeClosure to its dotted arcs, then to its full
arcs. The principle and algorithm for Nodeclosure are 
\ref{alg:FD-NodeClosure-Principle}, \ref{alg:FD-NodeClosure}. 

\vspace{1.2em}

\begin{algorithm}
\KwIn{
$L_i$: set of nodes for which there exists dotted (resp. full) arcs $(i, j)$}
\KwOut{$L_i^+$: the dotted (resp. full) closure of $i$}

\BlankLine
\BlankLine

Initialize a list of nodes to treat $S_i$ to $L_i$ \;
\While{there is a node $j$ to treat in $S_i$}{
	remove $j$ from $S_i$ \;
	\If{$j$ is simple node}{
		\ForAll{compound node $m$ \belemp{except $i$}, $j$ appears in}{
			increase $q_m$ by 1 \;
			\If {$q_m$ = number of outgoing \belemp{dotted} arcs from $m$}{
				$m$ is reachable from $i$ by \aliemp{union} \;
				$m$ must be treated, add it to $S_i$ \;
			}
		}
		
	}

	add $j$ to the closure $L_i^+$ \;
	
	\ForAll{nodes $k$ for each there is an arc $(j, \ k)$}{
		\If{$k$ is not yet in the closure $L_i^+$ or in the \belemp{dotted}
			closure $L_i^{1+}$ of $i$}{
			$k$ is reachable from $i$ by \aliemp{transitivity} \;
			$k$ must be treated, add it to $S_i$ \;
		}
	}

return $L_i^{+}$ \;
}


\caption{NodeClosure (Principle)}
\label{alg:FD-NodeClosure-Principle}
\end{algorithm}

We would like to provide some observations on top of their description. Namely 
on the \aliemp{union} step and $q_m$ counters. Say $i \imp m$ where $m$ is a 
compound node is a valid implication in a FD-graph. Furthermore say $m = 
\bigcup_i m_i$ where $m_i$'s are simple nodes. The union step models the fact 
that if we have $i \imp m_i$ for all $m_i$ in $m$, then we must have $i \imp m$ 
also. The counter $q_m$ ensures that we indeed reached all $m_i$'s in $m$. 
Also, the algorithm has access to all the structures we described above (nodes, 
sets of arcs, and so forth). Parameters are thus lists we are going to modify 
somewhat.

\newpage

\begin{algorithm}
\KwIn{
$L_i$: set of nodes for which there exists dotted (resp. full) arcs $(i, j)$}
\KwOut{$L_i^+$: the dotted (resp. full) closure of $i$}

\BlankLine
\BlankLine	

$S_i := L_i$ \;

\While{$S_i \neq \emptyset$}{
	select $j$ from $S_i$ \;
	\If{$j \in V^0$}{
		\ForAll{$m \in D_j - \{ i \}$}{
			$q_m := q_m + 1$ \;
			\If{$q_m = |L_m^1|$}{
				$S_i := S_i \cup \{ m \}$ \;
			}
			
		}
	}

	$S_i^+ := S_i^+ \cup \{ j \}$ \;
	
	\ForAll{$k \in L_j^0 \cup L_j^1$}{
		\If{$k \not\in S_i^+ \cup L_i^{1+} \cup \{ i\}$}{
			$S_i := S_i \cup \{ k \}$ \;
		}
	}
}
return $L_i^+$ \;
\caption{NodeClosure}
\label{alg:FD-NodeClosure}
\end{algorithm}

\vspace{1.2em}

Next, we present the principle and pseudo-code for the closure of an FD-graph
\ref{alg:FD-Closure-Principle}, \ref{alg:FD-Closure}. Mostly, the principle is 
the idea we described previously. There is just one observation to make about 
setting a counter $q_m$ to 1. This variable helps to see whether we can use 
union rule as we saw in procedure NodeClosure 
(\ref{alg:FD-NodeClosure-Principle}, \ref{alg:FD-NodeClosure}). We initialize it
in case $i$ is indeed part of some compound node so that we do not omit to count
it when dealing with $S_i$ (because $S_i$ does not contain $i$). 

\begin{algorithm}
\KwIn{$V_0$, $V_1$ and $\forall i \in V$ $D_i$, $L_i^0$, $L_i^1$}
\KwOut{$\forall i \in V$ $L_i^{0+}$, $L_i^{1+}$}

\BlankLine
\BlankLine

\ForAll{node $i$ in $V$ with outgoing arcs}{
	
	\If{$i$ is an attribute of a compound node $m$}{
		set a counter $q_m$ to $1$ \;	
	}
	
	initialize the closure of $i$ to $\emptyset$ \;
	
	\If{$i$ is a compound node}{
		determine \belemp{dotted} arcs in the closure of $i$ \;
	}
	
	determine \belemp{full} arcs in the closure of $i$ \;
	
}

\caption{Closure (Principle)}
\label{alg:FD-Closure-Principle}
\end{algorithm}

\begin{algorithm}
\KwIn{$V_0$, $V_1$ and $\forall i \in V$ $D_i$, $L_i^0$, $L_i^1$}
\KwOut{$\forall i \in V$ $L_i^{0+}$, $L_i^{1+}$}

\BlankLine
\BlankLine

\ForAll{$i \in V$ with $L_i^0  \cup L_i^1 \neq \emptyset$}{

	\ForAll{$m \in V^1$}{
		\uIf{$m \in D_i$}{
			$q_m := 1$ \;
			
		} \Else {
			$q_m := 0$ \;
			
		}
	}

	$L_i^{1+} := \emptyset$ \;
	$L_i^{0+} := \emptyset$ \;
	
	\If{$i \in V^1$}{
		$L_i^{1+} := $ NodeClosure($L_i^{1}$) \;	
	}

	$L_i^{0+} := $ NodeClosure($L_i^{0} - L_i^{1+}$) \;

} 

\caption{Closure}
\label{alg:FD-Closure}
\end{algorithm}

\vspace{1.2em}

Now that algorithms for computing the closure of a FD-graph have been set, we
can move to the minimization part.

\paragraph{Minimization algorithm for FD-Graphs}

We have two steps in this algorithm. First, we need to remove redundant nodes, 
then superfluous nodes. To delete redundant nodes, the claim of 
\cite{ausiello_graph_1983, ausiello_minimal_1986} is that removing nodes with
dotted arcs only in the closure of an FD-Graph is sufficient. Indeed, if a node
$i$ has only dotted paths in the closure of an FD-graph, it means that for every
$i \imp j$ holding, we can find a subset of $k$ of $i$ such that $k \imp j$. In
our terms, it is saying that $\I(i) = \I^{-}$ where $\I^{-}$ denotes $\I$ from
which we removed the implication having $i$ as a body.


\begin{algorithm}
\KwIn{$G_{\I}$: the FD-Graph of some basis $\I$}
\KwOut{$G_{\I_c}$: the associated minimum FD-Graph}

\BlankLine
\BlankLine

$G_{\I}^{+} = Closure(G_{\I})$ \;

\BlankLine

\midemp{Redundancy Elimination} \\
\ForAll{$i \in V^1$}{
	\If{$L_i^{1+} = \emptyset$}{
		remove $i$ and its outgoing arcs from $G_{\I}$ \;	
	}
} 

\BlankLine

\midemp{Superfluousness Elimination} \\
\ForAll{$i \in V^1$}{
	find an equivalent node $j$ \;
	\If{$j$ exists}{
		$L_j^{0+} := L_j^{0+} \cup (L_i^{0+} \cap L_j^{1+})$ \;
		$L_j^{1+} := L_j^{1+} - (L_i^{0+} \cap L_j^{1+})$ \;
	}

	remove $i$ from the closure \;
	add $(i, j)$ to a list $L$ \; 
}

\ForAll{$(i, j) \in L$}{
	$L_j^0 := L_j^0 \cup (L_i^0 - L_j^1)$ \;
	remove $i$ from $G_{\I}$ \;
}	

\caption{Ausiello Minimization Algorithm}
\label{alg:ausiello-min}
\end{algorithm}

\vspace{1.2em} 

Let us try to investigate the complexity of this algorithm. According to the 
article, Closure is achieved in $O(|\body{\I}| \cdot |\I|)$. If the graph is
represented as adjacency lists, deleting a node and its outgoing arc can be
$O(1)$ (it consists in freeing the node and its associated two lists). Running 
over all compound nodes is $O(|\body{\I}|)$. In the case we would have to 
remove arcs one by one, note that we would have at most $O(|\Sg|)$ arcs to 
delete. This would yield a $O(|\body{\I}| \cdot |\Sg|) = O(|\I|)$
complexity. \aliemp{How to remove compound nodes that have disappeared from the
closure of other nodes ? It is time consuming}. It is $O(|\body{\I}| \cdot 
|\I^{+}|)$. We don't count for intern data structure operation. We can consider
that adding and removing can be done in $O(1)$. The first loop of 
superfluousness elimination goes over all compounds nodes, and finding an 
equivalent node is $O(V)$ Moving elements in lists associated to $j$ is also
$O(V)$ thus the first loop is $O(V^1 \cdot V)$. The second loop may need to 
go over all the full arcs of the FD-graph, which account for a $O(|\I|)$ 
complexity. As a consequence the whole algorithm may run in $O(|\body{\I}| 
\cdot |\I|)$ (because of the closure computation).



\subsection{Elements of proofs}

In this section we write various claims which help us prove correctness and 
equivalence of Maier/Ausiello's algorithms. Knowledge about FD-graphs is 
assumed. Equivalence of redundancy elimination is not shown (1st step of 
the algorithm). We place ourselves in a non-redundant context to match
definitions of direct determination. Non-redundancy does not affect superfluous
definition.

\begin{proposition} \label{prop:maier.equiv_sup_sub}
	A node $i$ in a FD-graph is superfluous with respect to $j$ if and only if 
	$i \equiv j$ and there exists a proper subset (non-empty) $k$ of $i$ such 
	that $\I \models k \imp j$. 
\end{proposition}

\begin{proof} This is a translation of the definition of superfluous node in 
	terms of implications.	
\end{proof}

\begin{proposition} \label{prop:maier.equiv_ssup_dd}
	the following statements are equivalent, for $A, B$ bodies of $\I$:
	\begin{itemize}
		\item[(i)] $A \ddv B$,
		\item[(ii)] the node $A$ is superfluous with respect to $B$, and there 
		exists 
		a dotted FD-path from $A$ to $B$ not using any outgoing full arcs of
		nodes equivalent to $A$.
	\end{itemize}
	
\end{proposition}

\begin{proof} (i) $\imp$ (ii). We will use claim 1 to help us. Suppose $ A \ddv 
	B$. By	definition of direct determination, we have that the nodes $A$ and 
	$B$ 
	are equivalent. By hypothesis, we have $ \I^{*} = \I - E_{\I}(A) \models A 
	\imp 
	B$.	We will distinguish two cases, either $B \subseteq A$, or $B \nsubseteq 
	A$.
	
	\vspace{1.2em}
	
	\paragraph{$B \subseteq A$.} In this case, direct determination is straight 
	forward, and if $A$ and $B$ are bodies of $\I$, there exists dotted arcs 
	between $A$ and $B$ forming a dotted FD-path. Then $A$ is indeed 
	superfluous 
	with respect to $B$. Furthermore, because we used only dotted arcs from 
	$A$, we 
	did not use any full arcs outgoing from $A$.
	
	\paragraph{$B \nsubseteq A$.} Since we don't use any implications with left 
	side equivalent to $A$ in direct determination, but still have $A \imp B$, 
	we 
	must be able to find a non-equivalent proper subset $X$ of $A$ such that $X 
	\imp B$, otherwise we would contradict direct determination (because we 
	would 
	not have $B \subseteq \I^{~}(A)$ ). Using claim 1, we can conclude that $A$ 
	is 
	indeed superfluous with respect to $B$. Moreover, notice that using an 
	outgoing 
	full arc from a node equivalent $C$ to $A$ is exactly using an implication 
	with 
	left hand side equivalent to $A$. Therefore, if there is not dotted FD-path 
	from $A$ to $B$ not using those arcs, we would contradict direct 
	determination.
	
	\vspace{1.2em}
	
	(ii) $\imp$ (i) Suppose $A$ is superfluous with respect to $B$ and there 
	exists a dotted FD-path from $A$ to $B$ not using any outgoing full arcs 
	from 
	nodes equivalent to $A$. Those full arcs represent exactly the implications 
	contained in $E_{\I}(A)$. Since we don't use them, the path still holds in 
	$\I^{*}$ (we would remove compound nodes without outgoing full arcs of 
	course, 
	but this would only make the path stops to attributes instead of compound 
	node). Having this path in $\I^{~}$ means that $\I^{*} \models A \imp B$.
	
\end{proof} 

\begin{proposition} \label{prop:maier.equiv_sup_ssup}
	The following two statement are equivalent, given the FD-graph of $\I$:
	\begin{itemize}
		\item[(i)] $A$ is a superfluous node,
		\item[(ii)] $A$ is superfluous with respect to $B$, and there exists a
		dotted path from $A$ to $B$ not using any outgoing full arcs from nodes
		equivalent to $A$.
	\end{itemize}
	
\end{proposition}

\begin{proof} (ii) $\imp$ (i) is trivial. let's focus on (i) $\imp$ (ii). If 
	$A$ is superfluous, then there exists $B$ such that $A \equiv B$ and there 
	has a dotted path from $A$ to $B$.
	
	\vspace{1.2em}
	
	If $B \subseteq A$, the dotted path is straight forward. If it is not the 
	case, path from $A$ to $B$ uses outgoing full arcs from nodes equivalent to 
	$A$, or it does not. If it does not we are done. Now Suppose this
	path uses an outgoing full arc from a node $C$ equivalent to $A$. This 
	means, that $A \equiv B \equiv C$ by definition and therefore, there exists 
	a
	dotted path from $A$ to $C$ (because we need to reach $C$, so to derive it,
	to use its outgoing arcs). We can reiterate this reasoning until reaching 
	$A$. Getting to $A$ or one of its subset would contradict our assumptions 
	meaning that we stopped finding used equivalent arcs earlier.
	
\end{proof}

From the previous claims, we can yield the following one

\begin{proposition} \label{prop:maier.equiv_sup_dd_sub}
	The following states are equivalent:
	\begin{itemize}
		\item[(i)] $A \ddv B$,
		\item[(ii)] $A$ is superfluous,
		\item[(iii)] $A \equiv B$ and there exists $X \subset A$ such that $\I 
		\models X \imp B$.
	\end{itemize}
\end{proposition}

\begin{proof} (i) $\longleftrightarrow$ (ii) comes from propositions 
	\ref{prop:maier.equiv_ssup_dd} and \ref{prop:maier.equiv_sup_ssup}. (ii) 
	$\longleftrightarrow$ (iii) comes from proposition 
	\ref{prop:maier.equiv_sup_sub}.
	
\end{proof}

Those claims help to see the relation between operations of the algorithms. 
Indeed, the last claim states that finding a direct determination is the same
as finding a superfluous node, therefore the two algorithms are looking for
the same structures in different terminologies. This emphasizes the fact they
work on the same computations. Remark: Maier algorithm do it in a special 
order (some kinds of minimal paths in terms of FD-graph. Can we remove them in 
any order? It seems to since this is what the Ausiello algorithm does, and
the previous proposition states that direct determination is equivalent to the
existence of superfluous nodes.

\vspace{1.2em}

The next claim provides an argument in this way. It also explains in what 
extent the operation of removal in those algorithm is exact.

\begin{proposition} Let $A, B$ be bodies of $\I$. If $A \equiv B$ and
	there exists $X \subset A$ such that $\I \models X \imp B$ then $\I$ is not
	minimum.
	
\end{proposition}

\begin{proof} Let $\I$ be an implication basis with bodies $A, B$ equivalent and
	such that there exists $X \subset A$ with $\I \models X \imp B$. Let us 
	build
	$\I_c$ by removing $A \imp E$ from $\I$, and replacing $B \imp F$ by $B \imp
	E \cup F$. We must show that $I_c \models A \imp E$. Since there exists $X 
	\subset A$ such that $\I \models X \imp B$ and we only removed $A \imp E$, 
	we
	have $\I_c \models X \imp E$ hence $I_c \models X \imp B$. Consequently, 
	$\I_c \models A \imp E$ since $X \subset A \imp \I_c(X) \subseteq \I_c(A)$.
\end{proof}

\vspace{1.2em}

This proposition can be stated as its contraposition. That is, if a basis is
minimum then we cannot find equivalent bodies with one to be removed. This 
states that the second step of Ausiello and Maier algorithms end up with a 
body minimal basis, with help of equivalences proved in proposition 
\ref{prop:maier.equiv_sup_sub}.

















% 2-steps algorithm
\section{"Usual" Algorithm}

Compute the canonical (or Duquenne-Guigues) basis $\I_c$ given $\I$. 
\cite{guigues_familles_1986, b._ganter_conceptual_2016}. 

\begin{algorithm}[H]
\KwIn{$\I$ an implication basis}
\KwOut{$\I_c$ the canonical basis derived from $\I$}

\BlankLine
\BlankLine

\ForEach{$A \imp B$ in $\I$}{
	$\I$ = $\I - \{A \imp B \}$ \;
	$B$ = $\I(A \cup B)$ \;
	$\I$ = $\I \cup \{A \imp B \}$ \;
}

\ForEach{$A \imp B$}{
	$\I$ = $\I - \{A \imp B \}$ \;
	$A$ = $\I(A)$ \;
	
	\If{$A \neq B$}{
		$\I$ = $\I \cup \{ A \imp B \}$ \;
	}
}

\caption{Canonical Cover}	
\end{algorithm}


\vspace{1.2em}

\noindent The first loop maximizes each heads, that is $A \imp B$ becomes $A 
\imp \I(A)$. We summarize the knowledge. The second loop maximizes left side
of implications with $\I^{-}(A)$. The second step kills redundancy.

\vspace{1.2em}

\noindent Notes: few tips to see that the resulting basis is equivalent to the 
input. By the end of the first loop, since we replaced $A \imp B$ by $A \imp 
\I(A)$, and by definition of $A \imp B$ , $A \imp B$ still holds for all $B$
($B \subseteq \I(A)$). For the second loop, observe that we omit only redundant
implications from the basis, because if $\I^{-}(A) = \I(A)$ it means that all
the implications $A \imp B$ true in $\I$ are true in $\I^{-}$. 


% Letter algorithm
\section{Strong duality in Horn Minimization}

\cite{KristofLetter}, \cite{BorosES}, \cite{BorosMinMax}.

Pure Horn function (PHF). Implicates. Subsuming. Given $h$ a PHF, denote by 
$\mathcal{I}(h)$ the set of implicates of $h$. $\cal{P} \subseteq \cal{I}$ is
the set of prime implicates of $h$. Forward chaining is the closure operator. 

The essential set $\cal{E}_S$ of a subset $S$ of variables $V$ is defined as
follows:

	\[ \cal{E}_S = \{ A \imp B | A \subseteq S, B \nsubseteq S \} \]
	
It describes a subset of $h$ for which $S$ is a false point. In fact, it points
out the clauses falsified by $S$. So more than indicating whether $S$ is a model
or not of the function, it also says where does it falsify the function.

Lemma 4 states that a CNF $\phi$ represents a function $h$ if for each non-model
$S$ of $h$, $\phi$ contains at least one of the clause falsified by $h$. That is
$h$ and $\phi$ are false at the same points.

$ess(h)$ allows to partition the non-models of $h$. It is the number of maximum
pairwise disjoint essential sets of $h$. This can be viewed as the number of
subsets sufficient to disjointly describes false point of $h$. Body-disjoint 
essential set is a stronger criterion. Two essential sets $\cal{E}$, $\cal{E}'$
are body-disjoint if the intersection of the left sides they contain is empty.

sketch of proof for lemma 5. 

\begin{lemma} Let $h$ be PHF over $V$. For $P, Q \subseteq V$, we have the 
following equivalences:

\begin{itemize}
	\item[(i)] $\mathcal{E}_P$ and $\mathcal{E}_Q$ are disjoint iff $F_h(P \cap
		Q) \subseteq P \cup Q$,
	
	\item[(ii)] $\cal{E}_P$ and $\cal{E}_Q$ are body-disjoint iff $F_h(P
	 	\cap Q) \subseteq P xor Q$.
\end{itemize}
	
\end{lemma} 

\begin{proof}
(i), $\imp$. Note that for $P \subseteq V$, $F_h(P)$ is exactly the set of
variables contained in $\mathcal{E}_P \cup P$. Also, note that if $P, Q$, 
$\mathcal{E}_{P \cap Q} \subseteq \mathcal{E}_P \cap \mathcal{E}_Q$. If the
two last essential sets are disjoint, then their intersection is empty, 
leading to $\mathcal{E}_{P \cap Q} = \emptyset$. In other words, $P \cap Q$ is
a model of $h$, and by definition $F_h(P \cap Q) = P \cap Q \subseteq P \cup Q$

\vspace{1.2em}

(i), $\rimp$. Let us denote by $V(\mathcal{E}_P)$ the variables contained in 
$\mathcal{E}_P$. We reason by contraposition. Suppose $V(\mathcal{E}_{P \cap Q})
\neq \emptyset$. Then, 

\begin{align*}
	F_h(P \cap Q) & = V(\mathcal{E}_{P \cap Q}) \cup (P \cap Q) \\
	& = (V(\mathcal{E}_{P \cap Q}) \cup P) \cap
		(V(\mathcal{E}_{P \cap Q}) \cup Q) \\
	& = (A \cup X \cup P) \cap (A \cup X \cup Q) \\
	& = (P \cup X) \cap (Q \cup X) \\
	& = (P \cap Q) \cup X \nsubseteq P \cup Q
\end{align*}


\noindent where $X$ is a non-empty set derived from the definition of 
$\mathcal{E}_{P \cap Q}$ if it is non-empty. Indeed if $\mathcal{E}$ is 
non-empty then it contains at least one implication $B \imp u$ such that $B 
\subseteq P \cap Q$ and $u \notin P \cap Q$. The set of such $u$ is $X$. Note
that in this case, since $\mathcal{E}$ is derived from the essential sets of
$P$ and $Q$, $X$ does belong neither to $P$ nor $Q$. In other words $X 
\nsubseteq P \cup Q$. 

$A$ is a subset of $P \cap Q$. Indeed, it is possible not to have $P \cap Q
\subseteq V$ if the implications in $V$ only use parts of $P \cap Q$. $A$ is 
non-empty, and is be absorbed by $P$ or $Q$ in the previous derivations, 
through $P \cup A$ and $Q \cup A$. 

\vspace{1.2em}

(ii) $\imp$.
\end{proof}


% Angluin Algorithm
\section{Angluin Algorithm}

Include two refs, the Angluin paper, and the one on poylnomial explanation of
it. \cite{Angluin1992}, \cite{PolyAngluin}.


\chapter{Efficiency, comparison}



\bibliographystyle{acm}
	\bibliography{Biblio.bib}

\end{document}