\documentclass[a4paper, 10pt]{report}

\input{../Subfiles/Packages.tex}

\subfloat[Average time (in $s$), $|\Sg| = 100$]{
	\hspace{-3em}
	\scalebox{0.44}{%% Creator: Matplotlib, PGF backend
%%
%% To include the figure in your LaTeX document, write
%%   \input{<filename>.pgf}
%%
%% Make sure the required packages are loaded in your preamble
%%   \usepackage{pgf}
%%
%% Figures using additional raster images can only be included by \input if
%% they are in the same directory as the main LaTeX file. For loading figures
%% from other directories you can use the `import` package
%%   \usepackage{import}
%% and then include the figures with
%%   \import{<path to file>}{<filename>.pgf}
%%
%% Matplotlib used the following preamble
%%   \usepackage{fontspec}
%%   \setmainfont{DejaVu Serif}
%%   \setsansfont{DejaVu Sans}
%%   \setmonofont{DejaVu Sans Mono}
%%
\begingroup%
\makeatletter%
\begin{pgfpicture}%
\pgfpathrectangle{\pgfpointorigin}{\pgfqpoint{8.100000in}{6.600000in}}%
\pgfusepath{use as bounding box, clip}%
\begin{pgfscope}%
\pgfsetbuttcap%
\pgfsetmiterjoin%
\definecolor{currentfill}{rgb}{1.000000,1.000000,1.000000}%
\pgfsetfillcolor{currentfill}%
\pgfsetlinewidth{0.000000pt}%
\definecolor{currentstroke}{rgb}{1.000000,1.000000,1.000000}%
\pgfsetstrokecolor{currentstroke}%
\pgfsetdash{}{0pt}%
\pgfpathmoveto{\pgfqpoint{0.000000in}{0.000000in}}%
\pgfpathlineto{\pgfqpoint{8.100000in}{0.000000in}}%
\pgfpathlineto{\pgfqpoint{8.100000in}{6.600000in}}%
\pgfpathlineto{\pgfqpoint{0.000000in}{6.600000in}}%
\pgfpathclose%
\pgfusepath{fill}%
\end{pgfscope}%
\begin{pgfscope}%
\pgfsetbuttcap%
\pgfsetmiterjoin%
\definecolor{currentfill}{rgb}{1.000000,1.000000,1.000000}%
\pgfsetfillcolor{currentfill}%
\pgfsetlinewidth{0.000000pt}%
\definecolor{currentstroke}{rgb}{0.000000,0.000000,0.000000}%
\pgfsetstrokecolor{currentstroke}%
\pgfsetstrokeopacity{0.000000}%
\pgfsetdash{}{0pt}%
\pgfpathmoveto{\pgfqpoint{1.012500in}{0.726000in}}%
\pgfpathlineto{\pgfqpoint{7.290000in}{0.726000in}}%
\pgfpathlineto{\pgfqpoint{7.290000in}{5.808000in}}%
\pgfpathlineto{\pgfqpoint{1.012500in}{5.808000in}}%
\pgfpathclose%
\pgfusepath{fill}%
\end{pgfscope}%
\begin{pgfscope}%
\pgfsetbuttcap%
\pgfsetroundjoin%
\definecolor{currentfill}{rgb}{0.000000,0.000000,0.000000}%
\pgfsetfillcolor{currentfill}%
\pgfsetlinewidth{0.803000pt}%
\definecolor{currentstroke}{rgb}{0.000000,0.000000,0.000000}%
\pgfsetstrokecolor{currentstroke}%
\pgfsetdash{}{0pt}%
\pgfsys@defobject{currentmarker}{\pgfqpoint{0.000000in}{-0.048611in}}{\pgfqpoint{0.000000in}{0.000000in}}{%
\pgfpathmoveto{\pgfqpoint{0.000000in}{0.000000in}}%
\pgfpathlineto{\pgfqpoint{0.000000in}{-0.048611in}}%
\pgfusepath{stroke,fill}%
}%
\begin{pgfscope}%
\pgfsys@transformshift{1.012500in}{0.726000in}%
\pgfsys@useobject{currentmarker}{}%
\end{pgfscope}%
\end{pgfscope}%
\begin{pgfscope}%
\pgftext[x=1.012500in,y=0.628778in,,top]{\sffamily\fontsize{10.000000}{12.000000}\selectfont \(\displaystyle 0\)}%
\end{pgfscope}%
\begin{pgfscope}%
\pgfsetbuttcap%
\pgfsetroundjoin%
\definecolor{currentfill}{rgb}{0.000000,0.000000,0.000000}%
\pgfsetfillcolor{currentfill}%
\pgfsetlinewidth{0.803000pt}%
\definecolor{currentstroke}{rgb}{0.000000,0.000000,0.000000}%
\pgfsetstrokecolor{currentstroke}%
\pgfsetdash{}{0pt}%
\pgfsys@defobject{currentmarker}{\pgfqpoint{0.000000in}{-0.048611in}}{\pgfqpoint{0.000000in}{0.000000in}}{%
\pgfpathmoveto{\pgfqpoint{0.000000in}{0.000000in}}%
\pgfpathlineto{\pgfqpoint{0.000000in}{-0.048611in}}%
\pgfusepath{stroke,fill}%
}%
\begin{pgfscope}%
\pgfsys@transformshift{1.611784in}{0.726000in}%
\pgfsys@useobject{currentmarker}{}%
\end{pgfscope}%
\end{pgfscope}%
\begin{pgfscope}%
\pgftext[x=1.611784in,y=0.628778in,,top]{\sffamily\fontsize{10.000000}{12.000000}\selectfont \(\displaystyle 2000\)}%
\end{pgfscope}%
\begin{pgfscope}%
\pgfsetbuttcap%
\pgfsetroundjoin%
\definecolor{currentfill}{rgb}{0.000000,0.000000,0.000000}%
\pgfsetfillcolor{currentfill}%
\pgfsetlinewidth{0.803000pt}%
\definecolor{currentstroke}{rgb}{0.000000,0.000000,0.000000}%
\pgfsetstrokecolor{currentstroke}%
\pgfsetdash{}{0pt}%
\pgfsys@defobject{currentmarker}{\pgfqpoint{0.000000in}{-0.048611in}}{\pgfqpoint{0.000000in}{0.000000in}}{%
\pgfpathmoveto{\pgfqpoint{0.000000in}{0.000000in}}%
\pgfpathlineto{\pgfqpoint{0.000000in}{-0.048611in}}%
\pgfusepath{stroke,fill}%
}%
\begin{pgfscope}%
\pgfsys@transformshift{2.211068in}{0.726000in}%
\pgfsys@useobject{currentmarker}{}%
\end{pgfscope}%
\end{pgfscope}%
\begin{pgfscope}%
\pgftext[x=2.211068in,y=0.628778in,,top]{\sffamily\fontsize{10.000000}{12.000000}\selectfont \(\displaystyle 4000\)}%
\end{pgfscope}%
\begin{pgfscope}%
\pgfsetbuttcap%
\pgfsetroundjoin%
\definecolor{currentfill}{rgb}{0.000000,0.000000,0.000000}%
\pgfsetfillcolor{currentfill}%
\pgfsetlinewidth{0.803000pt}%
\definecolor{currentstroke}{rgb}{0.000000,0.000000,0.000000}%
\pgfsetstrokecolor{currentstroke}%
\pgfsetdash{}{0pt}%
\pgfsys@defobject{currentmarker}{\pgfqpoint{0.000000in}{-0.048611in}}{\pgfqpoint{0.000000in}{0.000000in}}{%
\pgfpathmoveto{\pgfqpoint{0.000000in}{0.000000in}}%
\pgfpathlineto{\pgfqpoint{0.000000in}{-0.048611in}}%
\pgfusepath{stroke,fill}%
}%
\begin{pgfscope}%
\pgfsys@transformshift{2.810352in}{0.726000in}%
\pgfsys@useobject{currentmarker}{}%
\end{pgfscope}%
\end{pgfscope}%
\begin{pgfscope}%
\pgftext[x=2.810352in,y=0.628778in,,top]{\sffamily\fontsize{10.000000}{12.000000}\selectfont \(\displaystyle 6000\)}%
\end{pgfscope}%
\begin{pgfscope}%
\pgfsetbuttcap%
\pgfsetroundjoin%
\definecolor{currentfill}{rgb}{0.000000,0.000000,0.000000}%
\pgfsetfillcolor{currentfill}%
\pgfsetlinewidth{0.803000pt}%
\definecolor{currentstroke}{rgb}{0.000000,0.000000,0.000000}%
\pgfsetstrokecolor{currentstroke}%
\pgfsetdash{}{0pt}%
\pgfsys@defobject{currentmarker}{\pgfqpoint{0.000000in}{-0.048611in}}{\pgfqpoint{0.000000in}{0.000000in}}{%
\pgfpathmoveto{\pgfqpoint{0.000000in}{0.000000in}}%
\pgfpathlineto{\pgfqpoint{0.000000in}{-0.048611in}}%
\pgfusepath{stroke,fill}%
}%
\begin{pgfscope}%
\pgfsys@transformshift{3.409636in}{0.726000in}%
\pgfsys@useobject{currentmarker}{}%
\end{pgfscope}%
\end{pgfscope}%
\begin{pgfscope}%
\pgftext[x=3.409636in,y=0.628778in,,top]{\sffamily\fontsize{10.000000}{12.000000}\selectfont \(\displaystyle 8000\)}%
\end{pgfscope}%
\begin{pgfscope}%
\pgfsetbuttcap%
\pgfsetroundjoin%
\definecolor{currentfill}{rgb}{0.000000,0.000000,0.000000}%
\pgfsetfillcolor{currentfill}%
\pgfsetlinewidth{0.803000pt}%
\definecolor{currentstroke}{rgb}{0.000000,0.000000,0.000000}%
\pgfsetstrokecolor{currentstroke}%
\pgfsetdash{}{0pt}%
\pgfsys@defobject{currentmarker}{\pgfqpoint{0.000000in}{-0.048611in}}{\pgfqpoint{0.000000in}{0.000000in}}{%
\pgfpathmoveto{\pgfqpoint{0.000000in}{0.000000in}}%
\pgfpathlineto{\pgfqpoint{0.000000in}{-0.048611in}}%
\pgfusepath{stroke,fill}%
}%
\begin{pgfscope}%
\pgfsys@transformshift{4.008920in}{0.726000in}%
\pgfsys@useobject{currentmarker}{}%
\end{pgfscope}%
\end{pgfscope}%
\begin{pgfscope}%
\pgftext[x=4.008920in,y=0.628778in,,top]{\sffamily\fontsize{10.000000}{12.000000}\selectfont \(\displaystyle 10000\)}%
\end{pgfscope}%
\begin{pgfscope}%
\pgfsetbuttcap%
\pgfsetroundjoin%
\definecolor{currentfill}{rgb}{0.000000,0.000000,0.000000}%
\pgfsetfillcolor{currentfill}%
\pgfsetlinewidth{0.803000pt}%
\definecolor{currentstroke}{rgb}{0.000000,0.000000,0.000000}%
\pgfsetstrokecolor{currentstroke}%
\pgfsetdash{}{0pt}%
\pgfsys@defobject{currentmarker}{\pgfqpoint{0.000000in}{-0.048611in}}{\pgfqpoint{0.000000in}{0.000000in}}{%
\pgfpathmoveto{\pgfqpoint{0.000000in}{0.000000in}}%
\pgfpathlineto{\pgfqpoint{0.000000in}{-0.048611in}}%
\pgfusepath{stroke,fill}%
}%
\begin{pgfscope}%
\pgfsys@transformshift{4.608204in}{0.726000in}%
\pgfsys@useobject{currentmarker}{}%
\end{pgfscope}%
\end{pgfscope}%
\begin{pgfscope}%
\pgftext[x=4.608204in,y=0.628778in,,top]{\sffamily\fontsize{10.000000}{12.000000}\selectfont \(\displaystyle 12000\)}%
\end{pgfscope}%
\begin{pgfscope}%
\pgfsetbuttcap%
\pgfsetroundjoin%
\definecolor{currentfill}{rgb}{0.000000,0.000000,0.000000}%
\pgfsetfillcolor{currentfill}%
\pgfsetlinewidth{0.803000pt}%
\definecolor{currentstroke}{rgb}{0.000000,0.000000,0.000000}%
\pgfsetstrokecolor{currentstroke}%
\pgfsetdash{}{0pt}%
\pgfsys@defobject{currentmarker}{\pgfqpoint{0.000000in}{-0.048611in}}{\pgfqpoint{0.000000in}{0.000000in}}{%
\pgfpathmoveto{\pgfqpoint{0.000000in}{0.000000in}}%
\pgfpathlineto{\pgfqpoint{0.000000in}{-0.048611in}}%
\pgfusepath{stroke,fill}%
}%
\begin{pgfscope}%
\pgfsys@transformshift{5.207488in}{0.726000in}%
\pgfsys@useobject{currentmarker}{}%
\end{pgfscope}%
\end{pgfscope}%
\begin{pgfscope}%
\pgftext[x=5.207488in,y=0.628778in,,top]{\sffamily\fontsize{10.000000}{12.000000}\selectfont \(\displaystyle 14000\)}%
\end{pgfscope}%
\begin{pgfscope}%
\pgfsetbuttcap%
\pgfsetroundjoin%
\definecolor{currentfill}{rgb}{0.000000,0.000000,0.000000}%
\pgfsetfillcolor{currentfill}%
\pgfsetlinewidth{0.803000pt}%
\definecolor{currentstroke}{rgb}{0.000000,0.000000,0.000000}%
\pgfsetstrokecolor{currentstroke}%
\pgfsetdash{}{0pt}%
\pgfsys@defobject{currentmarker}{\pgfqpoint{0.000000in}{-0.048611in}}{\pgfqpoint{0.000000in}{0.000000in}}{%
\pgfpathmoveto{\pgfqpoint{0.000000in}{0.000000in}}%
\pgfpathlineto{\pgfqpoint{0.000000in}{-0.048611in}}%
\pgfusepath{stroke,fill}%
}%
\begin{pgfscope}%
\pgfsys@transformshift{5.806772in}{0.726000in}%
\pgfsys@useobject{currentmarker}{}%
\end{pgfscope}%
\end{pgfscope}%
\begin{pgfscope}%
\pgftext[x=5.806772in,y=0.628778in,,top]{\sffamily\fontsize{10.000000}{12.000000}\selectfont \(\displaystyle 16000\)}%
\end{pgfscope}%
\begin{pgfscope}%
\pgfsetbuttcap%
\pgfsetroundjoin%
\definecolor{currentfill}{rgb}{0.000000,0.000000,0.000000}%
\pgfsetfillcolor{currentfill}%
\pgfsetlinewidth{0.803000pt}%
\definecolor{currentstroke}{rgb}{0.000000,0.000000,0.000000}%
\pgfsetstrokecolor{currentstroke}%
\pgfsetdash{}{0pt}%
\pgfsys@defobject{currentmarker}{\pgfqpoint{0.000000in}{-0.048611in}}{\pgfqpoint{0.000000in}{0.000000in}}{%
\pgfpathmoveto{\pgfqpoint{0.000000in}{0.000000in}}%
\pgfpathlineto{\pgfqpoint{0.000000in}{-0.048611in}}%
\pgfusepath{stroke,fill}%
}%
\begin{pgfscope}%
\pgfsys@transformshift{6.406056in}{0.726000in}%
\pgfsys@useobject{currentmarker}{}%
\end{pgfscope}%
\end{pgfscope}%
\begin{pgfscope}%
\pgftext[x=6.406056in,y=0.628778in,,top]{\sffamily\fontsize{10.000000}{12.000000}\selectfont \(\displaystyle 18000\)}%
\end{pgfscope}%
\begin{pgfscope}%
\pgfsetbuttcap%
\pgfsetroundjoin%
\definecolor{currentfill}{rgb}{0.000000,0.000000,0.000000}%
\pgfsetfillcolor{currentfill}%
\pgfsetlinewidth{0.803000pt}%
\definecolor{currentstroke}{rgb}{0.000000,0.000000,0.000000}%
\pgfsetstrokecolor{currentstroke}%
\pgfsetdash{}{0pt}%
\pgfsys@defobject{currentmarker}{\pgfqpoint{0.000000in}{-0.048611in}}{\pgfqpoint{0.000000in}{0.000000in}}{%
\pgfpathmoveto{\pgfqpoint{0.000000in}{0.000000in}}%
\pgfpathlineto{\pgfqpoint{0.000000in}{-0.048611in}}%
\pgfusepath{stroke,fill}%
}%
\begin{pgfscope}%
\pgfsys@transformshift{7.005340in}{0.726000in}%
\pgfsys@useobject{currentmarker}{}%
\end{pgfscope}%
\end{pgfscope}%
\begin{pgfscope}%
\pgftext[x=7.005340in,y=0.628778in,,top]{\sffamily\fontsize{10.000000}{12.000000}\selectfont \(\displaystyle 20000\)}%
\end{pgfscope}%
\begin{pgfscope}%
\pgftext[x=4.151250in,y=0.438809in,,top]{\sffamily\fontsize{12.000000}{14.400000}\selectfont \(\displaystyle |\mathcal{B}|\)}%
\end{pgfscope}%
\begin{pgfscope}%
\pgfsetbuttcap%
\pgfsetroundjoin%
\definecolor{currentfill}{rgb}{0.000000,0.000000,0.000000}%
\pgfsetfillcolor{currentfill}%
\pgfsetlinewidth{0.803000pt}%
\definecolor{currentstroke}{rgb}{0.000000,0.000000,0.000000}%
\pgfsetstrokecolor{currentstroke}%
\pgfsetdash{}{0pt}%
\pgfsys@defobject{currentmarker}{\pgfqpoint{-0.048611in}{0.000000in}}{\pgfqpoint{0.000000in}{0.000000in}}{%
\pgfpathmoveto{\pgfqpoint{0.000000in}{0.000000in}}%
\pgfpathlineto{\pgfqpoint{-0.048611in}{0.000000in}}%
\pgfusepath{stroke,fill}%
}%
\begin{pgfscope}%
\pgfsys@transformshift{1.012500in}{0.956312in}%
\pgfsys@useobject{currentmarker}{}%
\end{pgfscope}%
\end{pgfscope}%
\begin{pgfscope}%
\pgftext[x=0.845833in,y=0.903550in,left,base]{\sffamily\fontsize{10.000000}{12.000000}\selectfont \(\displaystyle 0\)}%
\end{pgfscope}%
\begin{pgfscope}%
\pgfsetbuttcap%
\pgfsetroundjoin%
\definecolor{currentfill}{rgb}{0.000000,0.000000,0.000000}%
\pgfsetfillcolor{currentfill}%
\pgfsetlinewidth{0.803000pt}%
\definecolor{currentstroke}{rgb}{0.000000,0.000000,0.000000}%
\pgfsetstrokecolor{currentstroke}%
\pgfsetdash{}{0pt}%
\pgfsys@defobject{currentmarker}{\pgfqpoint{-0.048611in}{0.000000in}}{\pgfqpoint{0.000000in}{0.000000in}}{%
\pgfpathmoveto{\pgfqpoint{0.000000in}{0.000000in}}%
\pgfpathlineto{\pgfqpoint{-0.048611in}{0.000000in}}%
\pgfusepath{stroke,fill}%
}%
\begin{pgfscope}%
\pgfsys@transformshift{1.012500in}{1.163097in}%
\pgfsys@useobject{currentmarker}{}%
\end{pgfscope}%
\end{pgfscope}%
\begin{pgfscope}%
\pgftext[x=0.776388in,y=1.110336in,left,base]{\sffamily\fontsize{10.000000}{12.000000}\selectfont \(\displaystyle 10\)}%
\end{pgfscope}%
\begin{pgfscope}%
\pgfsetbuttcap%
\pgfsetroundjoin%
\definecolor{currentfill}{rgb}{0.000000,0.000000,0.000000}%
\pgfsetfillcolor{currentfill}%
\pgfsetlinewidth{0.803000pt}%
\definecolor{currentstroke}{rgb}{0.000000,0.000000,0.000000}%
\pgfsetstrokecolor{currentstroke}%
\pgfsetdash{}{0pt}%
\pgfsys@defobject{currentmarker}{\pgfqpoint{-0.048611in}{0.000000in}}{\pgfqpoint{0.000000in}{0.000000in}}{%
\pgfpathmoveto{\pgfqpoint{0.000000in}{0.000000in}}%
\pgfpathlineto{\pgfqpoint{-0.048611in}{0.000000in}}%
\pgfusepath{stroke,fill}%
}%
\begin{pgfscope}%
\pgfsys@transformshift{1.012500in}{1.369883in}%
\pgfsys@useobject{currentmarker}{}%
\end{pgfscope}%
\end{pgfscope}%
\begin{pgfscope}%
\pgftext[x=0.776388in,y=1.317122in,left,base]{\sffamily\fontsize{10.000000}{12.000000}\selectfont \(\displaystyle 20\)}%
\end{pgfscope}%
\begin{pgfscope}%
\pgfsetbuttcap%
\pgfsetroundjoin%
\definecolor{currentfill}{rgb}{0.000000,0.000000,0.000000}%
\pgfsetfillcolor{currentfill}%
\pgfsetlinewidth{0.803000pt}%
\definecolor{currentstroke}{rgb}{0.000000,0.000000,0.000000}%
\pgfsetstrokecolor{currentstroke}%
\pgfsetdash{}{0pt}%
\pgfsys@defobject{currentmarker}{\pgfqpoint{-0.048611in}{0.000000in}}{\pgfqpoint{0.000000in}{0.000000in}}{%
\pgfpathmoveto{\pgfqpoint{0.000000in}{0.000000in}}%
\pgfpathlineto{\pgfqpoint{-0.048611in}{0.000000in}}%
\pgfusepath{stroke,fill}%
}%
\begin{pgfscope}%
\pgfsys@transformshift{1.012500in}{1.576669in}%
\pgfsys@useobject{currentmarker}{}%
\end{pgfscope}%
\end{pgfscope}%
\begin{pgfscope}%
\pgftext[x=0.776388in,y=1.523907in,left,base]{\sffamily\fontsize{10.000000}{12.000000}\selectfont \(\displaystyle 30\)}%
\end{pgfscope}%
\begin{pgfscope}%
\pgfsetbuttcap%
\pgfsetroundjoin%
\definecolor{currentfill}{rgb}{0.000000,0.000000,0.000000}%
\pgfsetfillcolor{currentfill}%
\pgfsetlinewidth{0.803000pt}%
\definecolor{currentstroke}{rgb}{0.000000,0.000000,0.000000}%
\pgfsetstrokecolor{currentstroke}%
\pgfsetdash{}{0pt}%
\pgfsys@defobject{currentmarker}{\pgfqpoint{-0.048611in}{0.000000in}}{\pgfqpoint{0.000000in}{0.000000in}}{%
\pgfpathmoveto{\pgfqpoint{0.000000in}{0.000000in}}%
\pgfpathlineto{\pgfqpoint{-0.048611in}{0.000000in}}%
\pgfusepath{stroke,fill}%
}%
\begin{pgfscope}%
\pgfsys@transformshift{1.012500in}{1.783455in}%
\pgfsys@useobject{currentmarker}{}%
\end{pgfscope}%
\end{pgfscope}%
\begin{pgfscope}%
\pgftext[x=0.776388in,y=1.730693in,left,base]{\sffamily\fontsize{10.000000}{12.000000}\selectfont \(\displaystyle 40\)}%
\end{pgfscope}%
\begin{pgfscope}%
\pgfsetbuttcap%
\pgfsetroundjoin%
\definecolor{currentfill}{rgb}{0.000000,0.000000,0.000000}%
\pgfsetfillcolor{currentfill}%
\pgfsetlinewidth{0.803000pt}%
\definecolor{currentstroke}{rgb}{0.000000,0.000000,0.000000}%
\pgfsetstrokecolor{currentstroke}%
\pgfsetdash{}{0pt}%
\pgfsys@defobject{currentmarker}{\pgfqpoint{-0.048611in}{0.000000in}}{\pgfqpoint{0.000000in}{0.000000in}}{%
\pgfpathmoveto{\pgfqpoint{0.000000in}{0.000000in}}%
\pgfpathlineto{\pgfqpoint{-0.048611in}{0.000000in}}%
\pgfusepath{stroke,fill}%
}%
\begin{pgfscope}%
\pgfsys@transformshift{1.012500in}{1.990240in}%
\pgfsys@useobject{currentmarker}{}%
\end{pgfscope}%
\end{pgfscope}%
\begin{pgfscope}%
\pgftext[x=0.776388in,y=1.937479in,left,base]{\sffamily\fontsize{10.000000}{12.000000}\selectfont \(\displaystyle 50\)}%
\end{pgfscope}%
\begin{pgfscope}%
\pgfsetbuttcap%
\pgfsetroundjoin%
\definecolor{currentfill}{rgb}{0.000000,0.000000,0.000000}%
\pgfsetfillcolor{currentfill}%
\pgfsetlinewidth{0.803000pt}%
\definecolor{currentstroke}{rgb}{0.000000,0.000000,0.000000}%
\pgfsetstrokecolor{currentstroke}%
\pgfsetdash{}{0pt}%
\pgfsys@defobject{currentmarker}{\pgfqpoint{-0.048611in}{0.000000in}}{\pgfqpoint{0.000000in}{0.000000in}}{%
\pgfpathmoveto{\pgfqpoint{0.000000in}{0.000000in}}%
\pgfpathlineto{\pgfqpoint{-0.048611in}{0.000000in}}%
\pgfusepath{stroke,fill}%
}%
\begin{pgfscope}%
\pgfsys@transformshift{1.012500in}{2.197026in}%
\pgfsys@useobject{currentmarker}{}%
\end{pgfscope}%
\end{pgfscope}%
\begin{pgfscope}%
\pgftext[x=0.776388in,y=2.144264in,left,base]{\sffamily\fontsize{10.000000}{12.000000}\selectfont \(\displaystyle 60\)}%
\end{pgfscope}%
\begin{pgfscope}%
\pgfsetbuttcap%
\pgfsetroundjoin%
\definecolor{currentfill}{rgb}{0.000000,0.000000,0.000000}%
\pgfsetfillcolor{currentfill}%
\pgfsetlinewidth{0.803000pt}%
\definecolor{currentstroke}{rgb}{0.000000,0.000000,0.000000}%
\pgfsetstrokecolor{currentstroke}%
\pgfsetdash{}{0pt}%
\pgfsys@defobject{currentmarker}{\pgfqpoint{-0.048611in}{0.000000in}}{\pgfqpoint{0.000000in}{0.000000in}}{%
\pgfpathmoveto{\pgfqpoint{0.000000in}{0.000000in}}%
\pgfpathlineto{\pgfqpoint{-0.048611in}{0.000000in}}%
\pgfusepath{stroke,fill}%
}%
\begin{pgfscope}%
\pgfsys@transformshift{1.012500in}{2.403812in}%
\pgfsys@useobject{currentmarker}{}%
\end{pgfscope}%
\end{pgfscope}%
\begin{pgfscope}%
\pgftext[x=0.776388in,y=2.351050in,left,base]{\sffamily\fontsize{10.000000}{12.000000}\selectfont \(\displaystyle 70\)}%
\end{pgfscope}%
\begin{pgfscope}%
\pgfsetbuttcap%
\pgfsetroundjoin%
\definecolor{currentfill}{rgb}{0.000000,0.000000,0.000000}%
\pgfsetfillcolor{currentfill}%
\pgfsetlinewidth{0.803000pt}%
\definecolor{currentstroke}{rgb}{0.000000,0.000000,0.000000}%
\pgfsetstrokecolor{currentstroke}%
\pgfsetdash{}{0pt}%
\pgfsys@defobject{currentmarker}{\pgfqpoint{-0.048611in}{0.000000in}}{\pgfqpoint{0.000000in}{0.000000in}}{%
\pgfpathmoveto{\pgfqpoint{0.000000in}{0.000000in}}%
\pgfpathlineto{\pgfqpoint{-0.048611in}{0.000000in}}%
\pgfusepath{stroke,fill}%
}%
\begin{pgfscope}%
\pgfsys@transformshift{1.012500in}{2.610597in}%
\pgfsys@useobject{currentmarker}{}%
\end{pgfscope}%
\end{pgfscope}%
\begin{pgfscope}%
\pgftext[x=0.776388in,y=2.557836in,left,base]{\sffamily\fontsize{10.000000}{12.000000}\selectfont \(\displaystyle 80\)}%
\end{pgfscope}%
\begin{pgfscope}%
\pgfsetbuttcap%
\pgfsetroundjoin%
\definecolor{currentfill}{rgb}{0.000000,0.000000,0.000000}%
\pgfsetfillcolor{currentfill}%
\pgfsetlinewidth{0.803000pt}%
\definecolor{currentstroke}{rgb}{0.000000,0.000000,0.000000}%
\pgfsetstrokecolor{currentstroke}%
\pgfsetdash{}{0pt}%
\pgfsys@defobject{currentmarker}{\pgfqpoint{-0.048611in}{0.000000in}}{\pgfqpoint{0.000000in}{0.000000in}}{%
\pgfpathmoveto{\pgfqpoint{0.000000in}{0.000000in}}%
\pgfpathlineto{\pgfqpoint{-0.048611in}{0.000000in}}%
\pgfusepath{stroke,fill}%
}%
\begin{pgfscope}%
\pgfsys@transformshift{1.012500in}{2.817383in}%
\pgfsys@useobject{currentmarker}{}%
\end{pgfscope}%
\end{pgfscope}%
\begin{pgfscope}%
\pgftext[x=0.776388in,y=2.764621in,left,base]{\sffamily\fontsize{10.000000}{12.000000}\selectfont \(\displaystyle 90\)}%
\end{pgfscope}%
\begin{pgfscope}%
\pgfsetbuttcap%
\pgfsetroundjoin%
\definecolor{currentfill}{rgb}{0.000000,0.000000,0.000000}%
\pgfsetfillcolor{currentfill}%
\pgfsetlinewidth{0.803000pt}%
\definecolor{currentstroke}{rgb}{0.000000,0.000000,0.000000}%
\pgfsetstrokecolor{currentstroke}%
\pgfsetdash{}{0pt}%
\pgfsys@defobject{currentmarker}{\pgfqpoint{-0.048611in}{0.000000in}}{\pgfqpoint{0.000000in}{0.000000in}}{%
\pgfpathmoveto{\pgfqpoint{0.000000in}{0.000000in}}%
\pgfpathlineto{\pgfqpoint{-0.048611in}{0.000000in}}%
\pgfusepath{stroke,fill}%
}%
\begin{pgfscope}%
\pgfsys@transformshift{1.012500in}{3.024169in}%
\pgfsys@useobject{currentmarker}{}%
\end{pgfscope}%
\end{pgfscope}%
\begin{pgfscope}%
\pgftext[x=0.706944in,y=2.971407in,left,base]{\sffamily\fontsize{10.000000}{12.000000}\selectfont \(\displaystyle 100\)}%
\end{pgfscope}%
\begin{pgfscope}%
\pgfsetbuttcap%
\pgfsetroundjoin%
\definecolor{currentfill}{rgb}{0.000000,0.000000,0.000000}%
\pgfsetfillcolor{currentfill}%
\pgfsetlinewidth{0.803000pt}%
\definecolor{currentstroke}{rgb}{0.000000,0.000000,0.000000}%
\pgfsetstrokecolor{currentstroke}%
\pgfsetdash{}{0pt}%
\pgfsys@defobject{currentmarker}{\pgfqpoint{-0.048611in}{0.000000in}}{\pgfqpoint{0.000000in}{0.000000in}}{%
\pgfpathmoveto{\pgfqpoint{0.000000in}{0.000000in}}%
\pgfpathlineto{\pgfqpoint{-0.048611in}{0.000000in}}%
\pgfusepath{stroke,fill}%
}%
\begin{pgfscope}%
\pgfsys@transformshift{1.012500in}{3.230954in}%
\pgfsys@useobject{currentmarker}{}%
\end{pgfscope}%
\end{pgfscope}%
\begin{pgfscope}%
\pgftext[x=0.706944in,y=3.178193in,left,base]{\sffamily\fontsize{10.000000}{12.000000}\selectfont \(\displaystyle 110\)}%
\end{pgfscope}%
\begin{pgfscope}%
\pgfsetbuttcap%
\pgfsetroundjoin%
\definecolor{currentfill}{rgb}{0.000000,0.000000,0.000000}%
\pgfsetfillcolor{currentfill}%
\pgfsetlinewidth{0.803000pt}%
\definecolor{currentstroke}{rgb}{0.000000,0.000000,0.000000}%
\pgfsetstrokecolor{currentstroke}%
\pgfsetdash{}{0pt}%
\pgfsys@defobject{currentmarker}{\pgfqpoint{-0.048611in}{0.000000in}}{\pgfqpoint{0.000000in}{0.000000in}}{%
\pgfpathmoveto{\pgfqpoint{0.000000in}{0.000000in}}%
\pgfpathlineto{\pgfqpoint{-0.048611in}{0.000000in}}%
\pgfusepath{stroke,fill}%
}%
\begin{pgfscope}%
\pgfsys@transformshift{1.012500in}{3.437740in}%
\pgfsys@useobject{currentmarker}{}%
\end{pgfscope}%
\end{pgfscope}%
\begin{pgfscope}%
\pgftext[x=0.706944in,y=3.384979in,left,base]{\sffamily\fontsize{10.000000}{12.000000}\selectfont \(\displaystyle 120\)}%
\end{pgfscope}%
\begin{pgfscope}%
\pgfsetbuttcap%
\pgfsetroundjoin%
\definecolor{currentfill}{rgb}{0.000000,0.000000,0.000000}%
\pgfsetfillcolor{currentfill}%
\pgfsetlinewidth{0.803000pt}%
\definecolor{currentstroke}{rgb}{0.000000,0.000000,0.000000}%
\pgfsetstrokecolor{currentstroke}%
\pgfsetdash{}{0pt}%
\pgfsys@defobject{currentmarker}{\pgfqpoint{-0.048611in}{0.000000in}}{\pgfqpoint{0.000000in}{0.000000in}}{%
\pgfpathmoveto{\pgfqpoint{0.000000in}{0.000000in}}%
\pgfpathlineto{\pgfqpoint{-0.048611in}{0.000000in}}%
\pgfusepath{stroke,fill}%
}%
\begin{pgfscope}%
\pgfsys@transformshift{1.012500in}{3.644526in}%
\pgfsys@useobject{currentmarker}{}%
\end{pgfscope}%
\end{pgfscope}%
\begin{pgfscope}%
\pgftext[x=0.706944in,y=3.591764in,left,base]{\sffamily\fontsize{10.000000}{12.000000}\selectfont \(\displaystyle 130\)}%
\end{pgfscope}%
\begin{pgfscope}%
\pgfsetbuttcap%
\pgfsetroundjoin%
\definecolor{currentfill}{rgb}{0.000000,0.000000,0.000000}%
\pgfsetfillcolor{currentfill}%
\pgfsetlinewidth{0.803000pt}%
\definecolor{currentstroke}{rgb}{0.000000,0.000000,0.000000}%
\pgfsetstrokecolor{currentstroke}%
\pgfsetdash{}{0pt}%
\pgfsys@defobject{currentmarker}{\pgfqpoint{-0.048611in}{0.000000in}}{\pgfqpoint{0.000000in}{0.000000in}}{%
\pgfpathmoveto{\pgfqpoint{0.000000in}{0.000000in}}%
\pgfpathlineto{\pgfqpoint{-0.048611in}{0.000000in}}%
\pgfusepath{stroke,fill}%
}%
\begin{pgfscope}%
\pgfsys@transformshift{1.012500in}{3.851311in}%
\pgfsys@useobject{currentmarker}{}%
\end{pgfscope}%
\end{pgfscope}%
\begin{pgfscope}%
\pgftext[x=0.706944in,y=3.798550in,left,base]{\sffamily\fontsize{10.000000}{12.000000}\selectfont \(\displaystyle 140\)}%
\end{pgfscope}%
\begin{pgfscope}%
\pgfsetbuttcap%
\pgfsetroundjoin%
\definecolor{currentfill}{rgb}{0.000000,0.000000,0.000000}%
\pgfsetfillcolor{currentfill}%
\pgfsetlinewidth{0.803000pt}%
\definecolor{currentstroke}{rgb}{0.000000,0.000000,0.000000}%
\pgfsetstrokecolor{currentstroke}%
\pgfsetdash{}{0pt}%
\pgfsys@defobject{currentmarker}{\pgfqpoint{-0.048611in}{0.000000in}}{\pgfqpoint{0.000000in}{0.000000in}}{%
\pgfpathmoveto{\pgfqpoint{0.000000in}{0.000000in}}%
\pgfpathlineto{\pgfqpoint{-0.048611in}{0.000000in}}%
\pgfusepath{stroke,fill}%
}%
\begin{pgfscope}%
\pgfsys@transformshift{1.012500in}{4.058097in}%
\pgfsys@useobject{currentmarker}{}%
\end{pgfscope}%
\end{pgfscope}%
\begin{pgfscope}%
\pgftext[x=0.706944in,y=4.005336in,left,base]{\sffamily\fontsize{10.000000}{12.000000}\selectfont \(\displaystyle 150\)}%
\end{pgfscope}%
\begin{pgfscope}%
\pgfsetbuttcap%
\pgfsetroundjoin%
\definecolor{currentfill}{rgb}{0.000000,0.000000,0.000000}%
\pgfsetfillcolor{currentfill}%
\pgfsetlinewidth{0.803000pt}%
\definecolor{currentstroke}{rgb}{0.000000,0.000000,0.000000}%
\pgfsetstrokecolor{currentstroke}%
\pgfsetdash{}{0pt}%
\pgfsys@defobject{currentmarker}{\pgfqpoint{-0.048611in}{0.000000in}}{\pgfqpoint{0.000000in}{0.000000in}}{%
\pgfpathmoveto{\pgfqpoint{0.000000in}{0.000000in}}%
\pgfpathlineto{\pgfqpoint{-0.048611in}{0.000000in}}%
\pgfusepath{stroke,fill}%
}%
\begin{pgfscope}%
\pgfsys@transformshift{1.012500in}{4.264883in}%
\pgfsys@useobject{currentmarker}{}%
\end{pgfscope}%
\end{pgfscope}%
\begin{pgfscope}%
\pgftext[x=0.706944in,y=4.212121in,left,base]{\sffamily\fontsize{10.000000}{12.000000}\selectfont \(\displaystyle 160\)}%
\end{pgfscope}%
\begin{pgfscope}%
\pgfsetbuttcap%
\pgfsetroundjoin%
\definecolor{currentfill}{rgb}{0.000000,0.000000,0.000000}%
\pgfsetfillcolor{currentfill}%
\pgfsetlinewidth{0.803000pt}%
\definecolor{currentstroke}{rgb}{0.000000,0.000000,0.000000}%
\pgfsetstrokecolor{currentstroke}%
\pgfsetdash{}{0pt}%
\pgfsys@defobject{currentmarker}{\pgfqpoint{-0.048611in}{0.000000in}}{\pgfqpoint{0.000000in}{0.000000in}}{%
\pgfpathmoveto{\pgfqpoint{0.000000in}{0.000000in}}%
\pgfpathlineto{\pgfqpoint{-0.048611in}{0.000000in}}%
\pgfusepath{stroke,fill}%
}%
\begin{pgfscope}%
\pgfsys@transformshift{1.012500in}{4.471668in}%
\pgfsys@useobject{currentmarker}{}%
\end{pgfscope}%
\end{pgfscope}%
\begin{pgfscope}%
\pgftext[x=0.706944in,y=4.418907in,left,base]{\sffamily\fontsize{10.000000}{12.000000}\selectfont \(\displaystyle 170\)}%
\end{pgfscope}%
\begin{pgfscope}%
\pgfsetbuttcap%
\pgfsetroundjoin%
\definecolor{currentfill}{rgb}{0.000000,0.000000,0.000000}%
\pgfsetfillcolor{currentfill}%
\pgfsetlinewidth{0.803000pt}%
\definecolor{currentstroke}{rgb}{0.000000,0.000000,0.000000}%
\pgfsetstrokecolor{currentstroke}%
\pgfsetdash{}{0pt}%
\pgfsys@defobject{currentmarker}{\pgfqpoint{-0.048611in}{0.000000in}}{\pgfqpoint{0.000000in}{0.000000in}}{%
\pgfpathmoveto{\pgfqpoint{0.000000in}{0.000000in}}%
\pgfpathlineto{\pgfqpoint{-0.048611in}{0.000000in}}%
\pgfusepath{stroke,fill}%
}%
\begin{pgfscope}%
\pgfsys@transformshift{1.012500in}{4.678454in}%
\pgfsys@useobject{currentmarker}{}%
\end{pgfscope}%
\end{pgfscope}%
\begin{pgfscope}%
\pgftext[x=0.706944in,y=4.625693in,left,base]{\sffamily\fontsize{10.000000}{12.000000}\selectfont \(\displaystyle 180\)}%
\end{pgfscope}%
\begin{pgfscope}%
\pgfsetbuttcap%
\pgfsetroundjoin%
\definecolor{currentfill}{rgb}{0.000000,0.000000,0.000000}%
\pgfsetfillcolor{currentfill}%
\pgfsetlinewidth{0.803000pt}%
\definecolor{currentstroke}{rgb}{0.000000,0.000000,0.000000}%
\pgfsetstrokecolor{currentstroke}%
\pgfsetdash{}{0pt}%
\pgfsys@defobject{currentmarker}{\pgfqpoint{-0.048611in}{0.000000in}}{\pgfqpoint{0.000000in}{0.000000in}}{%
\pgfpathmoveto{\pgfqpoint{0.000000in}{0.000000in}}%
\pgfpathlineto{\pgfqpoint{-0.048611in}{0.000000in}}%
\pgfusepath{stroke,fill}%
}%
\begin{pgfscope}%
\pgfsys@transformshift{1.012500in}{4.885240in}%
\pgfsys@useobject{currentmarker}{}%
\end{pgfscope}%
\end{pgfscope}%
\begin{pgfscope}%
\pgftext[x=0.706944in,y=4.832478in,left,base]{\sffamily\fontsize{10.000000}{12.000000}\selectfont \(\displaystyle 190\)}%
\end{pgfscope}%
\begin{pgfscope}%
\pgfsetbuttcap%
\pgfsetroundjoin%
\definecolor{currentfill}{rgb}{0.000000,0.000000,0.000000}%
\pgfsetfillcolor{currentfill}%
\pgfsetlinewidth{0.803000pt}%
\definecolor{currentstroke}{rgb}{0.000000,0.000000,0.000000}%
\pgfsetstrokecolor{currentstroke}%
\pgfsetdash{}{0pt}%
\pgfsys@defobject{currentmarker}{\pgfqpoint{-0.048611in}{0.000000in}}{\pgfqpoint{0.000000in}{0.000000in}}{%
\pgfpathmoveto{\pgfqpoint{0.000000in}{0.000000in}}%
\pgfpathlineto{\pgfqpoint{-0.048611in}{0.000000in}}%
\pgfusepath{stroke,fill}%
}%
\begin{pgfscope}%
\pgfsys@transformshift{1.012500in}{5.092026in}%
\pgfsys@useobject{currentmarker}{}%
\end{pgfscope}%
\end{pgfscope}%
\begin{pgfscope}%
\pgftext[x=0.706944in,y=5.039264in,left,base]{\sffamily\fontsize{10.000000}{12.000000}\selectfont \(\displaystyle 200\)}%
\end{pgfscope}%
\begin{pgfscope}%
\pgfsetbuttcap%
\pgfsetroundjoin%
\definecolor{currentfill}{rgb}{0.000000,0.000000,0.000000}%
\pgfsetfillcolor{currentfill}%
\pgfsetlinewidth{0.803000pt}%
\definecolor{currentstroke}{rgb}{0.000000,0.000000,0.000000}%
\pgfsetstrokecolor{currentstroke}%
\pgfsetdash{}{0pt}%
\pgfsys@defobject{currentmarker}{\pgfqpoint{-0.048611in}{0.000000in}}{\pgfqpoint{0.000000in}{0.000000in}}{%
\pgfpathmoveto{\pgfqpoint{0.000000in}{0.000000in}}%
\pgfpathlineto{\pgfqpoint{-0.048611in}{0.000000in}}%
\pgfusepath{stroke,fill}%
}%
\begin{pgfscope}%
\pgfsys@transformshift{1.012500in}{5.298811in}%
\pgfsys@useobject{currentmarker}{}%
\end{pgfscope}%
\end{pgfscope}%
\begin{pgfscope}%
\pgftext[x=0.706944in,y=5.246050in,left,base]{\sffamily\fontsize{10.000000}{12.000000}\selectfont \(\displaystyle 210\)}%
\end{pgfscope}%
\begin{pgfscope}%
\pgfsetbuttcap%
\pgfsetroundjoin%
\definecolor{currentfill}{rgb}{0.000000,0.000000,0.000000}%
\pgfsetfillcolor{currentfill}%
\pgfsetlinewidth{0.803000pt}%
\definecolor{currentstroke}{rgb}{0.000000,0.000000,0.000000}%
\pgfsetstrokecolor{currentstroke}%
\pgfsetdash{}{0pt}%
\pgfsys@defobject{currentmarker}{\pgfqpoint{-0.048611in}{0.000000in}}{\pgfqpoint{0.000000in}{0.000000in}}{%
\pgfpathmoveto{\pgfqpoint{0.000000in}{0.000000in}}%
\pgfpathlineto{\pgfqpoint{-0.048611in}{0.000000in}}%
\pgfusepath{stroke,fill}%
}%
\begin{pgfscope}%
\pgfsys@transformshift{1.012500in}{5.505597in}%
\pgfsys@useobject{currentmarker}{}%
\end{pgfscope}%
\end{pgfscope}%
\begin{pgfscope}%
\pgftext[x=0.706944in,y=5.452835in,left,base]{\sffamily\fontsize{10.000000}{12.000000}\selectfont \(\displaystyle 220\)}%
\end{pgfscope}%
\begin{pgfscope}%
\pgfsetbuttcap%
\pgfsetroundjoin%
\definecolor{currentfill}{rgb}{0.000000,0.000000,0.000000}%
\pgfsetfillcolor{currentfill}%
\pgfsetlinewidth{0.803000pt}%
\definecolor{currentstroke}{rgb}{0.000000,0.000000,0.000000}%
\pgfsetstrokecolor{currentstroke}%
\pgfsetdash{}{0pt}%
\pgfsys@defobject{currentmarker}{\pgfqpoint{-0.048611in}{0.000000in}}{\pgfqpoint{0.000000in}{0.000000in}}{%
\pgfpathmoveto{\pgfqpoint{0.000000in}{0.000000in}}%
\pgfpathlineto{\pgfqpoint{-0.048611in}{0.000000in}}%
\pgfusepath{stroke,fill}%
}%
\begin{pgfscope}%
\pgfsys@transformshift{1.012500in}{5.712383in}%
\pgfsys@useobject{currentmarker}{}%
\end{pgfscope}%
\end{pgfscope}%
\begin{pgfscope}%
\pgftext[x=0.706944in,y=5.659621in,left,base]{\sffamily\fontsize{10.000000}{12.000000}\selectfont \(\displaystyle 230\)}%
\end{pgfscope}%
\begin{pgfscope}%
\pgftext[x=0.651388in,y=3.267000in,,bottom,rotate=90.000000]{\sffamily\fontsize{12.000000}{14.400000}\selectfont seconds}%
\end{pgfscope}%
\begin{pgfscope}%
\pgfpathrectangle{\pgfqpoint{1.012500in}{0.726000in}}{\pgfqpoint{6.277500in}{5.082000in}}%
\pgfusepath{clip}%
\pgfsetrectcap%
\pgfsetroundjoin%
\pgfsetlinewidth{1.505625pt}%
\definecolor{currentstroke}{rgb}{0.172549,0.243137,0.313725}%
\pgfsetstrokecolor{currentstroke}%
\pgfsetdash{}{0pt}%
\pgfpathmoveto{\pgfqpoint{1.312142in}{0.970076in}}%
\pgfpathlineto{\pgfqpoint{1.611784in}{1.009013in}}%
\pgfpathlineto{\pgfqpoint{1.911426in}{1.080293in}}%
\pgfpathlineto{\pgfqpoint{2.211068in}{1.184390in}}%
\pgfpathlineto{\pgfqpoint{2.510710in}{1.283974in}}%
\pgfpathlineto{\pgfqpoint{2.810352in}{1.427928in}}%
\pgfpathlineto{\pgfqpoint{3.109994in}{1.610385in}}%
\pgfpathlineto{\pgfqpoint{3.409636in}{1.797896in}}%
\pgfpathlineto{\pgfqpoint{3.709278in}{2.054201in}}%
\pgfpathlineto{\pgfqpoint{4.008920in}{2.319644in}}%
\pgfpathlineto{\pgfqpoint{4.308562in}{2.294087in}}%
\pgfpathlineto{\pgfqpoint{4.608204in}{2.474652in}}%
\pgfpathlineto{\pgfqpoint{4.907846in}{2.582923in}}%
\pgfpathlineto{\pgfqpoint{5.207488in}{2.934833in}}%
\pgfpathlineto{\pgfqpoint{5.507130in}{3.347313in}}%
\pgfpathlineto{\pgfqpoint{5.806772in}{3.850360in}}%
\pgfpathlineto{\pgfqpoint{6.106414in}{4.259382in}}%
\pgfpathlineto{\pgfqpoint{6.406056in}{4.379649in}}%
\pgfpathlineto{\pgfqpoint{6.705698in}{4.993616in}}%
\pgfpathlineto{\pgfqpoint{7.005340in}{5.577000in}}%
\pgfusepath{stroke}%
\end{pgfscope}%
\begin{pgfscope}%
\pgfpathrectangle{\pgfqpoint{1.012500in}{0.726000in}}{\pgfqpoint{6.277500in}{5.082000in}}%
\pgfusepath{clip}%
\pgfsetrectcap%
\pgfsetroundjoin%
\pgfsetlinewidth{1.505625pt}%
\definecolor{currentstroke}{rgb}{0.086275,0.627451,0.521569}%
\pgfsetstrokecolor{currentstroke}%
\pgfsetdash{}{0pt}%
\pgfpathmoveto{\pgfqpoint{1.312142in}{0.957000in}}%
\pgfpathlineto{\pgfqpoint{1.611784in}{0.958645in}}%
\pgfpathlineto{\pgfqpoint{1.911426in}{0.961737in}}%
\pgfpathlineto{\pgfqpoint{2.211068in}{0.966354in}}%
\pgfpathlineto{\pgfqpoint{2.510710in}{0.970532in}}%
\pgfpathlineto{\pgfqpoint{2.810352in}{0.976822in}}%
\pgfpathlineto{\pgfqpoint{3.109994in}{0.984729in}}%
\pgfpathlineto{\pgfqpoint{3.409636in}{0.992454in}}%
\pgfpathlineto{\pgfqpoint{3.709278in}{1.004108in}}%
\pgfpathlineto{\pgfqpoint{4.008920in}{1.016118in}}%
\pgfpathlineto{\pgfqpoint{4.308562in}{1.012648in}}%
\pgfpathlineto{\pgfqpoint{4.608204in}{1.019145in}}%
\pgfpathlineto{\pgfqpoint{4.907846in}{1.024302in}}%
\pgfpathlineto{\pgfqpoint{5.207488in}{1.037882in}}%
\pgfpathlineto{\pgfqpoint{5.507130in}{1.055243in}}%
\pgfpathlineto{\pgfqpoint{5.806772in}{1.078164in}}%
\pgfpathlineto{\pgfqpoint{6.106414in}{1.096758in}}%
\pgfpathlineto{\pgfqpoint{6.406056in}{1.100500in}}%
\pgfpathlineto{\pgfqpoint{6.705698in}{1.127999in}}%
\pgfpathlineto{\pgfqpoint{7.005340in}{1.157340in}}%
\pgfusepath{stroke}%
\end{pgfscope}%
\begin{pgfscope}%
\pgfsetrectcap%
\pgfsetmiterjoin%
\pgfsetlinewidth{0.803000pt}%
\definecolor{currentstroke}{rgb}{0.000000,0.000000,0.000000}%
\pgfsetstrokecolor{currentstroke}%
\pgfsetdash{}{0pt}%
\pgfpathmoveto{\pgfqpoint{1.012500in}{0.726000in}}%
\pgfpathlineto{\pgfqpoint{1.012500in}{5.808000in}}%
\pgfusepath{stroke}%
\end{pgfscope}%
\begin{pgfscope}%
\pgfsetrectcap%
\pgfsetmiterjoin%
\pgfsetlinewidth{0.803000pt}%
\definecolor{currentstroke}{rgb}{0.000000,0.000000,0.000000}%
\pgfsetstrokecolor{currentstroke}%
\pgfsetdash{}{0pt}%
\pgfpathmoveto{\pgfqpoint{7.290000in}{0.726000in}}%
\pgfpathlineto{\pgfqpoint{7.290000in}{5.808000in}}%
\pgfusepath{stroke}%
\end{pgfscope}%
\begin{pgfscope}%
\pgfsetrectcap%
\pgfsetmiterjoin%
\pgfsetlinewidth{0.803000pt}%
\definecolor{currentstroke}{rgb}{0.000000,0.000000,0.000000}%
\pgfsetstrokecolor{currentstroke}%
\pgfsetdash{}{0pt}%
\pgfpathmoveto{\pgfqpoint{1.012500in}{0.726000in}}%
\pgfpathlineto{\pgfqpoint{7.290000in}{0.726000in}}%
\pgfusepath{stroke}%
\end{pgfscope}%
\begin{pgfscope}%
\pgfsetrectcap%
\pgfsetmiterjoin%
\pgfsetlinewidth{0.803000pt}%
\definecolor{currentstroke}{rgb}{0.000000,0.000000,0.000000}%
\pgfsetstrokecolor{currentstroke}%
\pgfsetdash{}{0pt}%
\pgfpathmoveto{\pgfqpoint{1.012500in}{5.808000in}}%
\pgfpathlineto{\pgfqpoint{7.290000in}{5.808000in}}%
\pgfusepath{stroke}%
\end{pgfscope}%
\begin{pgfscope}%
\pgfsetbuttcap%
\pgfsetmiterjoin%
\definecolor{currentfill}{rgb}{1.000000,1.000000,1.000000}%
\pgfsetfillcolor{currentfill}%
\pgfsetfillopacity{0.800000}%
\pgfsetlinewidth{1.003750pt}%
\definecolor{currentstroke}{rgb}{0.800000,0.800000,0.800000}%
\pgfsetstrokecolor{currentstroke}%
\pgfsetstrokeopacity{0.800000}%
\pgfsetdash{}{0pt}%
\pgfpathmoveto{\pgfqpoint{1.109722in}{5.289174in}}%
\pgfpathlineto{\pgfqpoint{2.257970in}{5.289174in}}%
\pgfpathquadraticcurveto{\pgfqpoint{2.285748in}{5.289174in}}{\pgfqpoint{2.285748in}{5.316952in}}%
\pgfpathlineto{\pgfqpoint{2.285748in}{5.710778in}}%
\pgfpathquadraticcurveto{\pgfqpoint{2.285748in}{5.738556in}}{\pgfqpoint{2.257970in}{5.738556in}}%
\pgfpathlineto{\pgfqpoint{1.109722in}{5.738556in}}%
\pgfpathquadraticcurveto{\pgfqpoint{1.081944in}{5.738556in}}{\pgfqpoint{1.081944in}{5.710778in}}%
\pgfpathlineto{\pgfqpoint{1.081944in}{5.316952in}}%
\pgfpathquadraticcurveto{\pgfqpoint{1.081944in}{5.289174in}}{\pgfqpoint{1.109722in}{5.289174in}}%
\pgfpathclose%
\pgfusepath{stroke,fill}%
\end{pgfscope}%
\begin{pgfscope}%
\pgfsetrectcap%
\pgfsetroundjoin%
\pgfsetlinewidth{1.505625pt}%
\definecolor{currentstroke}{rgb}{0.172549,0.243137,0.313725}%
\pgfsetstrokecolor{currentstroke}%
\pgfsetdash{}{0pt}%
\pgfpathmoveto{\pgfqpoint{1.137500in}{5.626088in}}%
\pgfpathlineto{\pgfqpoint{1.415278in}{5.626088in}}%
\pgfusepath{stroke}%
\end{pgfscope}%
\begin{pgfscope}%
\pgftext[x=1.526389in,y=5.577477in,left,base]{\sffamily\fontsize{10.000000}{12.000000}\selectfont \textsc{Linclosure}}%
\end{pgfscope}%
\begin{pgfscope}%
\pgfsetrectcap%
\pgfsetroundjoin%
\pgfsetlinewidth{1.505625pt}%
\definecolor{currentstroke}{rgb}{0.086275,0.627451,0.521569}%
\pgfsetstrokecolor{currentstroke}%
\pgfsetdash{}{0pt}%
\pgfpathmoveto{\pgfqpoint{1.137500in}{5.422231in}}%
\pgfpathlineto{\pgfqpoint{1.415278in}{5.422231in}}%
\pgfusepath{stroke}%
\end{pgfscope}%
\begin{pgfscope}%
\pgftext[x=1.526389in,y=5.373620in,left,base]{\sffamily\fontsize{10.000000}{12.000000}\selectfont \textsc{Closure}}%
\end{pgfscope}%
\end{pgfpicture}%
\makeatother%
\endgroup%
}
}
\subfloat[Average time (in $s$), $|\B| = 1000$]{
	\scalebox{0.5}{%% Creator: Matplotlib, PGF backend
%%
%% To include the figure in your LaTeX document, write
%%   \input{<filename>.pgf}
%%
%% Make sure the required packages are loaded in your preamble
%%   \usepackage{pgf}
%%
%% Figures using additional raster images can only be included by \input if
%% they are in the same directory as the main LaTeX file. For loading figures
%% from other directories you can use the `import` package
%%   \usepackage{import}
%% and then include the figures with
%%   \import{<path to file>}{<filename>.pgf}
%%
%% Matplotlib used the following preamble
%%   \usepackage{fontspec}
%%   \setmainfont{DejaVu Serif}
%%   \setsansfont{DejaVu Sans}
%%   \setmonofont{DejaVu Sans Mono}
%%
\begingroup%
\makeatletter%
\begin{pgfpicture}%
\pgfpathrectangle{\pgfpointorigin}{\pgfqpoint{7.140000in}{5.340000in}}%
\pgfusepath{use as bounding box, clip}%
\begin{pgfscope}%
\pgfsetbuttcap%
\pgfsetmiterjoin%
\definecolor{currentfill}{rgb}{1.000000,1.000000,1.000000}%
\pgfsetfillcolor{currentfill}%
\pgfsetlinewidth{0.000000pt}%
\definecolor{currentstroke}{rgb}{1.000000,1.000000,1.000000}%
\pgfsetstrokecolor{currentstroke}%
\pgfsetdash{}{0pt}%
\pgfpathmoveto{\pgfqpoint{0.000000in}{0.000000in}}%
\pgfpathlineto{\pgfqpoint{7.140000in}{0.000000in}}%
\pgfpathlineto{\pgfqpoint{7.140000in}{5.340000in}}%
\pgfpathlineto{\pgfqpoint{0.000000in}{5.340000in}}%
\pgfpathclose%
\pgfusepath{fill}%
\end{pgfscope}%
\begin{pgfscope}%
\pgfsetbuttcap%
\pgfsetmiterjoin%
\definecolor{currentfill}{rgb}{1.000000,1.000000,1.000000}%
\pgfsetfillcolor{currentfill}%
\pgfsetlinewidth{0.000000pt}%
\definecolor{currentstroke}{rgb}{0.000000,0.000000,0.000000}%
\pgfsetstrokecolor{currentstroke}%
\pgfsetstrokeopacity{0.000000}%
\pgfsetdash{}{0pt}%
\pgfpathmoveto{\pgfqpoint{0.892500in}{0.587400in}}%
\pgfpathlineto{\pgfqpoint{6.426000in}{0.587400in}}%
\pgfpathlineto{\pgfqpoint{6.426000in}{4.699200in}}%
\pgfpathlineto{\pgfqpoint{0.892500in}{4.699200in}}%
\pgfpathclose%
\pgfusepath{fill}%
\end{pgfscope}%
\begin{pgfscope}%
\pgfsetbuttcap%
\pgfsetroundjoin%
\definecolor{currentfill}{rgb}{0.000000,0.000000,0.000000}%
\pgfsetfillcolor{currentfill}%
\pgfsetlinewidth{0.803000pt}%
\definecolor{currentstroke}{rgb}{0.000000,0.000000,0.000000}%
\pgfsetstrokecolor{currentstroke}%
\pgfsetdash{}{0pt}%
\pgfsys@defobject{currentmarker}{\pgfqpoint{0.000000in}{-0.048611in}}{\pgfqpoint{0.000000in}{0.000000in}}{%
\pgfpathmoveto{\pgfqpoint{0.000000in}{0.000000in}}%
\pgfpathlineto{\pgfqpoint{0.000000in}{-0.048611in}}%
\pgfusepath{stroke,fill}%
}%
\begin{pgfscope}%
\pgfsys@transformshift{1.041360in}{0.587400in}%
\pgfsys@useobject{currentmarker}{}%
\end{pgfscope}%
\end{pgfscope}%
\begin{pgfscope}%
\pgftext[x=1.041360in,y=0.490178in,,top]{\sffamily\fontsize{10.000000}{12.000000}\selectfont \(\displaystyle 0\)}%
\end{pgfscope}%
\begin{pgfscope}%
\pgfsetbuttcap%
\pgfsetroundjoin%
\definecolor{currentfill}{rgb}{0.000000,0.000000,0.000000}%
\pgfsetfillcolor{currentfill}%
\pgfsetlinewidth{0.803000pt}%
\definecolor{currentstroke}{rgb}{0.000000,0.000000,0.000000}%
\pgfsetstrokecolor{currentstroke}%
\pgfsetdash{}{0pt}%
\pgfsys@defobject{currentmarker}{\pgfqpoint{0.000000in}{-0.048611in}}{\pgfqpoint{0.000000in}{0.000000in}}{%
\pgfpathmoveto{\pgfqpoint{0.000000in}{0.000000in}}%
\pgfpathlineto{\pgfqpoint{0.000000in}{-0.048611in}}%
\pgfusepath{stroke,fill}%
}%
\begin{pgfscope}%
\pgfsys@transformshift{1.554672in}{0.587400in}%
\pgfsys@useobject{currentmarker}{}%
\end{pgfscope}%
\end{pgfscope}%
\begin{pgfscope}%
\pgftext[x=1.554672in,y=0.490178in,,top]{\sffamily\fontsize{10.000000}{12.000000}\selectfont \(\displaystyle 10000\)}%
\end{pgfscope}%
\begin{pgfscope}%
\pgfsetbuttcap%
\pgfsetroundjoin%
\definecolor{currentfill}{rgb}{0.000000,0.000000,0.000000}%
\pgfsetfillcolor{currentfill}%
\pgfsetlinewidth{0.803000pt}%
\definecolor{currentstroke}{rgb}{0.000000,0.000000,0.000000}%
\pgfsetstrokecolor{currentstroke}%
\pgfsetdash{}{0pt}%
\pgfsys@defobject{currentmarker}{\pgfqpoint{0.000000in}{-0.048611in}}{\pgfqpoint{0.000000in}{0.000000in}}{%
\pgfpathmoveto{\pgfqpoint{0.000000in}{0.000000in}}%
\pgfpathlineto{\pgfqpoint{0.000000in}{-0.048611in}}%
\pgfusepath{stroke,fill}%
}%
\begin{pgfscope}%
\pgfsys@transformshift{2.067984in}{0.587400in}%
\pgfsys@useobject{currentmarker}{}%
\end{pgfscope}%
\end{pgfscope}%
\begin{pgfscope}%
\pgftext[x=2.067984in,y=0.490178in,,top]{\sffamily\fontsize{10.000000}{12.000000}\selectfont \(\displaystyle 20000\)}%
\end{pgfscope}%
\begin{pgfscope}%
\pgfsetbuttcap%
\pgfsetroundjoin%
\definecolor{currentfill}{rgb}{0.000000,0.000000,0.000000}%
\pgfsetfillcolor{currentfill}%
\pgfsetlinewidth{0.803000pt}%
\definecolor{currentstroke}{rgb}{0.000000,0.000000,0.000000}%
\pgfsetstrokecolor{currentstroke}%
\pgfsetdash{}{0pt}%
\pgfsys@defobject{currentmarker}{\pgfqpoint{0.000000in}{-0.048611in}}{\pgfqpoint{0.000000in}{0.000000in}}{%
\pgfpathmoveto{\pgfqpoint{0.000000in}{0.000000in}}%
\pgfpathlineto{\pgfqpoint{0.000000in}{-0.048611in}}%
\pgfusepath{stroke,fill}%
}%
\begin{pgfscope}%
\pgfsys@transformshift{2.581295in}{0.587400in}%
\pgfsys@useobject{currentmarker}{}%
\end{pgfscope}%
\end{pgfscope}%
\begin{pgfscope}%
\pgftext[x=2.581295in,y=0.490178in,,top]{\sffamily\fontsize{10.000000}{12.000000}\selectfont \(\displaystyle 30000\)}%
\end{pgfscope}%
\begin{pgfscope}%
\pgfsetbuttcap%
\pgfsetroundjoin%
\definecolor{currentfill}{rgb}{0.000000,0.000000,0.000000}%
\pgfsetfillcolor{currentfill}%
\pgfsetlinewidth{0.803000pt}%
\definecolor{currentstroke}{rgb}{0.000000,0.000000,0.000000}%
\pgfsetstrokecolor{currentstroke}%
\pgfsetdash{}{0pt}%
\pgfsys@defobject{currentmarker}{\pgfqpoint{0.000000in}{-0.048611in}}{\pgfqpoint{0.000000in}{0.000000in}}{%
\pgfpathmoveto{\pgfqpoint{0.000000in}{0.000000in}}%
\pgfpathlineto{\pgfqpoint{0.000000in}{-0.048611in}}%
\pgfusepath{stroke,fill}%
}%
\begin{pgfscope}%
\pgfsys@transformshift{3.094607in}{0.587400in}%
\pgfsys@useobject{currentmarker}{}%
\end{pgfscope}%
\end{pgfscope}%
\begin{pgfscope}%
\pgftext[x=3.094607in,y=0.490178in,,top]{\sffamily\fontsize{10.000000}{12.000000}\selectfont \(\displaystyle 40000\)}%
\end{pgfscope}%
\begin{pgfscope}%
\pgfsetbuttcap%
\pgfsetroundjoin%
\definecolor{currentfill}{rgb}{0.000000,0.000000,0.000000}%
\pgfsetfillcolor{currentfill}%
\pgfsetlinewidth{0.803000pt}%
\definecolor{currentstroke}{rgb}{0.000000,0.000000,0.000000}%
\pgfsetstrokecolor{currentstroke}%
\pgfsetdash{}{0pt}%
\pgfsys@defobject{currentmarker}{\pgfqpoint{0.000000in}{-0.048611in}}{\pgfqpoint{0.000000in}{0.000000in}}{%
\pgfpathmoveto{\pgfqpoint{0.000000in}{0.000000in}}%
\pgfpathlineto{\pgfqpoint{0.000000in}{-0.048611in}}%
\pgfusepath{stroke,fill}%
}%
\begin{pgfscope}%
\pgfsys@transformshift{3.607919in}{0.587400in}%
\pgfsys@useobject{currentmarker}{}%
\end{pgfscope}%
\end{pgfscope}%
\begin{pgfscope}%
\pgftext[x=3.607919in,y=0.490178in,,top]{\sffamily\fontsize{10.000000}{12.000000}\selectfont \(\displaystyle 50000\)}%
\end{pgfscope}%
\begin{pgfscope}%
\pgfsetbuttcap%
\pgfsetroundjoin%
\definecolor{currentfill}{rgb}{0.000000,0.000000,0.000000}%
\pgfsetfillcolor{currentfill}%
\pgfsetlinewidth{0.803000pt}%
\definecolor{currentstroke}{rgb}{0.000000,0.000000,0.000000}%
\pgfsetstrokecolor{currentstroke}%
\pgfsetdash{}{0pt}%
\pgfsys@defobject{currentmarker}{\pgfqpoint{0.000000in}{-0.048611in}}{\pgfqpoint{0.000000in}{0.000000in}}{%
\pgfpathmoveto{\pgfqpoint{0.000000in}{0.000000in}}%
\pgfpathlineto{\pgfqpoint{0.000000in}{-0.048611in}}%
\pgfusepath{stroke,fill}%
}%
\begin{pgfscope}%
\pgfsys@transformshift{4.121231in}{0.587400in}%
\pgfsys@useobject{currentmarker}{}%
\end{pgfscope}%
\end{pgfscope}%
\begin{pgfscope}%
\pgftext[x=4.121231in,y=0.490178in,,top]{\sffamily\fontsize{10.000000}{12.000000}\selectfont \(\displaystyle 60000\)}%
\end{pgfscope}%
\begin{pgfscope}%
\pgfsetbuttcap%
\pgfsetroundjoin%
\definecolor{currentfill}{rgb}{0.000000,0.000000,0.000000}%
\pgfsetfillcolor{currentfill}%
\pgfsetlinewidth{0.803000pt}%
\definecolor{currentstroke}{rgb}{0.000000,0.000000,0.000000}%
\pgfsetstrokecolor{currentstroke}%
\pgfsetdash{}{0pt}%
\pgfsys@defobject{currentmarker}{\pgfqpoint{0.000000in}{-0.048611in}}{\pgfqpoint{0.000000in}{0.000000in}}{%
\pgfpathmoveto{\pgfqpoint{0.000000in}{0.000000in}}%
\pgfpathlineto{\pgfqpoint{0.000000in}{-0.048611in}}%
\pgfusepath{stroke,fill}%
}%
\begin{pgfscope}%
\pgfsys@transformshift{4.634542in}{0.587400in}%
\pgfsys@useobject{currentmarker}{}%
\end{pgfscope}%
\end{pgfscope}%
\begin{pgfscope}%
\pgftext[x=4.634542in,y=0.490178in,,top]{\sffamily\fontsize{10.000000}{12.000000}\selectfont \(\displaystyle 70000\)}%
\end{pgfscope}%
\begin{pgfscope}%
\pgfsetbuttcap%
\pgfsetroundjoin%
\definecolor{currentfill}{rgb}{0.000000,0.000000,0.000000}%
\pgfsetfillcolor{currentfill}%
\pgfsetlinewidth{0.803000pt}%
\definecolor{currentstroke}{rgb}{0.000000,0.000000,0.000000}%
\pgfsetstrokecolor{currentstroke}%
\pgfsetdash{}{0pt}%
\pgfsys@defobject{currentmarker}{\pgfqpoint{0.000000in}{-0.048611in}}{\pgfqpoint{0.000000in}{0.000000in}}{%
\pgfpathmoveto{\pgfqpoint{0.000000in}{0.000000in}}%
\pgfpathlineto{\pgfqpoint{0.000000in}{-0.048611in}}%
\pgfusepath{stroke,fill}%
}%
\begin{pgfscope}%
\pgfsys@transformshift{5.147854in}{0.587400in}%
\pgfsys@useobject{currentmarker}{}%
\end{pgfscope}%
\end{pgfscope}%
\begin{pgfscope}%
\pgftext[x=5.147854in,y=0.490178in,,top]{\sffamily\fontsize{10.000000}{12.000000}\selectfont \(\displaystyle 80000\)}%
\end{pgfscope}%
\begin{pgfscope}%
\pgfsetbuttcap%
\pgfsetroundjoin%
\definecolor{currentfill}{rgb}{0.000000,0.000000,0.000000}%
\pgfsetfillcolor{currentfill}%
\pgfsetlinewidth{0.803000pt}%
\definecolor{currentstroke}{rgb}{0.000000,0.000000,0.000000}%
\pgfsetstrokecolor{currentstroke}%
\pgfsetdash{}{0pt}%
\pgfsys@defobject{currentmarker}{\pgfqpoint{0.000000in}{-0.048611in}}{\pgfqpoint{0.000000in}{0.000000in}}{%
\pgfpathmoveto{\pgfqpoint{0.000000in}{0.000000in}}%
\pgfpathlineto{\pgfqpoint{0.000000in}{-0.048611in}}%
\pgfusepath{stroke,fill}%
}%
\begin{pgfscope}%
\pgfsys@transformshift{5.661166in}{0.587400in}%
\pgfsys@useobject{currentmarker}{}%
\end{pgfscope}%
\end{pgfscope}%
\begin{pgfscope}%
\pgftext[x=5.661166in,y=0.490178in,,top]{\sffamily\fontsize{10.000000}{12.000000}\selectfont \(\displaystyle 90000\)}%
\end{pgfscope}%
\begin{pgfscope}%
\pgfsetbuttcap%
\pgfsetroundjoin%
\definecolor{currentfill}{rgb}{0.000000,0.000000,0.000000}%
\pgfsetfillcolor{currentfill}%
\pgfsetlinewidth{0.803000pt}%
\definecolor{currentstroke}{rgb}{0.000000,0.000000,0.000000}%
\pgfsetstrokecolor{currentstroke}%
\pgfsetdash{}{0pt}%
\pgfsys@defobject{currentmarker}{\pgfqpoint{0.000000in}{-0.048611in}}{\pgfqpoint{0.000000in}{0.000000in}}{%
\pgfpathmoveto{\pgfqpoint{0.000000in}{0.000000in}}%
\pgfpathlineto{\pgfqpoint{0.000000in}{-0.048611in}}%
\pgfusepath{stroke,fill}%
}%
\begin{pgfscope}%
\pgfsys@transformshift{6.174477in}{0.587400in}%
\pgfsys@useobject{currentmarker}{}%
\end{pgfscope}%
\end{pgfscope}%
\begin{pgfscope}%
\pgftext[x=6.174477in,y=0.490178in,,top]{\sffamily\fontsize{10.000000}{12.000000}\selectfont \(\displaystyle 100000\)}%
\end{pgfscope}%
\begin{pgfscope}%
\pgftext[x=3.659250in,y=0.300209in,,top]{\sffamily\fontsize{12.000000}{14.400000}\selectfont \(\displaystyle |\Sigma|\)}%
\end{pgfscope}%
\begin{pgfscope}%
\pgfsetbuttcap%
\pgfsetroundjoin%
\definecolor{currentfill}{rgb}{0.000000,0.000000,0.000000}%
\pgfsetfillcolor{currentfill}%
\pgfsetlinewidth{0.803000pt}%
\definecolor{currentstroke}{rgb}{0.000000,0.000000,0.000000}%
\pgfsetstrokecolor{currentstroke}%
\pgfsetdash{}{0pt}%
\pgfsys@defobject{currentmarker}{\pgfqpoint{-0.048611in}{0.000000in}}{\pgfqpoint{0.000000in}{0.000000in}}{%
\pgfpathmoveto{\pgfqpoint{0.000000in}{0.000000in}}%
\pgfpathlineto{\pgfqpoint{-0.048611in}{0.000000in}}%
\pgfusepath{stroke,fill}%
}%
\begin{pgfscope}%
\pgfsys@transformshift{0.892500in}{0.769417in}%
\pgfsys@useobject{currentmarker}{}%
\end{pgfscope}%
\end{pgfscope}%
\begin{pgfscope}%
\pgftext[x=0.725833in,y=0.716655in,left,base]{\sffamily\fontsize{10.000000}{12.000000}\selectfont \(\displaystyle 0\)}%
\end{pgfscope}%
\begin{pgfscope}%
\pgfsetbuttcap%
\pgfsetroundjoin%
\definecolor{currentfill}{rgb}{0.000000,0.000000,0.000000}%
\pgfsetfillcolor{currentfill}%
\pgfsetlinewidth{0.803000pt}%
\definecolor{currentstroke}{rgb}{0.000000,0.000000,0.000000}%
\pgfsetstrokecolor{currentstroke}%
\pgfsetdash{}{0pt}%
\pgfsys@defobject{currentmarker}{\pgfqpoint{-0.048611in}{0.000000in}}{\pgfqpoint{0.000000in}{0.000000in}}{%
\pgfpathmoveto{\pgfqpoint{0.000000in}{0.000000in}}%
\pgfpathlineto{\pgfqpoint{-0.048611in}{0.000000in}}%
\pgfusepath{stroke,fill}%
}%
\begin{pgfscope}%
\pgfsys@transformshift{0.892500in}{0.923365in}%
\pgfsys@useobject{currentmarker}{}%
\end{pgfscope}%
\end{pgfscope}%
\begin{pgfscope}%
\pgftext[x=0.725833in,y=0.870603in,left,base]{\sffamily\fontsize{10.000000}{12.000000}\selectfont \(\displaystyle 1\)}%
\end{pgfscope}%
\begin{pgfscope}%
\pgfsetbuttcap%
\pgfsetroundjoin%
\definecolor{currentfill}{rgb}{0.000000,0.000000,0.000000}%
\pgfsetfillcolor{currentfill}%
\pgfsetlinewidth{0.803000pt}%
\definecolor{currentstroke}{rgb}{0.000000,0.000000,0.000000}%
\pgfsetstrokecolor{currentstroke}%
\pgfsetdash{}{0pt}%
\pgfsys@defobject{currentmarker}{\pgfqpoint{-0.048611in}{0.000000in}}{\pgfqpoint{0.000000in}{0.000000in}}{%
\pgfpathmoveto{\pgfqpoint{0.000000in}{0.000000in}}%
\pgfpathlineto{\pgfqpoint{-0.048611in}{0.000000in}}%
\pgfusepath{stroke,fill}%
}%
\begin{pgfscope}%
\pgfsys@transformshift{0.892500in}{1.077312in}%
\pgfsys@useobject{currentmarker}{}%
\end{pgfscope}%
\end{pgfscope}%
\begin{pgfscope}%
\pgftext[x=0.725833in,y=1.024551in,left,base]{\sffamily\fontsize{10.000000}{12.000000}\selectfont \(\displaystyle 2\)}%
\end{pgfscope}%
\begin{pgfscope}%
\pgfsetbuttcap%
\pgfsetroundjoin%
\definecolor{currentfill}{rgb}{0.000000,0.000000,0.000000}%
\pgfsetfillcolor{currentfill}%
\pgfsetlinewidth{0.803000pt}%
\definecolor{currentstroke}{rgb}{0.000000,0.000000,0.000000}%
\pgfsetstrokecolor{currentstroke}%
\pgfsetdash{}{0pt}%
\pgfsys@defobject{currentmarker}{\pgfqpoint{-0.048611in}{0.000000in}}{\pgfqpoint{0.000000in}{0.000000in}}{%
\pgfpathmoveto{\pgfqpoint{0.000000in}{0.000000in}}%
\pgfpathlineto{\pgfqpoint{-0.048611in}{0.000000in}}%
\pgfusepath{stroke,fill}%
}%
\begin{pgfscope}%
\pgfsys@transformshift{0.892500in}{1.231260in}%
\pgfsys@useobject{currentmarker}{}%
\end{pgfscope}%
\end{pgfscope}%
\begin{pgfscope}%
\pgftext[x=0.725833in,y=1.178498in,left,base]{\sffamily\fontsize{10.000000}{12.000000}\selectfont \(\displaystyle 3\)}%
\end{pgfscope}%
\begin{pgfscope}%
\pgfsetbuttcap%
\pgfsetroundjoin%
\definecolor{currentfill}{rgb}{0.000000,0.000000,0.000000}%
\pgfsetfillcolor{currentfill}%
\pgfsetlinewidth{0.803000pt}%
\definecolor{currentstroke}{rgb}{0.000000,0.000000,0.000000}%
\pgfsetstrokecolor{currentstroke}%
\pgfsetdash{}{0pt}%
\pgfsys@defobject{currentmarker}{\pgfqpoint{-0.048611in}{0.000000in}}{\pgfqpoint{0.000000in}{0.000000in}}{%
\pgfpathmoveto{\pgfqpoint{0.000000in}{0.000000in}}%
\pgfpathlineto{\pgfqpoint{-0.048611in}{0.000000in}}%
\pgfusepath{stroke,fill}%
}%
\begin{pgfscope}%
\pgfsys@transformshift{0.892500in}{1.385208in}%
\pgfsys@useobject{currentmarker}{}%
\end{pgfscope}%
\end{pgfscope}%
\begin{pgfscope}%
\pgftext[x=0.725833in,y=1.332446in,left,base]{\sffamily\fontsize{10.000000}{12.000000}\selectfont \(\displaystyle 4\)}%
\end{pgfscope}%
\begin{pgfscope}%
\pgfsetbuttcap%
\pgfsetroundjoin%
\definecolor{currentfill}{rgb}{0.000000,0.000000,0.000000}%
\pgfsetfillcolor{currentfill}%
\pgfsetlinewidth{0.803000pt}%
\definecolor{currentstroke}{rgb}{0.000000,0.000000,0.000000}%
\pgfsetstrokecolor{currentstroke}%
\pgfsetdash{}{0pt}%
\pgfsys@defobject{currentmarker}{\pgfqpoint{-0.048611in}{0.000000in}}{\pgfqpoint{0.000000in}{0.000000in}}{%
\pgfpathmoveto{\pgfqpoint{0.000000in}{0.000000in}}%
\pgfpathlineto{\pgfqpoint{-0.048611in}{0.000000in}}%
\pgfusepath{stroke,fill}%
}%
\begin{pgfscope}%
\pgfsys@transformshift{0.892500in}{1.539155in}%
\pgfsys@useobject{currentmarker}{}%
\end{pgfscope}%
\end{pgfscope}%
\begin{pgfscope}%
\pgftext[x=0.725833in,y=1.486394in,left,base]{\sffamily\fontsize{10.000000}{12.000000}\selectfont \(\displaystyle 5\)}%
\end{pgfscope}%
\begin{pgfscope}%
\pgfsetbuttcap%
\pgfsetroundjoin%
\definecolor{currentfill}{rgb}{0.000000,0.000000,0.000000}%
\pgfsetfillcolor{currentfill}%
\pgfsetlinewidth{0.803000pt}%
\definecolor{currentstroke}{rgb}{0.000000,0.000000,0.000000}%
\pgfsetstrokecolor{currentstroke}%
\pgfsetdash{}{0pt}%
\pgfsys@defobject{currentmarker}{\pgfqpoint{-0.048611in}{0.000000in}}{\pgfqpoint{0.000000in}{0.000000in}}{%
\pgfpathmoveto{\pgfqpoint{0.000000in}{0.000000in}}%
\pgfpathlineto{\pgfqpoint{-0.048611in}{0.000000in}}%
\pgfusepath{stroke,fill}%
}%
\begin{pgfscope}%
\pgfsys@transformshift{0.892500in}{1.693103in}%
\pgfsys@useobject{currentmarker}{}%
\end{pgfscope}%
\end{pgfscope}%
\begin{pgfscope}%
\pgftext[x=0.725833in,y=1.640341in,left,base]{\sffamily\fontsize{10.000000}{12.000000}\selectfont \(\displaystyle 6\)}%
\end{pgfscope}%
\begin{pgfscope}%
\pgfsetbuttcap%
\pgfsetroundjoin%
\definecolor{currentfill}{rgb}{0.000000,0.000000,0.000000}%
\pgfsetfillcolor{currentfill}%
\pgfsetlinewidth{0.803000pt}%
\definecolor{currentstroke}{rgb}{0.000000,0.000000,0.000000}%
\pgfsetstrokecolor{currentstroke}%
\pgfsetdash{}{0pt}%
\pgfsys@defobject{currentmarker}{\pgfqpoint{-0.048611in}{0.000000in}}{\pgfqpoint{0.000000in}{0.000000in}}{%
\pgfpathmoveto{\pgfqpoint{0.000000in}{0.000000in}}%
\pgfpathlineto{\pgfqpoint{-0.048611in}{0.000000in}}%
\pgfusepath{stroke,fill}%
}%
\begin{pgfscope}%
\pgfsys@transformshift{0.892500in}{1.847051in}%
\pgfsys@useobject{currentmarker}{}%
\end{pgfscope}%
\end{pgfscope}%
\begin{pgfscope}%
\pgftext[x=0.725833in,y=1.794289in,left,base]{\sffamily\fontsize{10.000000}{12.000000}\selectfont \(\displaystyle 7\)}%
\end{pgfscope}%
\begin{pgfscope}%
\pgfsetbuttcap%
\pgfsetroundjoin%
\definecolor{currentfill}{rgb}{0.000000,0.000000,0.000000}%
\pgfsetfillcolor{currentfill}%
\pgfsetlinewidth{0.803000pt}%
\definecolor{currentstroke}{rgb}{0.000000,0.000000,0.000000}%
\pgfsetstrokecolor{currentstroke}%
\pgfsetdash{}{0pt}%
\pgfsys@defobject{currentmarker}{\pgfqpoint{-0.048611in}{0.000000in}}{\pgfqpoint{0.000000in}{0.000000in}}{%
\pgfpathmoveto{\pgfqpoint{0.000000in}{0.000000in}}%
\pgfpathlineto{\pgfqpoint{-0.048611in}{0.000000in}}%
\pgfusepath{stroke,fill}%
}%
\begin{pgfscope}%
\pgfsys@transformshift{0.892500in}{2.000998in}%
\pgfsys@useobject{currentmarker}{}%
\end{pgfscope}%
\end{pgfscope}%
\begin{pgfscope}%
\pgftext[x=0.725833in,y=1.948237in,left,base]{\sffamily\fontsize{10.000000}{12.000000}\selectfont \(\displaystyle 8\)}%
\end{pgfscope}%
\begin{pgfscope}%
\pgfsetbuttcap%
\pgfsetroundjoin%
\definecolor{currentfill}{rgb}{0.000000,0.000000,0.000000}%
\pgfsetfillcolor{currentfill}%
\pgfsetlinewidth{0.803000pt}%
\definecolor{currentstroke}{rgb}{0.000000,0.000000,0.000000}%
\pgfsetstrokecolor{currentstroke}%
\pgfsetdash{}{0pt}%
\pgfsys@defobject{currentmarker}{\pgfqpoint{-0.048611in}{0.000000in}}{\pgfqpoint{0.000000in}{0.000000in}}{%
\pgfpathmoveto{\pgfqpoint{0.000000in}{0.000000in}}%
\pgfpathlineto{\pgfqpoint{-0.048611in}{0.000000in}}%
\pgfusepath{stroke,fill}%
}%
\begin{pgfscope}%
\pgfsys@transformshift{0.892500in}{2.154946in}%
\pgfsys@useobject{currentmarker}{}%
\end{pgfscope}%
\end{pgfscope}%
\begin{pgfscope}%
\pgftext[x=0.725833in,y=2.102184in,left,base]{\sffamily\fontsize{10.000000}{12.000000}\selectfont \(\displaystyle 9\)}%
\end{pgfscope}%
\begin{pgfscope}%
\pgfsetbuttcap%
\pgfsetroundjoin%
\definecolor{currentfill}{rgb}{0.000000,0.000000,0.000000}%
\pgfsetfillcolor{currentfill}%
\pgfsetlinewidth{0.803000pt}%
\definecolor{currentstroke}{rgb}{0.000000,0.000000,0.000000}%
\pgfsetstrokecolor{currentstroke}%
\pgfsetdash{}{0pt}%
\pgfsys@defobject{currentmarker}{\pgfqpoint{-0.048611in}{0.000000in}}{\pgfqpoint{0.000000in}{0.000000in}}{%
\pgfpathmoveto{\pgfqpoint{0.000000in}{0.000000in}}%
\pgfpathlineto{\pgfqpoint{-0.048611in}{0.000000in}}%
\pgfusepath{stroke,fill}%
}%
\begin{pgfscope}%
\pgfsys@transformshift{0.892500in}{2.308893in}%
\pgfsys@useobject{currentmarker}{}%
\end{pgfscope}%
\end{pgfscope}%
\begin{pgfscope}%
\pgftext[x=0.656388in,y=2.256132in,left,base]{\sffamily\fontsize{10.000000}{12.000000}\selectfont \(\displaystyle 10\)}%
\end{pgfscope}%
\begin{pgfscope}%
\pgfsetbuttcap%
\pgfsetroundjoin%
\definecolor{currentfill}{rgb}{0.000000,0.000000,0.000000}%
\pgfsetfillcolor{currentfill}%
\pgfsetlinewidth{0.803000pt}%
\definecolor{currentstroke}{rgb}{0.000000,0.000000,0.000000}%
\pgfsetstrokecolor{currentstroke}%
\pgfsetdash{}{0pt}%
\pgfsys@defobject{currentmarker}{\pgfqpoint{-0.048611in}{0.000000in}}{\pgfqpoint{0.000000in}{0.000000in}}{%
\pgfpathmoveto{\pgfqpoint{0.000000in}{0.000000in}}%
\pgfpathlineto{\pgfqpoint{-0.048611in}{0.000000in}}%
\pgfusepath{stroke,fill}%
}%
\begin{pgfscope}%
\pgfsys@transformshift{0.892500in}{2.462841in}%
\pgfsys@useobject{currentmarker}{}%
\end{pgfscope}%
\end{pgfscope}%
\begin{pgfscope}%
\pgftext[x=0.656388in,y=2.410080in,left,base]{\sffamily\fontsize{10.000000}{12.000000}\selectfont \(\displaystyle 11\)}%
\end{pgfscope}%
\begin{pgfscope}%
\pgfsetbuttcap%
\pgfsetroundjoin%
\definecolor{currentfill}{rgb}{0.000000,0.000000,0.000000}%
\pgfsetfillcolor{currentfill}%
\pgfsetlinewidth{0.803000pt}%
\definecolor{currentstroke}{rgb}{0.000000,0.000000,0.000000}%
\pgfsetstrokecolor{currentstroke}%
\pgfsetdash{}{0pt}%
\pgfsys@defobject{currentmarker}{\pgfqpoint{-0.048611in}{0.000000in}}{\pgfqpoint{0.000000in}{0.000000in}}{%
\pgfpathmoveto{\pgfqpoint{0.000000in}{0.000000in}}%
\pgfpathlineto{\pgfqpoint{-0.048611in}{0.000000in}}%
\pgfusepath{stroke,fill}%
}%
\begin{pgfscope}%
\pgfsys@transformshift{0.892500in}{2.616789in}%
\pgfsys@useobject{currentmarker}{}%
\end{pgfscope}%
\end{pgfscope}%
\begin{pgfscope}%
\pgftext[x=0.656388in,y=2.564027in,left,base]{\sffamily\fontsize{10.000000}{12.000000}\selectfont \(\displaystyle 12\)}%
\end{pgfscope}%
\begin{pgfscope}%
\pgfsetbuttcap%
\pgfsetroundjoin%
\definecolor{currentfill}{rgb}{0.000000,0.000000,0.000000}%
\pgfsetfillcolor{currentfill}%
\pgfsetlinewidth{0.803000pt}%
\definecolor{currentstroke}{rgb}{0.000000,0.000000,0.000000}%
\pgfsetstrokecolor{currentstroke}%
\pgfsetdash{}{0pt}%
\pgfsys@defobject{currentmarker}{\pgfqpoint{-0.048611in}{0.000000in}}{\pgfqpoint{0.000000in}{0.000000in}}{%
\pgfpathmoveto{\pgfqpoint{0.000000in}{0.000000in}}%
\pgfpathlineto{\pgfqpoint{-0.048611in}{0.000000in}}%
\pgfusepath{stroke,fill}%
}%
\begin{pgfscope}%
\pgfsys@transformshift{0.892500in}{2.770736in}%
\pgfsys@useobject{currentmarker}{}%
\end{pgfscope}%
\end{pgfscope}%
\begin{pgfscope}%
\pgftext[x=0.656388in,y=2.717975in,left,base]{\sffamily\fontsize{10.000000}{12.000000}\selectfont \(\displaystyle 13\)}%
\end{pgfscope}%
\begin{pgfscope}%
\pgfsetbuttcap%
\pgfsetroundjoin%
\definecolor{currentfill}{rgb}{0.000000,0.000000,0.000000}%
\pgfsetfillcolor{currentfill}%
\pgfsetlinewidth{0.803000pt}%
\definecolor{currentstroke}{rgb}{0.000000,0.000000,0.000000}%
\pgfsetstrokecolor{currentstroke}%
\pgfsetdash{}{0pt}%
\pgfsys@defobject{currentmarker}{\pgfqpoint{-0.048611in}{0.000000in}}{\pgfqpoint{0.000000in}{0.000000in}}{%
\pgfpathmoveto{\pgfqpoint{0.000000in}{0.000000in}}%
\pgfpathlineto{\pgfqpoint{-0.048611in}{0.000000in}}%
\pgfusepath{stroke,fill}%
}%
\begin{pgfscope}%
\pgfsys@transformshift{0.892500in}{2.924684in}%
\pgfsys@useobject{currentmarker}{}%
\end{pgfscope}%
\end{pgfscope}%
\begin{pgfscope}%
\pgftext[x=0.656388in,y=2.871923in,left,base]{\sffamily\fontsize{10.000000}{12.000000}\selectfont \(\displaystyle 14\)}%
\end{pgfscope}%
\begin{pgfscope}%
\pgfsetbuttcap%
\pgfsetroundjoin%
\definecolor{currentfill}{rgb}{0.000000,0.000000,0.000000}%
\pgfsetfillcolor{currentfill}%
\pgfsetlinewidth{0.803000pt}%
\definecolor{currentstroke}{rgb}{0.000000,0.000000,0.000000}%
\pgfsetstrokecolor{currentstroke}%
\pgfsetdash{}{0pt}%
\pgfsys@defobject{currentmarker}{\pgfqpoint{-0.048611in}{0.000000in}}{\pgfqpoint{0.000000in}{0.000000in}}{%
\pgfpathmoveto{\pgfqpoint{0.000000in}{0.000000in}}%
\pgfpathlineto{\pgfqpoint{-0.048611in}{0.000000in}}%
\pgfusepath{stroke,fill}%
}%
\begin{pgfscope}%
\pgfsys@transformshift{0.892500in}{3.078632in}%
\pgfsys@useobject{currentmarker}{}%
\end{pgfscope}%
\end{pgfscope}%
\begin{pgfscope}%
\pgftext[x=0.656388in,y=3.025870in,left,base]{\sffamily\fontsize{10.000000}{12.000000}\selectfont \(\displaystyle 15\)}%
\end{pgfscope}%
\begin{pgfscope}%
\pgfsetbuttcap%
\pgfsetroundjoin%
\definecolor{currentfill}{rgb}{0.000000,0.000000,0.000000}%
\pgfsetfillcolor{currentfill}%
\pgfsetlinewidth{0.803000pt}%
\definecolor{currentstroke}{rgb}{0.000000,0.000000,0.000000}%
\pgfsetstrokecolor{currentstroke}%
\pgfsetdash{}{0pt}%
\pgfsys@defobject{currentmarker}{\pgfqpoint{-0.048611in}{0.000000in}}{\pgfqpoint{0.000000in}{0.000000in}}{%
\pgfpathmoveto{\pgfqpoint{0.000000in}{0.000000in}}%
\pgfpathlineto{\pgfqpoint{-0.048611in}{0.000000in}}%
\pgfusepath{stroke,fill}%
}%
\begin{pgfscope}%
\pgfsys@transformshift{0.892500in}{3.232579in}%
\pgfsys@useobject{currentmarker}{}%
\end{pgfscope}%
\end{pgfscope}%
\begin{pgfscope}%
\pgftext[x=0.656388in,y=3.179818in,left,base]{\sffamily\fontsize{10.000000}{12.000000}\selectfont \(\displaystyle 16\)}%
\end{pgfscope}%
\begin{pgfscope}%
\pgfsetbuttcap%
\pgfsetroundjoin%
\definecolor{currentfill}{rgb}{0.000000,0.000000,0.000000}%
\pgfsetfillcolor{currentfill}%
\pgfsetlinewidth{0.803000pt}%
\definecolor{currentstroke}{rgb}{0.000000,0.000000,0.000000}%
\pgfsetstrokecolor{currentstroke}%
\pgfsetdash{}{0pt}%
\pgfsys@defobject{currentmarker}{\pgfqpoint{-0.048611in}{0.000000in}}{\pgfqpoint{0.000000in}{0.000000in}}{%
\pgfpathmoveto{\pgfqpoint{0.000000in}{0.000000in}}%
\pgfpathlineto{\pgfqpoint{-0.048611in}{0.000000in}}%
\pgfusepath{stroke,fill}%
}%
\begin{pgfscope}%
\pgfsys@transformshift{0.892500in}{3.386527in}%
\pgfsys@useobject{currentmarker}{}%
\end{pgfscope}%
\end{pgfscope}%
\begin{pgfscope}%
\pgftext[x=0.656388in,y=3.333765in,left,base]{\sffamily\fontsize{10.000000}{12.000000}\selectfont \(\displaystyle 17\)}%
\end{pgfscope}%
\begin{pgfscope}%
\pgfsetbuttcap%
\pgfsetroundjoin%
\definecolor{currentfill}{rgb}{0.000000,0.000000,0.000000}%
\pgfsetfillcolor{currentfill}%
\pgfsetlinewidth{0.803000pt}%
\definecolor{currentstroke}{rgb}{0.000000,0.000000,0.000000}%
\pgfsetstrokecolor{currentstroke}%
\pgfsetdash{}{0pt}%
\pgfsys@defobject{currentmarker}{\pgfqpoint{-0.048611in}{0.000000in}}{\pgfqpoint{0.000000in}{0.000000in}}{%
\pgfpathmoveto{\pgfqpoint{0.000000in}{0.000000in}}%
\pgfpathlineto{\pgfqpoint{-0.048611in}{0.000000in}}%
\pgfusepath{stroke,fill}%
}%
\begin{pgfscope}%
\pgfsys@transformshift{0.892500in}{3.540475in}%
\pgfsys@useobject{currentmarker}{}%
\end{pgfscope}%
\end{pgfscope}%
\begin{pgfscope}%
\pgftext[x=0.656388in,y=3.487713in,left,base]{\sffamily\fontsize{10.000000}{12.000000}\selectfont \(\displaystyle 18\)}%
\end{pgfscope}%
\begin{pgfscope}%
\pgfsetbuttcap%
\pgfsetroundjoin%
\definecolor{currentfill}{rgb}{0.000000,0.000000,0.000000}%
\pgfsetfillcolor{currentfill}%
\pgfsetlinewidth{0.803000pt}%
\definecolor{currentstroke}{rgb}{0.000000,0.000000,0.000000}%
\pgfsetstrokecolor{currentstroke}%
\pgfsetdash{}{0pt}%
\pgfsys@defobject{currentmarker}{\pgfqpoint{-0.048611in}{0.000000in}}{\pgfqpoint{0.000000in}{0.000000in}}{%
\pgfpathmoveto{\pgfqpoint{0.000000in}{0.000000in}}%
\pgfpathlineto{\pgfqpoint{-0.048611in}{0.000000in}}%
\pgfusepath{stroke,fill}%
}%
\begin{pgfscope}%
\pgfsys@transformshift{0.892500in}{3.694422in}%
\pgfsys@useobject{currentmarker}{}%
\end{pgfscope}%
\end{pgfscope}%
\begin{pgfscope}%
\pgftext[x=0.656388in,y=3.641661in,left,base]{\sffamily\fontsize{10.000000}{12.000000}\selectfont \(\displaystyle 19\)}%
\end{pgfscope}%
\begin{pgfscope}%
\pgfsetbuttcap%
\pgfsetroundjoin%
\definecolor{currentfill}{rgb}{0.000000,0.000000,0.000000}%
\pgfsetfillcolor{currentfill}%
\pgfsetlinewidth{0.803000pt}%
\definecolor{currentstroke}{rgb}{0.000000,0.000000,0.000000}%
\pgfsetstrokecolor{currentstroke}%
\pgfsetdash{}{0pt}%
\pgfsys@defobject{currentmarker}{\pgfqpoint{-0.048611in}{0.000000in}}{\pgfqpoint{0.000000in}{0.000000in}}{%
\pgfpathmoveto{\pgfqpoint{0.000000in}{0.000000in}}%
\pgfpathlineto{\pgfqpoint{-0.048611in}{0.000000in}}%
\pgfusepath{stroke,fill}%
}%
\begin{pgfscope}%
\pgfsys@transformshift{0.892500in}{3.848370in}%
\pgfsys@useobject{currentmarker}{}%
\end{pgfscope}%
\end{pgfscope}%
\begin{pgfscope}%
\pgftext[x=0.656388in,y=3.795608in,left,base]{\sffamily\fontsize{10.000000}{12.000000}\selectfont \(\displaystyle 20\)}%
\end{pgfscope}%
\begin{pgfscope}%
\pgfsetbuttcap%
\pgfsetroundjoin%
\definecolor{currentfill}{rgb}{0.000000,0.000000,0.000000}%
\pgfsetfillcolor{currentfill}%
\pgfsetlinewidth{0.803000pt}%
\definecolor{currentstroke}{rgb}{0.000000,0.000000,0.000000}%
\pgfsetstrokecolor{currentstroke}%
\pgfsetdash{}{0pt}%
\pgfsys@defobject{currentmarker}{\pgfqpoint{-0.048611in}{0.000000in}}{\pgfqpoint{0.000000in}{0.000000in}}{%
\pgfpathmoveto{\pgfqpoint{0.000000in}{0.000000in}}%
\pgfpathlineto{\pgfqpoint{-0.048611in}{0.000000in}}%
\pgfusepath{stroke,fill}%
}%
\begin{pgfscope}%
\pgfsys@transformshift{0.892500in}{4.002318in}%
\pgfsys@useobject{currentmarker}{}%
\end{pgfscope}%
\end{pgfscope}%
\begin{pgfscope}%
\pgftext[x=0.656388in,y=3.949556in,left,base]{\sffamily\fontsize{10.000000}{12.000000}\selectfont \(\displaystyle 21\)}%
\end{pgfscope}%
\begin{pgfscope}%
\pgfsetbuttcap%
\pgfsetroundjoin%
\definecolor{currentfill}{rgb}{0.000000,0.000000,0.000000}%
\pgfsetfillcolor{currentfill}%
\pgfsetlinewidth{0.803000pt}%
\definecolor{currentstroke}{rgb}{0.000000,0.000000,0.000000}%
\pgfsetstrokecolor{currentstroke}%
\pgfsetdash{}{0pt}%
\pgfsys@defobject{currentmarker}{\pgfqpoint{-0.048611in}{0.000000in}}{\pgfqpoint{0.000000in}{0.000000in}}{%
\pgfpathmoveto{\pgfqpoint{0.000000in}{0.000000in}}%
\pgfpathlineto{\pgfqpoint{-0.048611in}{0.000000in}}%
\pgfusepath{stroke,fill}%
}%
\begin{pgfscope}%
\pgfsys@transformshift{0.892500in}{4.156265in}%
\pgfsys@useobject{currentmarker}{}%
\end{pgfscope}%
\end{pgfscope}%
\begin{pgfscope}%
\pgftext[x=0.656388in,y=4.103504in,left,base]{\sffamily\fontsize{10.000000}{12.000000}\selectfont \(\displaystyle 22\)}%
\end{pgfscope}%
\begin{pgfscope}%
\pgfsetbuttcap%
\pgfsetroundjoin%
\definecolor{currentfill}{rgb}{0.000000,0.000000,0.000000}%
\pgfsetfillcolor{currentfill}%
\pgfsetlinewidth{0.803000pt}%
\definecolor{currentstroke}{rgb}{0.000000,0.000000,0.000000}%
\pgfsetstrokecolor{currentstroke}%
\pgfsetdash{}{0pt}%
\pgfsys@defobject{currentmarker}{\pgfqpoint{-0.048611in}{0.000000in}}{\pgfqpoint{0.000000in}{0.000000in}}{%
\pgfpathmoveto{\pgfqpoint{0.000000in}{0.000000in}}%
\pgfpathlineto{\pgfqpoint{-0.048611in}{0.000000in}}%
\pgfusepath{stroke,fill}%
}%
\begin{pgfscope}%
\pgfsys@transformshift{0.892500in}{4.310213in}%
\pgfsys@useobject{currentmarker}{}%
\end{pgfscope}%
\end{pgfscope}%
\begin{pgfscope}%
\pgftext[x=0.656388in,y=4.257451in,left,base]{\sffamily\fontsize{10.000000}{12.000000}\selectfont \(\displaystyle 23\)}%
\end{pgfscope}%
\begin{pgfscope}%
\pgfsetbuttcap%
\pgfsetroundjoin%
\definecolor{currentfill}{rgb}{0.000000,0.000000,0.000000}%
\pgfsetfillcolor{currentfill}%
\pgfsetlinewidth{0.803000pt}%
\definecolor{currentstroke}{rgb}{0.000000,0.000000,0.000000}%
\pgfsetstrokecolor{currentstroke}%
\pgfsetdash{}{0pt}%
\pgfsys@defobject{currentmarker}{\pgfqpoint{-0.048611in}{0.000000in}}{\pgfqpoint{0.000000in}{0.000000in}}{%
\pgfpathmoveto{\pgfqpoint{0.000000in}{0.000000in}}%
\pgfpathlineto{\pgfqpoint{-0.048611in}{0.000000in}}%
\pgfusepath{stroke,fill}%
}%
\begin{pgfscope}%
\pgfsys@transformshift{0.892500in}{4.464161in}%
\pgfsys@useobject{currentmarker}{}%
\end{pgfscope}%
\end{pgfscope}%
\begin{pgfscope}%
\pgftext[x=0.656388in,y=4.411399in,left,base]{\sffamily\fontsize{10.000000}{12.000000}\selectfont \(\displaystyle 24\)}%
\end{pgfscope}%
\begin{pgfscope}%
\pgfsetbuttcap%
\pgfsetroundjoin%
\definecolor{currentfill}{rgb}{0.000000,0.000000,0.000000}%
\pgfsetfillcolor{currentfill}%
\pgfsetlinewidth{0.803000pt}%
\definecolor{currentstroke}{rgb}{0.000000,0.000000,0.000000}%
\pgfsetstrokecolor{currentstroke}%
\pgfsetdash{}{0pt}%
\pgfsys@defobject{currentmarker}{\pgfqpoint{-0.048611in}{0.000000in}}{\pgfqpoint{0.000000in}{0.000000in}}{%
\pgfpathmoveto{\pgfqpoint{0.000000in}{0.000000in}}%
\pgfpathlineto{\pgfqpoint{-0.048611in}{0.000000in}}%
\pgfusepath{stroke,fill}%
}%
\begin{pgfscope}%
\pgfsys@transformshift{0.892500in}{4.618108in}%
\pgfsys@useobject{currentmarker}{}%
\end{pgfscope}%
\end{pgfscope}%
\begin{pgfscope}%
\pgftext[x=0.656388in,y=4.565347in,left,base]{\sffamily\fontsize{10.000000}{12.000000}\selectfont \(\displaystyle 25\)}%
\end{pgfscope}%
\begin{pgfscope}%
\pgftext[x=0.600833in,y=2.643300in,,bottom,rotate=90.000000]{\sffamily\fontsize{12.000000}{14.400000}\selectfont seconds}%
\end{pgfscope}%
\begin{pgfscope}%
\pgfpathrectangle{\pgfqpoint{0.892500in}{0.587400in}}{\pgfqpoint{5.533500in}{4.111800in}}%
\pgfusepath{clip}%
\pgfsetrectcap%
\pgfsetroundjoin%
\pgfsetlinewidth{1.505625pt}%
\definecolor{currentstroke}{rgb}{0.172549,0.243137,0.313725}%
\pgfsetstrokecolor{currentstroke}%
\pgfsetdash{}{0pt}%
\pgfpathmoveto{\pgfqpoint{1.144023in}{1.473390in}}%
\pgfpathlineto{\pgfqpoint{1.246685in}{1.509978in}}%
\pgfpathlineto{\pgfqpoint{1.349347in}{1.552530in}}%
\pgfpathlineto{\pgfqpoint{1.452010in}{1.490252in}}%
\pgfpathlineto{\pgfqpoint{1.554672in}{1.519575in}}%
\pgfpathlineto{\pgfqpoint{1.657334in}{1.651224in}}%
\pgfpathlineto{\pgfqpoint{1.759997in}{1.685646in}}%
\pgfpathlineto{\pgfqpoint{1.862659in}{1.714343in}}%
\pgfpathlineto{\pgfqpoint{1.965321in}{1.757064in}}%
\pgfpathlineto{\pgfqpoint{2.067984in}{1.875940in}}%
\pgfpathlineto{\pgfqpoint{2.170646in}{1.885128in}}%
\pgfpathlineto{\pgfqpoint{2.273308in}{1.966913in}}%
\pgfpathlineto{\pgfqpoint{2.375971in}{2.045498in}}%
\pgfpathlineto{\pgfqpoint{2.478633in}{2.175416in}}%
\pgfpathlineto{\pgfqpoint{2.581295in}{2.221504in}}%
\pgfpathlineto{\pgfqpoint{2.683958in}{2.155908in}}%
\pgfpathlineto{\pgfqpoint{2.786620in}{2.185255in}}%
\pgfpathlineto{\pgfqpoint{2.889282in}{2.244668in}}%
\pgfpathlineto{\pgfqpoint{2.991945in}{2.365084in}}%
\pgfpathlineto{\pgfqpoint{3.094607in}{2.441596in}}%
\pgfpathlineto{\pgfqpoint{3.197269in}{2.544895in}}%
\pgfpathlineto{\pgfqpoint{3.299932in}{2.563677in}}%
\pgfpathlineto{\pgfqpoint{3.402594in}{2.666853in}}%
\pgfpathlineto{\pgfqpoint{3.505256in}{2.783129in}}%
\pgfpathlineto{\pgfqpoint{3.607919in}{2.758852in}}%
\pgfpathlineto{\pgfqpoint{3.710581in}{2.962540in}}%
\pgfpathlineto{\pgfqpoint{3.813244in}{2.945005in}}%
\pgfpathlineto{\pgfqpoint{3.915906in}{3.122922in}}%
\pgfpathlineto{\pgfqpoint{4.018568in}{3.204900in}}%
\pgfpathlineto{\pgfqpoint{4.121231in}{3.343591in}}%
\pgfpathlineto{\pgfqpoint{4.223893in}{3.351073in}}%
\pgfpathlineto{\pgfqpoint{4.326555in}{3.163226in}}%
\pgfpathlineto{\pgfqpoint{4.429218in}{3.238275in}}%
\pgfpathlineto{\pgfqpoint{4.531880in}{3.321392in}}%
\pgfpathlineto{\pgfqpoint{4.634542in}{3.390083in}}%
\pgfpathlineto{\pgfqpoint{4.737205in}{3.465471in}}%
\pgfpathlineto{\pgfqpoint{4.839867in}{3.508577in}}%
\pgfpathlineto{\pgfqpoint{4.942529in}{3.609536in}}%
\pgfpathlineto{\pgfqpoint{5.045192in}{3.676903in}}%
\pgfpathlineto{\pgfqpoint{5.147854in}{3.746287in}}%
\pgfpathlineto{\pgfqpoint{5.250516in}{3.864534in}}%
\pgfpathlineto{\pgfqpoint{5.353179in}{3.918339in}}%
\pgfpathlineto{\pgfqpoint{5.455841in}{3.999839in}}%
\pgfpathlineto{\pgfqpoint{5.558503in}{4.075781in}}%
\pgfpathlineto{\pgfqpoint{5.661166in}{4.158236in}}%
\pgfpathlineto{\pgfqpoint{5.763828in}{4.214504in}}%
\pgfpathlineto{\pgfqpoint{5.866490in}{4.300253in}}%
\pgfpathlineto{\pgfqpoint{5.969153in}{4.401134in}}%
\pgfpathlineto{\pgfqpoint{6.071815in}{4.470796in}}%
\pgfpathlineto{\pgfqpoint{6.174477in}{4.512300in}}%
\pgfusepath{stroke}%
\end{pgfscope}%
\begin{pgfscope}%
\pgfpathrectangle{\pgfqpoint{0.892500in}{0.587400in}}{\pgfqpoint{5.533500in}{4.111800in}}%
\pgfusepath{clip}%
\pgfsetrectcap%
\pgfsetroundjoin%
\pgfsetlinewidth{1.505625pt}%
\definecolor{currentstroke}{rgb}{0.086275,0.627451,0.521569}%
\pgfsetstrokecolor{currentstroke}%
\pgfsetdash{}{0pt}%
\pgfpathmoveto{\pgfqpoint{1.144023in}{0.782503in}}%
\pgfpathlineto{\pgfqpoint{1.246685in}{0.781829in}}%
\pgfpathlineto{\pgfqpoint{1.349347in}{0.782286in}}%
\pgfpathlineto{\pgfqpoint{1.452010in}{0.779279in}}%
\pgfpathlineto{\pgfqpoint{1.554672in}{0.778437in}}%
\pgfpathlineto{\pgfqpoint{1.657334in}{0.778846in}}%
\pgfpathlineto{\pgfqpoint{1.759997in}{0.777716in}}%
\pgfpathlineto{\pgfqpoint{1.862659in}{0.776609in}}%
\pgfpathlineto{\pgfqpoint{1.965321in}{0.776104in}}%
\pgfpathlineto{\pgfqpoint{2.067984in}{0.776393in}}%
\pgfpathlineto{\pgfqpoint{2.170646in}{0.775022in}}%
\pgfpathlineto{\pgfqpoint{2.273308in}{0.774853in}}%
\pgfpathlineto{\pgfqpoint{2.375971in}{0.774853in}}%
\pgfpathlineto{\pgfqpoint{2.478633in}{0.775214in}}%
\pgfpathlineto{\pgfqpoint{2.581295in}{0.775286in}}%
\pgfpathlineto{\pgfqpoint{2.683958in}{0.774468in}}%
\pgfpathlineto{\pgfqpoint{2.786620in}{0.774468in}}%
\pgfpathlineto{\pgfqpoint{2.889282in}{0.774396in}}%
\pgfpathlineto{\pgfqpoint{2.991945in}{0.774324in}}%
\pgfpathlineto{\pgfqpoint{3.094607in}{0.774781in}}%
\pgfpathlineto{\pgfqpoint{3.197269in}{0.775407in}}%
\pgfpathlineto{\pgfqpoint{3.299932in}{0.774300in}}%
\pgfpathlineto{\pgfqpoint{3.402594in}{0.774613in}}%
\pgfpathlineto{\pgfqpoint{3.505256in}{0.775310in}}%
\pgfpathlineto{\pgfqpoint{3.607919in}{0.774492in}}%
\pgfpathlineto{\pgfqpoint{3.710581in}{0.774877in}}%
\pgfpathlineto{\pgfqpoint{3.813244in}{0.775046in}}%
\pgfpathlineto{\pgfqpoint{3.915906in}{0.775166in}}%
\pgfpathlineto{\pgfqpoint{4.018568in}{0.775214in}}%
\pgfpathlineto{\pgfqpoint{4.121231in}{0.775599in}}%
\pgfpathlineto{\pgfqpoint{4.223893in}{0.775070in}}%
\pgfpathlineto{\pgfqpoint{4.326555in}{0.774877in}}%
\pgfpathlineto{\pgfqpoint{4.429218in}{0.774709in}}%
\pgfpathlineto{\pgfqpoint{4.531880in}{0.774709in}}%
\pgfpathlineto{\pgfqpoint{4.634542in}{0.774998in}}%
\pgfpathlineto{\pgfqpoint{4.737205in}{0.775142in}}%
\pgfpathlineto{\pgfqpoint{4.839867in}{0.774926in}}%
\pgfpathlineto{\pgfqpoint{4.942529in}{0.775166in}}%
\pgfpathlineto{\pgfqpoint{5.045192in}{0.775311in}}%
\pgfpathlineto{\pgfqpoint{5.147854in}{0.775311in}}%
\pgfpathlineto{\pgfqpoint{5.250516in}{0.775623in}}%
\pgfpathlineto{\pgfqpoint{5.353179in}{0.775238in}}%
\pgfpathlineto{\pgfqpoint{5.455841in}{0.775527in}}%
\pgfpathlineto{\pgfqpoint{5.558503in}{0.775575in}}%
\pgfpathlineto{\pgfqpoint{5.661166in}{0.775575in}}%
\pgfpathlineto{\pgfqpoint{5.763828in}{0.775744in}}%
\pgfpathlineto{\pgfqpoint{5.866490in}{0.775912in}}%
\pgfpathlineto{\pgfqpoint{5.969153in}{0.776417in}}%
\pgfpathlineto{\pgfqpoint{6.071815in}{0.776321in}}%
\pgfpathlineto{\pgfqpoint{6.174477in}{0.775888in}}%
\pgfusepath{stroke}%
\end{pgfscope}%
\begin{pgfscope}%
\pgfsetrectcap%
\pgfsetmiterjoin%
\pgfsetlinewidth{0.803000pt}%
\definecolor{currentstroke}{rgb}{0.000000,0.000000,0.000000}%
\pgfsetstrokecolor{currentstroke}%
\pgfsetdash{}{0pt}%
\pgfpathmoveto{\pgfqpoint{0.892500in}{0.587400in}}%
\pgfpathlineto{\pgfqpoint{0.892500in}{4.699200in}}%
\pgfusepath{stroke}%
\end{pgfscope}%
\begin{pgfscope}%
\pgfsetrectcap%
\pgfsetmiterjoin%
\pgfsetlinewidth{0.803000pt}%
\definecolor{currentstroke}{rgb}{0.000000,0.000000,0.000000}%
\pgfsetstrokecolor{currentstroke}%
\pgfsetdash{}{0pt}%
\pgfpathmoveto{\pgfqpoint{6.426000in}{0.587400in}}%
\pgfpathlineto{\pgfqpoint{6.426000in}{4.699200in}}%
\pgfusepath{stroke}%
\end{pgfscope}%
\begin{pgfscope}%
\pgfsetrectcap%
\pgfsetmiterjoin%
\pgfsetlinewidth{0.803000pt}%
\definecolor{currentstroke}{rgb}{0.000000,0.000000,0.000000}%
\pgfsetstrokecolor{currentstroke}%
\pgfsetdash{}{0pt}%
\pgfpathmoveto{\pgfqpoint{0.892500in}{0.587400in}}%
\pgfpathlineto{\pgfqpoint{6.426000in}{0.587400in}}%
\pgfusepath{stroke}%
\end{pgfscope}%
\begin{pgfscope}%
\pgfsetrectcap%
\pgfsetmiterjoin%
\pgfsetlinewidth{0.803000pt}%
\definecolor{currentstroke}{rgb}{0.000000,0.000000,0.000000}%
\pgfsetstrokecolor{currentstroke}%
\pgfsetdash{}{0pt}%
\pgfpathmoveto{\pgfqpoint{0.892500in}{4.699200in}}%
\pgfpathlineto{\pgfqpoint{6.426000in}{4.699200in}}%
\pgfusepath{stroke}%
\end{pgfscope}%
\begin{pgfscope}%
\pgfsetbuttcap%
\pgfsetmiterjoin%
\definecolor{currentfill}{rgb}{1.000000,1.000000,1.000000}%
\pgfsetfillcolor{currentfill}%
\pgfsetfillopacity{0.800000}%
\pgfsetlinewidth{1.003750pt}%
\definecolor{currentstroke}{rgb}{0.800000,0.800000,0.800000}%
\pgfsetstrokecolor{currentstroke}%
\pgfsetstrokeopacity{0.800000}%
\pgfsetdash{}{0pt}%
\pgfpathmoveto{\pgfqpoint{0.989722in}{4.180374in}}%
\pgfpathlineto{\pgfqpoint{2.137970in}{4.180374in}}%
\pgfpathquadraticcurveto{\pgfqpoint{2.165748in}{4.180374in}}{\pgfqpoint{2.165748in}{4.208152in}}%
\pgfpathlineto{\pgfqpoint{2.165748in}{4.601978in}}%
\pgfpathquadraticcurveto{\pgfqpoint{2.165748in}{4.629756in}}{\pgfqpoint{2.137970in}{4.629756in}}%
\pgfpathlineto{\pgfqpoint{0.989722in}{4.629756in}}%
\pgfpathquadraticcurveto{\pgfqpoint{0.961944in}{4.629756in}}{\pgfqpoint{0.961944in}{4.601978in}}%
\pgfpathlineto{\pgfqpoint{0.961944in}{4.208152in}}%
\pgfpathquadraticcurveto{\pgfqpoint{0.961944in}{4.180374in}}{\pgfqpoint{0.989722in}{4.180374in}}%
\pgfpathclose%
\pgfusepath{stroke,fill}%
\end{pgfscope}%
\begin{pgfscope}%
\pgfsetrectcap%
\pgfsetroundjoin%
\pgfsetlinewidth{1.505625pt}%
\definecolor{currentstroke}{rgb}{0.172549,0.243137,0.313725}%
\pgfsetstrokecolor{currentstroke}%
\pgfsetdash{}{0pt}%
\pgfpathmoveto{\pgfqpoint{1.017500in}{4.517288in}}%
\pgfpathlineto{\pgfqpoint{1.295278in}{4.517288in}}%
\pgfusepath{stroke}%
\end{pgfscope}%
\begin{pgfscope}%
\pgftext[x=1.406389in,y=4.468677in,left,base]{\sffamily\fontsize{10.000000}{12.000000}\selectfont \textsc{Linclosure}}%
\end{pgfscope}%
\begin{pgfscope}%
\pgfsetrectcap%
\pgfsetroundjoin%
\pgfsetlinewidth{1.505625pt}%
\definecolor{currentstroke}{rgb}{0.086275,0.627451,0.521569}%
\pgfsetstrokecolor{currentstroke}%
\pgfsetdash{}{0pt}%
\pgfpathmoveto{\pgfqpoint{1.017500in}{4.313431in}}%
\pgfpathlineto{\pgfqpoint{1.295278in}{4.313431in}}%
\pgfusepath{stroke}%
\end{pgfscope}%
\begin{pgfscope}%
\pgftext[x=1.406389in,y=4.264820in,left,base]{\sffamily\fontsize{10.000000}{12.000000}\selectfont \textsc{Closure}}%
\end{pgfscope}%
\end{pgfpicture}%
\makeatother%
\endgroup%
}
}

\centering

\subfloat[Some landmarks times, $|\Sg| = 100$]{
\begin{tabular}{| c || c | c |}
	\hline \rowcolor{clouds}
	$|\B|$ & \textsc{LinClosure} & \textsc{Closure} \\ \hline
	5000  & 15.846  & 0.688 \\ \hline
	10000 & 65.930  & 2.724 \\ \hline
	15000 & 115.627 & 4.784 \\ \hline 
	20000 & 223.453 & 9.721 \\ \hline 
\end{tabular}
}\qquad
\subfloat[Some landmarks times]{
	\begin{tabular}{| c || c | c |}
		\hline \rowcolor{clouds}
		$|\Sigma|$ & \textsc{LinClosure} & \textsc{Closure} \\ \hline
		10000  & 4.873  & 0.058 \\ \hline
		20000  & 7.187  & 0.045 \\ \hline
		30000  & 9.432  & 0.038 \\ \hline 
		40000  & 10.862 & 0.034 \\ \hline 
		50000  & 12.923 & 0.032 \\ \hline
		60000  & 16.721 & 0.040 \\ \hline
		70000  & 17.023 & 0.036 \\ \hline
		80000  & 19.337 & 0.038 \\ \hline 
		90000  & 22.013 & 0.040 \\ \hline 
		100000 & 24.312 & 0.042 \\ \hline
	\end{tabular}
}


\caption{Comparison of closure operators for \textsc{MinCover}}
\label{fig:MinCover-Prune}
\input{../Subfiles/Languages.tex}



% header / footer
\fancyhead[C]{Implication theories minimization} 
\fancyhead[L]{HSE - ISIMA}
\fancyhead[R]{Page : \thepage}

\renewcommand{\footrulewidth}{0.5pt}
\fancyfoot[C]{2017 - 2018} 
\fancyfoot[L]{Vilmin}
\fancyfoot[R]{Data Science}

\pagestyle{fancy}


\begin{document}
	
	%\includepdf[pages=-]{Title.pdf}
	


\begin{titlepage}

       \begin{minipage}{0.4\textwidth}
        \begin{flushleft}
            \vspace{-25mm}
            \includegraphics[scale = 0.32]{Pictures/ISIMA.png}
            \textsc{\vspace{1.2em} \small \\ Institut Supérieur d'Informatique,
            	de Modélisation et de leurs Applications \\}
            \textsc{\small \\ $1$ rue de la Chebarde \\}
            \textsc{\small Aubières, $63178$, France \\}
            
        \end{flushleft}
    \end{minipage}
    ~
    \begin{minipage}{0.5\textwidth}
        \vspace{1cm} % mega potard
        \begin{flushright}

            \vspace{-25mm}
            \includegraphics[scale = 0.1]{Pictures/HSE.png}
            \textsc{\vspace{1.2em} \small \\ Higher School of Economics \\}
            \textsc{\small \\ Kochnovskiy Proyezd, $3$ \\}
            \textsc{\small Moscow, $125319$, Russia \\}
            
        \end{flushright}
    \end{minipage}
    
    \vspace{25mm}

    \begin{center}
    
        % "pré-titres"
        
        Master Thesis report: \\
        Data science and 3rd year of computer science 
        engineering \\
        
        % Titre principal de la page, encadré de
        % deux lignes
        
        \TLine \\
        \vspace{1.2em}
        \textsc{\LARGE{Implication systems Minimization: an overview of existing algorithms}}
        \TLine \\
        
        \vspace{5mm}
        \noindent
        
        % Encadrant et Auteurs
        % le ~ sert à superposer les deux minipages.
        
        \begin{minipage}{0.4\textwidth}
            \begin{flushleft}
                \hspace{-20mm}
                \vspace{10mm}
                \textit{\textbf{Author}} : Simon \textsc{Vilmin}\\  
                
                \vspace{4mm}
                
                \hspace{-20mm}
                \emph{Academic Supervisor :} Sergei A. \textsc{Obiedkov} \\
                \vspace{1.2em}
                \hspace{-20mm}
                % \emph{Responsable(s) de l'entreprise :} Julien 
                %\textsc{Fayolle}
                    
            \end{flushleft}
        \end{minipage}
        ~
        \begin{minipage}{0.4\textwidth}
            \begin{flushright}
                \vspace{19mm}
                \emph{Restitution}: May, 21, 2018 \\
                \emph{Held}: June, 6, 2018 \\
                \vspace{1.2em}
                % \emph{Durée du stage : } 720h
            \end{flushright}
        \end{minipage}
        
        \vspace{15mm}
        
        % \Huge{\color{alizarine} CONFIDENTIEL}
        
        \vspace{20mm}
        
        \hspace{-15mm}
        % \includegraphics[scale = 0.58]{Pictures/logo.png}
        
    \end{center}
    
    % logo de l'ubp et de l'isima en fin de page
    
    

    
    % Permet de combler les blancs en fonctions du contenu.
    \vfill
    
    

\end{titlepage}

\begin{comment}

\end{comment}



	
\pagenumbering{roman}

\input{Chapter/Acknowledgement.tex}
\addcontentsline{toc}{chapter}{Acknowledgement}	

Before any discussion, I would like to thank people having helped me within this master thesis. First of all, I am deeply grateful to my academic supervisor Mr. Sergei A. Obiedkov for his availability and precious advices, very interesting discussions and above all his patience with me relentlessly knocking at his door.

I am also thankful toward people having given me feedback on this report.

Eventually, since this master thesis is for me a representative of a year studying in Russia, I am thankful to Ekaterina Pavlova and Sergei Kusnetsov for their support, and my friends here, Betty and Irina for memories they contributed to build.

\newpage
\listoffigures
\newpage

\newpage
\listoftables
\newpage

\newpage
\listofalgorithms
\newpage

\input{Chapter/Abstract.tex}
\addcontentsline{toc}{chapter}{Abstract}

This document is a report of a master thesis made at HSE computer science faculty (Moscow). It has been conducted in the context of a double-diploma program with the ISIMA (France). The topic was implication theories, or Horn, minimization, used in database applications for instance. More precisely the aim was to provide a review of existing algorithms for performing minimization and implement them to see how do they behave under practical test cases.

\vspace{1.2em}

By the end of the period, we found several algorithms and reviewed them within the context of closure systems. On top of providing explanations on their operation and complexity analysis, we implemented those algorithms using C++.
Using randomly generated systems and real data, we determined which closure operator matches the best each algorithm and which algorithm or steps are likely to be used in practice. Those results being valid within the context
of our experiments, suggest further research and experiments to lead in future work.

\vspace{1.2em}

\textbf{Keywords}: theoretical computer science, implications, closure systems, minimization, canonical basis, algorithmic.

\tableofcontents

% Introduction
\input{Chapter/Intro.tex}
\addcontentsline{toc}{chapter}{Introduction}

This document reports a master thesis conducted at the faculty of Computer Science of the Higher School of Economics in Moscow, department of Data Analysis and Artificial Intelligence. It took place within the context of double-diploma ISIMA (France)/HSE in the Data Science master programme, second year.

\vspace{1.2em}

The topic during this master thesis was to review and implement several algorithms for implication theories minimization, also known as Horn minimization. Such systems model knowledge deduction through correlation relations known as implications. The aim of algorithms we studied is to minimize the size of those implicational basis while
keeping all the information they provide. Such relations arise in various applications such as formal concept analysis, relational databases, conceptual exploration linguistics or learning spaces for instance. The present
document is relative to a theoretical study, hence applications will not be extensively discussed.

\vspace{1.2em}

To expose our work, we will stick to an outline divided in three chapters. First, we will review the overhaul definition of our problem and mathematical elements we shall need to understand subsequent algorithms. The second part will be dedicated to the review of algorithms we effectively implemented. In
the last chapter, we will focus on implementation of procedures (in C++) and their experimental results in terms of execution time.

\pagenumbering{arabic}


% Chapter I: Definitions, and so forth
\chapter{Introduction to implications through closure systems}

In this first chapter, we will be involved in presenting our topic of 
minimization. For this ground to be understandable by as much readers as 
possible, we will heavily rely on toy examples to illustrate and provide 
intuition on the various notions we will introduce. To be more precise on the
path we are about to follow in this chapter, we are first to expose an informal
small example of the task we want to achieve. Then, we shall investigate the
history of research on our topic, to act as an exposition of the actual 
knowledge on the question and to give a context to our study. For the rest of
this chapter we will get familiar with mathematical objects called 
\midemp{closure operators} and \midemp{closure systems} modelling our problem.
As we shall observe, the topic of minimization can be described in several 
mathematical frameworks. However, even if we describe briefly other objects
in next chapters, we will stick to our closure framework in all the report in
order to have a leading light among various different terminologies.

\section{Implications and minimization: first meeting}

Let us imagine we are some specialist of flowers and plants in general. As such,
we are interested in studying \belemp{correlations} between plant 
characteristics. Some possible traits are: \textit{colourful, bloom, wither, 
aquatic, seasonal, climbing, scented, flower, perennial} and so forth. Having
observed countless plants during our studies, we are able to draw relations
among all those \belemp{attributes}. For instance, we know that a plant having
the attribute \textit{flower} is likely to have traits \textit{scent, bloom, 
wither} while a plant being \textit{perennial} (i.e: does not need a lot of
water to survive, like a cactus) is not likely to be \textit{aquatic}. 

\vspace{1.2em}

Those relations \textit{"if we have some attributes, we get those ones too"}
depict correlation between attributes (not cause/consequence!). It is important
to stress on the knowledge those relations bring. They just indicate that 
whenever we have say \textit{flower}, we have also \textit{colourful}. This is 
very different from saying that \textit{because} some plant is a flower, it 
will be colourful. We call those correlation relations \belemp{implication}
and use $ flower \imp colourful$ to denote \textit{"if we have the attribute
flower, then we have colourful"}. Now let us give some implications:

\begin{center}
	\textit{(colourful, bloom $\imp$ seasonal), (colourful, wither $\imp$ 
		seasonal), (bloom $\imp$ wither)}
\end{center}

\noindent All those implications represent a certain amount of knowledge. While
in our example they are not numerous we could imagine having tons of them. Hence
we would wonder whether there is a way to reduce the number of implications 
while keeping all the knowledge they represent. This question is 
\belemp{minimization}. Actually, in our small example we can reduce
the number of implications. Take \textit{(colourful, bloom $\imp$ seasonal)}. 
We can derive this implication relation only with the two other 
ones. Indeed, because a plant \textit{blooming} is likely to \textit{wither} 
(3rd implication), we have \textit{(colourful, bloom $\imp$ wither)}, but since 
we now have \textit{wither} and \textit{colourful} we also have 
\textit{seasonal} (2nd implication). That is,
the implication \textit{(colourful, bloom $\imp$ seasonal)} is useless (or 
\belemp{redundant}) in our context and can be removed. Our set of implications 
will then be smaller, but pointing out the same relations as before. 

\vspace{1.2em}

To summarize, we have seen that out of a set of \belemp{attributes} we can draw
several relations called \belemp{implications} providing some knowledge. We also
realized that sometimes, some implications are not necessary. Consequently, 
the set of implications we are given can be \belemp{minimized} without 
altering the information it contains. This is the topic we were interested 
during this master thesis. In the next section, we will trace back the overhaul
knowledge on this question.

\section{Research on implications theories minimization}

This section is intended to supply the reader with a general overview of the
minimization topic. After a short contextual information, we focus on some 
relevant results on the question by providing references to algorithms and 
properties dedicated to our problem. Eventually, we situate our work within this
context.

\vspace{1.2em}

The question of minimization has been discussed and developed through various 
frameworks, and several computer scientists communities. Notice that in order 
not to make this synthesis too long, we will stay within the context of 
minimization and will not trace the field of implication theories in general. 
For a survey of this domain anyway, the reader should refer to 
\cite{wild_joy_2017}. Also, note that minimality in general terms is not 
unique. Indeed, one can define several type of minimality among implication 
systems. For instance, not only we can define minimality with respect to the 
number of implication within a system (which is our interest) but also with 
respect to the number of attributes in each implications. The former one is 
called \textit{canonical} in relational database field, and \textit{hyperarc 
	minimum} within the graph context. Especially in the graph-theoretic and 
boolean logic settings, one can derive more types of minimality. For general 
introduction to boolean logic notations, we invite the reader to see 
\cite{cori_mathematical_2000}. In terms of propositional logic, implications 
are represented through Horn formulae. Interestingly, the minimization problem 
we are going to consider is the only one being polynomial time solvable. Other 
problems are proved to be NP-Complete or NP-Hard. For more discussion on other 
minimality definitions and their computational complexity, the reader should 
refer to \cite{boros_strong_2017, ausiello_directed_2017, 
	b._ganter_conceptual_2016, ausiello_minimal_1986, wild_joy_2017, 
	boros_horn_1998}. In particular for NP-Completeness in the canonical case, 
	one 
can see \cite{hammer_optimal_1993}. In subsequent explanations, we will refer 
to minimization with respect to the number of implications.

\vspace{1.2em}

To the best of our knowledge, the two first fields in which algorithms and 
properties of minimality arose are Formal Concept Analysis (FCA) (see 
\cite{ganter_formal_1999, 
	ganter_two_2010} for an introduction) and Database Theory (DB) (see 
\cite{maier_theory_1983}). Both sides were developed independently in the early 
80's. For the first domain, characterization of minimality goes to Duquenne and 
Guigues \cite{guigues_j.l_familles_1986}, in which they describe the so-called 
\textit{canonical basis} (also called \textit{Duquenne-Guigues basis} after its 
authors) relying on the notion of pseudo-closed sets. For the database part, 
study of implications is made by Maier through FD's (\cite{maier_theory_1983, 
	david_minimum_1980}). The polynomial time algorithm he gives for 
	minimization 
heavily relies on a fast subroutine discovered by Beeri and Bernstein in 
\cite{beeri_computational_1979}, 1979.

\vspace{1.2em}

From then on, knowledge increased over years and spread out over domains. 
Another algorithm based on a minimality theorem is given by Shock in 1986 
(\cite{shock_computing_1986}). Unfortunately, as we shall see and as already 
discussed by Wild in \cite{wild_computations_1995} the algorithm may not be
correct in general, even though the underlying theorem is. During the same 
period, Ausiello and al. brought the problem to graph-theoretic ground, and 
provided new structure known as \textit{FD-Graph} and algorithm to represent 
and work on implication systems in \cite{ausiello_directed_2017, 
	ausiello_graph_1983, ausiello_minimal_1986}. This approach has been seen in 
graph theory as an extension of the transitive closure in graphs 
(\cite{aho_transitive_2006}), but no consideration equivalent to minimization 
task seems to have been taken beforehand, as far as we know. Still in the 1980 
decade, Ganter expressed the canonical basis formalized by Duquenne and Guigues 
in his paper related to algorithms in FCA, \cite{ganter_two_2010} through 
closure systems, pseudo-closed and quasi-closed sets. Next, Wild 
(\cite{wild_implicational_1989, wild_theory_1994, wild_computations_1995}) 
linked within this set-theoretic framework both the relational databases, 
formal concept analysis and lattice-theoretic approach. In relating those 
fields, he describes an algorithm for minimizing a basis, similar to algorithms 
of Day and, somehow, Shock (resp. \cite{day_lattice_1992},  
\cite{shock_computing_1986}). This framework is the one we will use for our 
study, and can be found in more recent work by Ganter \& Obiedkov in 
\cite{b._ganter_conceptual_2016}. Also, the works of Maier and Duquenne-Guigues 
have been used in the lattice-theoretic context by Day in 
\cite{day_lattice_1992} to derive an algorithm based on congruence relations. 
For in-depth knowledge of implication system within lattice terminology, we can 
see \cite{davey_introduction_2002} as an introduction and 
\cite{bertet_lattices_2016} for a survey. Later, Duquenne proposed some 
variations in Day's work with another algorithm in 
\cite{duquenne_variations_2007}. More recently, Bor\`os and al. by 
working in a boolean logic framework, exhibited a theorem on the size of
canonical basis \cite{boros_exclusive_2010, boros_strong_2017}. They also gave
a general theoretic approach that algorithm should do one way or another on
reduction purpose. Out of these papers, Berczi \& al. derived a new 
minimization procedure based on hypergraphs in \cite{berczi_directed_2017}. 
Furthermore, an algorithm for computing the canonical basis starting from any 
system is given in \cite{b._ganter_conceptual_2016}.

\vspace{1.2em}

Even though the work we are going to cite is not designed to answer this 
question of minimization, it must also be exposed as the algorithm is 
intimately related to DG basis and can be used for base reduction. The paper
of Angluin and al. in query learning, see \cite{angluin_learning_1992}, provides
an algorithm for learning a Horn representation of an unknown initial formula. 
It has been shown later by Ariàs and Alcazar (\cite{arias_canonical_2009}) that
the output of Angluin algorithm was always the Duquennes-Guigues basis.

\vspace{1.2em}

Our purpose with this master thesis is to review and implement as much as 
possible the algorithms we exposed to provide a comparison. This comparison 
shall act as both theoretical and experimental statement of algorithm 
efficiency. As we already mentioned we will focus on closure theory framework.
The reason for this choice is our starting point. Because we start from the
algorithms provided by Wild and because the closure framework is the one we 
are the most familiar with, we focus on clearly explain this terminology with
examples. However, once we will be comfortable with those definitions, we will 
relate other frameworks to our main approach in the next chapter, to explain and
draw parallels with other algorithms. In the next section we will focus on 
theoretical definitions we shall need to understand the algorithms we have 
implemented.

\section{Implications and minimization: theoretic approach}

Here we will dive into mathematical representation of the task we gave
in the first section of this chapter. For the recall, our aim here is to
get familiar with the representation being closest from closure systems.  Most 
of the notions initially come from \cite{guigues_j.l_familles_1986, 
ganter_two_2010, wild_theory_1994,	ganter_formal_1999} but the reader can 
also find more than sufficient explanations in \cite{b._ganter_conceptual_2016, 
wild_joy_2017}. Readers with knowledge in relational databases will recognize 
most of functional dependency notations. The reason is close vicinity between 
implications and functional dependencies. Talking about our needs, we can 
consider them as equivalent notations. Actually, the real-life application our 
set up will be the closest from is FCA (\cite{ganter_formal_1999}) as we shall
see in the last chapter.

\subsection{Implications and closure systems}

The easiest object to project onto mathematical definitions is our attribute
set. For all the report, we fix $\Sg$ to be a set of \belemp{attributes}. 
Usually, we will denote attributes by small letters: \textit{a, b, c, \dots} 
and subsets of $\Sg$ (groups of attributes) will be denoted by capital letters: 
\textit{A, B, C, \dots} We assume the reader to have few background in 
elementary set-theoretic notations. 

\begin{definition}[\midemp{Implication, implication system}] An 
\belemp{implication} over $\Sg$ is a pair $(A, B)$ with $A, B \subseteq \Sg$. 
It is usually denoted by $A \imp B$. A set $\I$ of implications is called an 
\belemp{implication system}, \belemp{implication theory} or 
\belemp{implication(al) base(is)}.
\end{definition}

\noindent Note that given as is, this definition seems to lose the semantic
relation we depicted earlier. But we should keep in mind that in our set up, we
will be given implications more than an attribute set. Hence, implications will
make sense on their own, independently from the attribute set they are drawn 
from. Quickly, remark that implications in logical terms are expressed as
\textit{Horn formulae} giving another of its names to implication theories. 
Also, in $A \imp B$, $A$ is said to be the \belemp{premise} (or \belemp{body}) 
and $B$ the \belemp{conclusion} (\belemp{head}).

\begin{definition}[\midemp{Model}] Let $\I$ be an implication system over 
	$\Sg$, and $M \subseteq \Sg$. Then:
	\begin{itemize}
		\item[(i)] $M$ is a \belemp{model} of an implication $A \imp B$, 
		written 
		$M \models A \imp B$, if $B \subseteq M$ or $A \nsubseteq M$,
		\item[(ii)] $M$ is a \belemp{model} of $\I$ if $M \models A \imp B$ for 
		all
		$A \imp B \in \I$.
	\end{itemize}
	
\end{definition}

\noindent The notion of model may seem disarming at first sight. But $M$ being
a model of $A \imp B$ simply means that, if $A$ is included in $M$, then for
the implication $A \imp B$ to hold in $M$, we must have $B$ in $M$ too. This 
still suits the intuitive notion of premise/conclusion. Placed in the context
of $M$, $A \imp B$ says \textit{"whenever we have A, we must also have B"}.
Reader with some background in mathematical logic should be familiar with the
notation $\models$, denoting semantic entailment, as opposed to $\vdash$ for
syntactic deduction (see \cite{cori_mathematical_2000}). By a fortunate twist of
fate, semantic entailment is our next step:

\begin{definition}[\midemp{Semantic entailment}] We say that an implication 
	$A \imp B$ \belemp{semantically follows} from $\I$, denoted $\I \models A 
	\imp 
	B$, if all models $M$ of $\I$ are models of $A \imp B$.
	
\end{definition} 

\noindent Because next definitions are going to be on a slightly different 
structure, even though closely related to implication systems of course, let us
rest for a while and illustrate our definitions with an example.

\vspace{1.2em}

\paragraph{Example} Consider again our plant properties. Let $\Sg = $ 
\{\textit{colourful, bloom, wither, seasonal, aquatic, perennial, flower, 
	scented}\}. An implication could be \textit{flower $\imp$ scented}, or even
\textit{(bloom, aquatic) $\imp$ colourful} if we get rid off semantic 
interpretations. An implication basis $\I$ is for instance:

\begin{center}
	\textit{(colourful, bloom $\imp$ seasonal), (colourful, wither $\imp$ 
		seasonal), (bloom $\imp$ wither)}
\end{center}

\noindent and $M = $\textit{(colourful, bloom, seasonal)} is a model of 
\textit{colourful, bloom $\imp$ seasonal} because both the head and the body
of the implication belong to $M$. Also, $M$ is not a model of $\I$ because it
is not a model of \textit{bloom $\imp$ wither}. A model of $\I$ could be 
\textit{(bloom, wither)} or even the empty set $\emptyset$.

\vspace{1.2em}

Next definitions are about closure operators, and closure systems. We need 
to ground ourselves in those definitions before returning to implications.
$2^{\Sg}$ is the set of all subsets of $\Sg$, also named the \belemp{power set}
of $\Sg$.

\begin{definition}[\midemp{Closure operator}] Let $\Sg$ be a set and $\phi : 
	2^{\Sg} \imp 2^{\Sg}$ an application on the power set of $\Sg$. $\phi$ is
	a \belemp{closure operator} if $\forall X, Y \subseteq \Sg$:
	\begin{itemize}
		\item[(i)] $X \subseteq \phi(X)$ \midemp{(extensive)},
		\item[(ii)] $X \subseteq Y \imp \phi(X) \subseteq \phi(Y)$
		\midemp{(monotone)},
		\item[(iii)] $\phi(X) = \phi(\phi(X))$ \midemp{(idempotent)}.
	\end{itemize}
	$X \subseteq \Sg$ is called \belemp{closed} if $X = \phi(X)$.
\end{definition}

\begin{definition}[\midemp{Closure system}] Let $\Sg$ be a set, and $\Sg^{\phi}
	\subseteq 2^{\Sg}$. $\Sg^{\phi}$ is called a \belemp{closure system} if:
	\begin{itemize}
		\item[(i)] $\Sg \in \Sg^{\phi}$,
		\item[(ii)] if $\cal{S} \subseteq \Sg^{\phi}$, then $\bigcap \cal{S} 
		\in 
		\Sg^{\phi}$ \quad \midemp{(closed under intersection)}.
	\end{itemize}
	
\end{definition}

\noindent In the second definition, it is worth stressing on the fact that
$\Sg^{\phi}$ is a set of sets. Also, the notation $\Sg^{\phi}$ may seem 
surprising, but it has been chosen purposefully. Indeed, to each closure system
$\Sg^{\phi}$ over $\Sg$, we can associate a closure operator $\phi$ and 
vice-versa:
\begin{itemize}
	\item from $\phi$ to $\Sg^{\phi}$: compute all closed sets of $\phi$ to
	obtain $\Sg^{\phi}$,
	\item from $\Sg^{\phi}$ to $\phi$: define $\phi(X)$ as the smallest element
	of $\Sg^{\phi}$ (inclusion-wise) containing $X$. Observe that such a set
	always exists in $\Sg^{\phi}$ because $\Sg \in \Sg^{\phi}$.
\end{itemize}
 In any case, this notation used for clear exposition of the link
between closure systems and closure operators will be adapted to our context
of implication systems as we shall see later on. Notice that one can encounter 
another object, \belemp{closure space}, being a pair ($\Sg$, $\phi$) where
$\Sg$ is a set and $\phi$ a closure operator over $\Sg$. We are likely to find
this notation notably in \cite{wild_implicational_1989, 
	wild_theory_1994} where a general theory of closure spaces is addressed.

\vspace{1.2em}

\paragraph{Example} Let us imagine we have four people: \textit{Jezabel, Neige, 
Seraphin} and \textit{Narcisse}. Let us assume they all know each other and then
define a relation \textit{"like"} between them. For instance, say 
\textit{Séraphin likes Jezabel}. this relation is a \belemp{binary relation}: 
it relates pairs of elements. We can represent this relation by a graph where 
nodes are people and edges represent relations:

\begin{figure}[ht]
	\centering

\begin{tikzpicture}

\node[Vertex, label=above left:{$Neige$}] (ng) at (-1, 1) {};
\node[Vertex, label=below left:{$Seraphin$}] (s) at (-1, -1) {};
\node[Vertex, label=above right:{$Narcisse$}] (na) at (1, 1) {};
\node[Vertex, label=below right:{$Jezabel$}] (j) at (1, -1) {};

\draw[->] (ng) to[bend right=20] (j);
\draw[->] (s) -- (j);
\draw[->] (s) -- (ng);
\draw[->] (s) -- (na);
\draw[->] (j) -- (na);
\draw[->] (j) to[bend right=20] (ng);
\draw[->] (na) to[in=45, out=135, loop] (na);


\end{tikzpicture}

\caption{Graph of "like" relation}
\label{fig:I-love}
\end{figure}

The arrow from \textit{Seraphin} to \textit{Jezabel} stands for 
\textit{"Seraphin likes Jezabel"} and the arrow from \textit{Narcisse} to itself
means equivalently \textit{"Narcisse likes Narcisse"}. With this clear, let 
us introduce an operation of gathering people. Starting from any group $A$ of 
persons presented here, let's add to $A$ every person liked by at least one 
element of $A$, until we can no more add people. For instance:
\begin{itemize}
	\item[-] if we start from \textit{Neige}, because \textit{Neige} likes
	\textit{Jezabel} and \textit{Jezabel} likes \textit{Narcisse} we will add
	both of them to the group of \textit{Neige},
	\item[-] because \textit{Narcisse} only likes himself, we have no people
	to add in his group.
\end{itemize}
\noindent Now observe that this operation of gathering people is in fact a
closure operator:
\begin{itemize}
	\item[(i)] it is \midemp{extensive}: starting from any group of people,
	we can only add new ones, hence either the group does not change (e.g: 
	\textit{Narcisse}) or it grows,
	\item[(ii)] it is \midemp{monotone}: if we start from a group $A$ containing
	a group $B$, it is clear that we will at least gather in $A$ all the people
	we would add with $B$,
	\item[(iii)] \midemp{idempotency}: once we added all the people we had to
	reach, then trying to find new people is useless by definition. Hence the
	group will remain the same if we apply our operation once more.
\end{itemize}

\vspace{1.2em}

We are going to get back to our main implication purpose to illustrate the 
notion of closure in our context. It turns out that given a basis
$\I$ over some set $\Sg$, the set of models of $\I$, $\Sg^{\I}$, is a closure 
system. Moreover, the operator $\I : 2^{\Sg} \imp 2^{\Sg}$ associating to a 
subset $X$ of $\Sg$ the smallest model (inclusion wise) containing $X$ is 
a closure operator. Furthermore, the closure system it defines is 
$\Sg^{\I}$. An interesting point is the mathematical computation of 
$\I(X)$ given $\I$ as a set of implications. We rely on 
\cite{wild_implicational_1989, b._ganter_conceptual_2016} to this end. Let 
us define a temporary operation $\circ : 2^{\Sg} \imp 2^{\Sg}$ as follows:

\[ X^{\circ} = 
X \cup \bigcup \{ B \; | \; A \imp B \in \I, \; A \subseteq X \} \]

\noindent Applying this operator up to stability provides $\I(X)$. In other 
words $\I(X) 
= X^{\circ \circ \dots}$. It is clear that we have a finite amount of 
iterations since $X$ cannot grow more than $\Sg$. Readers with background in
logic (see \cite{boros_strong_2017}) or graph theory 
(\cite{berczi_directed_2017}) might see this operation as the marking or 
forward chaining procedure.

\vspace{1.2em}

\paragraph{Example} Let's stick to our vegetable example, but reducing $\Sg$ to 
\{\textit{bloom, flower, colourful} \} (abbreviated \textit{b, f, c}) for the 
sake of simplicity. Furthermore, let $\I =$ \{\textit{((colourful, bloom) 
	$\imp$ flower), (flower $\imp$ bloom)}\}, abbreviated then $cb \imp f$, $f 
\imp b$. For instance, because $f \imp b \in \I$, the smallest model of $\I$ 
containing $f$ is $bf$, and $bf$ is closed. More precisely, the set of closed
sets is the following:

	\[ \Sg^{\I} = \{ \emptyset, \ b, \ c, \ bf, \ bcf \} \]
	
\noindent Pouet.

\begin{comment}
we will place $X$ under $Y$ and draw an arc from $X$ to $Y$, except we do not 
display transitive arcs. This representation is related to partially ordered 
set and is sometimes known as Hasse diagram, see \cite{davey_introduction_2002} 
for more details. We rely on figure \ref{fig:I-def-CS}.

\begin{figure}[ht]
	\input{Pictures/I-def-CS.tex}
\end{figure}

\vspace{1.2em}

On the left side of the picture, one can find the boolean cube associated to 
$\Sg$, or equivalently the power set of $\Sg$ ordered by inclusion. On the 
right, the set of closed sets of $\I$. On the left side, all elements in the 
same cluster have the same closure: it defines an \belemp{equivalence class} 
under $\I$. Each dotted arc corresponds to application 
of the closure operator we described earlier.  
If a cluster contains only one element, this element is closed. This drawing 
shows the relation between a closure operator and its associated system, in 
particular in implication basis context, where the closure describes models. 
Finally, one can graphically note that the set of models is indeed closed 
under intersection. While this representation is graphically appealing, it is
clearly not tractable for larger attribute set: we have to draw
a diagram with an exponential number of elements (one for all $X \in 2^{\Sg}$). 
Thus, all Hasse diagrams we are going to draw only aim at providing some 
intuition of the various notions and not as an efficient representation.

\end{comment}

\vspace{1.2em}

Having presented the main definitions we shall need, we are to investigate 
practical computation of closures and more elaborated structures like the 
canonical basis (or Duquenne-Guigues basis) in the next section.

\begin{comment}

\gls{pseudo-closed set} pouet pouet. \gls{closure} hus hus. \gls{a} and 
\gls{z}.

In this chapter, we will focus on giving definitions of the mathematical 
objects we may need later. As we shall see, each section giving definitions is 
a different way to represent the same idea. Then, we will introduce the Horn 
Minimization task without any algorithm, to provide the reader with some 
intuition and simple examples of our problem. A more detailed description of 
the minimization task and existing solutions or studies will be given in 
chapter 2. 

\vspace{1.2em}

To be more precise, we develop first the main framework we may use to reason:
closure systems over attribute sets. We will often try to think of various 
algorithms and proofs within this framework to have a red line to follow. Then,
we approach propositional logic and few elements of graph theory. To emphasize
the links between various aspects, we will try to use the same notations on 
equivalent notions. 

% BD, Implicational Basis and stuff
\section{Order theory, Closure systems pov}

In this part, we will be considering definitions of set theory, with relation to
closure systems. The reader may refer to \cite{CExp}, \cite{Lat&Ord} for more 
details and further definitions.

\begin{definition}[\belemp{Closure operator}, \belemp{closure system}] Let
$\Sigma$ be a set, and define $\phi : \Sigma \longrightarrow \Sigma$ an
application. $\phi$ is a \belemp{closure} if it has the three following
properties for all $X, Y \subseteq \Sigma$:
\begin{itemize}
	\item[(i)] $X \subseteq \phi(X)$ (\midemp{extensive})
	\item[(ii)] $X \subseteq Y \longrightarrow \phi(X) \subseteq \phi(Y)$ 
		(\midemp{monotone})
	\item[(iii)] $\phi(\phi(X)) = \phi(X)$ (\midemp{idempotent})
\end{itemize}

\noindent Then, the pair $(\Sigma, \phi)$ is called a \belemp{closure system}.
	
\end{definition}


\begin{definition}[\belemp{Closed set}] Let $(\Sigma, \phi)$ be a closure 
system. A subset $X$ of $\Sigma$ is a \belemp{closed set} (with respect to 
$\phi$) if $\phi(X) = X$. We will denote by $\Sigma_{\phi}$ the set of all 
closed sets of $(\Sigma, \phi)$, that is:
	
	\[ \Sigma_{\phi} = \left\{ X \subseteq \Sigma \; | \; \phi(X) = X 
	\right\}  \]

\noindent and $\Sigma_{\phi}$ has the following properties:
\begin{itemize}
	\item[(i)] $\Sigma \in \Sigma_{\phi}$,
	\item[(ii)] if $X, Y \in \Sigma_{\phi}$, so does $X \cap Y$ 
		(\midemp{$\Sigma_{\phi}$ is closed under intersection}).
\end{itemize}
	
\end{definition}

\noindent Note that a closure system can be characterized either by its closure
operator, or by its set of closed sets. In other words, we can derive 
$\Sigma_{\phi}$ from $\phi$, as $\phi$ from $\Sigma_{\phi}$. The closed set 
associated to $X$ is the smallest closed set containing $X$, i.e:

	\[ \phi(X) = \bigcap \{Y \in \Sigma_\phi \; | \; X \subseteq Y \} \]

\noindent Since $\Sigma_{\phi}$ is closed under intersection, the resultant of
the intersection is also a closed set. The set of closed sets ordered by 
inclusion defines a complete lattice as defined previously. 


\paragraph{Example} Let $\Sigma = \llbracket 1 \; ; \; 4 \rrbracket$ and 
$\phi(X) = X \cup \{4 \}$. The pair $(\Sigma, \phi)$ is a closure system whose
closed sets are all the subsets containing 4. Let us prove that $\phi$ is indeed
a closure operator:
\begin{itemize}
	\item[(e)] for all subsets $X$ of $\Sigma$, either $4 \in X$ and 
	$\phi(X) = X \cup \{4 \} = X$ so $X \subseteq \phi(X)$, either $4 \notin X$
	and $X \subseteq X \cup \{4 \} = \phi(X)$.
	\item[(m)]
	\item[(i)] by definition $\phi(X)$ contains 4, so it will not be
	changed by applying $\phi$ another time.
\end{itemize}

\noindent Another interesting definition of $\Sigma_{\phi}$ is:

	\[ \Sigma_{\phi} = \{ \phi(X) \; | \; X \subseteq \Sigma \} \]

\noindent Then, in our example:

	\[ \Sigma_{\phi} = \{ \{ 4\}, \{ 1, 4\}, \{ 2, 4\},
		\{ 3, 4\}, \{ 1, 2, 4\}, \{ 1, 3, 4\},\{ 2, 3, 4\}, 
		\{ 1, 2, 3, 4\} \}
	\]
	
\noindent The lattice $(\Sigma_{\phi}, \subseteq)$ is:



	 
\section{Implication Basis}

\begin{definition}[\belemp{Implication}] Let $\Sg$ be a set, and $A, B 
\subseteq 
\Sg$. An \belemp{implication} over $\Sg$ is a relation/pair between $A$ and 
$B$, denoted by $A \imp B$. $A$ is the \belemp{premise}, $B$ the 
\belemp{conclusion}. 	
\end{definition}

\begin{definition}[\belemp{Model of an implication}] A subset $M \subseteq 
\Sg$ is a model of $A \imp B$, written $M \models A 
\imp B$ if $B \subseteq M \lor A \nsubseteq M$ (think of logical 
approach).
\end{definition}

\begin{definition}[\belemp{Model of an implication basis}] Let $\I$ be
a set of implication (or an \midemp{implication basis}) over $\Sg$. A subset
$M \subseteq \Sg$ is a model of $\I$ if $M \models A \imp
B$ for each $A \imp B$ in $\I$. It is denoted by $M \models \I$.
\end{definition}

The set of models of $\I$ over $\Sg$ describes a closure system $(\Sg, \phi)$. 
Then, An implication $A \imp B$ holds in $\I$ iff $B \subseteq \phi(A)$. 
Conversely, if $(\Sg, \phi)$ is a closure system, then we can define an 
implication basis $\I$ such that the models $M(\I)$ of $\I$ are the closed sets
$\Sg_\phi$ of $(\Sg, \phi)$. An implication $A \imp B$ \belemp{semantically 
follows} from $\I$ if for all $M \subseteq \Sg$ s.t $M \models \I$, we also have
$M \models A \imp B$. Then we write $\I \models A \imp B$.

\begin{definition}[\belemp{Completeness}] $\I$ over $\Sg$ is called 
\belemp{complete} with respect to a closure system if every implication holding
in the closure system follows from $\I$.
	
\end{definition}

Given $\I$, we denote by $\I(M)$ the closure of $M$ in $\I$, that is the 
smallest model of $\I$ containing $M$.

\begin{definition}[\belemp{Properties of $\I$}] Let $\I$ be a set of 
implications over $\Sg$, and $(\Sg, \phi)$ be a closure system. $\I$ is:
\begin{itemize}
	\item[(i)] \belemp{sound} if each implications of $\I$ holds in $\Sg_\phi$
	\item[(ii)] \belemp{complete} if each implications holding in $\Sg_\phi$ 
	follows from $\I$
	\item[(iii)] \belemp{nonredundant} if no implication in $\I$ follows from 
	other implications of $\I$.
\end{itemize}
	
\end{definition}

So our goal is to find such an implication basis which concentrates all possible
knowledge in the minimal number of implications. Finding (and minimizing!) such
an implication set can be very Time-Consuming.

\begin{definition}[\belemp{Pseudo-closed set}] Let $(\Sg, \phi)$ be a closure 
system. A subset $M \subseteq \Sg$ is \belemp{pseudo-closed} iff 
\begin{itemize}
	\item $\phi(M) \neq M$ ($M$ is not closed),
	\item if $Q \subset M$ is pseudo-closed, then $\phi(Q) \subseteq M$.
\end{itemize}

\end{definition}

Note that the empty set $\emptyset$ is pseudo-closed iff it is not closed. One 
can also define pseudo-closed set with help of \midemp{quasi-closed} sets. We
can prove that a pseudo-closed set is some kind of discontinuous structure 
within a lattice.

\begin{definition}[\belemp{Duquenne-Guigues Basis}] Let $(\Sg, \phi)$ be a 
closure system. The \belemp{Duquenne-Guigues} basis or \belemp{canonical} basis
$\I$ is:

	\[ \I = \{ M \imp \phi(M) \; | \; M \subseteq \Sg, \, M \;
	 	\text{pseudo-closed} \} \]
	
\noindent and $\I$ is complete, sound and nonredundant.
\end{definition}


\begin{definition}[\belemp{Preclosed set}] Given a closure system $(\Sg, \phi)$,
a subset $M \subseteq \Sg$ is \belemp{preclosed} iff it contains the closure of
all its proper preclosed subsets.
	
\end{definition}


% Logical POV
\section{Propositional logic and Horn formulas}

This section is dedicated to the introduction of some propositional logic 
notations and notions. Again, we assume the reader has some background in
propositional logic anyway (we are not going to cover the meaning of 
disjunction, conjunction and so forth). The reader can refer to 
\cite{cori_mathematical_2000} for an introduction out of our scope.

\vspace{1.2em}

Before going into definitions, we set up some notations. Let us denote by
$\Sg$ the set of propositional variables. Disjunction is written with $\lor$,
conjunction with $\land$, and negation $\lnot$. Truth values are 0 (resp. 
$\bot$) and 1 (resp. $\top$). 

Our aim is to build so called Horn formulas. To this purpose we must first 
introduce what we call clauses, and Horn clauses. 


\begin{definition}[\midn{clause}] Let $x_1, \ dots, \ x_n$, $n \in \Ens{N}$
be variables of $\Sg$. A \belemp{clause} $\mathcal{C}$ over $x_i$'s is a 
\belemp{disjunction} of literals $p_i$:

	\[ \mathcal{C} = \bigvee_{i = 1}^{n} p_i \]

\noindent where $p_i \in \{x_i, \lnot x_i \}$.
\end{definition}

\begin{definition}[\midn{(pure) Horn clause}] A clause $C$ over variables of 
$\Sg$
is said \belemp{Horn} (resp. \belemp{pure Horn}) if it contains at most (resp. 
exactly one) positive literal.

\end{definition}

\paragraph{Example} To clarify our idea let us give a simple example. Let $x_1,
x_2, \dots, x_n$ be boolean variables. We have the following:
\begin{itemize}
	\item $\emptyset$ is a clause (true),
	\item $(x_1 \lor x_2 \lor \lnot x_3 \lor \lnot x_4)$ is a clause,
	\item $(\lnot x_1 \lor \lnot x_2)$ is a Horn clause,
	\item $(\lnot x_1 \lor \lnot x_2 \lor x_3)$ is pure Horn.
\end{itemize}


\vspace{1.2em}

To begin to draw a link with the previous section, one can note that we can 
write Horn clauses with disjunction, or with logical implication $\imp$. Indeed
remind that ($x_1 \lor \lnot x_2 $) can be equivalently rewritten as $x2 \imp 
x_1$. Hence, the Horn clauses defined in the previous examples become:
\begin{itemize}
	\item $(x_1 \land x_2) \imp \bot$,
	\item $(x_1 \land x_2) \imp x_3$.
\end{itemize}
\noindent We can use terms body, head the same way as with implicational basis. 
Before going any further, remind that any boolean formula $h$ can be 
rewritten in terms of \belemp{Conjunctive Normal Form} (or CNF). 

\begin{definition}[\midn{(pure) Horn CNF}] A \belemp{Horn CNF} (resp. 
\belemp{pure 
Horn CNF}) $\I$ over $\Sg$ is a conjunction of Horn clauses (resp. pure Horn
clauses) $\mathcal{C}_i$, $i \in \Ens{N}$:

	\[ \I = \bigwedge_{i} \mathcal{C}_i\]
	
\end{definition}

\begin{definition}[\midn{(pure) Horn formula}] A boolean function $h$ over 
$\Sg$
is a \belemp{Horn formula} (resp. \belemp{pure Horn formula}) if it can be 
represented with a Horn CNF $\I$ (resp. pure Horn CNF).
	
\end{definition}

Here is the link with our implication basis. A Horn CNF $\I$ can be seen as an
implicational basis. This relies on some notes:
\begin{itemize}
	\item for $P, Q, R$ propositional formulas, $(P \imp Q) \land (P \imp R)$ 
	is equivalent to $P \imp (Q \land R)$,
	\item representing sets with their \belemp{characteristic vectors} (or 
	bitmaps), translating $\cup$ by $\lor$ and $\cap$ by $\land$ give almost
	a one-to-one correspondence between sets and boolean formulas and their 
	models.
\end{itemize}

\paragraph{Example} To clarify let's think of short example taken from the 
previous section: $\Sg = \{ a, b, c, d, e \}$ and 

\[ \I = \{ ab \imp de, \ a \imp c, \ ce \imp b  \} \]

\noindent If we think of \textit{a, b, c, d, e} as boolean variables indicating
their presence into a set, we can derive the following translation of $\I$:

	\[ ((a \land b) \imp (d \land e)) \land (a \imp c) \land
		 ((c \land e) \imp b) \]

\noindent which can be represented by a (pure) Horn CNF, say $\I_h$:

\begin{align*}
	\I_h & = ((a \land b) \imp d) \land ((a \land b) \imp e) \land (a \imp c) 
	\land ((c \land e) \imp b) \\
	& \equiv (d \lor \lnot a \lor \lnot b) \land (e \lor \lnot a \lor \lnot b) 
	\land 
	(c \lor \lnot a) \land (b \lor \lnot e \lor \lnot c) 
\end{align*}

\noindent We could also have gone from $\I_h$ to $\I$.

\vspace{1.2em}

For this reason, we can also represent a Horn CNF as a set of clauses and 
clauses as pairs (body, head).

\begin{definition}[\midn{semantic entailment}] A formula $Q$ 
\belemp{semantically follows} from a formula $F$, denoted $F \models Q$ if 
every model of $F$ satisfies $Q$.
	
\end{definition}

Since the meaning of semantic entailment is the same as encountered with sets,
we can use the $\models$ the same way for both. This concludes our overview of
logical needs. We have seen the strong link between sets and logic which will 
allow us to go from one representation to another without any requirements. The
next section is dedicated to a short presentation of hypergraphs.





















% (Hyper)-graphs POV
\section{Directed graphs and Hypergraphs}



% Horn minimization task
\section{Horn Minimization task}

Previously we settled down mathematical tools we shall use. In the next 
paragraphs we will focus on describing "with hands" the problem of Horn 
minimization in our sense. Here we suppose $\I$ is in so-called \belemp{reduced
form}, that is for all distinct implications $A \imp B, C \imp D$ of $\I$,
$A \neq C$. That is we do not have distinct implications with same bodies. This
said, we state one of the \aliemp{most important} of our problem.

\begin{definition}[\midn{Minimality of $\I$}] Let $\I$ be a set of 
implications over $\Sg$, and $(\Sg, \phi)$ be a closure system. $\I$ is:
\begin{itemize}
	\item[(i)] \belemp{sound} if each implications of $\I$ holds in 
	$\Sg_\phi$
	\item[(ii)] \belemp{complete} if each implications holding in 
	$\Sg_\phi$ follows from $\I$
	\item[(iii)] \belemp{nonredundant} if no implication in $\I$ follows 
	from other implications of $\I$.
\end{itemize}
\noindent An implicational basis with such properties is called 
\belemp{minimal}.
\end{definition}

Note that here, minimality is defined with relation to a closure system. In
our case we will just be given $\Sg$ and $\I$, without an associated $(\Sg, 
\phi)$. Therefore, our basis will automatically be sound and complete because
the closure system we will consider is the set of models of $\I$. Hence, we
can provide a simpler definition of minimality (which is in fact nonredundancy):

\begin{definition}[\midn{Mimimality}] A reduced implicational basis $\I$ is
said \belemp{minimal} if we cannot remove any implication without altering
its closure.
	
\end{definition}


As we shall see later, there exists several definition of minimality. Our 
point of view could also be called \belemp{body minimality}. More intuitively, 
our aim is to concentrate all the knowledge in a minimal number of implications.

\paragraph{Example} Let us recall our music example. We had music styles:
\textit{shoegaze, electronic, coldwave, pop, rock, dream-pop, jazz, 
experimental, atmospheric, contemporary jazz} and some implications:
\begin{itemize}
	\item[ ] \textit{jazz, experimental} $\imp$ \textit{contemporary jazz},
	\item[ ] \textit{coldwave, rock}  $\imp$ \textit{shoegaze},
	\item[ ] \textit{atmospheric} $\imp$ \textit{experimental, electronic},
	\item[ ] \textit{shoegaze, pop} $\imp$ \textit{dream-pop}.
\end{itemize}
\noindent Unfortunately, because all the bodies are disjoint, this basis is 
already minimal. But let's imagine we have the following implications instead:
\begin{itemize}
	\item[ ] \textit{coldwave} $\imp$ \textit{rock},
	\item[ ] \textit{shoegaze}  $\imp$ \textit{rock, coldwave, dream-pop},
	\item[ ] \textit{shoegaze, dream-pop} $\imp$ \textit{rock, coldwave},
	\item[ ] \textit{rock, dream-pop} $\imp$ \textit{atmospheric, shoegaze, 
	coldwave}.
\end{itemize}
\noindent Here, the third implication can be removed. Indeed, from tags 
\textit{shoegaze, dreampop} following the implications we can reach 
\textit{rock, coldwave, atmospheric}. But say we remove the third implication,
and thus keep:
\begin{itemize}
	\item[ ] \textit{coldwave} $\imp$ \textit{rock},
	\item[ ] \textit{shoegaze}  $\imp$ \textit{rock, coldwave, dream-pop},
	\item[ ] \textit{rock, dream-pop} $\imp$ \textit{atmospheric, shoegaze, 
	coldwave}.
\end{itemize}
What can we reach starting from \textit{shoegaze, dreampop}? Using the second
implication we get \textit{rock, coldwave}, and because we have \textit{rock, 
dreampop}, we also get \textit{atmospheric}. That is, we can get the same 
knowledge than in the previous case but with one less implication.

\vspace{0.5em}

Let us present a less practical but more visual example. Suppose $\Sg = \{ a, 
\  s,\  c, \  r, \  d\}$
and $\I$ as follows:

	\[ c \imp r, \; s \imp rcd, \; rd \imp asc, \; sd \imp rc \]

\noindent This basis is not minimal. Indeed, let us remove the fourth 
implication and call $\I^{-}$ the new basis. We want to know whether the 
implication we removed still holds in $\I^{-}$ so that its closure is kept (and
therefore its "knowledge"). To show that $\I^{-} \models sd \imp rc$, it is
enough to show that $rc \subseteq \I^{-}(sd)$ (see proposition 
\ref{prop:def.equiv_imp_clos}). Because we have $s \imp rcd$, then $rd \imp 
asc$ we conclude that $\I^{-}(sd) = acdrs$. Thus $\I^{-} \models sd \imp rc$ 
and $\I^{-}$ is smaller than $\I$, with $\I^{-} \equiv \I$. Also, $\I^{-}$ is 
minimal.

\vspace{0.5em}

Note that the latter example is in fact the same as the musical one. Take the 
first letter of each style and we end up with the basis described in the second
paragraph.


\vspace{1.2em}

To go in details, we will introduce some elements and objects related to body 
minimality, namely \belemp{Pseudo-closed sets} and \belemp{Duquenne-Guigues 
basis}.

\begin{definition}[\midn{Pseudo-closed set}] Let $(\Sg, \phi)$ be a closure 
system. A subset $M \subseteq \Sg$ is \belemp{pseudo-closed} if and only if 
\begin{itemize}
	\item $\phi(M) \neq M$ ($M$ is not closed),
	\item if $Q \subset M$ is pseudo-closed, then $\phi(Q) \subseteq M$.
\end{itemize}
	
\end{definition}

Note that the empty set $\emptyset$ is pseudo-closed if and only if it is not 
closed. Also, one can note that if $P$ is pseudo-closed and $P'$ is covered
(inclusion wise, in terms of partial ordering) then $P'$ cannot be 
pseudo-closed. For the next definition, one can refer to 
\cite{b._ganter_conceptual_2016, guigues_familles_1986}.

\begin{definition}[\midn{Duquenne-Guigues Basis}] Let $(\Sg, \phi)$ be a 
closure system. The \belemp{Duquenne-Guigues} basis or \belemp{canonical} 
basis $\I$ is:

	\[ \I = \{ M \imp \phi(M) \; | \; M \subseteq \Sg, \, M \;
		\text{pseudo-closed} \} \]

\noindent and $\I$ is complete, sound and nonredundant.
\end{definition}

\noindent Thus, the Duquenne-Guigues basis is body-minimal. In this section 
we presented the minimization problem we will worked on during this thesis. 
Another remark, as we shall see in the next chapter, this problem is solvable
in polynomial time.

\space{1.2em}

To conclude this first chapter, we introduced in general our problem of 
minimizing implicational basis as our way of working. We developed an overview 
of the mathematical tools we needed to 


\end{comment}


% Chapter II: A review on algorithms for horn minimization
\chapter{Minimization algorithms for implication theories}

In the first chapter, we settled the context of our study. We introduced our 
subject of interest and defined the theoretical ground it is built on. In this
chapter, we discuss several algorithms and their complexity. Our aim with this
review is to provide understanding of the algorithms, study their complexity.
We also describe them to prepare their implementation in the next chapter.  Our first section within this part will be dedicated to give a general outline of our review. Next, we will effectively dive into in-depth studies of minimization procedure. 


\section{Overview of the study}

Here, we draw the main lines of our subsequent explanations. First and foremost, we shall try as much as possible to express all algorithms in our framework of closure systems and orders. Nevertheless, to draw parallels
and understand the translation from one framework to another we may proceed to 
some travels in other terminologies such as query learning or directed graphs.

\vspace{1.2em}

The study will be divided into three parts. First, we will study algorithms 
coming from the FCA (Formal Concept Analysis) community. This includes 
algorithms \textsc{MinCover} from \cite{bazhanov_optimizations_2014, day_lattice_1992, wild_implicational_1989} and \textsc{DuquenneMinimization}
being a variation of \textsc{MinCover} (more precisely, of the algorithm
provided by Day in a lattice-theoretic framework). This second procedure can
be found in \cite{duquenne_variations_2007}. Even though \textsc{ShockMinimization} is an algorithm from FCA community, we may include
discussion about it in this part because of the theorem it is based on. Secondly, we will dive into DB (Database) and graphs domains by working on
\textsc{MaierMinimization} (see \cite{maier_theory_1983, maier_minimum_1980}) and FD-Graphs, an extension of minimization to graphs provided by Ausiello and al; in \cite{ausiello_graph_1983, ausiello_minimal_1986, ausiello_directed_2017}. Eventually we get into algorithms coming out of 
boolean logic and query learning communities with \textsc{BercziMinimization}
and \textsc{AFPMinimization} (see \cite{berczi_directed_2017} and \cite{angluin_learning_1992, arias_canonical_2009} resp.).

\vspace{1.2em}

Regarding the order in which we will study the algorithms, there is no particular choice but our proximity with the domains. Because \textsc{MinCover} 
is the algorithm we are starting from, it seemed logical for us to explain
it first. Furthermore, we try as much as possible to provide hands made elements
of proof so that the study would be self-sufficient. Still, we may find in all
the papers we cite more technical definitions and proofs extending 
the knowledge we summarize here.




% ---------------------------------------------------------------------------- %
% ==== Section: MinCover, Shock, Day, Duquenne 
% ---------------------------------------------------------------------------- %

\section{Algorithms on closure systems}

In this section we will be interested in algorithms arising from Formal Concept
Analysis. More precisely we will rely on the work provided by Ganter, Day, Wild,
Duquenne-Guigues in \cite{ganter_formal_1999, ganter_two_2010, day_lattice_1992, wild_implicational_1989, wild_computations_1995, guigues_familles_1986, duquenne_variations_2007}. In this framework we will heavily rely on pseudo-closed and quasi-closed sets.


% ==== Minimal Cover ==== %

\subsection{Minimal Cover}

The minimization procedure we will describe in this section, soberly called
\textsc{MinCover} is the starting point for this master thesis. It can be found
in \cite{ganter_conceptual_2016}. As we shall explain, it has roots in
Day lattice-based algorithm \cite{day_lattice_1992}, and more surprisingly, 
\textit{"unknown"} ancestor in Shock algorithm \cite{shock_computing_1986}. 

\vspace{1.2em}

The principle is to perform \belemp{right-saturation}, and then \belemp{body 
redundancy} elimination. In fact, not only this is the general idea recently 
issued in \cite{boros_strong_2017}, but it is also the main theorem of Shock in 
\cite{shock_computing_1986} and the part of a theorem by Wild 
(\cite{wild_implicational_1989, wild_theory_1994}). This procedure has the 
advantage to be somehow intuitive. Indeed, right-saturation means replacing the 
conclusion of an implication by the closure of its premise:

\begin{center} $A \imp B$ becomes $A \imp \I(A)$ (of course $B \subseteq 
	\I(A)$)
\end{center}

\noindent hence, it means that we associate to $A$, all the information we can 
reach starting from $A$. Then, we perform body redundancy elimination. That is, 
for each right-closed implication, we check whether the amount of knowledge 
represented by $\I(A)$ depends necessarily on $A$. In other words, we remove
$A \imp \I(A)$ from $\I$, and if starting from $A$ we still get the same amount 
of information ($\I^{-}(A) = \I(A)$), then $A$ is not required to get $\I(A)$: $A \imp \I(A)$ can be removed. The second loop can also be named \belemp{left-saturation} since we are maximizing a premise up to its implication. As we will see, because implications are right-closed, this is left-saturation is pseudo-closing  or closing a premise. Now that the principle is explained in words, let us introduce the pseudo-code (see 
algorithm \ref{alg:MinCover}).

\vspace{1.2em}

\begin{algorithm}
	\KwIn{$\I$: an implication base}
	\KwOut{the canonical base of $\I$}
	
	\BlankLine
	\BlankLine
	
	\ForEach{$A \imp B \in \I$}{
		$\I := \I - \{ A \imp B \}$ \;
		$B  := \I(A \cup B)$ \;
		$\I := \I \cup \{ A \imp B \}$ \;
	}
	
	\BlankLine
	
	\ForEach{$A \imp B \in \I$}{
		$\I := \I - \{ A \imp B \}$ \;
		$A  := \I(A)$ \;
		\If{$A \neq B$}{
			$\I := \I \cup \{ A \imp B \}$ \;	
		}
		
	}
	
	\caption{\textsc{MinCover}}
	\label{alg:MinCover}	
\end{algorithm}

\textsc{MinCover} ends up on the canonical basis. Assuming that 
closure are computed with \textsc{LinClosure}, the overhaul complexity of
the algorithm is $O(|\B||\I|)$. To see correctness of the algorithm, observe
that the resulting $\I_c$ is equivalent to $\I$ at the end of the algorithm.
Indeed, at the end of the first loop, we replaced $B$ in every implications $A 
\imp B$ of $\I$ by $\I(A)$. But by proposition 1, $\I \models A \imp B$ if and 
only if $B \subseteq \I(A)$. This is in particular the case for $B = \I(A)$.
In the second loop we remove an implication only if it is redundant, thus
the resulting $\I_c$ is indeed equivalent to $\I$. The main question is 
minimality of $\I_c$. Recall that the DQ-basis, being minimal is based on 
pseudo-closed sets. Hence, if we can show that we keep an implication in the 
second loop only if the premise $\I^{-}(A)$ is pseudo-closed, we are done. This
is the purpose of next \textit{"hand-made"} proposition:

\begin{proposition} Let $\I$ be a \belemp{right-closed} implication theory. 
	Denote $\I^{-}(A) := (\I - \{ A \imp \I(A) \})(A)$, the following holds for 
	all $A \imp \I(A) \in \I$:
	\begin{itemize}
		\item[(i)] if $\I(A) = \I^{-}(A)$, $A \imp \I(A)$ is redundant in $\I$,
		\item[(ii)] if $\I(A) \neq \I^{-}(A)$, $\I^{-}(A)$ is pseudo-closed.
	\end{itemize}
	
\end{proposition}

\begin{proof} \textit{(i)} is trivial by definition. For \textit{(ii)}, let us
show that $\I^{-}(A)$ is quasi-closed, and then minimal among quasi-closed 
sets in its equivalence class. Suppose $\I^{-}(A)$ is not quasi-closed, then 
there must exist $B \subseteq \I^{-}(A)$ such that $\I(B) \nsubseteq \I^{-}(A)$ 
and $\I(B) \neq \I(\I^{-}(A)) = \I(A)$. Because $B \subseteq \I^{-}(A)$, either 
$\I(B) \subset \I(A)$ or $\I(B) = \I(A)$. If we are in the equality case, 
we are done. So let $B \subseteq \I^{-}(A)$ and $\I(B) \subset \I(A)$. By 
definition of $\I^{-}$, if there exists such $B$, either it is closed in 
$\I$ and we are done, or there exist implications $C_i \imp \I(C_i)$ such that 
$C_i \subseteq B$, $\I(C_i) \nsubseteq B$ with $\bigcup \I(C_i) = \I(B)$. But 
for all such implications, $C_i \subseteq \I^{-}(A)$, and by construction of 
$\I^{-}(A)$, $\I(B) = \bigcup \I(C_i) \subseteq \I^{-}(A)$. Hence, 
$\I^{-}(A)$ is indeed quasi-closed. Now, let us show that it is minimal among 
quasi-closed sets in its equivalence class. If $A$ is closed in $\I^{-}$, the 
result is direct, because for all $C \imp \I(C)$ in $\I$, either $C \nsubseteq 
A$ or $(C \subseteq A) \ \land \ (\I(C) \subseteq \I^{-}(A) = A)$. Assume the 
presence of some $B$ such that $A \subseteq B \subseteq \I^{-}(A)$ with $B$ 
being quasi-closed. Note that if $A$ is not closed under $\I^{-}$, it
cannot be quasi-closed. If $B \imp \I(B) = \I(A)$ is in $\I$, then 
$\I^{-}(A) = \I(A)$ and we have a contradiction. If $B \imp \I(B) \not\in \I$, 
then we have $\I^{-}(A) = B$ because $B$ contains $A$ and will be closed under 
$\I^{-}$, which concludes the proof.
\end{proof}

\vspace{1.2em}

In fact, the second loop of \textsc{MinCover} is performing somehow the following operation: $\I^{-}(A) := A \cup \bigcup \{\I(B) \ | \ B \subset A\}$
related to \midemp{proper implications} we shall encounter in the third chapter.
Still, This proposition is sufficient for the algorithm to correctly end up on the canonical basis. Interestingly, the main idea of \textsc{MinCover} is similar to the theorem 2.1 of Shock in \cite{shock_computing_1986}, but the algorithm in practice is much closer from the procedure given by Day in section 6 of \cite{day_lattice_1992}. Before moving to their work, let us settle down an example of trace for \textsc{MinCover}.

\vspace{1.2em}

\paragraph{Example} Let us discuss the following example:
\begin{itemize}
	\item[-] $\Sg = \{a, \ b, \ c, \ d, \ e, \ f \}$,
	\item[-] $\I =$ \{\textit{ab $\imp$ cde, cd $\imp$ f, c $\imp$ a, d $\imp$ b, abcd $\imp$ ef} \} 
\end{itemize}
We will present a trace of \textsc{MinCover} by a sequence of vectors 
representing $\I$ after modifications:

\[
\begin{pmatrix}
ab  \imp  cde  \\
cd  \imp   f    \\
c  \imp  a     \\
d  \imp  b     \\
abcd \imp ef \\
\end{pmatrix}
\imp
\begin{pmatrix}
ab \imp \text{\midemp{abcdef}}   \\
cd \imp \text{\midemp{abcdef}}   \\
c \imp \text{\midemp{ac}}        \\
d \imp \text{\midemp{bd}}        \\
abcd \imp \text{\midemp{abcdef}} \\
\end{pmatrix}
\imp
\begin{pmatrix}
ab \imp abcdef \\
\text{\aliemp{abcdef $\imp$ abcdef}} \\
c \imp ac \\
d \imp bd \\
\text{\aliemp{abcdef $\imp$ abcdef}} \\
\end{pmatrix}
\imp
\begin{pmatrix}
ab \imp abcdef \\
c \imp ac \\
d \imp bd \\
\end{pmatrix}
\]

The first vector is the initial basis. Then we perform right-saturation. The
third vector differs a bit from true execution of \textsc{MinCover}, but it
illustrates replacement of $A$ by $\I^{-}(A)$ in $A \imp \I(A)$. As we can see,
two implications have the same premises and conclusion: they are useless and
hence removed in the resulting $\I$, being the last vector.

\vspace{1.2em}

Now that things should be a bit clearer, let us discuss the two other 
algorithms previously cited. Remark that we will not explain the procedure
given by Wild in \cite{wild_implicational_1989, wild_theory_1994} because it
is strictly \textsc{MinCover}:
\begin{enumerate}
	\item right-close all implications of $\I$,
	\item find a minimal non-redundant subfamily of implications in $\I$
	right-closed, i.e redundancy elimination. 
\end{enumerate}
\noindent Hence, the procedure given by Shock is presented in algorithm 
\ref{alg:Shock}.

\begin{algorithm}
	\KwIn{$\I$: a theory to minimize}
	\KwOut{a minimum cover for $\I$}
	
	\BlankLine
	\BlankLine
	
	\ForEach{$A \imp B \in \I$}{
		$\I := \I - \{ A \imp B \}$ \;
		\If{$B \nsubseteq \I(A)$}{
			$\I := \I \cup \{A \imp \I(B) \}$ \;	
		}
		
	}
	\caption{\textsc{ShockMinimization}}
	\label{alg:Shock}
\end{algorithm}

This routine, co-issued with the theorem we discussed previously is 
quite different from \textsc{MinCover}. Even though the conditional statement 
$B \nsubseteq \I(A)$ is equivalent to $\I(A) \neq \I^{-}(A)$ and 
replacing $A \imp B$ by $A \imp \I^{-}(B)$ is about right-closing $A \imp B$, 
the resulting basis of this algorithm may not be minimal in
general:
\begin{itemize}
	\item[-] if $\I = \{ \emptyset \imp a, \ a \imp b \}$ (in this order), 
	\textsc{ShockMinimization} will produce $\emptyset \imp ab$ which is right,
	\item[-] if $\I = \{ a \imp b, \ \emptyset \imp b \}$, the result will
	be $\{ a \imp ab, \ \emptyset \imp ab \}$ being redundant.
\end{itemize} 
In fact, this error has already been pointed out in 1995 by Wild in 
\cite{wild_computations_1995}. However, the implications given by Shock may have quasi-closed premises and present similarities with the algorithm for generating quasi-closed sets \textsc{FindCrucialGenerators} of Day (\cite{day_lattice_1992}). In Wild work, we can also find another proof for the theorem of Shock, jointly with minimality of DG-basis (theorem 5 of \cite{wild_theory_1994, wild_computations_1995}). Let us discuss now the algorithm proposed by A. Day in 1992 (see pseudo-code \ref{alg:Day}: \textsc{DayMinimization}). We express it through our framework of closure systems and implications. Since it would only complicate our explanations, we do not integrate definitions of lattice theory. However, for the reader with few background in lattice theory, we can mention Day's overhaul framework in some sentences. At the best of our understanding, the main idea is to focus on the partial order $(2^{\Sg}, \subseteq)$ being a complete join-semilattice (or just complete lattice when we consider finite $\Sg$) where the closure operator $\I(\cdot)$ is in fact a $\bigvee$-morphism from $(2^{\Sg}, \subseteq)$ to $(\Sg^{\I}, \subseteq)$. In $(2^{\Sg}, \subseteq)$, the join operation ($\lor$) is set union. Consequently, the equivalence classes of $\I$ define a congruence on the powerset of $\Sg$ being the kernel of $\I(\cdot)$. The approach developed by A. Day is then to build congruences (hence closure operators) through ordered pairs (or quotients) of the same equivalence class, representing implications.

\begin{algorithm}
	\KwIn{$\I$: a theory to minimize}
	\KwOut{canonical basis of $\I$}
	
	\BlankLine
	\BlankLine
	
	\ForEach{$A \imp B \in \I$}{
		$\I := \I - \{ A \imp B \}$ \;
		$A := \I(A)$ \;
		$B := \I(A \cup B)$ \;
		\If{$A \neq B$}{
			$\I := \I \cup \{ A \imp B \}$ \;	
		}
		
	}
	
	\caption{\textsc{DayMinimization}}
	\label{alg:Day}
\end{algorithm}

\noindent Here, equivalence with \textsc{MinCover} is clear. The only difference
is the order in which operations are performed. Even though the 2 algorithms 
rely on the same computation, it is worth noting a particular case where the
algorithm by Day may fail. Consider a system not being reduced, e.g:

\[ \I =  \{ b \imp ac, \ c \imp a, \  c \imp b \} \]

\noindent we have:
\begin{itemize}
	\item[-] \textsc{MinCover} output: $\{ b \imp ac, \ c \imp abc \}$,
	\item[-] \textsc{DayMinimization} output: $ \{ b \imp ac, \ ac \imp abc, \  c \imp abc \}$
\end{itemize}
In general, performing right-closure before redundancy elimination, as in 
\textsc{MinCover} avoids this problem. 

\vspace{1.2em}

In this section we reviewed the algorithm acting as our starting point, by 
studying its complexity and principle. We also linked it to research we made
and other algorithms we found. As exposed, \textsc{MinCover} summarizes and 
corrects all material we covered here, and hence justifies not to implement
all of them. The next section defines a slightly different algorithm, being a
variation of the work from \cite{day_lattice_1992}.

\subsection{Duquenne algorithm}

Later, also based on Day lattice theoretic work and its own approach, Duquenne
proposed a variation of \textsc{MinCover} based on first computing quasi-closed
sets and then using the recursive characterization of pseudo-closed sets to iteratively build the Duquenne-Guigues basis. Recall that pseudo-closed sets
are particular open sets that can be defined by two means:
\begin{itemize}
	\item[-] a set $P$ is pseudo-closed if it is larger than all the closure of any	sub pseudo-closed set,
	\item[-] $P$ is pseudo-closed if it is quasi-closed and minimal among quasi-closed sets of $\left[ P \right]_{\I}$.
\end{itemize}
We will call this procedure \textsc{DuquenneMinimization} (see algorithm \ref{alg:Duquenne-min}). In fact, It uses the algorithm 2 from \cite{day_lattice_1992} to compute quasi-closed sets from premises of $\I$,
this is the first loop. All pseudo-closed sets are included in the resulting $\I$ after this first step. Then, we use \belemp{lectic ordering} to have a $\subseteq$-compatible way to process implications before building $\I_c$.


\begin{algorithm}
	\KwIn{$\I$ a theory to minimize}
	\KwOut{$\I_c$ the DQ-basis of $\I$}
	
	\BlankLine
	\BlankLine
	
	\ForEach{$A \imp B \in \I$}{
		$\I = \I - \{ A \imp B \}$ \;
		$A := \I(A)$ \;
		
		\If{$B \not\subseteq A$}{
			$B = B \cup A$ \;
			$\I := \I \cup \{ A \imp B \}$ \;	
		}

	}

	\BlankLine
	
	\textsc{LecticOrder}($\I$) \;
	$\I_c := \emptyset$ \;
	
	\BlankLine
	
	\ForEach{$A \imp B \in \I$}{
		\ForEach{$\alpha \imp \beta \in \I_c$}{
			\If{$\alpha \subset A \land \beta \nsubseteq A$}{
				$\I = \I - \{ A \imp B \}$ \;
				\textbf{goto next $A \imp B \in \I$} \;
		
			}
	
		}
	
		$B = \I(B)$ \;
		$\I_c := \I_c \cup \{ A \imp B \}$ \;
	}

	\BlankLine
	
	return $\I_c$ \;
	
	\caption{\textsc{DuquenneMinimization}}
	\label{alg:Duquenne-min}
\end{algorithm}

Before proving the algorithm, let us define lectic ordering $\leq_{\Sg}$. First, we must assume that $\Sg$ can be assigned a total order $\leq$. For the recall, an order is total if for all pairs $(x, \ y)$ of $\Sg$, $x \leq y$ or $y \leq x$. Hence, provided $\Sg$ is a chain, we can define $\leq_{\Sg}$ on
$2^{\Sg}$ as follows: $\forall A, B \subseteq \Sg$ $A \leq_{\Sg} B$ if the
smallest element in which $A$ and $B$ differ belongs to $B$. We say that $A$ is \belemp{lectically smaller than} B. Note that $\leq_{\Sg}$ is \belemp{$\subseteq$-compatible}, that is $A \subseteq B \imp A \leq_{\Sg} B$.
The opposite direction however does not hold since $\subseteq$ is a partial 
ordering, while $\leq_{\Sg}$ is total. 

\paragraph{Example} Consider $\Sg = \{ a, b, c \}$ with $a < b < c$. In
this setting, $b \leq_{\Sg} a$, $b \leq_{\Sg} ab$ and $b \leq_{\Sg} bc$ for instance. To observe $\subseteq$-compatibility, we can refer to figure \ref{fig:II-LecticSubset} illustrating the two orderings side-by-side. 

\begin{figure}[ht]
	\centering
\subfloat[Partial order $(2^{\Sg}, \subseteq)$]{
	\begin{tikzpicture}[scale=0.7]
	\node[Vertex, label=below:{$\emptyset$}] (epty) at (0, -3) {};
	\node[Vertex, label=left:{$a$}] (a) at (-2, -1) {};
	\node[Vertex, label=right:{$b$}] (b) at (0, -1) {};
	\node[Vertex, label=right:{\quad $c$}] (c) at (2, -1) {};
	\node[Vertex, label=left:{$ab$}] (ab) at (-2, 1) {};
	\node[Vertex, label=left:{$ac$}] (ac) at (0, 1) {};
	\node[Vertex, label=right:{\quad $bc$}] (bc) at (2, 1) {};
	\node[Vertex, label=above:{$abc$}] (abc) at (0, 3) {};
	
	\draw (epty) -- (a) -- (ac) -- (c) -- (epty) --
	(b) -- (ab) -- (abc) -- (bc) -- (b);
	\draw (ab) -- (a);
	\draw (bc) -- (c);
	\draw (abc) --(ac);
	
	
	\end{tikzpicture}
}\qquad
\subfloat[Total order $(2^{\Sg}, \leq_{\Sg})$]{
\begin{tikzpicture}[scale=0.7]
\node[Vertex, label=below:{$\emptyset$}] (epty) at (0, -3) {};
\node[Vertex, label=right:{$c$}] (c) at (0, -2) {};
\node[Vertex, label=right:{$b$}] (b) at (0, -1) {};
\node[Vertex, label=right:{$bc$}] (bc) at (0, 0) {};
\node[Vertex, label=right:{$a$}] (a) at (0, 1) {};
\node[Vertex, label=right:{$ac$}] (ac) at (0, 2) {};
\node[Vertex, label=right:{$ab$}] (ab) at (0, 3) {};
\node[Vertex, label=above:{$abc$}] (abc) at (0, 4) {};

\draw (epty) -- (c) -- (b) -- (bc) -- (a) -- (ac) -- (ab) -- (abc);

\node (b1) at (-2, 0) {};
\node (b2) at (2, 0) {};

\end{tikzpicture}
}

\caption{Subset and lectic ordering}
\label{fig:II-LecticSubset}
\end{figure}

\vspace{1.2em}

The advantage of lectic ordering is to allow for easier checking of pseudo-closeness recursive property: if we test sets in lectic order,
we are sure not to avoid any pseudo-closed subset of a given set when
looking at previously considered ones. Furthermore, because this is 
a total order, we can use fast logarithmic sorting procedure to order 
efficiently sets. Next, let us give some elements of proof for this 
algorithm.


\begin{proposition} The following statements hold for all $A \imp B \in \I$, $\I^{-} = \I - \{ A \imp B \}$:
	\begin{itemize}
		\item[(i)] $\I^{-}(A) = \I(A)$, then $A \imp B$ is redundant,
		\item[(ii)] $\I^{-}(A) \neq \I(A)$, then $\I^{-}(A)$ is quasi-closed in $\I$.
	\end{itemize}
\end{proposition}

\begin{proof} \textit{(i)}. $\I^{-}(A) = \I(A)$ is equivalent to $B \subseteq \I^{-}(A)$, that is $\I - \{ A \imp B \} \models A \imp B$. \textit{(ii)}. Assume $\I^{-}(A) \neq \I(A)$ and $\I^{-}(A)$ not quasi-closed. Then we must be
able to find an implication $\alpha \imp \beta$ such that $\alpha \subseteq \I^{-}(A)$ and $ \I(\alpha) \nsubseteq \I^{-}(A)$. We have also $\I(\alpha) \subset \I(A)$. $\I^{-}$ is a closure operator, hence $\alpha \subseteq \I^{-}(A) \imp \I^{-}(\alpha) \subseteq \I^{-}(A)$. $\I^{-}(\alpha) \subseteq \I^{-}(A)$ and $ \I(\alpha) \nsubseteq \I^{-}(A)$ leads to $\I^{-}(\alpha) 
\subset \I(\alpha)$. Because of $\I^{-}$ computations, this leads to $A \subseteq \I(\alpha)$, hence by monotonicity of $\I$, $\I(A) \subseteq \I(\alpha)$ contradicting $\I(\alpha) \subset \I(A)$.

\end{proof}

Next, we will assume a lemma given in \cite{duquenne_variations_2007} from where
the algorithm comes:

\begin{lemma} For lists $\I$, $\mathcal{H}$ of implications, with $\mathcal{H} \subseteq \I$, the following statements are equivalent:
\begin{itemize}
	\item[(i)] $\I \equiv \mathcal{H}$,
	\item[(ii)] $\I$ has the same canonical basis as $\mathcal{H}$,
	\item[(iii)] for every pseudo-closed set $P$ of $\I$, there is at least
	one $A \imp B \in \mathcal{H}$ for which $A \subseteq P \subset \I(P) = \mathcal{H}(A)$.
\end{itemize}
\end{lemma}

Because $\I^{-}$ is a closure operator, if $\I^{-}(A)$ is not closed under $\I$,
it is then the smallest quasi-closed set containing $A$. Furthermore, both $A$
and $\I^{-}(A)$ belong to $\left[ A \right]_{\I}$. With lemma 1 where $\mathcal{H} = \I$, one has that every pseudo-closed sets of $\I$ must be a premise of $\I$ after the first loop.

\begin{proposition} At the end of \textsc{DuquenneMinimization} $\I_c$ is the
DG-basis of $\I$.
\end{proposition}

\begin{proof} We proved that after the first step, all pseudo-closed sets
	appeared as premise of $\I$. Because we use a $\subseteq$-compatible ordering on premises of $\I$ with lectic order, the nested loop on $\I_c$
	checks the recursive property of being pseudo-closed. Hence we only add
	implications $P \imp \I(P)$ where $P$ is $\I$-pseudo-closed to $\I_c$.
\end{proof}

\paragraph{Example} In order to clarify the algorithm, let us execute it on a
small example. Because we may have a lazy imagination, consider the example we used for \textsc{MinCover}:
\begin{itemize}
	\item[-] $\Sg = \{a, \ b, \ c, \ d, \ e, \ f \}$,
	\item[-] $\I =$ \{\textit{ab $\imp$ cde, cd $\imp$ f, c $\imp$ a, d $\imp$ b, 
		abcd $\imp$ ef} \} 
\end{itemize}
We proceed by steps:
\begin{enumerate}
	\item \midemp{left-saturation}. For all implications $A \imp B$ of $\I$,
	we compare $\I(A)$ and $\I^{-}(A)$ where $\I^{-}$ is $\I$ from which we removed $A \imp B$. We should keep in mind that $\I^{-}$ is different 
	for every $A \imp B$ then, not only because previous implications are altered or remove, but also since at each step we delete a different implication from $\I$. Let us present results of this step through table \ref{tab:duq-red}.
	
\begin{table}[ht]
	\begin{tabular}{| c || p{2.7cm} | c | p{2.5cm} | c | c | p{2.7cm} | }
	\hline \rowcolor{clouds}
	$\B(\I)$ & $\I$ before & $\I(\cdot)$ & $\I^{-}$ & $\I^{-}(\cdot)$ & imp ? & $\I$ after \\ \hline
	
	$ab$ & $ab \smimp cde, cd \smimp f,$ \newline $c \smimp a, d \smimp b,$ \newline  $abcd \smimp ef$ & $abcdef$ & $cd \smimp f, c \smimp a,$ \newline $d \smimp b,$ \newline $abcd \smimp ef$ & $ab$ & $ab \imp abcde$ & $ab \smimp abcde,$ \newline $cd \smimp f, c \smimp a,$ \newline $d \smimp b, abcd \smimp ef$ \\	\hline
	 
	$cd$ & $ab \smimp abcde,$ \newline $cd \smimp f, c \smimp a,$ \newline $d \smimp b, abcd \smimp ef$ &	$abcdef$ & $ab \smimp abcde,$ \newline $c \smimp a, d \smimp b,$ \newline $abcd \smimp ef$ & $abcdef$ & \aliemp{removed} & $ab \smimp abcde, c \smimp a,$ \newline $d \smimp b, abcd \smimp ef$ \\ \hline
	
	$c$ & $ab \smimp abcde, c \smimp a,$ \newline $d \smimp b, abcd \smimp ef$ & $ca$ & $ab \smimp abcde$ \newline $d \smimp b, abcd \smimp ef$ & $c$ & 
	$c \imp ca$ & $ab \smimp abcde,$ \newline $c \smimp ca, d \smimp b,$ \newline $abcd \smimp ef$ \\ \hline
	
	$d$ & $ab \smimp abcde,$ \newline $c \smimp ca, d \smimp b,$ \newline $abcd \smimp ef$ & $db$ & $ab \smimp abcde,$ \newline $c \smimp ca,$ \newline $abcd \smimp ef$ & $d$ & $d \imp db$ & $ab \smimp abcde,$ \newline $c \smimp ca, d \smimp db,$ \newline $abcd \smimp ef$ \\ \hline
	
	$abcd$ & $ab \smimp abcde,$ \newline $c \smimp ca, d \smimp db,$ \newline $abcd \smimp ef$ & $abcdef$ & $ab \smimp abcde,$ \newline $c \smimp ca, d \smimp db$ & $abcde$ & $abcde \imp abcdef$ & $ab \smimp abcde,$ \newline $c \smimp ca, d \smimp db,$ \newline $abcde \smimp abcdef$ \\ \hline
		
	\end{tabular}

\caption{First step of \textsc{DuquenneMinimization}}
\label{tab:duq-red}
\end{table}
Observe that at the second step, when we consider $cd$, because $\I(cd) = \I^{-}(cd)$, the implication $cd \imp f$ is removed from $\I$. The basis we get
after this step is: $ab \imp abcde, \ c \imp ca, \ d \imp db,$  $abcde \imp abcdef$.

\item \midemp{lectic ordering} Here we will not follow the sorting procedure, but instead, we will illustrate how does lectic ordering work in practice, and how to compute it at hands easily. Say we have the following \belemp{total} order in elements of $\Sg$: $a < b < c < d < e < f$. We can imagine represent
a subset of $\Sg$ by a binary string of size $|\Sg|$ where the least significant bit (on the right) corresponds to $f$, and the most significant bit
(on the left) matches $a$. The binary string associated to some $A \subset \Sg$ will have ones in place of elements it contains. For instance, the subset 
$acef$ will have $101011$ as binary string:

\[\begin{matrix}
	$a$ & $b$ & $c$ & $d$ & $e$ & $f$ \\
	1 & 0 & 1 & 0 & 1 & 1 \\
\end{matrix}\]

From this point of view, lectic ordering is just binary enumeration. A premise $A_1$ will be lower than $A_2$ under lectic order if the binary string
associated to $A_1$ comes before the word related to $A_2$  when enumerating naturally binary numbers. In our case, we have:
\begin{itemize}
	\item[-] $ab$: $110000$,
	\item[-] $c$: $001000$,
	\item[-] $d$: $000100$,
	\item[-] $abcde$: $111110$.
\end{itemize}
Ordering those premises by their order of appearance under binary counting, we
get $\I$ shuffled as follows: $ d \imp db, \ c \imp ca, \ ab \imp abcde, \ abcde \imp abcdef$.

\item \midemp{getting pseudo-closed implications} For each implication of $\I$, we want to check whether its premise is pseudo-closed or not. To do this, we will build iteratively our resulting basis $\I_c$, containing left-pseudo-closed and right-closed implications (see table \ref{tab:duq-PC}).

\begin{table}[ht]
	\centering
	\begin{tabular}{| c | p{3cm} | c | p{3cm} |}
		\hline \rowcolor{clouds}
		implication & input $\I_c$ & pseudo-closed ? & output $\I_c$ \\ \hline
		$d \smimp db$ & $\emptyset$ & $\lor$ & $d \smimp db$ \\ \hline
		$c \smimp ca$ & $d \smimp db$ & $\lor$ & $d \smimp db, c \smimp ca$ \\ \hline
		$ab \smimp abcde$ & $d \smimp db, c \smimp ca$ & $\lor$ & 
		$d \smimp db, c \smimp ca$ \newline $ab \smimp abcdef$ \\ \hline
		$abcde \smimp abcdef$ & $d \smimp db, c \smimp ca$ \newline $ab \smimp abcdef$ & $\times$: \aliemp{ab $\smimp$ abcdef} & $d \smimp db, c \smimp ca$ \newline $ab \smimp abcdef$ \\ \hline
	\end{tabular}
			
\caption{\textsc{DuquenneMinimization} third step}
\label{tab:duq-PC}
\end{table}

As one can see, when we add an implication from $\I$ to $\I_c$ we perform 
right-closing. This is necessary to check pseudo-closeness. Consequently for
the last step, because $ab \imp \I(abcde) = abcdef \in \I_c$, $abcde$ is not
pseudo-closed: $ab \subseteq abcde$ and $\I(ab) = abcdef \nsubseteq abcde$. Thus we do not add it to $\I_c$. At the end of the algorithm we have $\I_c = 
d \imp db, \ c \imp ca, \ ab \imp abcdef$.
\end{enumerate}

Regarding the complexity of the algorithm, one may note that the first loop has
the same complexity as the second step of \textsc{MinCover}, that is $O(|\B||\I|)$ provided we use \textsc{LinClosure} for lowering theoretical complexity of closure computations. Observe then that the lectic order is a total order, hence we can use a logarithmic sorting function as quick-sort for
\textsc{LecticOrder}, resulting in $O(|\Sg||\B|\log_2(|\B|))$ due to $O(|\Sg|)$
comparison of two sets. Eventually, for each $A \imp B$, the nested for each loop may require $O(|\I|)$ since we are performing set operations on at most as much implications as $|\B|$. Then we may perform a closure under $\I$, being
$O(|\I|)$ also. Therefore, the overhaul loop should require $O(|\B||\I|)$ time.
From the three steps we have, we can conclude that the whole algorithm has complexity $O(|\B||\I|)$ as \textsc{MinCover}. However, the first loop of 
\textsc{DuquenneMinimization} act as a redundancy elimination which helps to
reduce the cost of closure computations in the whole algorithm. This could be
an improvement of \textsc{MinCover} to test in practical implementation.

\vspace{1.2em}

In this section we were interested in studying algorithms built on the work of
Wild, Day, Duquenne-Guigues and Ganter mainly relying on left and right-saturation of implications to produce the canonical basis. In the next section we will be involved in studying algorithms based on the work of Maier
and later, Ausiello.


% ---------------------------------------------------------------------------- %
% ==== Section: DB algorithms
% ---------------------------------------------------------------------------- %


\section{Algorithms based on Maier's database approach}

Here, we deal with algorithms based on Functional Dependencies (FD's). In database theory, functional dependencies are also implications between sets
of attributes. The difference lies in the value of those attributes. Up to know we have been considering \textit{"binary"} attributes and implications: either
we have $x$ or we do not have it. In DB, an attribute can be multi-valued (imagine an attribute \textit{"age"} for our plant example). Therefore, the 
notion of implication is stricter: if we are given two attributes $x$ and $y$,
$x \imp y$ will be a valid functional dependency if the \belemp{value} (not the presence) of $x$ determines the value of $y$. For more precision, see \cite{maier_theory_1983}. Actually, when it comes at minimization this semantic difference does not matter. Therefore, the algorithms we are about to study suits also our framework. For this reason we will explain algorithms within our
usual context.

\subsection{First algorithm: Maier's algorithm on FDs}

Here we will consider one of the first algorithm given for minimization 
purpose. It has been proposed by Maier in \cite{maier_theory_1983, 
maier_minimum_1980} and relies notably on the algorithm \textsc{LinClosure} issued in \cite{beeri_computational_1979}. For understandability we will explain this algorithm through implication theories framework while drawing parallel with Maier's notation and definitions. As a soft introduction, we will develop an interesting and simple example in Maier's algorithm context.


\paragraph{Example} Let $\Sg$ and $\I$ be as follows (in fact, same example as 
in previous section):
\begin{itemize}
	\item[-] $\Sg = \{a, \ b, \ c, \ d, \ e, \ f \}$,
	\item[-] $\I =$ \{\textit{ab $\imp$ cde, cd $\imp$ f, c $\imp$ a, d $\imp$ b, abcd $\imp$ ef} \} 
\end{itemize}
\noindent Let us try to minimize it \textit{"with hands"}. First, we see \textit{abcd $\imp$ ef} to be redundant. Indeed, if we remove it from $\I$, we still have \textit{ab $\imp$ cde} and \textit{cd $\imp$ f}, thus $\I^{-} := \I - \{ abcd \imp ef \} \models abcd \imp ef$. $\I^{-}$ is not redundant any more. 
Nevertheless, we can still remove an implication. Indeed, not only we can reach 
$ab$ from $cd$, but also $cd$ from $ab$. Consequently, we could remove $cd \imp 
f$ from $\I^{-}$ while adding to the head of $ab \imp cde$ the element
$f$ (the head of $cd \imp f$) to avoid loss of informations. Hence, we 
would end up with 
\begin{center}
	$\I =$ \{ \textit{c $\imp$ a, d $\imp$ b, ab $\imp$ cdef} \}
\end{center}

\vspace{1.2em}

Those steps of redundancy elimination and equivalence manipulation are the core 
manipulations of Maier algorithm. For the recall, Maier worked with functional
dependencies, but this makes no difference when it comes as implications. 

\paragraph{Redundancy elimination} As mentioned in the first chapter, given 
$\I$, $A \imp B \in \I$ is redundant if $\I^{-} := \I - \{ A \imp B \} \models A \imp B$ or equivalently if $B \subseteq \I^{-}(A)$. Thus get rid of 
redundancy elimination can be done in the following procedure:

\begin{algorithm}
	\KwIn{$\I$: an implication theory}
	\KwOut{$\I$ without redundant implications}
	
	\BlankLine
	\BlankLine
	
	\ForEach{$A \imp B \in \I$}{
		\If{$\I - \{ A \imp B \} \models A \imp B$}{
			remove $A \imp B$ from $\I$ \;
		}
		
	}
	
	\caption{\textsc{RedundancyElimination}}
	\label{alg:Maier-RE}	
\end{algorithm}

\noindent Checking for redundancy is done with \textsc{LinClosure}. Because this
is done for all implications of $\I$, complexity of redundancy elimination is
$O(|\B| \times |\I|)$.

\vspace{1.2em}

\paragraph{Equivalence classes} As briefly described previously, we can identify
equivalence classes within $\Sg^{\I}$. For $X \subseteq \Sg$, we can set up
$\left[ X \right]_{\I} = \{ Y \subseteq \Sg \ | \ \I(Y) = \I(X) \}$. Recall 
that this is the definition of an \belemp{equivalence class} we presented in the first chapter. More than this, we can limit those equivalence classes to premises of $\I$, i.e for some $A \subseteq \Sg$ let
\begin{itemize}
	\item[(i)] $E_{\I}(A) = \{ X \imp Y \in \I \ | \ X \in \left[ A 
	\right]_{\I} \}$ 
	\item[(ii)] $e_{\I}(A) =\{ X \ | \ X \in \B(\I) \cap \left[ A \right]_{\I} 
	\}$
\end{itemize}
\noindent Plus, we say that $A$ \belemp{directly determines} $B$, denoted $A 
\ddv B$, if $\I - E_{\I}(A) \models A \imp B$. Now, the minimization process
in \cite{maier_theory_1983, maier_minimum_1980} is the following: 

\begin{proposition} Let $\I$ be an irredundant theory. If $A \imp B, \ C \imp D 
\in \I$ (distinct) are such that $C \equiv_{\I} A$ and $A \ddv C$, then we can remove $A \imp B$ from $\I$ and replace $C \imp D$ by $C \imp D \cup B$ without 
altering $\Sg^{\I}$.
\end{proposition}

\begin{proof} Suppose we removed $A \imp B$ and modified $C \imp D$ to $C \imp D \cup B$. Put $\I^{-}$ as the system we obtained. The main point is to show that we still have $\I^{-} \models A \imp B$. Recall $A \ddv C$, thus:

\begin{align*}
	(\I - E_{\I}(A) \models A \imp C) & \; \imp \I - A \imp B \models A \imp 
	C \\
	& \; \imp \I^{-} \models A \imp C \\ 
\end{align*}

\noindent Because we changed $C \imp D$ to $C \imp D \cup B$, we have then

\[ (\I^{-} \models A \imp C) \land (\I^{-} \models C \imp D \cup B) \imp
	(\I^{-} \models A \imp B) \]

\noindent by transitivity. Note that equivalence of $A$ and $C$ is 
preserved, because in $\I$, by transitivity $C \imp A \imp B$. Removing $A \imp 
B$ but moving $B$ to $C \imp D \cup B$ preserves $C \imp B$. Also, taking 
equivalent premise is important in order not to alter the closure system 
$\Sg^{\I}$. If $A$ and $C$ were not equivalent, we may have changed the system 
by adding $B$ to the closure of $C$ even though $\I \not\models C \imp B$.
\end{proof}

\vspace{1.2em}

This is worth noting we gave a \textit{"light"} definition of direct determination more relying on a property proved by Maier than the strict original definition. Also, Wild in \cite{wild_implicational_1989, wild_theory_1994} drawn a parallel between quasi-closeness property and equivalence classes $E_{\I}(\cdot)$. Indeed, a set $A$ will be quasi-closed if and only if $\I(A) = \I^{-}_{E_{\I}(A)}(A)$. Finally, the last proposition gives us an algorithmic test for minimization: given an equivalence class $E_{\I}(X)$, one can run across all its implications and successively remove useless ones. Actually it says that if an non-redundant basis contains direct determinations, it is not minimal. Hence contrapositive yields a condition for minimality ensuring correctness and end of the operation.

\vspace{1.2em}

The main question is: how to get the equivalence classes efficiently? It turns
out this can be done using a modified version of \textsc{LinClosure}. It is 
sufficient to embed in the function a vector of implied premises. That is, 
for a given premise $X$, we provide to \textsc{LinClosure} a bit-vector 
$implied$ of size $|\B|$. Within the procedure, whenever we reach $count\left[A \imp B \right] = 0$ for some $A \imp B \in \I$, then $A$ is implied by $X$ under $\I$. Hence we set $implied\left[ A \imp B\right]$ to 1. Doing this operation for all implications in $\I$ provide a matrix $M$ of size $|\B| \times |\B|$. Then, to compute equivalence classes, a travel over the matrix is enough. Two implications $A \imp B$ and $C \imp D$ of $\I$ belong to the same equivalence class if 

\[ M\left[A \imp B, \ C \imp D \right] = 1 \quad \text{and} \quad 
M\left[C \imp D, \ A \imp B \right] = 1 \]

\noindent Building the matrix requires $|\B|$ executions of \textsc{LinClosure}
in any case, thus has $O(|\B||\I|)$ time complexity. Running across $M$ is 
of course $O(|\B|^2)$. Hence the whole operation can be done in $O(|\B||\I|)$,
since $|\B||\I| = |\B|^2|\Sg| > |\B|^2$. We will not rewrite 
\textsc{LinClosure} altered since the modification is about one line in the 
algorithm and quite simple to understand as is. We will not write the run over 
$M$ for conciseness, since principle seems sufficient for understanding. All 
those steps will be summarized as \textsc{EquivClasses}($\I$) in subsequent 
algorithm. Finally, the whole Maier minimization process is given in algorithm
\ref{alg:Maier-Min}.

\vspace{1.2em}

\begin{algorithm}
	\KwIn{$\I$ : a theory to minimize}
	\KwOut{$\I$ minimized}
	
	\BlankLine
	\BlankLine
	
	\ForEach{$A \imp B \in \I$}{
		\If{$\I - \{ A \imp B \} \models A \imp B$}{
			remove $A \imp B$ from $\I$ \;
		}
		
	}
	
	\BlankLine
	
	$E_{\I} := $ \textsc{EquivClasses}($\I$) \;
	
	\BlankLine
	
	\ForEach{$E_{\I}(X) \in E_{\I}$}{
		\ForEach{$A \imp B \in E_{\I}(X)$}{
			\If{$\exists C \imp D \in E_{\I}(X)$ s.t $A \ddv C$}{
				remove $A \imp B$ from $\I$ \;
				replace $C \imp D$ by $C \imp D \cup B$ \;	
			}
		}
		
	}
	
	\caption{\textsc{MaierMinimization}}
	\label{alg:Maier-Min}
\end{algorithm}

A question we could have is about complexity of removing direct determinations.
In fact, we can use again a modified version of \textsc{LinClosure} to find
direct determination. For each implications $A \imp B$, we would have to 
provide \textsc{LinClosure} with a vector of implications with premises 
equivalent to $A$. The first one we reach (i.e the first one for which the 
counter goes to 0) is necessarily an example of direct determination. Moreover,
note that equivalence classes in $E_{\I}$ defines a partition of $\I$. That is,
we will have to compute at most $|\B|$ closures to get rid off direct 
determinations. So, the last step of the algorithm requires $O(|\B||\I|)$,
which is consequently the complexity of \textsc{MaierMinimization} by 
previous explanations. It is important to mention that even though the base
we obtain is minimal, it is not the canonical basis, as we shall see in the
next example, acting as a trace.

\vspace{1.2em}

\paragraph{Example} Let us use the example we presented when studying the first algorithm, but this time, applying explicitly Maier's algorithm on it. As a reminder, we had:
\begin{itemize}
	\item[-] $\Sg = \{a, \ b, \ c, \ d, \ e, \ f \}$,
	\item[-] $\I =$ \{\textit{ab $\imp$ cde, cd $\imp$ f, c $\imp$ a, d $\imp$ b, abcd $\imp$ ef} \} 
\end{itemize}
\noindent We will proceed by steps.
\begin{enumerate}
	\item \midemp{redundancy elimination}: in this step, we compute the closure
	of each premise to see whether there exists $A \imp B \in \I$ such that 
	$B \subseteq \I^{-}(A)$. It turns out that the only one for which
	this happens is $abcd \imp ef$. After this step, we have:
	
	\begin{center}
		$\I =$ \{\textit{ab $\imp$ cde, cd $\imp$ f, c $\imp$ a, d $\imp$ b} \} 
	\end{center}
	
	\item \midemp{getting equivalence classes}: here is a more interesting step.
	First, we have to compute the matrix $M$ (see table \ref{tab:Maier-Mat}).
	The table does not represent the way computations are done, but is still of
	interest. On the left-hand side, we described the closure of each premise
	of $\I$. On the right-hand side, we gave the matrix $M$. We can see that
	an element $M(i, j)$ of $M$ equals 1 if the closure of $i$ contains the
	premise of $j$.
	
	\begin{table}[ht]
		\centering
		\subfloat[closures of $\B(\I)$]{
			\begin{tabular}{| c | c |}
				\hline \rowcolor{clouds}
				$\B(\I)$ & $\I(\cdot)$ \\ \hline
				$ab$ & $abcdef$ \\ \hline
				$cd$ & $abcdef$ \\ \hline
				$c$  & $ac$ \\ \hline
				$d$  & $bd$ \\ \hline
			\end{tabular}
		}\quad
		\subfloat[matrix $M$]{
			\begin{tabular}{c | c c c c}
				& $ab \imp cde$ & $cd \imp f$ & $c \imp a$ & $d \imp b$ \\ 
				\hline
				$ab \imp cde$ & 1 & 1 & 1 & 1 \\
				$cd \imp f$   & 1 & 1 & 1 & 1 \\
				$c \imp a$    & 0 & 0 & 1 & 0 \\
				$d \imp b$    & 0 & 0 & 0 & 1 \\    
			\end{tabular} 
			
		}
		\caption{Computing matrix $M$ of implied premises}
		\label{tab:Maier-Mat}
	\end{table}
	
	\noindent Then, we need to derive out of $M$ the different equivalence 
	classes. For all pairs of implications $(i, j)$, if $M(i, j) = M(j, i) = 1$,
	then they belong to the same equivalence class. In our case, we will 
	partition $\I$ in 3 classes:
	\begin{itemize}
		\item[-] $E_{\I}(ab) = \{ ab \imp cde, \ cd \imp f \}$ (= $E_{\I}(cd)$),
		\item[-] $E_{\I}(c) = \{ c \imp a \}$,
		\item[-] $E_{\I}(d) = \{ d \imp b \}$
	\end{itemize}
	\item \midemp{removing direct determination}: last step. We have to look in
	all equivalence classes for distinct implications with direct determination.
	Because $E_{\I}(c)$ and $E_{\I}(d)$ are of size 1, they cannot be reduced.
	However, $E_{\I}(ab)$ is more interesting. We do not have $ab \ddv cd$. 
	Indeed, the only way to reach $cd$ from $ab$ is to use $ab \imp cde$, that 
	is, an element of $E_{\I}(ab)$. Nevertheless, $cd \ddv ab$ because if we 
	restrict ourselves to $\I - E_{\I}(ab) = \{ c \imp a, \ d \imp b \}$, $cd 
	\imp ab$ holds. Consequently, we can apply our modifications: we remove 
	$cd \imp f$ from $\I$, and $ab \imp cde$ becomes $ab \imp cdef$.
\end{enumerate}
After applying this algorithm, we end up with a minimal $\I$ being:

\[ \I = \{ c \imp a, \ d \imp b, \ ab \imp cdef \} \]

\vspace{1.2em}

In this section we provided a theoretical study of the algorithm 
proposed by Maier in \cite{maier_theory_1983, maier_minimum_1980} for finding a 
minimal cover of a basis $\I$. Based on his results, we stated that the  
asymptotic complexity of this algorithm was $O(|\B||\I|)$. In the next section, 
we will develop another procedure coming from the graph theory community.

\subsection{Graph-theoretic approach to Maier's algorithm}

This section is dedicated to a minimization algorithm relying on graphs. 
It has been set up by Ausiello and al. in \cite{ausiello_directed_2017, 
ausiello_graph_1983, ausiello_minimal_1986}. Starting from a directed 
hypergraph representation of functional dependencies, it builds a special kind 
of directed graph, called \belemp{FD-Graph} with which it reduces the initial 
hypergraph. In order, we are going to define what is a FD-graph, provide the 
general idea for the algorithm as explained in \cite{ausiello_minimal_1986} and 
then go into further details and more precise algorithms for such computations 
as exposed in \cite{ausiello_graph_1983}.

\subsubsection{FD-Graphs and minimum covers}

As we already mentioned, the graph framework developed by Ausiello and al. in
\cite{ausiello_graph_1983, ausiello_minimal_1986} comes from the work of Maier
in database theory over functional dependencies (see \cite{maier_theory_1983}).
Moreover we already discussed the closeness of FD and implications in our 
context, hence we can still consider the algorithms we are about to study from
an implication point of view. This leads to no alteration. Furthermore, the
hypergraph representation of some theory $\I$ is no more than worth mentioning
for us, since it just presents an attractive graphical description of $\I$.
Because the structure presented by Ausiello is a particular kind of directed
graph, let us try to keep explanations as simple as possible and stick to this
one. It might be first interesting to recall what are graphs (undirected and 
directed) as an introduction.

\begin{definition}[\midemp{graph}] A \belemp{graph} $G = (V, E)$ is a pair
of sets where $V$ is a set of \belemp{nodes} or \belemp{vertices} and $E$ is
a set of unordered pair called \belemp{edges} or \belemp{arcs} from $V^2$.
\end{definition}

\begin{definition}[\midemp{directed graph}] A graph $G = (V, E)$ where $E$ is
a set of \belemp{ordered} pairs from $V^2$ is called a \belemp{directed graph}.	
\end{definition}

\paragraph{Example} Let us illustrate those notions with examples. First, let us
imagine a graph (not directed) $G_1 = (V_1, E_1)$ where $V_1 = \{
a, b, c, d, e\}$ is a set of cities and
$E_1$ would be railways between them, as $(b, c)$ and $(a, d)$ for example. 
Because railways are bidirectional, if we can go from one city to another, then 
the other way around is valid too. One possible \textit{"map"} is represented 
on the left side of figure \ref{fig:II-graph}.

\begin{figure}[ht]
	\centering
\subfloat[undirected graph $G_1$]{
\begin{tikzpicture}

\node[Vertex, label=above left:{$a$}] (a) at (0, 1) {};
\node[Vertex, label=below left:{$b$}] (b) at (-1.5, 0) {};
\node[Vertex, label=below left:{$c$}] (c) at (-1,-1.5) {};
\node[Vertex, label=below right:{$d$}] (d) at (1, -1.5) {};
\node[Vertex, label=below right:{$e$}] (e) at (1.5, 0) {};

\draw (a) -- (e) -- (d) -- (c) -- (b) -- (e) -- (c);
\draw (d) -- (a);


\end{tikzpicture}
}\qquad
\subfloat[directed graph $G_2$]{
\begin{tikzpicture}

\node[Vertex, label=above left:{\textit{Neige}}] (ng) at (-1, 1) {};
\node[Vertex, label=below left:{\textit{Seraphin}}] (s) at (-1, -1) {};
\node[Vertex, label=above right:{\textit{Narcisse}}] (na) at (1, 1) {};
\node[Vertex, label=below right:{\textit{Jezabel}}] (j) at (1, -1) {};

\draw[->] (ng) to[bend right=20] (j);
\draw[->] (s) -- (j);
\draw[->] (s) -- (ng);
\draw[->] (s) -- (na);
\draw[->] (j) -- (na);
\draw[->] (j) to[bend right=20] (ng);
\draw[->] (na) to[in=45, out=135, loop] (na);

\end{tikzpicture}
}

\caption{Representation of graphs $G_1$, $G_2$}
\label{fig:II-graph}

\end{figure}

For an example of directed graph, recall our \textit{"like"} binary relation from chapter 1. The associated graph is $G_2 = (V_2, E_2)$ where $V_2 = $ 
\{\textit{Narcisse, Neige, Jezabel, Seraphin}\} and for instance, 
(\textit{Seraphin, Jezabel}) is an edge of $E_2$ while (\textit{Jezabel, 
Seraphin}) is not. See right-hand side of figure \ref{fig:II-graph} for an 
illustration.

\vspace{1.2em}

Now that the notion of graph may be clearer, let us introduce a particular
kind of directed graph issued in \cite{ausiello_graph_1983, 
ausiello_minimal_1986} as an improvement of the structure proposed in 
\cite{ausiello_graph_1980}. It deserves to represent implication theories within
a framework simpler than hypergraphs. It has been recently re-issued in 
\cite{ausiello_directed_2017} being a survey.

\begin{definition}[\midemp{FD-Graph}] Given a theory $\I$ over some $\Sg$, 
the directed graph $G_{\I} = (V, \  E)$ such that:
\begin{itemize}
	\item[-] $V = V_0 \cup V_1$ is the set of nodes where:
	\begin{itemize}
		\item[$\bullet$] $V_0 = \Sg$ is the set of \belemp{simple} nodes (a node
		per attribute in $\Sg$),
		\item[$\bullet$] $V_1 = \{X | X \in \body{\I} \}$ is the set of 
		\belemp{compound} nodes (a node per distinct body in $\I$),
	\end{itemize}
	
	\item[-] $E = E_0 \cup E_1$ is the set of arcs where:
	\begin{itemize}
		\item[$\bullet$] $E_0$ is the set of \belemp{full} arcs. We have a full arc $(X, i)$ in	$E_0$ if $(X, i)$ is an hyperarc of $\I$,
		\item[$\bullet$] $E_1$ the set of \belemp{dotted} arcs. For each compound node $X$ of $V^1$, we have a dotted arc $(X, \ i)$ to every attributes $i$ of $X$,
	\end{itemize}
	
\end{itemize}
\noindent is the \belemp{Functionnal Dependency Graph} or \belemp{FD-Graph} 
associated to $\I$.
\end{definition}


Again, the definition may be quite confusing. $\B(\I)$ is still the set of premises of $\I$. Therefore, let us pause our explanations with some toy examples, presented in figure \ref{fig:FD-Graph-1}. 
From those graphs, we can give \textit{"handy"} way
to build an FD-graph out of some theory $\I$:
\begin{itemize}
	\item[-] every single attribute of $\Sg$ is a node, as every premise of $\I$,
	\item[-] for each $A \imp B$ of $\I$ we draw a \belemp{full} arc from the node
	$A$ to \textit{every attribute} of $B$,
	\item[-] for each compound node $A$, we draw a \belemp{dotted} arc from
	$A$ to \textit{all of its attribute}.
\end{itemize}
\noindent This is indeed what we formally defined previously. Furthermore, for
this algorithm, we consider a basis $\I$ over an attribute set $\Sg$, such that:
\begin{itemize}
	\item[-] there is no $A \imp B$, $A' \imp B'$ in $\I$ such that $A = A'$ when
	$B \neq B'$,
	\item[-] for all $A \imp B$ of $\I$, $A \cap B = \emptyset$
\end{itemize}
\noindent A theory in this form is said \belemp{reduced}.

\begin{figure}[ht]
	\centering
\subfloat[FD-Graph of $a \imp bc$]{
\begin{tikzpicture}
\node[Vertex, label=left:{$a$}] (1) at (-0.5, 0) {};
\node[Vertex, label=right:{$b$}] (2) at (0.5, 0.5) {};
\node[Vertex, label=right:{$c$}] (3) at (0.5, -0.5) {};

\draw[->] (1) -- (2);
\draw[->] (1) -- (3);
\end{tikzpicture}
}\qquad
\subfloat[FD-Graph of $bc \imp a$]{
\begin{tikzpicture}
\node[Vertex, label=below:{$bc$}] (23) at (0, 0) {};
\node[Vertex, label=right:{$a$}] (1) at (1, 0) {};
\node[Vertex, label=left:{$b$}] (2) at (-1, 0.5) {};
\node[Vertex, label=left:{$c$}] (3) at (-1, -0.5) {};

\draw[->, dotted] (23) -- (2);
\draw[->, dotted] (23) -- (3);
\draw[->] (23) -- (1);
\end{tikzpicture}		
}

\subfloat[FD-Graph of $ab \imp cd$]{
\begin{tikzpicture}
\node[Vertex, label=below:{$ab$}] (12) at (0, 0) {};
\node[Vertex, label=left:{$a$}] (1) at (-1, 0.5) {};
\node[Vertex, label=left:{$b$}] (2) at (-1, -0.5) {};
\node[Vertex, label=right:{$c$}] (3) at (1, -0.5) {};
\node[Vertex, label=right:{$d$}] (4) at (1, 0.5) {};

\draw[->, dotted] (12) -- (2);
\draw[->, dotted] (12) -- (1);
\draw[->] (12) -- (3);
\draw[->] (12) -- (4);
\end{tikzpicture}	
}

\caption{Representation of some FD-graph}
\label{fig:FD-Graph-1}
\end{figure}

\vspace{1.2em}

The next definition is about describing a graph-theoretic way to combine 
implications to derive new ones. This notion is essential for all the following 
material and is called \belemp{FD-paths}.

\begin{definition}[\midemp{FD-Path}] Given an FD-Graph $G_{\I} = (V, E)$, an
\belemp{FD-Path} $\langle i, \ j \rangle$ is a minimal subgraph 
$\bar{G}_{\I} = (\bar{V}, \bar{E})$ of $G_{\I}$ such that $i, j \in \bar{V}$ 
and either $(i, j) \in \bar{E}$ or one of the following holds:
\begin{itemize}
	\item[-] $j$ is a simple node and there exists $k \in \bar{V}$ such that 
	$(k, j) \in \bar{E}$ and there exists a FD-Path $\langle i, \ k \rangle$ 
	included in $\bar{G}$, 
	 
	\item[-] $j = \bigcup_{k = 1}^n j_k$ is a compound node and there exists 
	FD-paths $\langle i, \ j_k \rangle$ included in $\bar{G}$, for all $k = 
	1, \ \dots, \ n$.
\end{itemize}
	
\end{definition}

\noindent Informally, an FD-path from a node $i$ to $j$ describes the 
implications we use to derive $i \imp j$. Intuitively, 
directed paths are FD-paths. But there is also one case in which
we can go \textit{"backward"} in the graph. For better understanding, see examples of figure \ref{fig:FD-Graph-3} based on the theory described in figure 
\ref{fig:FD-Graph-2}. To be more precise, $\I = \{ ab \imp f, \ af \imp g,
\ a \imp c, \ b \imp d, \ cd \imp e, \ c \imp h, \ cd \imp e \}$.

\begin{figure}[ht]
	\begin{figure}[ht]\centering
\begin{tikzpicture}
\node[Vertex, label=left:{ab}] (ab) at (-2, 0) {};
\node[Vertex, label=left:{f}] (f) at (-2, 2) {};
\node[Vertex, label=right:{af}] (af) at (-1, 3) {};
\node[Vertex, label=above:{g}] (g) at (-1, 4) {};
\node[Vertex, label=below:{a}] (a) at (-1, 1) {};
\node[Vertex, label=below:{b}] (b) at (-1, -1) {};
\node[Vertex, label=below:{c}] (c) at (0, 1) {};
\node[Vertex, label=below:{d}] (d) at (0, -1) {};
\node[Vertex, label=above:{h}] (h) at (1, 2) {};
\node[Vertex, label=left:{cd}] (cd) at (1, 0) {};
\node[Vertex, label=below:{e}] (e) at (2, 0) {};

\draw[->, dotted] (af) -- (f);
\draw[->, dotted] (af) -- (a);
\draw[->, dotted] (ab) -- (a);
\draw[->, dotted] (ab) -- (b);
\draw[->, dotted] (cd) -- (c);
\draw[->, dotted] (cd) -- (d);
\draw[->] (af) -- (g);
\draw[->] (ab) -- (f);
\draw[->] (a) -- (c);
\draw[->] (b) -- (d);
\draw[->] (c) -- (h);
\draw[->] (cd) -- (e);
\end{tikzpicture}

\caption{FD-Graph of some implicational basis}
\label{fig:FD-Graph-2}
\end{figure}
\end{figure}

There are either \belemp{dotted} or \belemp{full} paths. A path $\langle i, j
\rangle$ is dotted if all arcs leaving $i$ are dotted, it is full otherwise.

\begin{figure}[ht]
	\begin{figure}[ht]\centering
\subfloat[][dotted FD-path for $ab \imp e$]{
\begin{tikzpicture}
\node[Vertex, label=left:{ab}] (ab) at (-2, 0) {};
\node[Vertex, label=below:{a}] (a) at (-1, 1) {};
\node[Vertex, label=below:{b}] (b) at (-1, -1) {};
\node[Vertex, label=below:{c}] (c) at (0, 1) {};
\node[Vertex, label=below:{d}] (d) at (0, -1) {};
\node[Vertex, label=below:{cd}] (cd) at (1, 0) {};
\node[Vertex, label=right:{e}] (e) at (2, 0) {};

\draw[->, dotted] (ab) -- (a);
\draw[->, dotted] (ab) -- (b);
\draw[->, dotted] (cd) -- (c);
\draw[->, dotted] (cd) -- (d);
\draw[->] (a) -- (c);
\draw[->] (b) -- (d);
\draw[->] (cd) -- (e);

\end{tikzpicture}
}\qquad
\subfloat[][full FD-path for $ab \imp g$]{
\begin{tikzpicture}
\node[Vertex, label=below:{ab}] (ab) at (0, 0) {};
\node[Vertex, label=right:{a}] (a) at (1, 1) {};
\node[Vertex, label=left:{f}] (f) at (0, 2) {};
\node[Vertex, label=right:{af}] (af) at (1, 3) {};
\node[Vertex, label=right:{g}] (g) at (1, 4) {};

\draw[->, dotted] (ab) -- (a);
\draw[->, dotted] (af) -- (a);
\draw[->, dotted] (af) -- (f);
\draw[->] (ab) -- (f);
\draw[->] (af) -- (g);
\end{tikzpicture}		
}

\caption{Representation of some FD-paths}
\label{fig:FD-Graph-3}
\end{figure}
\end{figure}

\vspace{1.2em}

Having explained FD-Graphs, we will now move to explanations of the algorithm
developed by Ausiello and al. The procedure finds from a given basis 
its minimal representation in our terms (see alorithm \ref{alg:Ausiello}).

\vspace{1.2em}

\begin{algorithm}[H]
	\KwIn{$\I$ an implication basis}
	\KwOut{$\I_c$ a minimal cover for $\I$}
	
	\BlankLine
	\BlankLine
	
	Find the \belemp{FD-Graph} of $\I$ \;
	Remove \belemp{redundant} nodes \;
	Remove \belemp{superfluous} nodes \;
	Remove \belemp{redundant} arc \;
	Derive $\I_c$ from the new graph \;
	
	\caption{\textsc{AusielloMinimization} (Overview, 1983)}
	\label{alg:Ausiello}
\end{algorithm}

\vspace{1.2em}

As we will see in detailed explanations, those steps are equivalent to Maier's
procedure. In fact, the last part, removing redundant arcs, goes beyond the 
scope of our needs since it deserves to reduce sizes of premises and 
conclusion, not the number of implications. This has also been studied in 
Maier's work, but for out of scope reason we did not reviewed it. For the same
argument here, we will not focus on it either. To help the reader see where we 
are heading, one should keep in mind that the two other steps of removing 
redundant nodes and superfluous nodes will be equivalent to removing redundant 
implication and direct determination respectively. However, we must first
dive into the closure of an FD-Graph (parallel to closure of implications) to
be able to perform removal steps.

\subsubsection{Closure of an FD-Graph}

The closure is based on the following data structures:
\begin{itemize}
	\item[-] $V_0$: set of \belemp{simple} nodes,
	\item[-] $V_1$: set of \belemp{compound} nodes,
	\item[-] $D_i$ ($\forall i \in V$): nodes from \belemp{incoming dotted} arcs
	$\{j \in V \ | \ (j, \  i) \text{ is a dotted arc} \}$,
	\item[-] $L_{i}^0$ ($\forall i \in V$): nodes from \belemp{outgoing full} arcs
	$\{j \in V \ | \ (i, \  j) \text{ is a full arc} \}$,
	\item[-] $L_{i}^1$ ($\forall i \in V$): nodes from \belemp{outgoing dotted} 	
	arcs $\{j \in V \ | \ (i, \  j) \text{ is a dotted arc} \}$,
	\item[-] $L_{i}^{0+}, L_{i}^{1+}$ ($\forall i \in V$): the respective closures
	of $L_i^0, L_i^1$,
	\item[-] $q_m$ ($\forall m \in V^1$): counter of nodes in $m$ belonging to 
	$L_i^{0+} \cup L_i^{1+}$ for some $i \in V$.
\end{itemize}

\noindent To make understanding easier, we first give pseudo-code closer from
principle than algorithms. From a general point of view, to determine the 
closure of a FD-graph, we must compute the closure of all its nodes. The 
closure of a node is described by its full and dotted outgoing arcs. Because we
put a priority on dotted possibilities, they will be computed before. Principle
are given in algorithmic/pseudo-code form so that identification between steps
of procedures and ideas of principle are easier to see.

\vspace{1.2em}

First, we introduce the procedure \textsc{NodeClosure} which computes the 
closure of a node with respect to a type of arc. In other words, to compute the 
full closure of a node, we must first apply \textsc{NodeClosure} to its dotted 
arcs, then to its full arcs. The principle and algorithm for 
\textsc{Nodeclosure} are procedures \ref{alg:FD-NodeClosure-Principle}, 
\ref{alg:FD-NodeClosure}. 

\vspace{1.2em}

\begin{algorithm}
	\KwIn{
		$L_i$: set of nodes for which there exists dotted (resp. full) arcs 
		$(i, j)$}
	\KwOut{$L_i^+$: the dotted (resp. full) closure of $i$}
	
	\BlankLine
	\BlankLine
	
	Initialize a list of nodes to treat $S_i$ to $L_i$ \;
	\While{there is a node $j$ to treat in $S_i$}{
		remove $j$ from $S_i$ \;
		\If{$j$ is simple node}{
			\ForAll{compound node $m$ \belemp{except $i$}, $j$ appears in}{
				increase $q_m$ by 1 \;
				\If {$q_m$ = number of outgoing \belemp{dotted} arcs from $m$}{
					$m$ is reachable from $i$ by \aliemp{union} \;
					$m$ must be treated, add it to $S_i$ \;
				}
			}
			
		}
		
		add $j$ to the closure $L_i^+$ \;
		
		\ForAll{nodes $k$ such that there is an arc $(j, \ k)$}{
			\If{$k$ is not yet in the closure $L_i^+$ or in the \belemp{dotted}
				closure $L_i^{1+}$ of $i$}{
				$k$ is reachable from $i$ by \aliemp{transitivity} \;
				$k$ must be treated, add it to $S_i$ \;
			}
		}
		
		return $L_i^{+}$ \;
	}
	
	
	\caption{\textsc{NodeClosure} (Principle)}
	\label{alg:FD-NodeClosure-Principle}
\end{algorithm}

We would like to provide some observations on top of their description. Namely 
on the \aliemp{union} step and $q_m$ counters. Say $i \imp m$ where $m$ is a 
compound node is a valid implication in a FD-graph. Furthermore say $m = 
\bigcup_i m_i$ where $m_i$'s are simple nodes. The union step models the fact 
that if we have $i \imp m_i$ for all $m_i$ in $m$, then we must have $i \imp m$ 
also. The counter $q_m$ ensures that we indeed reached all $m_i$'s in $m$. 
Also, the algorithm has access to all the structures we described above (nodes, 
sets of arcs, and so forth). Parameters are thus lists we are going to modify 
somewhat. The \textsc{NodeClosure} algorithm runs in time $O(|\I|)$. The first 
nested loop runs in at most $O(|\Sg| \times |\B|) = O(|\I|)$ because 
$S_i$ contains at most $|\Sg|$ elements, and the block referring to the 
\textit{union} rule runs over compound nodes, that is bodies of $\I$. For the 
second loop (transitivity) note that we can at most consider all the edges of 
the FD-graph. In fact, the cost of transitivity operation for all $j$ is 
$O(\sum_{j = 1}^n |L_j^0 \cup L_j^1 |)$. But by definition, those sets are 
disjoints, and therefore we cannot treat more than $| E |$ arcs (the total 
number of arcs in $G$), that is $|\I|$. It is important to note that we can
reach this complexity only if we consider the closure of a node to be a matrix in which accessing an element is $O(1)$. Hence, if we consider the FD-Graph to be an adjacency list so as to ensure $|G_{\I}| = O(|\I|)$ as mentioned in \cite{ausiello_graph_1983, ausiello_minimal_1986}, the closure is represented 
through an adjacency matrix and adding element to the closure may be understood as \textit{"setting the value to 1 in the vector corresponding to the closure of i"}. The size of the graph representing the closure is then $O((|\B| + |\Sg|)^2)$ since every node appears in the closure.

\begin{algorithm}
	\KwIn{
		$L_i$: set of nodes for which there exists dotted (resp. full) arcs 
		$(i, j)$}
	\KwOut{$S_i^+$: the dotted (resp. full) closure of $i$}
	
	\BlankLine
	\BlankLine	
	
	$S_i := L_i$ \;
	$S_i^{+} := \emptyset$ \;
	
	\While{$S_i \neq \emptyset$}{
		select $j$ from $S_i$ \;
		\If{$j \in V^0$}{
			\ForAll{$m \in D_j - \{ i \}$}{
				$q_m := q_m + 1$ \;
				\If{$q_m = |L_m^1|$}{
					$S_i := S_i \cup \{ m \}$ \;
				}
				
			}
		}
		
		$S_i^+ := S_i^+ \cup \{ j \}$ \;
		
		\ForAll{$k \in L_j^0 \cup L_j^1$}{
			\If{$k \not\in S_i^+ \cup L_i^{1+} \cup \{ i\}$}{
				$S_i := S_i \cup \{ k \}$ \;
			}
		}
	}
	return $S_i^+$ \;
	\caption{\textsc{NodeClosure}}
	\label{alg:FD-NodeClosure}
\end{algorithm}

\vspace{1.2em}

Next, we present the principle and pseudo-code for the closure of an FD-graph
\ref{alg:FD-Closure-Principle}, \ref{alg:FD-Closure}. Mostly, the principle is 
the idea we described previously. There is just one observation to make about 
setting a counter $q_m$ to 1. This variable helps to see whether we can use 
union rule as we saw in procedure \textsc{NodeClosure}
(\ref{alg:FD-NodeClosure-Principle}, \ref{alg:FD-NodeClosure}). We initialize it
in case $i$ is indeed part of some compound node so that we do not omit to count
it when dealing with $S_i$ (because $S_i$ does not contain $i$). In terms of
complexity, we are running \textsc{NodeClosure} on all nodes having outgoing
edges, that is $|\B|$ nodes (if a compound node is represented, it must
have at least one outgoing full arc). Since \textsc{NodeClosure} operates in
$O(|\I|)$, the whole closure algorithm must run in $O(|\B| \times |\I|)$.

\begin{algorithm}
	\KwIn{$V_0$, $V_1$ and $\forall i \in V$ $D_i$, $L_i^0$, $L_i^1$}
	\KwOut{$\forall i \in V$ $L_i^{0+}$, $L_i^{1+}$}
	
	\BlankLine
	\BlankLine
	
	\ForAll{node $i$ in $V$ with outgoing arcs}{
		
		\If{$i$ is an attribute of a compound node $m$}{
			set a counter $q_m$ to $1$ \;	
		}
		
		initialize the closure of $i$ to $\emptyset$ \;
		
		\If{$i$ is a compound node}{
			determine \belemp{dotted} arcs in the closure of $i$ \;
		}
		
		determine \belemp{full} arcs in the closure of $i$ \;
		
	}
	
	\caption{\textsc{GraphClosure} (Principle)}
	\label{alg:FD-Closure-Principle}
\end{algorithm}

\begin{algorithm}
	\KwIn{$V_0$, $V_1$ and $\forall i \in V$ $D_i$, $L_i^0$, $L_i^1$}
	\KwOut{$\forall i \in V$ $L_i^{0+}$, $L_i^{1+}$}
	
	\BlankLine
	\BlankLine
	
	\ForAll{$i \in V$ with $L_i^0  \cup L_i^1 \neq \emptyset$}{
		
		\ForAll{$m \in V^1$}{
			\uIf{$m \in D_i$}{
				$q_m := 1$ \;
				
			} \Else {
				$q_m := 0$ \;
				
			}
		}
		
		$L_i^{1+} := \emptyset$ \;
		$L_i^{0+} := \emptyset$ \;
		
		\If{$i \in V^1$}{
			$L_i^{1+} := $ \textsc{NodeClosure}($L_i^{1}$) \;	
		}
		
		$L_i^{0+} := $ \textsc{NodeClosure}($L_i^{0} - L_i^{1+}$) \;
		
	} 
	
	\caption{\textsc{GraphClosure}}
	\label{alg:FD-Closure}
\end{algorithm}

\vspace{1.2em}

Now that algorithms for computing the closure of a FD-graph have been set, we
can move to the minimization part.

\subsubsection{Removing redundant nodes}

The first step is about removing redundant implications. In terms of FD-graphs, 
we remove redundant nodes. A compound node (only) $i$ is said \belemp{redundant} if for each full arc $(i, j)$ leaving $i$
there exists a dotted path $\fdpath{i}{j}$. We give an example in the figure
\ref{fig:FD-Graph-4}.

\begin{figure}[ht]
\input{Pictures/II/RedExample.tex}
\end{figure}

In this example, the associated basis is $\I = { ab \imp cd \, ; \, a \imp c \, ; \, b \imp d}$. Indeed, in this case, $ab \imp cd$ is 
redundant because $\I - {ab \imp cd} \models ab \imp cd$. So removing a redundant node is removing exactly one implication in $\I$ since $\I$ is reduced. It is quite direct to see equivalence between redundancy of a node and
redundancy of the implication having this node as a premise. We give a 
proposition anyway to make everything clear.

\begin{proposition} An implication $A \imp B$ is redundant in $\I$ if and only if $A$ is a redundant node in the FD-graph $G_{\I}$ associated to $\I$.
\end{proposition}

\begin{proof} Assume $A \imp B$ is redundant in $\I$. Then $A \imp B$ still holds in $\I^{-} := \I - {A \imp B}$. The FD-graph $G_{\I^{-}}$ associated to 
$\I^{-}$ is in fact $G_{\I}$ where we got rid of node $A$ (being compound) and
of all its outgoing arcs, dotted and full. If $\I^{-} \models A \imp B$ we must be able to find implications $X_i \imp Y_i$, $X_i \subseteq A$ such that $\bigcup_{i} X_i \imp B$. In particular we could add a compound node $\bigcup_i X_i$ to $G_{\I^{-}}$ with only dotted arcs to its attribute so that
we would have only a dotted FD-path from $\bigcup_i X_i$ to $B$, hence from $A$
to $B$.

\vspace{1.2em}

Suppose $A$ is redundant node in $G_{\I}$. It has full outgoing arcs, and is
compound hence corresponds to a premise $A$ of $\I$. Because it is redundant, we
can remove all of its full outgoing arcs without any loss of information. Say
$A \imp B$ is the implication represented by the node $A$ and its full outgoing
arcs. Removing all is reducing $A \imp B$ to $A \imp \emptyset$ that is an
implication we can remove. Because there is still FD-path from $A$ to $B$ in this set up we have $A \imp B$ still holding in $\I$ where $A \imp B$ has been
replaced by $A \imp \emptyset$ equivalent to $\I - \{ A \imp B \}$.

\end{proof}

To remove redundant nodes, Ausiello and al. observed that a redundant node will
only have dotted arcs in the closure of a FD-Graph. Hence its minimization procedure for some $\I$ and associated $G_{\I}$ suggests to determine the closure of $G_{\I}$ and then to remove all redundant nodes by checking it. However, let us consider the following case:

\[ \I = \{ ab \imp d, \ bc \imp d, a \imp c, c \imp a \} \]

with the associated FD-Graph presented in figure \ref{fig:FD-Red-CountEx}. On the right-hand side of the figure we gave the closure of the FD-graph. We can
observe that two nodes are redundant, namely $ab$ and $bc$. Indeed, we have:
\begin{itemize}
	\item[-] $\I - \{ab \imp d \} \models ab \imp d$
	\item[-] $\I - \{bc \imp d \} \models bc \imp d$
\end{itemize}
\noindent Nevertheless, this does not mean we can remove the two of them. Indeed
those two implications are somehow \textit{"mutually redundant"}: if we remove
one, the other is not redundant anymore. For instance, consider removing $ab \imp d$ from $\I$. Then, $\I = \{ bc \imp d, a \imp c, c \imp a \}$. In this
case $\I - \{bc \imp d \} \not\models bc \imp d$ because even though $c \imp a$,
the lack of $ac \imp d$ prevent redundancy of $bc \imp d$. Therefore, the idea
proposed by Ausiello as we understood it would result in $\I = \{ a \imp c, \ c \imp a \}$ after redundancy elimination, being not correct. In Maier's term, this would be equivalent to first marking all redundant implications and then
removing all of them.

\begin{figure}[ht]
	\centering
\subfloat[FD-graph of $\I$]{
\begin{tikzpicture}

\node[Vertex, label=above left:{$ab$}] (ab) at (-2, 0) {};
\node[Vertex, label=below:{$b$}] (b) at (0, 0) {};
\node[Vertex, label=above right:{$bc$}] (bc) at (2, 0) {};
\node[Vertex, label=above:{$d$}] (d) at (0, 1) {};
\node[Vertex, label=below left:{$a$}] (a) at (-2, -1) {};
\node[Vertex, label=below right:{$c$}] (c) at (2, -1) {};

\draw[->] (ab) -- (d);
\draw[->] (bc) -- (d);
\draw[->] (a) to[bend right=10] (c);
\draw[->] (c) to[bend right=10] (a);

\draw[->, dotted, thick] (ab) -- (a);
\draw[->, dotted, thick] (ab) -- (b);
\draw[->, dotted, thick] (bc) -- (b);
\draw[->, dotted, thick] (bc) -- (c);

\end{tikzpicture}
}\qquad
\subfloat[Closure FD-graph of $\I$]{
\begin{tikzpicture}

\node[Vertex, label=above left:{$ab$}] (ab) at (-2, 0) {};
\node[Vertex, label=below:{$b$}] (b) at (0, 0) {};
\node[Vertex, label=above right:{$bc$}] (bc) at (2, 0) {};
\node[Vertex, label=above:{$d$}] (d) at (0, 1) {};
\node[Vertex, label=below left:{$a$}] (a) at (-2, -1) {};
\node[Vertex, label=below right:{$c$}] (c) at (2, -1) {};

\draw[->] (a) to[bend right=10] (c);
\draw[->] (c) to[bend right=10] (a);

\draw[->, dotted, thick] (ab) -- (a);
\draw[->, dotted, thick] (ab) -- (b);
\draw[->, dotted, thick] (bc) -- (b);
\draw[->, dotted, thick] (bc) -- (c);
\draw[->, dotted, thick] (ab) -- (c);
\draw[->, dotted, thick] (bc) -- (a);
\draw[->, dotted, thick] (ab) -- (d);
\draw[->, dotted, thick] (bc) -- (d);
\draw[->, dotted, thick] (ab) to[bend right=15] (bc);
\draw[->, dotted, thick] (bc) to[bend right=15] (ab);

\end{tikzpicture}
}

\caption{Representations of a redundant FD-graph and its closure}
\label{fig:FD-Red-CountEx}
\end{figure}

\vspace{1.2em}

FD-Graphs suffer from another drawback: representation of non-closed empty set.
To the best of our knowledge, this has not been discussed. While Maier's algorithm is flexible towards open empty set, FD-graphs may not, unless we missed informations. How to represent the basis $\I = \{ \emptyset \imp ab, a \imp b\}$? We thought of two possible choices:
\begin{itemize}
	\item[(i)] consider $\emptyset$ as a compound node without dotted arcs,
	hence like a simple node
	\item[(ii)] when $\emptyset$ is present, consider all simple nodes as 
	compound, and add a dotted arc for all of them into $\emptyset$
\end{itemize}
Let us investigate those two representations. As one may have noticed, $\I$
is redundant and we should only keep $\emptyset \imp ab$. The two ideas
are represented in figure \ref{fig:FD-empty}.

\begin{figure}[ht]
	\centering
\subfloat[Representation (i)]{
\begin{tikzpicture}

\node[Vertex, label=above left:{$a$}] (a) at (-1, 1) {};
\node[Vertex, label=above right:{$b$}] (b) at (1, 1) {};
\node[Vertex, label=below:{$\emptyset$}] (e) at (0, -1) {};

\draw[->] (a) -- (b);
\draw[->] (e) -- (a);
\draw[->] (e) -- (b);

\end{tikzpicture}
}\quad
\subfloat[Representation (ii)]{

\begin{tikzpicture}

\node[Vertex, label=above left:{$a$}] (a) at (-1, 1) {};
\node[Vertex, label=above right:{$b$}] (b) at (1, 1) {};
\node[Vertex, label=below:{$\emptyset$}] (e) at (0, -1) {};

\draw[->] (a) -- (b);
\draw[->] (e) -- (a);
\draw[->] (e) -- (b);

\draw[->, dotted, thick] (a) to[bend right=15] (e);
\draw[->, dotted, thick] (b) to[bend left=15] (e);

\end{tikzpicture}

}

\caption{Two possible representations of the empty set in FD-Graphs}
\label{fig:FD-empty}	
\end{figure}

\vspace{1.2em}

In the first representation, we will not remove any implication, since there
is no dotted arcs anywhere. On the right-hand side, we would remove simple
nodes, namely $a$, since we have indeed dotted FD-path from $a$ to $b$.
However, we would reduce our attribute set and consequently we would keep only
$\emptyset \imp b$ in $\I$. Therefore, none of those ideas is satisfying. In fact, one possible solution is to add a new element $\sigma$ to $\Sg$ acting
as the empty set. Then, for all premises $A$ of $\I$ we add $\sigma$ as a new
element of $A$. This however brings a solution in pre-processing our implication
theory, it does not solve the problem in a graph theoretic manner.

\vspace{1.2em}

That said, we could argue on two points. First, maybe we should doubt of our
interpretation of the algorithm. Second, let us consider we understood it well,
then a possible correction would be to compute both dotted and full closure for
all compound nodes of $\I$ to see whether they are redundant or not. If it is the case, we update the graph and its closure (among nodes already computed!) by removing the node. This would be strictly the same operation as Maier's redundancy elimination, plus the cost in time and memory of bidirectional translation from basis to FD-Graph. On top of that, one should consider the cost of removing only one node and all of its outgoing arcs within a graph. While this could be done quite easily in an adjacency matrix representation, the adjacency lists choice (which seems to be the one assumed somehow in Ausiello's work) would be more time consuming, namely $O(\I)$. 

\vspace{1.2em}

We shall discuss it again later on, but as for now, it seemed to us that  Ausiello algorithm requiring bidirectional translation, maybe misleading operations, or equal processing as in the Maier case was not worth implementing.


\subsubsection{Removing superfluous nodes}

From now on, assume we are given a nonredundant FD-Graph. Let us remove so-called superfluous nodes. A node $i$ is \belemp{superfluous} if there is an equivalent node $j$ and a dotted path from $i$ to $j$. Two nodes $i, j$ are \belemp{equivalent} if there are FD-paths $\fdpath{i}{j}$ and $\fdpath{j}{i}$.
It comes at no surprise that nodes are equivalent exactly when they have the
same closure in $\I$. From a theoretical point of view, the minimization algorithm suggests the following operation:
\begin{itemize}
	\item[-] find a superfluous node $i$, and an equivalent node $j$ with a dotted
	path from $i$ to $j$
	\item[-] for each full arc $ik$, we add a full arc $jk$
	\item[-] then we remove the node $i$ and all of its outgoing arcs from the 
	graph
	\item[-] repeat until no more superfluous nodes exist
\end{itemize}
\noindent An example of this procedure is given in the figure
\ref{fig:FD-Graph-5}. In this example $\I = {ab \imp e \, ; \, a \imp c
	\, ; \, b \imp d \, ; \, cd \imp ab}$. The node $ab$ is superfluous. Since 
our
basis are reduced, note that removing a superfluous node is removing exactly 
one implication in $\I$. In this case, the resulting $\I$ will be

\[ \I = {a \imp c, b \imp d, cd \imp abe} \] 


\begin{figure}[ht]
\input{Pictures/II/SupExample.tex}
\end{figure}


\noindent Now we may rewrite this operation in our terms. Let $A \imp B$ and 
for instance $C \imp D$ be part of $\I$ to be general. Then $A$ is superfluous
body if

\[ \I \models A \imp C, C \imp A \land \exists X \subset A \; s.t \;
\I \models X \imp C \]

\noindent In this case, we apply the following operations
\begin{itemize}
	\item[-] $C \imp D$ becomes $C \imp (D \cup B)$
	\item[-] we remove $A \imp B$ from $\I$
\end{itemize}

\noindent In order to prove correctness of this operation, we will show that
a node $A$ is superfluous exactly when $A$ directly determines some equivalent
$B$ in Maier's terms. Because in both case we are doing same replacement/deletion operation, if two statements are equivalent then the Ausiello algorithm is correct as Maier's one is.

\begin{proposition}\label{prop:maier.equiv_sup_sub} Let $A \imp C$ be the implication of $\I$ with $A$ as body. The following properties are equivalent:
\begin{itemize}
	\item[(i)] A node $A$ in a FD-graph is superfluous with respect to $B$,
	\item[(ii)] $A \equiv B$ and $\I - \{A \imp C \} \models A \imp B$.
\end{itemize}
\end{proposition}

\begin{proof} (i) $\imp$ (ii). If $A$ is superfluous, we have a dotted FD-Path
$\langle A, \ B \rangle$. Since it is dotted, let us remove this node $A$ 
and its outgoing arcs. Actually, none of the nodes pointed by dotted arcs of $A$
have been removed, thus we can still find the nodes $a_i$ (attributes of 
$a$) used in the dotted path $\langle A, \ B \rangle$ such that $\bigcup_i a_i 
\models B$. Because $\bigcup_i a_i \subseteq A$, we end up with $\I - \{A 
\imp C \} \models A \imp B$.

\vspace{1.2em}

(ii) $\imp$ (i). In the FD-Graph associated to $\I - \{A \imp C \}$, the 
node $A$ is not present. But still, $A \imp B$ holds. This means that we must be
able to find a list of proper subsets $A_i$ of $A$ (possibly single 
attributes) such that $\bigcup_{i} A_i \models B$. Adding $A \imp C$ will add the node $A$ and in particular dotted arcs from $A$ to each attributes of $\bigcup_{i} A_i \subseteq A$. Thus, we will have a dotted path from $A$ to $\bigcup_{i} A_i$ and consequently, to $B$. $A$ is indeed superfluous. Because $B$ is equivalent to $A$ by assumption, this property is preserved when adding a node.
	
\end{proof}


\begin{proposition} \label{prop:maier.equiv_ssup_dd}
the following statements are equivalent, for $A, B$ bodies of $\I$:
\begin{itemize}
	\item[(i)] $A \ddv B$ and $B \equiv A$,
	\item[(ii)] the node $A$ is superfluous with respect to $B$, and there 
	exists 
	a dotted FD-path from $A$ to $B$ not using any outgoing full arcs of
	nodes equivalent to $A$.
\end{itemize}
	
\end{proposition}


\begin{proof} (i) $\imp$ (ii). Using proposition
\ref{prop:maier.equiv_sup_sub} and the equivalence between $A \ddv B$
and $\I - E_{\I}(A) \models A \imp B$ (from Maier's terminology), showing that there is a direct determination starting from $A$ implies $A$ is a superfluous node in the FD-Graph is straightforward. If $\I - E_{\I}(A) \subseteq \I - \{A \imp C \} \models A \imp B$ so does $\I - \{A \imp C \}$. This holds in particular if $B \subseteq A$. Moreover, notice that using an outgoing full arc from a node $D$ equivalent to $A$ is exactly using an implication with left hand side equivalent to $A$. Therefore, if there is not dotted FD-path from $A$ to $B$ not using those arcs, we would contradict direct determination.

\vspace{1.2em}

(ii) $\imp$ (i). Suppose $A$ is superfluous and there exists a dotted FD-path from $A$ to $B$ not using any outgoing full arcs from nodes equivalent to $A$. Those full arcs represent exactly the implications contained in $E_{\I}(A)$. Since we don't use them, the path still holds in $\I - E_{\I}(A)$ (we would remove compound nodes without outgoing full arcs of course, but this would only make the path stops to attributes instead of compound node). Having this path in $\I - E_{\I}(A)$ means that $\I - E_{\I}(A) \models A \imp B$.
	
\end{proof}

With propositions 3, 6 and above remarks on operations done in FD-graphs, we
can conclude that \textsc{AusielloMinimization} performs from a graph-theoretic point of view the operations made by \textsc{MaierMinimization} in implications
framework. Hence both are correct and the resulting FD-graph represents a minimal basis in our terms. 

\vspace{1.2em}

In fact, the manipulations we described previously to remove superfluous nodes do not represent the true algorithm given by Ausiello and al for this purpose.
To be more precise, the papers \cite{ausiello_graph_1983, ausiello_minimal_1986} suggests to perform the computations of algorithm \ref{alg:Ausiello-Superfluous}.

\vspace{1.2em}

\begin{algorithm}
	\KwIn{$G_{\I}$: the FD-Graph of some non-redundant basis $\I$}
	\KwOut{$G_{\I}$: the associated minimum FD-Graph}
	
	\BlankLine
	\BlankLine
	
	\ForAll{$i \in V^1$}{
		find an equivalent node $j$ \;
		\If{$j$ exists}{
			$L_j^{0+} := L_j^{0+} \cup (L_i^{0+} \cap L_j^{1+})$ \;
			$L_j^{1+} := L_j^{1+} - (L_i^{0+} \cap L_j^{1+})$ \;
		}
		
		remove $i$ from the closure \;
		add $(i, j)$ to a list $L$ \; 
	}
	
	remove superfluous nodes \;
	move arcs to their final destination \;	
	
	\caption{\textsc{SuperfluousnessElimination}}
	\label{alg:Ausiello-Superfluous}
\end{algorithm}

For all compound nodes $i$, we check in its dotted closure $L_{i}^{1+}$ whether 
we can find an equivalent node $j$. This can be done in $O(|\B| + |\Sg|)$. If we found such $j$, then $i$ is superfluous and we move to the full closure of $j$ all nodes $k$ such that $(i, k)$ is full and $(j, k)$ dotted. This corresponds to the union operation we depicted earlier. When we remove an implication with a superfluous premise, we append to the consequence of the \textit{"target"} implication, the consequence of the superfluous one. This is also $O(|\B| + |\Sg|)$. Then, removing $i$ from the closure does also require a run over all nodes of the closure: $O((|\B| + |\Sg|))$. Therefore, the first part of this algorithm is $O(|\B|(|\B| + |\Sg|))$. The second part is about removing nodes and moving arcs. If run over $L$ from the end, we may be able to find the final destination of full arcs with only one run. Then, we can run over the graph, and for each arc, whether it points out to a superfluous node and we remove it, the node itself is redundant and the arc is full and we move it to its destination, or we do nothing. All these operations are $O(1)$ and the overhaul operation requires to visit all the edges, resulting in $O(|\I|)$ complexity. Therefore, the cost of removing superfluous node provided we have the closure of the FD-Graph we work on seems to be $O(|\B|(|\B| + |\Sg|))$, in
accordance with \cite{ausiello_graph_1983, ausiello_minimal_1986} being better than the removal of direct determination in Maier's algorithm.

\vspace{1.2em}

What we can conclude of the procedure issued by Ausiello is that it performs
the same work as Maier's algorithm but using advantage of the structure of
FD-Graph and its distinction between dotted/full arcs to enhance removing superfluous nodes / direct determination. However, as aforementioned, we did not implement this graph-theoretic approach. The main reason is some difficulties in clearly understanding the ideas of Ausiello, notably for the
redundancy step, empty set representation, or the data structures one should use to match theoretical complexity. Furthermore, as we exposed, the operations performed are somehow very similar to the work done by Maier even though FD-Graphs use their dotted-full arcs to slightly improve computations. Nevertheless, the whole algorithm may still require as much theoretical time as Maier since most of the properties of \textsc{AusielloMinimization} use the closure of an FD-Graph needing $O(|\B||\Sg|)$ to be computed. On top of that, to get from a basis to a graph, we may need (linear) translation time, as much as a large amount of memory (especially for the closure of a graph). Eventually, this algorithm could be considered as a weak point of our study since for all this reason taking time to investigate, we chose to focus on other algorithms. Plus, because of the similarities with \textsc{MaierMinimization}, limiting ourselves to the Maier's algorithm permits to have a first insight on the efficiency of core operations. Nevertheless, as all this section shows, we still provided some theoretical comparison to exhibit the heart of \textsc{AusielloMinimization} being a translation from a framework to another. Hence, with those explanations we leave the door open to a future implementation. The interest of truly implementing this graph algorithm will depend on the efficiency of \textsc{MaierMinimization} in practice.

\vspace{1.2em}

In this section we studied \textsc{MaierMinimization} and its graph-theoretic version \textsc{AusielloMinimization}, two algorithms designed for functional 
dependencies minimization. It remains now to focus on algorithms coming from
logical and query learning communities.


\section{Propositional logic based approach}

The last domain we will explore is boolean logic. Actually, boolean logic is just a background as previous domains and we will focus to the expression of the algorithms within our framework. The main studies we will focus on can be found in \cite{boros_strong_2017, berczi_directed_2017, angluin_learning_1992, arias_canonical_2009}.  Anyway, as an informal introduction, let us see the 
parallel between logic and our set-theoretic representation. This introduction is absolutely not mandatory for the understanding of next material, but it can be interesting for the reader with some knowledge in propositional logic.


\subsection{From sets to boolean logic}

We will need the usual logical notations $\lnot, \lor, \land, \imp$ (negation, disjunction, conjunction, implication for the recall) we defined in the first chapter. The reader can read \cite{cori_mathematical_2000} to have in-depth explanations. For a moment, let $\Sg$ be a set of \belemp{boolean variables}: variables being either \textit{true} ($\top$, $1$) or \textit{false} ($\bot$, $0$). From this
point of view, an implication is in fact a so-called \belemp{clause}. A clause
is a disjunction of literals. For instance:
\begin{itemize}
	\item[-] $(x_1 \lor x_2 \lor \lnot x_3)$ is a clause,
	\item[-] $((x_1 \lor x_2) \land x_3)$ is not (in our context).
\end{itemize}
\noindent And in particular, a clause in which there is \belemp{at most} one
positive literal is called a \belemp{Horn clause}. When there is exactly one
positive variable, it is a \belemp{definite} such clause. In our context, we only have definite Horn clauses:
\begin{itemize}
	\item[-] $(x_1 \lor \lnot x_2 \lor \lnot x_3)$ is a definite Horn clause,
	\item[-] $(\lnot x_1 \lor \lnot x_2)$ is a Horn clause not definite.
\end{itemize}
\noindent Because $x_1 \imp x_2$ is logically equivalent to $x_2 \lor \lnot x_1$
and thanks to de Morgan's law on negation (the negation of a disjunction is the conjunction of the negations and vice-versa), we may rewrite a clause $(x_1 \lor \lnot x_2 \lor \lnot x_3)$ as an implication $(x_2 \land x_3) \imp x_1$. 
Now, we can associate several clauses in a conjunction:

\[ ((x_2 \land x_3) \imp x_1) \land ((x_2 \land x_3 \land x_4) \imp x_5) 
\land (x_2 \land x_3) \imp x_4) \]

\noindent Such a conjunction is called a \belemp{Horn formula}. When written
under its clause form, it is also a formula under \belemp{Conjunctive Normal Form} (CNF): a conjunction of disjunctive clauses. To get back up on our feet
with sets, we only lack one step. Given a subset $A$ of an attribute set $\Sg$, we can use the \belemp{characteristic vector} $\psi_A$ of $A$. It is a binary vector of size $|\Sg|$ such that the $i$-th bit of $\psi_A$ is a boolean variable saying whether the $i$-th element of $\Sg$ is in $A$ or not. This supposes $\Sg$ to be totally ordered, which is never an issue. For instance,
let $\Sg = \{ a, b, c, d, e\}$ be an attribute set (not boolean variables),
and $A = abd$, then $\psi_A = 11010$. Now, we can associate propositional variables $x_a, x_b, x_c, x_d, x_e$ to $a, b, c, d, e$ respectively. Finally,
let us take an implication say $ab \imp de$. This attribute implication can be
transformed into logical one:

	\[ x_a \land x_b \imp x_d \land x_e = (x_a \land x_b \imp x_d)
	\land (x_a \land x_b \imp x_e) \]

\noindent We use conjunction because $ab$ contains $a$ \textit{and} $b$. This
very short paragraph concludes our quick parallel with logical background, we will now go back to our usual framework to study algorithms. With this procedure explained, it remains know to focus on query learning and Angluin algorithm.

\subsection{Iterative building of the canonical basis}

Here, we will study an algorithm proposed by Berczi and al. in 
\cite{berczi_directed_2017} following a paper by Boros and al. 
(\cite{boros_strong_2017}). Readers having a glance at the paper previously 
cited will see different notations and framework between our study and the one 
performed by the authors. This is because they use a graph-theoretic ground 
equivalent to ours to express logical process, but as we previously said, in order to stay in a somehow coherent set up all along this report, we will discuss in terms of implications and so forth.

\vspace{1.2em}

The main idea of the algorithm we should keep in mind, is to build iteratively
the DG basis. To describe briefly the procedure in words, having an initial 
basis $\I$: we initialize $\I_c = \emptyset$ and then at each step of the 
algorithm we add a new implication $A \imp \I(A)$ in $\I_c$ such that $A$ is 
pseudo-closed in $\I$. By construction then, we will terminate and end up with 
the DQ basis. Now that the process is defined, let us expose the procedure
and discuss it (see algorithm \ref{alg:Berczi-min})

\begin{algorithm}
	\KwIn{$\I$: an implication theory}
	\KwOut{$\I_c$: the DG-basis of $\I$}
	
	\BlankLine
	\BlankLine
	
	$\I_c := \emptyset$ \;
	
	\While{$\exists B \in \B(\I)$ s.t $\I_c(B) \neq \I(B)$}{
		$P := \min\{\I_c(B), \ B \in \B(\I) \text{ and } \I_c(B) \neq \I(B)\}$ 
		\;
		$\I_c := \I_c \cup \{P \imp \I(P) \}$ \;
		
	}
	
	return $\I_c$ \;
	
	\caption{\textsc{BercziMinimization}}
	\label{alg:Berczi-min}
\end{algorithm}

\noindent In \cite{berczi_directed_2017, boros_strong_2017}, pseudo-closure
is not explicitly considered. Instead, an implication of the form $P \imp \I(P)$ where $P$ is pseudo-closed, is called \belemp{left-right-saturated}. In fact, left-saturation stands for quasi-closure somehow. To stay close to our definition of the canonical basis, we provide an proposition for the correctness of this algorithm, based on pseudo-closed sets:

\begin{proposition} In algorithm \ref{alg:Berczi-min}, we add an implication
	$P \imp \I(P)$ only if $P$ is pseudo-closed.
	
\end{proposition}

\begin{proof} Let us prove this proposition by induction. 
	
	\paragraph{Initial Case} The initial case is the first implication we add
	to $\I_c$. Because $\I_c$ is empty, for all $B \in \B(\I)$, $\I_c(B) = B$.
	Thus, we add to $I_c$ an implication $B \imp \I(B)$ where $B$ is minimal
	inclusion-wise among bodies of $\I$. Recalling our definition of 
	pseudo-closure, $B$ is compelled to be pseudo-closed then. Note that in 
	fact,
	this argument will hold for all minimal bodies inclusion-wise. Hence, the 
	proposition is true for the initial case.
	
	\vspace{0.5em}
	
	\paragraph{Induction} Suppose we added only implications with pseudo-closed 
	sets as premises in $\I_c$. We will show that in the next implication $P 
	\imp 
	\I(P)$ we add, $P$ is pseudo-closed. Take $P$ as mentioned in the algorithm.
	First observe that taking the minimal non-closed sets of $\I$ closed in 
	$\I_c$
	generated by bodies of $\I$ is sufficient to have the minimal such sets in 
	general. Indeed, let $X$ be a closed set of $\I_c$ not closed in $\I$. Then,
	because bodies are minimal in non-closed sets of $\I$, there must exist 
	implications $\alpha \imp \beta$ in $\I$ such that $\alpha \subseteq X$.
	In particular, we must have an implication $\alpha_i \imp \beta_i$ such 
	that 
	$\alpha_i \subseteq X$ and $\beta_i \nsubseteq X$, because $X$ is not 
	closed. 
	Now by construction of the algorithm, we have the following for all 
	implications $A \imp B$ of $\I$: either $\I_c(A) \imp \I(A)$ belongs to 
	$\I_c$, 
	either it does not (here $\I_c(A)$ is the closure of $A$ before adding 
	$\I_c(A) \imp \I(A)$ to $\I_c$). Because $\beta_i \nsubseteq X = 
	\I_c(X)$ we can conclude that $\I_c(\alpha_i) \imp \I(\alpha_i) \not\in 
	\I_c$.
	Thus $X$ will not be minimal $\I_c$-closed $\I$-non-closed unless it is the 
	closure of some body of $\I$.
	Because we are sure to take a minimal $\I_c$-closed $\I$-non-closed set at 
	each step, we are sure to have all possible pseudo-closed sets $P_i \subset 
	P$
	when considering $P$. Furthermore, since we take $P$ to be the minimal close
	set of $\I_c$, $\I(P_i) \subseteq P$ for all $P_i$. Hence $P$ is indeed 
	pseudo-closed, which confirms the induction hypothesis and the property in
	general.
	
\end{proof}

This statement saying that if we add an implication, then its premise is 
pseudo-closed is enough to justify termination of the algorithm on DG
basis. The outer while loop will be executed at most $|\B|$ times since at
each step, we take out another body of $\I$. Computing and finding the next
pseudo-closed set in this case is done in $O(|\B||\I|)$ operations (an 
execution of \textsc{LinClosure} for each implication of $\I$), thus resulting
in an $O(|\B|^2|\I|)$ asymptotic complexity for the whole algorithm. Even though
simple in its form, it is much more time consuming than previous studied 
algorithms. Let us study an example.

\paragraph{Example} To be coherent, let us take again our perpetual example:
\begin{itemize}
	\item[-] $\Sg = \{a, \ b, \ c, \ d, \ e, \ f \}$,
	\item[-] $\I =$ \{\textit{ab $\imp$ cde, cd $\imp$ f, c $\imp$ a, d $\imp$ b, abcd $\imp$ ef} \} 
\end{itemize}
Let us illustrate the algorithm through a graphical trace (see figure 
\ref{fig:berczi-trace}). In this figure, we represented the 4 steps of 
\textsc{BercziMinimization} over $\I$ as follows: on the left-hand side of each 
step, one can find the closures of premises of $\I$ under $\I_c$, denoted 
$\I_c(\B(\I))$, ordered by inclusion ($\subseteq$). On the right-hand side, we 
have the closures of premises of $\I$ under $\I$, that is $\I(\B(\I))$, again 
ordered by inclusion.

\vspace{1.2em}

In fact, Berczi procedure is a matter of comparing those two orderings. At each
step, we should consider all premises $B$ of $\I$ such that $\I(B)$ is not an
element of $(\I_c(\B(\I)), \subseteq)$. Among those premises, we take one with
a minimal $\I_c$-closure. Then, adding $\I_c(B) \imp \I(B)$ to $\I_c$ ensures
in next steps, we will not have to consider elements of $\downarrow \I(B)$ in
$(\I(\B(\I)), \subseteq)$. In details (a point refers to a step in the figure):
\begin{itemize}
	\item[(a)] $\I_c = \emptyset$, so for all premises $B$ of $\I$, $\I_c(B) = 
	B$.	Hence we take $c$ as a premise with minimal $\I_c$-closure, and append 
	$c \imp ac$ to $\I_c$.
	\item[(b)] $\I_c = \{ c \imp ac \}$. $d$ is a premise of $\I$ being closed 
	in $\I_c$, hence minimal. Consequently, we add $d \imp bd$ to $\I_c$.
	\item[(c)] $\I_c = \{ c \imp ac, d \imp bd \}$, the closures of $c$ and 
	$d$, are the same in $\I_c$ and in $\I$. It remains then $ab$, $cd$ and 
	$abcd$. In $\I_c$, we have:
	\begin{itemize}
		\item $\I_c(ab) = ab$,
		\item $\I_c(cd) = abcd$,
		\item $\I_c(abcd) = abcd$,
	\end{itemize}
	thus the minimal one is $\I_c(ab) = ab$ and we add $ab \imp abcdef$ to 
	$\I_c$,
	\item[(d)] $\I_c = \{ c \imp ac, d \imp bd, ab \imp abcdef 
	\}$, for all $B \in \B(\I)$, $\I_c(B) = \I(B)$, the two orderings
	are identical (or \belemp{isomorphic}), $\I_c$ is equivalent to $\I$ and
	canonical whence minimal.
\end{itemize}


\begin{figure}[ht]
	\centering
\subfloat[$\I_c = \emptyset$ \midemp{($\I_c(c) = c$ and $\I(c) = ac$)}]{
\begin{tikzpicture}[scale=0.6]

\node (plc) at (-4, 4) {$(\I_c(\B(\I)), \  \subseteq)$};

\node[Vertex, label=below:{$d$}] (lcd) at (-2, -2) {};
\node[Vertex, label=below:{$c$}] (lcc) at (-4, -2) {};
\node[Vertex, label=right:{$cd$}] (lccd) at (-3, 0) {};
\node[Vertex, label=left:{$ab$}] (lcab) at (-5, 0) {};
\node[Vertex, label=above:{$abcd$}] (lcabcd) at (-4, 2) {};

\node (pl) at (3, 4) {$(\I(\B(\I)), \  \subseteq)$};

\node[Vertex, label=below:{$ac$}] (lc) at (2, -1) {};
\node[Vertex, label=below:{$bd$}] (ld) at (4, -1) {};
\node[Vertex, label=above:{$abcdef$}] (lab) at (3, 1) {};

\draw[turquoise, fill=turquoise, opacity=0.5, thick] (lcc) circle(0.5);
\draw[turquoise, fill=turquoise, opacity=0.5, thick] (lc) circle(0.5);

\draw (lcab) -- (lcabcd) -- (lccd) -- (lcc);
\draw (lccd) -- (lcd);

\draw (lc) -- (lab) -- (ld);

\end{tikzpicture}
}\qquad
\subfloat[$\I_c = \{ c \rightarrow ac \}$ \midemp{($\I_c(d) = d$, $\I(d) = 
bd$)}]{
\begin{tikzpicture}[scale=0.6]

\node[Vertex, label=below:{$d$}] (lcd) at (-2, -2) {};
\node[Vertex, label=below:{$ac$}] (lcc) at (-4, -2) {};
\node[Vertex, label=right:{$acd$}] (lccd) at (-3, 0) {};
\node[Vertex, label=left:{$ab$}] (lcab) at (-5, 0) {};
\node[Vertex, label=above:{$abcd$}] (lcabcd) at (-4, 2) {};


\node[Vertex, label=below:{$ac$}] (lc) at (2, -1) {};
\node[Vertex, label=below:{$bd$}] (ld) at (4, -1) {};
\node[Vertex, label=above:{$abcdef$}] (lab) at (3, 1) {};

\draw[turquoise, fill=turquoise, opacity=0.5, thick] (lcd) circle(0.5);
\draw[turquoise, fill=turquoise, opacity=0.5, thick] (ld) circle(0.5);

\draw (lcab) -- (lcabcd) -- (lccd) -- (lcc);
\draw (lccd) -- (lcd);

\draw (lc) -- (lab) -- (ld);

\end{tikzpicture}
}

\subfloat[$\I_c = \{ c \rightarrow ac, \ d \rightarrow db \}$ 
\midemp{($\I_c(ab) = abcd$ and 
$\I(ab) = \Sg$)}]{
\begin{tikzpicture}[scale=0.6]

\node[Vertex, label=below:{$ac$}] (lcc) at (-4, -1) {};
\node[Vertex, label=right:{$bd$}] (lcd) at (-3, -1) {};
\node[Vertex, label=left:{$ab$}] (lcab) at (-5, -1) {};
\node[Vertex, label=above:{$abcd$}] (lcabcd) at (-4, 1) {};

\node[Vertex, label=below:{$ac$}] (lc) at (2, -1) {};
\node[Vertex, label=below:{$bd$}] (ld) at (4, -1) {};
\node[Vertex, label=above:{$abcdef$}] (lab) at (3, 1) {};

\draw[turquoise, fill=turquoise, opacity=0.5, thick] (lcab) circle(0.5);
\draw[turquoise, fill=turquoise, opacity=0.5, thick] (lab) circle(0.5);

\draw (lcab) -- (lcabcd) -- (lcd);
\draw (lcabcd) -- (lcc);

\draw (lc) -- (lab) -- (ld);

\end{tikzpicture}
}\qquad
\subfloat[$\I_c =$ \{\textit{c $\rightarrow$ ac, d $\rightarrow$ db, ab 
$\rightarrow \Sg$}\}]{
\begin{tikzpicture}[scale=0.6]

\node[Vertex, label=below:{$ac$}] (lcc) at (-4, -1) {};
\node[Vertex, label=below:{$bd$}] (lcd) at (-2, -1) {};
\node[Vertex, label=above:{$abcdef$}] (lcab) at (-3, 1) {};

\node[Vertex, label=below:{$ac$}] (lc) at (2, -1) {};
\node[Vertex, label=below:{$bd$}] (ld) at (4, -1) {};
\node[Vertex, label=above:{$abcdef$}] (lab) at (3, 1) {};

\draw (lc) -- (lab) -- (ld);
\draw (lcc) -- (lcab) -- (lcd);

\end{tikzpicture}
}
\caption{Trace of \textsc{BercziMinimization} execution}
\label{fig:berczi-trace}
\end{figure}

As we said, this algorithm is much less 
efficient in theory than \textsc{MinCover} or even than \textsc{MaierMinimization}. The problem may come from the need to re-compute the closure of bodies in $\I$ under $\I_c$ at each step to find a possible minimum (see chapter 3).

\subsection{Angluin algorithm and AFP: Query Learning based approach}

Here, the method for building a minimal base is slightly different. We use 
so-called \belemp{query learning}. The idea is we formulate \belemp{queries}
to an \belemp{oracle} knowing the basis we are trying to learn. The oracle 
is assumed to provide an answer to our query in constant time. Depending on 
the query, it might also provide informations on the object we are looking for.
For the Angluin algorithm, we need 2 types of queries. Say we want to learn
a basis $\I$ over $\Sg$:
\begin{enumerate}
	\item \belemp{membership} query: is $M \subseteq \Sg$ a model of $\I$? The
	oracle may answer \textit{"yes"}, or \textit{"no"}.
	\item \belemp{equivalence} query: is a basis $\I'$ equivalent to $\I$? Again
	the answers are \textit{"yes"}, or \textit{"no"}. In the second case, the oracle provides a	\belemp{counterexample} either positive or negative:
	\begin{itemize}
		\item[(i)] \belemp{positive}: a model $M$ of $\I$ which is not a
		model of $\I'$,
		\item[(ii)] \belemp{negative}: a non-model $M$ of $\I$ being a model
		of $\I'$. 
	\end{itemize}
\end{enumerate}
\noindent To clarify, the terms negative/positive are related to the base $\I$
we want to learn. \textsc{AngluinAlgorithm} (\ref{alg:Angluin}) is the algorithm presented by Angluin, Frazer and Pitts in \cite{angluin_learning_1992} as \textsc{Horn1}. 
Initially, it is based on learning logical representation of implication theories: Horn clauses. This learning algorithm has been shown first to terminate on a minimal representation of the basis we want to learn (\cite{angluin_learning_1992}) and more than that, to end up on the canonical
basis by Arias and al. \cite{arias_canonical_2009}. It uses two operations allowing to reduce implications:
\begin{itemize}
	\item[-] $\textit{refine}(A \imp B, M)$: produces $M \imp \Sg$ if $B = \Sg$, $M \imp B \cup A - M$ otherwise,
	\item[-] $\textit{reduce}(A \imp B, M)$: produces $A \imp M - A$ if $B = \Sg$, $A \imp B \cap M$ otherwise.
\end{itemize}
\noindent The main idea is to ask the oracle whether the basis we are building ($\I_c$) is equivalent to $\I$ until it answers \textit{"yes"}. If it says \textit{"no"} then it provides an example on which $\I_c$ and $\I$ differ. If
the example is a model of $\I$, then we track implications in $\I_c$ falsified
by this example and correct them. If the example however is not a model of $\I$
we look for the first possible smaller example out of the one we got. The main
idea is to say that if we correct a smaller example, we are likely to correct
a larger one too. If we do not find any smaller example to correct, we add 
an implication in $\I_c$ addressing the problem. In practice, this procedure is likely to be somehow random because of the oracle. To get rid of non-determinism, one can derive from \textsc{AngluinAlgorithm} the algorithm \textsc{AFPMinimization} (\ref{alg:AFP}).

\vspace{1.2em}

\begin{algorithm}[H]
	\KwIn{$\I$ a theory to learn and an \textit{Oracle} with \textit{membership, equivalence} queries}
	\KwOut{$\I_c$ the canonical representation of $\I$}
	
	\BlankLine
	\BlankLine
	
	$\I_c = \emptyset$ \;
	\While{not equivalence($\I_c$)}{
		$M$ is the counterexample \;
		\If{$M$ is positive}{
			\ForEach{$A \imp B \in \I_c$ such that $M \not\models A \imp B$}{
				replace $A \imp B$ by $reduce(A \imp B, M)$ \;	
			}
			
		} \Else {
			\ForEach{$A \imp B \in \I_c$ such that $A \cap M \subset A$}{
				membership($M \cap A$) \;	
			}
			
			\If{Oracle replied "no" for at least one $A \imp B$}{
				Take the first such $A \imp B$ in $\I_c$ \;	
				replace $A \imp B$ by $refine(A \imp B, A \cap M)$ \;
				
			} \Else {
				add $M \imp \Sg$ to $\I_c$ \;
				
			}
			
		}
		
	}
	return $\I_c$ \;
	
	\caption{\textsc{AngluinAlgorithm}}
	\label{alg:Angluin}
\end{algorithm}



\vspace{1.2em}


\begin{algorithm}
	\KwIn{some theory $\I$ over $\Sg$}
	\KwOut{$\I_c$ the Duquenne-Guigues basis of $\I$}
	
	\BlankLine
	\BlankLine
	
	$\I_c := \emptyset$ \;
	Stack $\mathcal{S}$ \;
	
	\BlankLine
	
	\ForAll{$A \imp B \in \I$}{
		$\mathcal{S} := \left[ A \right]$ \;
		\Repeat{$\mathcal{S} = \emptyset$}{
			$X := \I_c(\text{pop}(\mathcal{S}))$ \;
			
			
			\If{$X \neq \I(X)$}{
				$found := \bot$ \;
				
				\ForAll{$\alpha \imp \beta \in \I_c$}{
					$C := \alpha \cap X$ \;
					\If{$C \neq \alpha$}{
						$D := \I(C)$ \;
						
						\If{$C \neq D$}{
							$found := \top$ \;
							change $\alpha \imp \beta$ by $C \imp D$ in $\I_c$\;
							push($X \cup D$, $\mathcal{S}$) \;
							
							\If{$\beta \neq D$}{
								push($\alpha$, $\mathcal{S}$)\;
							}
							\textbf{exit for}
						}
					}
				}
			
				\BlankLine
			
				\If{$found = \bot$}{
					$\I_c := \I_c \cup \{X \imp \I(X)\}$ \;	
				}
			}
		}

	}

	return $\I_c$ \;
	
	
	\caption{\textsc{AFPMinimization}}
	\label{alg:AFP}
\end{algorithm}

\noindent In this function, questions to and answers from the oracle are replaced by a stack and closure operations. Indeed, membership queries can
be done by computing closures under $\I$, $\I_c$. Regarding the stopping 
criterion equivalence query, observe that premises of $\I$ are a sufficient set of negative counter-example to find $\I_c$: when a premise is no more a negative counter-example, it means that it is neither closed in $\I$ nor in 
$\I_c$. Furthermore, since implications of $\I_c$ are $\I$-right-closed, every
premise $A$ of $\I$ has the same closure in $\I$ and $\I_c$. Hence, when there
is no premise left being negative counter-example, $I \equiv \I_c$. Note that this approach does not require positive counter-examples. The nested for all loop over $\I_c$ is similar to the operations we make in case of negative counter-example in \textsc{AngluinAlgorithm}: whenever we find a smaller counter-example out of $X$ and an implication of $\I_c$ we refine this implication and move to the next possible example. If we do not find any implication in $\I_c$ to refine (i.e the oracle replied "no" for all implications in $\I_c$ in other Angluin's algorithm), then we add $X \imp \I(X)$ to $\I_c$. 

\vspace{1.2em}

Up to know, we are not sure of the correctness of the algorithm, since we do
not know whether an implication can be refined twice, hence if the stack can
grow more than twice the number of implications in $\I$ (asymptotically). Still, if an implication is refined more than one time, the elements stacked shrink in size. Hence we are likely to have a loop requiring in the worst case $O(|\I|)$ to terminate. Because we are doubtful about this assumption, we do not integrate it in complexity estimation. This leads to an approximation of the overhaul time complexity. Indeed, even though we do not know the number of time one could iterate over the repeat loop, we still have nested
iterations over implications of $\I$ and $\I_c$. Furthermore, because the run
over implications of $\I_c$ contains computations of elements in $\I$, the 
overhaul algorithm may be cubic in the number of premises in $\I$ (that is $O(|\B|^2|\I|)$). Therefore, this would be the worst time complexity.

\vspace{1.2em}

Even though we do not have a proof for this algorithm, we can leave for further
research some ideas helping in approaching the correctness of \textsc{AFP}. Nevertheless, because they are somehow \textit{"abstract"} ideas, they may suffer from lack of precisions, or omit details:
\begin{itemize}
	\item[-] a set $A$ in $\mathcal{S}$ is always comprised between a premise
	of $\I$ and its closure $\I(A)$ in terms of inclusion,
	\item[-] the principle of \textsc{AFPMinimization} has echoes in \textsc{BercziMinimization}. In fact, one could rewrite the procedure 
	given by Berczi as follows: while we can find a premise in $\I$ generating a $\I_c$-closed $\I$-non-closed set (that is, a negative counter-example),
	take the minimal such example and add its full implication to $\I_c$. The minimality argument permits to limit the algorithm to appending full implications. Hence here is a possible idea for further work: it is likely that because in \textsc{AFP} we do not use minimality, we may need to update several implications. This processing relies on the main loop iterating over all implications and not being an equivalence query (as opposed to \textsc{BercziMinimization} and \textsc{AngluinAlgorithm}).
	\item[-] one should be interested in describing an invariant for the
	procedure. For instance, we may observe that at each step the remaining
	set of premises of $\I$ are a sufficient set of counter-examples for $\I_c$ with respect to $\I$. More than this, this may also be the case for $\mathcal{S} \cup \B_i$ where $B_i$ denotes the remaining set of premises of $\I$ at step $i$. If a premise is a counter example, it is likely to
	be pushed in $\mathcal{S}$ so that $mathcal{S} \cup \B_i$ does not change
	overhaul. Knowing that an element of the stack cannot be larger than its closure, and that if an implication can be refine more than one time, then 
	in the worst case all elements of the stack may shrink in size because $|\Sg|$ is finite. Therefore, termination of the repeat-until loop should be guaranteed, allowing for reducing the size of sufficient set of premises by 1 at each step of the main loop.
\end{itemize}
\noindent In spite of the lack of proof or counter-example, we can still provide a trace to see how does \textsc{AFP} work for real.

\paragraph{Example} Let us observe a trace with our toy example:
\begin{itemize}
	\item[-] $\Sg = \{a, \ b, \ c, \ d, \ e, \ f \}$,
	\item[-] $\I =$ \{\textit{ab $\imp$ cde, cd $\imp$ f, c $\imp$ a, d $\imp$ b, abcd $\imp$ ef} \} 
\end{itemize}
\noindent We will proceed by implications in $\I$:
\begin{enumerate}
	\item $ab \imp cde$, $\I_c = \emptyset$. We want to build a negative example with $ab$. The smallest closed set of $\I_c$ containing $ab$ is
	$\I_c(ab) = ab$. $ab$ is not closed in $\I$, hence it is itself a counter-example. Because $\I_c$ is empty, we can only add $ab \imp abcdef$ to $\I_c$.
	
	\item $cd \imp f$, $\I_c = \{ ab \imp abcdef \}$. Again, $cd$ is a negative
	counter example to $\I$: it is a model of $\I_c$, but not of $\I$. Furthermore, the only smaller example we build is $ab \cap cd = \emptyset$ 
	not being a negative one. Hence, we add $cd \imp abcdef$ to $\I_c$.
	
	\item $c \imp a$, $\I_c = \{ ab \imp abcdef, cd \imp abcdef \}$. $c$ is a
	negative example: $\I_c(c) = c \neq \I(c)$. For all implications of $\I_c$,
	we try to compute a new example:
	\begin{itemize}
		\item[-] $ab \cap c = \emptyset$, which is not a negative counter example,
		\item[-] $cd \cap c = c$. It is a negative example.
	\end{itemize}
	For $c$, we replace $cd \imp abcdef$ by $c \imp ca$ to correct $\I_c$. We push $ca$ in the stack since it may generate an counter-example. Furthermore, because we replaced $cd \imp abcdef$ by $c \imp ca$ and $abcdef \neq ca$, the closure of $cd$ under $\I_c$ is no more its closure under $\I$, it may also produce a counter-example, we push it into the stack.
	
	The next attribute set in the stack is $cd$. The corresponding example is
	$\I_c(cd) = acd$ not closed in $\I$:
	\begin{itemize}
		\item[-]$acd \cap ab = a$: $a$ is closed both under $\I$ and $\I_c$,
		\item[-]$acd \cap c = c$: $c$ has the same closure under $\I$ and $\I_c$.
	\end{itemize}
	we found no counter example, we add $acd \imp abcdef$ to $\I_c$.
	
	The last element of the stack is $ca$. It is closed under $\I$ hence it cannot be a counter example.
	
	\item $d \imp b$, $\I_c = \{ ab \imp abcdef, c \imp ca, acd \imp abcdef \}$.
	$\I_c(d) = d \neq \I(d)$ is a negative counter example:
	\begin{itemize}
		\item[-] $ab \cap d = \emptyset$
		\item[-] $c \cap d = \emptyset$
		\item[-] $acd \cap d = d$, being a negative counter example
	\end{itemize}
	For $d$, we replace $acd \imp abcdef$ by $d \imp bd$. Then we push it then stack $bd$ and because $bd \neq abcdef$, we also push $acd$.
	
	The next element in the stack is $acd$ for which the closure under $\I_c$
	is the same as the closure under $\I$: $abcdef$. It is not a counter example.
	
	The last element before the stack goes empty is $bd$ being a model of $\I$.
	
	\item $abcd \imp ef$, $\I_c = \{ ab \imp abcdef, c \imp ca, d \imp bd \}$.
	$\I_c(abcd) = abcdef = \I(abcd)$. It is not a counter example.
\end{enumerate}
\noindent The resulting basis $\I_c =  \{ ab \imp abcdef, c \imp ca, d \imp bd \}$ is indeed the DG basis of $\I$.

\vspace{1.2em}

Unlike \textsc{AusielloMinimization}, we implemented \textsc{AFPMinimization}
even though we cannot prove its correctness. The choice comes out of the following reasons. Actually, \textsc{AFPMinimization} has a principle very
different from all other algorithms, which is not the case of \textsc{AusielloAlgorithm} being somehow an optimization (at best) of \textsc{MaierMinimization}. Furthermore, because \textsc{AFPMinimization} can be expressed within closure systems,
it does not require memory overhead and can be implemented quickly, especially in our case where it was one of the last algorithms to be studied. Since we do not have a concrete idea of its complexity, implementing \textsc{AFPMinimization} allows to have a glimpse of its efficiency, as much as
an idea of the tractability of \textsc{AngluinAlgorithm} when we get rid of 
the randomness of an oracle.

\vspace{1.2em}

We discussed here algorithms coming from logical and query learning. Interestingly those two procedures have another approach to minimization: they try to build a minimal basis against an input theory. Unfortunately, it seems
like this approach may not be more efficient than minimizing
the input base itself, at least from a theoretical point of view. Before concluding this chapter, we will quickly discuss some expectations we can 
derive from our study.


\section{Theoretical expectations and conclusion}

All along previous sections, we discussed in detail several minimization
algorithms. To summarize, we studied:
\begin{itemize}
	\item[-] \textsc{MinCover} with complexity $O(|\B||\I|)$,
	\item[-] \textsc{DuquenneMinimization} with complexity $O(|\B||\I|)$,
	\item[-] \textsc{MaierMinimization}, $O(|\B||\I|)$,
	\item[-] \textsc{BercziMinimization}, $O(|\B|^2|\I|)$,
	\item[-] \textsc{AFPMinimization}, with a complexity supposed to be greater than $|\B|^2|\I|$.
\end{itemize}
\noindent Note that we did not truly listed all the algorithm we have been
talking about more than the algorithms we will indeed study in practical terms
in the next chapter. As one can observe, at least 3 of the algorithms have 
the same complexity even though being quite different. It turns out anyway that
among the first three algorithms, \textsc{MinCover, DuquenneMinimization} are
quite similar. Indeed they both rely on using quasi-closure and right-closure. Those steps first issued in works by Day in \cite{day_lattice_1992}, Duquenne-Guigues in \cite{duquenne_variations_2007, guigues_familles_1986}, Wild in \cite{wild_theory_1994, wild_computations_1995, wild_implicational_1989} have been recently discussed
by Boros and al in \cite{boros_strong_2017}. On the other hand \textsc{MaierMinimization} and \textsc{DuquenneMinimization} share redundancy elimination as a first step, being the last part of \textsc{MinCover}. The most striking difference of \textsc{MaierMinimization} with the two previous algorithms (or in fact all other algorithms but \textsc{AusielloMinimization} being based on \textsc{MaierMinimization}) is that the resulting implication base may not be the Duquenne-Guigues basis. Furthermore, where the two first functions need at most two closure computations per implication, we may need three for Maier's algorithm. This could suppose that \textsc{MaierMinimization} is worst in practice than \textsc{DuquenneMinimization} and \textsc{MinCover}. However, this argument is opposed to redundancy elimination of Duquenne's and Maier's functions. Removing several implications lighten subsequent loops and closure computations. Because \textsc{MinCover} does not perform redundancy elimination beforehand, there is also a chance that it is more expensive than the two others procedures, at least with non-minimal basis. Still, it uses redundancy elimination even though an implication being redundant in \textsc{MinCover} may not be considered as redundant in Maier's terms (because of right-closure in the latter one). Nevertheless, we still have a common point within those three first procedures: we are minimizing a given basis. Quite the opposite, \textsc{BercziMinimization} and \textsc{AFPMinimization} are building a canonical representation of an input basis. For \textsc{DuquenneMinimization} however, even if we are indeed minimizing $\I$ by removing its redundant implications, we are also building our result. Speaking of theoretical complexity, it seems like this building approach is not the most efficient. Still, we may keep in mind that we based our study of complexity on \textsc{LinClosure} which may not be the fastest closure algorithm in practice (see \cite{bazhanov_optimizations_2014}). 


\paragraph{Conclusion} In this chapter, we studied several algorithms for 
minimization task coming from various communities. Starting from algorithms
heavily using quasi-closure of FCA domain (\textsc{MinCover}, \textsc{DuquenneMinimization}), moving to Maier's database approach and Ausiello graphical representation with which we encountered some difficulties. Eventually we studied algorithms from boolean logic and query learning with \textsc{BercziMinimization} and \textsc{AFPMinimization}. After having studied their operations we will in the next chapter implement and compare them under practical computations.





% Chapter III: Implementation and results
\chapter{Implementation}



\section{Test set up}

\subsection{Tools}

Here, we understand by tools the general implementation ground we rely on:
the languages, some classes, file formatting and so forth. Let us begin by
a very large point of view. We use two languages: C++ (11) for implementing
the algorithms and testing. For data visualization we use python with libraries \lil{matplotlib}, \lil{numpy}, \lil{csv}. To be more precise about C++, 
we use MinGW-64 compiler and \lil{-O3} optimization flag. Our tests are timed
with release builds. We recorded CPU-time, not wall-clock time which consider also the time spent outside the program. CPU is an Intel I5, 2.4 GHz.

\vspace{1.2em}

We used the code from \url{https://github.com/yazevnul/fcai}, used in 
\cite{bazhanov_optimizations_2014} for experience on \textsc{LinClosure}. It has been made under FCA context. In this framework, sets are represented as \lil{boost::dynamic_bitset} instances (named \lil{BitSet}). Because \lil{boost}
was already present in the project and has various built-in tools, we used it
for timing and recording purposes with \lil{boost::timer} and \lil{boost::accumulators} respectively.

\vspace{1.2em}

On top of implementation of the algorithms we wrote few tools to ease I/O and
testing. Even though we will not go into details, let us introduce quickly
main use of those tools, especially the file format we chose. Because we may need to load or save some implication theories, we have to think of a file format able to represent such basis easily. For this aim, we made \lil{.imp}
files, structured as follows:
\begin{itemize}
	\item[-] the first line contains the number of attributes and the number of
	implications, separated by a space,
	\item[-] because of the \lil{Bitset} representation, an implication is 
	written as lists of indices, premise and conclusion, separated by ">". Indices are also space separated.
\end{itemize}
For instance, consider our running example from the previous chapter:
$\Sg = \{ a, \ b, \  c, \  d, \  e, \ f \}$ and $\I = \{ ab \imp cde, \ cd \imp f, \ c \imp a, \ d \imp b, \  abcd \imp  ef \}$. Considering $a$ indexed as 0 and $f$ indexed as 5, the corresponding \lil{.imp} file will be:

\begin{lstlisting}[language = inline, style = Light]
6 5
0 1 > 2 3 4
2 3 > 5
2 > 0
3 > 1
0 1 2 3 > 4 5
\end{lstlisting}

\noindent Using functions \lil{ReadFile} and \lil{WriteFile} in the namespace \lil{ImplicationTools}, one can easily read or write implication basis:

\begin{lstlisting}[language = CoreCpp, style = Light]
theory L, Lc;  // alias for std::vector<FCA::ImplicationInd>
ImplicationTools::ReadFile("input_file.imp", L);
Minimize(L, Lc);  // result of some minimization into Lc
ImplicationTools::WriteFile("output_file.imp", Lc);
\end{lstlisting}

\noindent Note that the meaning of attributes here is somehow lost. So far, this is not an issue and it may be fixed by keeping a trace of correspondence between indices and attributes names. To test minimization, we use a class
\lil{GridTester} writing CSV results.



\begin{itemize}
	\item csv exportation, bitfiles
	\item quickly describe the class GridTest
\end{itemize}

\subsection{Randomly generated data}

This paragraph is important for all following experiences.
We would like to put the emphasis on the way we randomly generate sets and 
implications. First, let us focus on set generation. We use \lil{boost::random} 
and \lil{time} libraries for generating pseudo-random numbers. In particular 
for a set $X$, we use discrete uniform random distribution on the interval 
$\lbrack 0 \ ; \ |\Sg| \rbrack$:
\begin{enumerate}
	\item determine the size $|X|$ of $X$ by drawing a number out of our 
	distribution,
	\item draw again $|X|$ numbers. Because of the interval, we are sure to
	obtain valid indices of elements to set in $X$.
\end{enumerate}
\noindent Note that an element can be drawn more than once, resulting in 
effective $|X|$ smaller than the one we got at the beginning. We do not consider
this as an issue. Generating theories then is as follows:
\begin{enumerate}
	\item generate a conclusion randomly,
	\item generate a premise. Because we want implications to be informative,
	we keep as a premise the difference \textit{premise $-$ conclusion}.
	\item because empty premise is likely to occur several times, resulting in
	$\emptyset \imp \Sg$, we allow for no more than one empty premise. 
	Nevertheless, in order not to loop forever with this condition, we fixed
	a maximal amount of re-roll with a variable \lil{MAX_ITER}. Passed this 
	number of failure, we accept an empty premise anyway.
\end{enumerate}
\noindent Note that this method can be discussed and probably improved. For us,
it seems like this method is sufficient to provide theories with informative
implications, and uniformly distributed sizes of premises and conclusions thus
a "good" representation of the theories space. We did not investigate further 
ways of generations since the main task was to test algorithms and not to study
implication structures in depth. This anyway is an interesting question for
further work.

\subsection{Real data}

In this part we will be interested in application to real datasets. The application we used is FCA since the framework we use is dedicated to FCA
testing. We will first present briefly FCA and its correlation with our
minimization issue, before describing some real datasets we have been using
and their characteristics.

\subsubsection{Introduction to FCA}

Formal Concept Analysis is a technique relying on array-like data and lattices
to describe hierarchies in data. It can be used in data mining, text mining or
chemistry for instance. Usually we are given a set $G$ of \belemp{objects} having some \belemp{attributes} of a set $M$. Between $G$ and $M$ we can define
a binary relation $I$. An object $g$ is related to an attribute $m$, $gIm$, if
$g$ has the attribute $m$. The tuple $(G, M, I)$ is called a \belemp{context}.

\vspace{1.2em}

Quite intuitively, we can define an operation $': 2^G \imp 2^M$, associating to a set of objects $A \subseteq G$ the set $B \subseteq M$ of attributes shared by
all elements of $A$. Conversely, we can set $': 2^M \imp 2^G$ yelling the set
$A$ of objects sharing all attributes of some set $B$. Formally:

\[ A' = \{ m \in M \ | \ \forall g \in A,\ gIm \} \quad \forall A \subseteq G \]
\[ B' = \{ g \in G \ | \ \forall m \in B,\ gIm \} \quad \forall B \subseteq M \]


\noindent When combined together those operators define a closure operator $'': 2^M \imp 2^M$. Very surprisingly, this operator is related with attribute implications. Indeed, given a context $(G, M, I)$ we can draw implications between subsets $A \imp B$ of attributes. Intuitively, an implication $A \imp B$ will be valid if every time we have all attributes from $A$, we also have
attributes from $B$: $A' \subseteq B'$ or equivalently, $B \subseteq A''$. A set
$\I$ of implications over $M$ will be complete and sound if the operator $''$ from the context coincide with $\I(\cdot)$ for all $A \subseteq M$. For the recall, $\I$ is \belemp{complete} with respect to a context $\Ens{K} = (G, M, I)$ if all valid implications of $\Ens{K}$ hold in $\I$, in other words $\I$ contains all informations of $\Ens{K}$. $\I$ is \belemp{sound} with respect to
$\Ens{K}$ if all valid implications of $\I$ are also valid in $\Ens{K}$, that is, $\I$ contains only true informations about $\Ens{K}$. 


\paragraph{Example} Pages and chapters ago, we were talking about flowers and
vegetables. Let us imagine we have the following:
\begin{itemize}
	\item[-] plants as $G$, $G = $ \{ \textit{cactus, water lily, apple tree, sea weed, birch} \},
	\item[-] attributes as $M$, $M = $ \{ \textit{aquatic, perennial, flower, seasonal} \}
\end{itemize}
We can represent a context $\Ens{K} = (G, M, I)$ as an array, see table \ref{tab:FCA-context}.

\begin{table}[ht]
	\centering
	\begin{tabular}{| >{\columncolor{clouds}}c | c | c | c | c |}
		\hline \rowcolor{clouds}
		& aquatic & perennial & flower & seasonal \\ \hline
		cactus & & $\times$ & $\times$ & $\times$ \\ \hline
		water lily & $\times$ & & $\times$ & $\times$ \\ \hline
		apple tree & & & $\times$ & $\times$ \\ \hline
		sea weed & $\times$ & & & \\ \hline
		birch & & & & $\times$ \\ \hline
	\end{tabular}
	
	\caption{Example of a small context}
	\label{tab:FCA-context}
\end{table}

\vspace{1.2em}

\noindent In this context we have for instance \{\textit{aquatic}\}$'$ $=$ \{\textit{water lily, sea weed}\} and \{\textit{cactus, apple tree, water lily}\}$' = $ \{\textit{flower, seasonal}\}. \{\textit{perennial}\} $\imp$ \{ \textit{flower, seasonal}\} is an example of valid implication, while 
\{\textit{aquatic}\} $\imp$ \{ \textit{flower, seasonal}\} is not since \textit{sea weed} is \textit{aquatic} but neither has \textit{flower} nor is \textit{seasonal}.

\vspace{1.2em}

As shown, contexts contain binary informations. Unfortunately, we may usually be
confronted to multi-valued data. If we stick to our green example, we could have
an attribute "size" having possibly many values. With such attributes we could get closer from relational databases. Because real datasets we are going to use
contains multi-valued attributes, we will need \belemp{FCA-scaling}. We will not
go into technical details. The idea however is the following:
\begin{itemize}
	\item[-] for continuous attributes (e.g: size of a plant) we can create
	disjoint classes based on intervals allowing for new binary attributes
	\item[-] for discrete attributes (e.g: zone of living) we can create an
	attribute per existing value.
\end{itemize}
\noindent In our case, considering discrete values is sufficient according to 
real datasets we will rely on.

\paragraph{Example} Let us retake our previous example but adding new attribute: \textit{zone of living}. The context may become table \ref{tab:FCA-context-MV}. Of course, remind that we do not assume
any true knowledge about biology, this stands only for appealing examples.

\begin{table}[ht]
	\centering
	\begin{tabular}{| >{\columncolor{clouds}}c | c | c | c | c | c |}
		\hline \rowcolor{clouds}
		& aquatic & perennial & flower & seasonal & zone of living \\ \hline
		cactus & & $\times$ & $\times$ & $\times$ & dry area \\ \hline
		water lily & $\times$ & & $\times$ & $\times$ & water \\ \hline
		apple tree & & & $\times$ & $\times$ & woods \\ \hline 
		sea weed & $\times$ & & & & water \\ \hline 
		birch & & & & $\times$& woods \\ \hline 
	\end{tabular}
	
	\caption{Example of a (discrete) multi-valued attribute}
	\label{tab:FCA-context-MV}
\end{table}

\noindent To be able to work in FCA framework, we may refactor the last attributes into 3 distinct ones, as in table \ref{tab:FCA-context-split}.

\begin{table}[ht]
	\centering
	\begin{tabular}{| >{\columncolor{clouds}}c | c | c | c | c | c | c | c |}
		\hline \rowcolor{clouds}
		& aquatic & perennial & flower & seasonal & dry area & water & woods \\ \hline
		cactus & & $\times$ & $\times$ & $\times$ & $\times$ & & \\ \hline
		water lily & $\times$ & & $\times$ & $\times$ & & $\times$ & \\ \hline
		apple tree & & & $\times$ & $\times$ & & & $\times$ \\ \hline 
		sea weed & $\times$ & & & & & $\times$ & \\ \hline 
		birch & & & & $\times$ & & & $\times$ \\ \hline 
	\end{tabular}
	
	\caption{Example of FCA-scaling for multi-valued attribute}
	\label{tab:FCA-context-split}
\end{table}

\vspace{1.2em}

Eventually, we shall discuss various possible basis one can build out of a context. We will use 4 of them: the canonical basis, the minimal basis resulting from Maier's algorithm, the basis of minimal generators and the
proper basis. We already discussed the two first ones. We derive them from the
non minimal ones:
\begin{itemize}
	\item[(i)] minimal generators: a set $X$ is a minimal generator of its closure $\I(X)$ if it is minimal in $[ X]_{\I}$,
	\item[(ii)] proper implications: implications $A \imp A^{\bullet}$ where 
	$\bullet$ is a saturation operator 
\end{itemize}


\subsubsection{Some real datasets}

For our tests, we took the datasets used in \cite{bazhanov_optimizations_2014} \url{https://archive.ics.uci.edu/ml/datasets.html}. They are the following ones:
\begin{itemize}
	\item[-] \midemp{"Zoo"}: it contains animals described by attributes such as "mammals", their number of legs and so forth. Because all attributes are discrete valued, the dataset is scaled as explained previously. For 101 animals, we go from 17 attributes to 28 by scaling.
	\item[-] \midemp{"Flare"}: describes various solar flares by their position, size, flare classification, activity... With 1389 objects, we ended up on 49 attributes by flattening all multi-valued ones.
	\item[-] \midemp{"Breast cancer"}:
	\item[-] \midemp{"Breast Wisconsin"}:
	\item[-] \midemp{"Post-operative"}: this dataset is dedicated to determine in which service should a patient go after its operation. All parameters are indicators on the metabolism of the patient such as body temperature,
	blood pressure and so forth. For 90 patients and 8 attributes, we scaled
	to 26 attributes.
	\item[-] \midemp{"SPECT"}:
	\item[-] \midemp{"Vote"}:
\end{itemize}

To summarize characteristics of all datasets, we may observe table \ref{tab:real-DS}. For the recall $|\Sg|$ is the number of attributes, and 
$|\B|$ the number of implications for each possible dataset.

\begin{table}[ht]
	\centering
	\begin{tabular}{| c | c | c | c |}
		\hline \rowcolor{clouds}
		\multicolumn{2}{|c|}{$\I$} & $|\Sg|$ & $|\B|$ \\ \hline
		\multirow{3}{*}{Zoo} & minimal & \multirow{3}{*}{28} & 141 \\
		& generators & & 874 \\
		& proper & & 554 \\ \hline
		
		\multirow{3}{*}{Flare} & minimal & \multirow{3}{*}{49} & 3382 \\
		& generators & & 39787 \\
		& proper & & 10692 \\ \hline
		
		\multirow{3}{*}{Breast cancer} & minimal & \multirow{3}{*}{43} & 3352 \\
		& generators & & 16137 \\
		& proper & & 11506 \\ \hline
		
		\multirow{3}{*}{Breast Wisconsin} & minimal & \multirow{3}{*}{91} & 10640 \\
		& generators & & 51118 \\
		& proper & & 45748 \\ \hline
		
		\multirow{3}{*}{Post operative} & minimal & \multirow{3}{*}{26} & 625 \\
		& generators & & 3044 \\
		& proper & & 1721 \\ \hline
		
		\multirow{3}{*}{SPECT} & minimal & \multirow{3}{*}{23} & 2169 \\
		& generators & & 44341 \\
		& proper & & 8358 \\ \hline
		
		\multirow{3}{*}{Vote} & minimal & \multirow{3}{*}{18} & 849 \\
		& generators & & 8367 \\
		& proper & & 2410 \\ \hline
	\end{tabular}
\caption{Summary of real datasets characteristics}
\label{tab:real-DS}
\end{table}


In all this section we have been interested in describing the tools we got or developed for implementing the algorithms and data we used. Regarding tools and language, we use C++ and MinGW-64 compiler for implementing and testing the algorithms, based on the code provided in \cite{bazhanov_optimizations_2014}. The CPU has 2.4 GHz frequency. Apart from generating random implications out
of \lil{boost} and uniform distribution, we took various real datasets from the UCI repository. Those real data we used are the same as ones used when testing the code we used on closure algorithms. In the next section we will study implementation of each algorithms, before concluding by a final comparison on 
the real datasets we evoked.

\section{Pruning the algorithms}

Because we can plug-in various closure procedures (\textsc{Closure}, \textsc{LinCLosure}), we are interested in knowing which configuration is
the most efficient for each minimization procedure. However, because we may
have several possible configurations and tests can be highly time consuming,
we will rely on the next assumption based on the result of \cite{bazhanov_optimizations_2014}. Actually, it is exhibited in this paper
that \textsc{LinClosure} performs worst in general than \textsc{Closure}. However, as we will see there are some possible optimizations for \textsc{LinClosure} dealing with initialization steps. Consequently, 
we will first give priority to \textsc{Closure}, and try to replace it 
by \textsc{LinClosure} whenever we find a possible optimized use of 
for this closure method. We will keep as the "best version" the most efficient
in the tests we will run.

\subsection{\textsc{MinCover}}

Recall that \textsc{MinCover} is a two-steps algorithm: right-closure and redundancy elimination. et $\I$ over $\Sg$ be the basis we are trying to minimize. In the first step, we do not remove or add any implications from
$\I$. Furthermore, we do not alter its premises. Therefore, when using
\textsc{LinClosure}, we may need to initialize counters and list only one
time. However, in the second loop, it is question of first removing an implication from $\I$. From our point of view, removing only one implication
$A \imp B$ in counters may be as complex as \textsc{LinClosure} itself:
\begin{itemize}
	\item[-] if the data structure used for \textit{list} in \textsc{LinClosure}
	is a chained list, then removing $A \imp B$ from some \textit{list}[$a$], $a \in A$ is $O(|\B|)$. Because this has to be done for all $a \in A$, the 
	overhaul operation should be $O(|\I|)$, like \textsc{LinClosure} and in particular, the initialization step
	\item[-] if the data structure is an array, there are again two possibilities. Either we store in \textit{list}[$a$] directly implications or indices of implications in $\I$, but then finding and removing an implication will be $O(|\I|)$. Or we can use marking procedure to store a boolean value at some index $i$ representing the $i-th$ implication to know
	whether or not the $i$-th implication should belong to \textit{list}[$a$]. Removing an implication then may be $O(|\Sg|)$. However, we should take care
	of updating all the boolean values  if we remove an implication from $\I$, because the $i$-th index may not correspond to the implication $i$ anymore. To avoid this update, we can in fact do not remove implications in $\I$ at
	all and put all redundant ones in some trash-marked state. This would require some more conditional statements in the nested for loop of the second step of \textsc{LinClosure}, but it may offer indeed a slight optimization.
\end{itemize}
\noindent We did not test the last possible optimizations due to lack of time,
but also because of the results we shall exhibit hereafter about \textsc{MinCover}, stressing on the bad behaviour of \textsc{LinClosure} in practice. Therefore, the two pseudo-codes we compared for \textsc{MinCover} are
algorithms \ref{alg:MinCoverClo}, \ref{alg:MinCoverLin}.

\vspace{1.2em}

\begin{minipage}[t]{0.4\textwidth}
	\begin{algorithm}[H]
		\TitleOfAlgo{\textsc{MinCoverClo}}
		\KwIn{$\I$: an implication base}
		\KwOut{the canonical base of $\I$}
		
		\BlankLine
		\BlankLine
		
		\ForEach{$A \imp B \in \I$}{
			$\I := \I - \{ A \imp B \}$ \;
			$B  := \textsc{Closure}(\I, A \cup B)$ \;
			$\I := \I \cup \{ A \imp B \}$ \;
		}
		
		\BlankLine
		
		\ForEach{$A \imp B \in \I$}{
			$\I := \I - \{ A \imp B \}$ \;
			$A  := \textsc{Closure}(\I, A)$ \;
			\If{$A \neq B$}{
				$\I := \I \cup \{ A \imp B \}$ \;	
			}
		}
	
		
		\label{alg:MinCoverClo}
	\end{algorithm}
\end{minipage}
~
\begin{minipage}[t]{0.4\textwidth}
	\begin{algorithm}[H]
		\TitleOfAlgo{\textsc{MinCoverLin}}
		\KwIn{$\I$: an implication base}
		\KwOut{the canonical base of $\I$}
		
		\BlankLine
		\BlankLine
		
		\textsc{LinClosureInit}$(\I)$ \;
		
		\ForEach{$A \imp B \in \I$}{
			$\I := \I - \{ A \imp B \}$ \;
			$B  := \textsc{LinClosure}(\I, A \cup B)$ \;
			$\I := \I \cup \{ A \imp B \}$ \;
		}
		
		\BlankLine
		
		\ForEach{$A \imp B \in \I$}{
			$\I := \I - \{ A \imp B \}$ \;
			$A  := \textsc{Closure}(\I, A)$ \;
			\If{$A \neq B$}{
				$\I := \I \cup \{ A \imp B \}$ \;	
			}
		}
	
		
		\label{alg:MinCoverLin}
	\end{algorithm}
\end{minipage}

\vspace{1.2em}

In \textsc{MinCoverLin} we used a function called \textsc{LinClosureInit}. In fact, this function is the initialization step of \textsc{LinClosure}. It sets up the containers \textit{list} and \textit{count}. Then, when we call \textsc{LinClosure}, we just consider pure computation of the closure for a given set. In both versions, redundancy elimination is done the same way.

\vspace{1.2em}

The main idea to test efficiency of each version of the algorithm is to perform
two series of tests corresponding to fixing one of the two parameters of its complexity : $|\Sg|$ or $|\B|$. Each series consists in fixing one of those parameters and testing a range of value for the other. We can observe figure
\ref{fig:MinCover-Prune} as results of those experiments.


\begin{figure}[ht]
	
\begin{minipage}{0.55\textwidth}
\subfloat[Average time (in $s$), $|\Sg| = 100$]{
	\scalebox{0.4}{%% Creator: Matplotlib, PGF backend
%%
%% To include the figure in your LaTeX document, write
%%   \input{<filename>.pgf}
%%
%% Make sure the required packages are loaded in your preamble
%%   \usepackage{pgf}
%%
%% Figures using additional raster images can only be included by \input if
%% they are in the same directory as the main LaTeX file. For loading figures
%% from other directories you can use the `import` package
%%   \usepackage{import}
%% and then include the figures with
%%   \import{<path to file>}{<filename>.pgf}
%%
%% Matplotlib used the following preamble
%%   \usepackage{fontspec}
%%   \setmainfont{DejaVu Serif}
%%   \setsansfont{DejaVu Sans}
%%   \setmonofont{DejaVu Sans Mono}
%%
\begingroup%
\makeatletter%
\begin{pgfpicture}%
\pgfpathrectangle{\pgfpointorigin}{\pgfqpoint{8.100000in}{6.600000in}}%
\pgfusepath{use as bounding box, clip}%
\begin{pgfscope}%
\pgfsetbuttcap%
\pgfsetmiterjoin%
\definecolor{currentfill}{rgb}{1.000000,1.000000,1.000000}%
\pgfsetfillcolor{currentfill}%
\pgfsetlinewidth{0.000000pt}%
\definecolor{currentstroke}{rgb}{1.000000,1.000000,1.000000}%
\pgfsetstrokecolor{currentstroke}%
\pgfsetdash{}{0pt}%
\pgfpathmoveto{\pgfqpoint{0.000000in}{0.000000in}}%
\pgfpathlineto{\pgfqpoint{8.100000in}{0.000000in}}%
\pgfpathlineto{\pgfqpoint{8.100000in}{6.600000in}}%
\pgfpathlineto{\pgfqpoint{0.000000in}{6.600000in}}%
\pgfpathclose%
\pgfusepath{fill}%
\end{pgfscope}%
\begin{pgfscope}%
\pgfsetbuttcap%
\pgfsetmiterjoin%
\definecolor{currentfill}{rgb}{1.000000,1.000000,1.000000}%
\pgfsetfillcolor{currentfill}%
\pgfsetlinewidth{0.000000pt}%
\definecolor{currentstroke}{rgb}{0.000000,0.000000,0.000000}%
\pgfsetstrokecolor{currentstroke}%
\pgfsetstrokeopacity{0.000000}%
\pgfsetdash{}{0pt}%
\pgfpathmoveto{\pgfqpoint{1.012500in}{0.726000in}}%
\pgfpathlineto{\pgfqpoint{7.290000in}{0.726000in}}%
\pgfpathlineto{\pgfqpoint{7.290000in}{5.808000in}}%
\pgfpathlineto{\pgfqpoint{1.012500in}{5.808000in}}%
\pgfpathclose%
\pgfusepath{fill}%
\end{pgfscope}%
\begin{pgfscope}%
\pgfsetbuttcap%
\pgfsetroundjoin%
\definecolor{currentfill}{rgb}{0.000000,0.000000,0.000000}%
\pgfsetfillcolor{currentfill}%
\pgfsetlinewidth{0.803000pt}%
\definecolor{currentstroke}{rgb}{0.000000,0.000000,0.000000}%
\pgfsetstrokecolor{currentstroke}%
\pgfsetdash{}{0pt}%
\pgfsys@defobject{currentmarker}{\pgfqpoint{0.000000in}{-0.048611in}}{\pgfqpoint{0.000000in}{0.000000in}}{%
\pgfpathmoveto{\pgfqpoint{0.000000in}{0.000000in}}%
\pgfpathlineto{\pgfqpoint{0.000000in}{-0.048611in}}%
\pgfusepath{stroke,fill}%
}%
\begin{pgfscope}%
\pgfsys@transformshift{1.012500in}{0.726000in}%
\pgfsys@useobject{currentmarker}{}%
\end{pgfscope}%
\end{pgfscope}%
\begin{pgfscope}%
\pgftext[x=1.012500in,y=0.628778in,,top]{\sffamily\fontsize{10.000000}{12.000000}\selectfont \(\displaystyle 0\)}%
\end{pgfscope}%
\begin{pgfscope}%
\pgfsetbuttcap%
\pgfsetroundjoin%
\definecolor{currentfill}{rgb}{0.000000,0.000000,0.000000}%
\pgfsetfillcolor{currentfill}%
\pgfsetlinewidth{0.803000pt}%
\definecolor{currentstroke}{rgb}{0.000000,0.000000,0.000000}%
\pgfsetstrokecolor{currentstroke}%
\pgfsetdash{}{0pt}%
\pgfsys@defobject{currentmarker}{\pgfqpoint{0.000000in}{-0.048611in}}{\pgfqpoint{0.000000in}{0.000000in}}{%
\pgfpathmoveto{\pgfqpoint{0.000000in}{0.000000in}}%
\pgfpathlineto{\pgfqpoint{0.000000in}{-0.048611in}}%
\pgfusepath{stroke,fill}%
}%
\begin{pgfscope}%
\pgfsys@transformshift{1.611784in}{0.726000in}%
\pgfsys@useobject{currentmarker}{}%
\end{pgfscope}%
\end{pgfscope}%
\begin{pgfscope}%
\pgftext[x=1.611784in,y=0.628778in,,top]{\sffamily\fontsize{10.000000}{12.000000}\selectfont \(\displaystyle 2000\)}%
\end{pgfscope}%
\begin{pgfscope}%
\pgfsetbuttcap%
\pgfsetroundjoin%
\definecolor{currentfill}{rgb}{0.000000,0.000000,0.000000}%
\pgfsetfillcolor{currentfill}%
\pgfsetlinewidth{0.803000pt}%
\definecolor{currentstroke}{rgb}{0.000000,0.000000,0.000000}%
\pgfsetstrokecolor{currentstroke}%
\pgfsetdash{}{0pt}%
\pgfsys@defobject{currentmarker}{\pgfqpoint{0.000000in}{-0.048611in}}{\pgfqpoint{0.000000in}{0.000000in}}{%
\pgfpathmoveto{\pgfqpoint{0.000000in}{0.000000in}}%
\pgfpathlineto{\pgfqpoint{0.000000in}{-0.048611in}}%
\pgfusepath{stroke,fill}%
}%
\begin{pgfscope}%
\pgfsys@transformshift{2.211068in}{0.726000in}%
\pgfsys@useobject{currentmarker}{}%
\end{pgfscope}%
\end{pgfscope}%
\begin{pgfscope}%
\pgftext[x=2.211068in,y=0.628778in,,top]{\sffamily\fontsize{10.000000}{12.000000}\selectfont \(\displaystyle 4000\)}%
\end{pgfscope}%
\begin{pgfscope}%
\pgfsetbuttcap%
\pgfsetroundjoin%
\definecolor{currentfill}{rgb}{0.000000,0.000000,0.000000}%
\pgfsetfillcolor{currentfill}%
\pgfsetlinewidth{0.803000pt}%
\definecolor{currentstroke}{rgb}{0.000000,0.000000,0.000000}%
\pgfsetstrokecolor{currentstroke}%
\pgfsetdash{}{0pt}%
\pgfsys@defobject{currentmarker}{\pgfqpoint{0.000000in}{-0.048611in}}{\pgfqpoint{0.000000in}{0.000000in}}{%
\pgfpathmoveto{\pgfqpoint{0.000000in}{0.000000in}}%
\pgfpathlineto{\pgfqpoint{0.000000in}{-0.048611in}}%
\pgfusepath{stroke,fill}%
}%
\begin{pgfscope}%
\pgfsys@transformshift{2.810352in}{0.726000in}%
\pgfsys@useobject{currentmarker}{}%
\end{pgfscope}%
\end{pgfscope}%
\begin{pgfscope}%
\pgftext[x=2.810352in,y=0.628778in,,top]{\sffamily\fontsize{10.000000}{12.000000}\selectfont \(\displaystyle 6000\)}%
\end{pgfscope}%
\begin{pgfscope}%
\pgfsetbuttcap%
\pgfsetroundjoin%
\definecolor{currentfill}{rgb}{0.000000,0.000000,0.000000}%
\pgfsetfillcolor{currentfill}%
\pgfsetlinewidth{0.803000pt}%
\definecolor{currentstroke}{rgb}{0.000000,0.000000,0.000000}%
\pgfsetstrokecolor{currentstroke}%
\pgfsetdash{}{0pt}%
\pgfsys@defobject{currentmarker}{\pgfqpoint{0.000000in}{-0.048611in}}{\pgfqpoint{0.000000in}{0.000000in}}{%
\pgfpathmoveto{\pgfqpoint{0.000000in}{0.000000in}}%
\pgfpathlineto{\pgfqpoint{0.000000in}{-0.048611in}}%
\pgfusepath{stroke,fill}%
}%
\begin{pgfscope}%
\pgfsys@transformshift{3.409636in}{0.726000in}%
\pgfsys@useobject{currentmarker}{}%
\end{pgfscope}%
\end{pgfscope}%
\begin{pgfscope}%
\pgftext[x=3.409636in,y=0.628778in,,top]{\sffamily\fontsize{10.000000}{12.000000}\selectfont \(\displaystyle 8000\)}%
\end{pgfscope}%
\begin{pgfscope}%
\pgfsetbuttcap%
\pgfsetroundjoin%
\definecolor{currentfill}{rgb}{0.000000,0.000000,0.000000}%
\pgfsetfillcolor{currentfill}%
\pgfsetlinewidth{0.803000pt}%
\definecolor{currentstroke}{rgb}{0.000000,0.000000,0.000000}%
\pgfsetstrokecolor{currentstroke}%
\pgfsetdash{}{0pt}%
\pgfsys@defobject{currentmarker}{\pgfqpoint{0.000000in}{-0.048611in}}{\pgfqpoint{0.000000in}{0.000000in}}{%
\pgfpathmoveto{\pgfqpoint{0.000000in}{0.000000in}}%
\pgfpathlineto{\pgfqpoint{0.000000in}{-0.048611in}}%
\pgfusepath{stroke,fill}%
}%
\begin{pgfscope}%
\pgfsys@transformshift{4.008920in}{0.726000in}%
\pgfsys@useobject{currentmarker}{}%
\end{pgfscope}%
\end{pgfscope}%
\begin{pgfscope}%
\pgftext[x=4.008920in,y=0.628778in,,top]{\sffamily\fontsize{10.000000}{12.000000}\selectfont \(\displaystyle 10000\)}%
\end{pgfscope}%
\begin{pgfscope}%
\pgfsetbuttcap%
\pgfsetroundjoin%
\definecolor{currentfill}{rgb}{0.000000,0.000000,0.000000}%
\pgfsetfillcolor{currentfill}%
\pgfsetlinewidth{0.803000pt}%
\definecolor{currentstroke}{rgb}{0.000000,0.000000,0.000000}%
\pgfsetstrokecolor{currentstroke}%
\pgfsetdash{}{0pt}%
\pgfsys@defobject{currentmarker}{\pgfqpoint{0.000000in}{-0.048611in}}{\pgfqpoint{0.000000in}{0.000000in}}{%
\pgfpathmoveto{\pgfqpoint{0.000000in}{0.000000in}}%
\pgfpathlineto{\pgfqpoint{0.000000in}{-0.048611in}}%
\pgfusepath{stroke,fill}%
}%
\begin{pgfscope}%
\pgfsys@transformshift{4.608204in}{0.726000in}%
\pgfsys@useobject{currentmarker}{}%
\end{pgfscope}%
\end{pgfscope}%
\begin{pgfscope}%
\pgftext[x=4.608204in,y=0.628778in,,top]{\sffamily\fontsize{10.000000}{12.000000}\selectfont \(\displaystyle 12000\)}%
\end{pgfscope}%
\begin{pgfscope}%
\pgfsetbuttcap%
\pgfsetroundjoin%
\definecolor{currentfill}{rgb}{0.000000,0.000000,0.000000}%
\pgfsetfillcolor{currentfill}%
\pgfsetlinewidth{0.803000pt}%
\definecolor{currentstroke}{rgb}{0.000000,0.000000,0.000000}%
\pgfsetstrokecolor{currentstroke}%
\pgfsetdash{}{0pt}%
\pgfsys@defobject{currentmarker}{\pgfqpoint{0.000000in}{-0.048611in}}{\pgfqpoint{0.000000in}{0.000000in}}{%
\pgfpathmoveto{\pgfqpoint{0.000000in}{0.000000in}}%
\pgfpathlineto{\pgfqpoint{0.000000in}{-0.048611in}}%
\pgfusepath{stroke,fill}%
}%
\begin{pgfscope}%
\pgfsys@transformshift{5.207488in}{0.726000in}%
\pgfsys@useobject{currentmarker}{}%
\end{pgfscope}%
\end{pgfscope}%
\begin{pgfscope}%
\pgftext[x=5.207488in,y=0.628778in,,top]{\sffamily\fontsize{10.000000}{12.000000}\selectfont \(\displaystyle 14000\)}%
\end{pgfscope}%
\begin{pgfscope}%
\pgfsetbuttcap%
\pgfsetroundjoin%
\definecolor{currentfill}{rgb}{0.000000,0.000000,0.000000}%
\pgfsetfillcolor{currentfill}%
\pgfsetlinewidth{0.803000pt}%
\definecolor{currentstroke}{rgb}{0.000000,0.000000,0.000000}%
\pgfsetstrokecolor{currentstroke}%
\pgfsetdash{}{0pt}%
\pgfsys@defobject{currentmarker}{\pgfqpoint{0.000000in}{-0.048611in}}{\pgfqpoint{0.000000in}{0.000000in}}{%
\pgfpathmoveto{\pgfqpoint{0.000000in}{0.000000in}}%
\pgfpathlineto{\pgfqpoint{0.000000in}{-0.048611in}}%
\pgfusepath{stroke,fill}%
}%
\begin{pgfscope}%
\pgfsys@transformshift{5.806772in}{0.726000in}%
\pgfsys@useobject{currentmarker}{}%
\end{pgfscope}%
\end{pgfscope}%
\begin{pgfscope}%
\pgftext[x=5.806772in,y=0.628778in,,top]{\sffamily\fontsize{10.000000}{12.000000}\selectfont \(\displaystyle 16000\)}%
\end{pgfscope}%
\begin{pgfscope}%
\pgfsetbuttcap%
\pgfsetroundjoin%
\definecolor{currentfill}{rgb}{0.000000,0.000000,0.000000}%
\pgfsetfillcolor{currentfill}%
\pgfsetlinewidth{0.803000pt}%
\definecolor{currentstroke}{rgb}{0.000000,0.000000,0.000000}%
\pgfsetstrokecolor{currentstroke}%
\pgfsetdash{}{0pt}%
\pgfsys@defobject{currentmarker}{\pgfqpoint{0.000000in}{-0.048611in}}{\pgfqpoint{0.000000in}{0.000000in}}{%
\pgfpathmoveto{\pgfqpoint{0.000000in}{0.000000in}}%
\pgfpathlineto{\pgfqpoint{0.000000in}{-0.048611in}}%
\pgfusepath{stroke,fill}%
}%
\begin{pgfscope}%
\pgfsys@transformshift{6.406056in}{0.726000in}%
\pgfsys@useobject{currentmarker}{}%
\end{pgfscope}%
\end{pgfscope}%
\begin{pgfscope}%
\pgftext[x=6.406056in,y=0.628778in,,top]{\sffamily\fontsize{10.000000}{12.000000}\selectfont \(\displaystyle 18000\)}%
\end{pgfscope}%
\begin{pgfscope}%
\pgfsetbuttcap%
\pgfsetroundjoin%
\definecolor{currentfill}{rgb}{0.000000,0.000000,0.000000}%
\pgfsetfillcolor{currentfill}%
\pgfsetlinewidth{0.803000pt}%
\definecolor{currentstroke}{rgb}{0.000000,0.000000,0.000000}%
\pgfsetstrokecolor{currentstroke}%
\pgfsetdash{}{0pt}%
\pgfsys@defobject{currentmarker}{\pgfqpoint{0.000000in}{-0.048611in}}{\pgfqpoint{0.000000in}{0.000000in}}{%
\pgfpathmoveto{\pgfqpoint{0.000000in}{0.000000in}}%
\pgfpathlineto{\pgfqpoint{0.000000in}{-0.048611in}}%
\pgfusepath{stroke,fill}%
}%
\begin{pgfscope}%
\pgfsys@transformshift{7.005340in}{0.726000in}%
\pgfsys@useobject{currentmarker}{}%
\end{pgfscope}%
\end{pgfscope}%
\begin{pgfscope}%
\pgftext[x=7.005340in,y=0.628778in,,top]{\sffamily\fontsize{10.000000}{12.000000}\selectfont \(\displaystyle 20000\)}%
\end{pgfscope}%
\begin{pgfscope}%
\pgftext[x=4.151250in,y=0.438809in,,top]{\sffamily\fontsize{12.000000}{14.400000}\selectfont \(\displaystyle |\mathcal{B}|\)}%
\end{pgfscope}%
\begin{pgfscope}%
\pgfsetbuttcap%
\pgfsetroundjoin%
\definecolor{currentfill}{rgb}{0.000000,0.000000,0.000000}%
\pgfsetfillcolor{currentfill}%
\pgfsetlinewidth{0.803000pt}%
\definecolor{currentstroke}{rgb}{0.000000,0.000000,0.000000}%
\pgfsetstrokecolor{currentstroke}%
\pgfsetdash{}{0pt}%
\pgfsys@defobject{currentmarker}{\pgfqpoint{-0.048611in}{0.000000in}}{\pgfqpoint{0.000000in}{0.000000in}}{%
\pgfpathmoveto{\pgfqpoint{0.000000in}{0.000000in}}%
\pgfpathlineto{\pgfqpoint{-0.048611in}{0.000000in}}%
\pgfusepath{stroke,fill}%
}%
\begin{pgfscope}%
\pgfsys@transformshift{1.012500in}{0.956312in}%
\pgfsys@useobject{currentmarker}{}%
\end{pgfscope}%
\end{pgfscope}%
\begin{pgfscope}%
\pgftext[x=0.845833in,y=0.903550in,left,base]{\sffamily\fontsize{10.000000}{12.000000}\selectfont \(\displaystyle 0\)}%
\end{pgfscope}%
\begin{pgfscope}%
\pgfsetbuttcap%
\pgfsetroundjoin%
\definecolor{currentfill}{rgb}{0.000000,0.000000,0.000000}%
\pgfsetfillcolor{currentfill}%
\pgfsetlinewidth{0.803000pt}%
\definecolor{currentstroke}{rgb}{0.000000,0.000000,0.000000}%
\pgfsetstrokecolor{currentstroke}%
\pgfsetdash{}{0pt}%
\pgfsys@defobject{currentmarker}{\pgfqpoint{-0.048611in}{0.000000in}}{\pgfqpoint{0.000000in}{0.000000in}}{%
\pgfpathmoveto{\pgfqpoint{0.000000in}{0.000000in}}%
\pgfpathlineto{\pgfqpoint{-0.048611in}{0.000000in}}%
\pgfusepath{stroke,fill}%
}%
\begin{pgfscope}%
\pgfsys@transformshift{1.012500in}{1.163097in}%
\pgfsys@useobject{currentmarker}{}%
\end{pgfscope}%
\end{pgfscope}%
\begin{pgfscope}%
\pgftext[x=0.776388in,y=1.110336in,left,base]{\sffamily\fontsize{10.000000}{12.000000}\selectfont \(\displaystyle 10\)}%
\end{pgfscope}%
\begin{pgfscope}%
\pgfsetbuttcap%
\pgfsetroundjoin%
\definecolor{currentfill}{rgb}{0.000000,0.000000,0.000000}%
\pgfsetfillcolor{currentfill}%
\pgfsetlinewidth{0.803000pt}%
\definecolor{currentstroke}{rgb}{0.000000,0.000000,0.000000}%
\pgfsetstrokecolor{currentstroke}%
\pgfsetdash{}{0pt}%
\pgfsys@defobject{currentmarker}{\pgfqpoint{-0.048611in}{0.000000in}}{\pgfqpoint{0.000000in}{0.000000in}}{%
\pgfpathmoveto{\pgfqpoint{0.000000in}{0.000000in}}%
\pgfpathlineto{\pgfqpoint{-0.048611in}{0.000000in}}%
\pgfusepath{stroke,fill}%
}%
\begin{pgfscope}%
\pgfsys@transformshift{1.012500in}{1.369883in}%
\pgfsys@useobject{currentmarker}{}%
\end{pgfscope}%
\end{pgfscope}%
\begin{pgfscope}%
\pgftext[x=0.776388in,y=1.317122in,left,base]{\sffamily\fontsize{10.000000}{12.000000}\selectfont \(\displaystyle 20\)}%
\end{pgfscope}%
\begin{pgfscope}%
\pgfsetbuttcap%
\pgfsetroundjoin%
\definecolor{currentfill}{rgb}{0.000000,0.000000,0.000000}%
\pgfsetfillcolor{currentfill}%
\pgfsetlinewidth{0.803000pt}%
\definecolor{currentstroke}{rgb}{0.000000,0.000000,0.000000}%
\pgfsetstrokecolor{currentstroke}%
\pgfsetdash{}{0pt}%
\pgfsys@defobject{currentmarker}{\pgfqpoint{-0.048611in}{0.000000in}}{\pgfqpoint{0.000000in}{0.000000in}}{%
\pgfpathmoveto{\pgfqpoint{0.000000in}{0.000000in}}%
\pgfpathlineto{\pgfqpoint{-0.048611in}{0.000000in}}%
\pgfusepath{stroke,fill}%
}%
\begin{pgfscope}%
\pgfsys@transformshift{1.012500in}{1.576669in}%
\pgfsys@useobject{currentmarker}{}%
\end{pgfscope}%
\end{pgfscope}%
\begin{pgfscope}%
\pgftext[x=0.776388in,y=1.523907in,left,base]{\sffamily\fontsize{10.000000}{12.000000}\selectfont \(\displaystyle 30\)}%
\end{pgfscope}%
\begin{pgfscope}%
\pgfsetbuttcap%
\pgfsetroundjoin%
\definecolor{currentfill}{rgb}{0.000000,0.000000,0.000000}%
\pgfsetfillcolor{currentfill}%
\pgfsetlinewidth{0.803000pt}%
\definecolor{currentstroke}{rgb}{0.000000,0.000000,0.000000}%
\pgfsetstrokecolor{currentstroke}%
\pgfsetdash{}{0pt}%
\pgfsys@defobject{currentmarker}{\pgfqpoint{-0.048611in}{0.000000in}}{\pgfqpoint{0.000000in}{0.000000in}}{%
\pgfpathmoveto{\pgfqpoint{0.000000in}{0.000000in}}%
\pgfpathlineto{\pgfqpoint{-0.048611in}{0.000000in}}%
\pgfusepath{stroke,fill}%
}%
\begin{pgfscope}%
\pgfsys@transformshift{1.012500in}{1.783455in}%
\pgfsys@useobject{currentmarker}{}%
\end{pgfscope}%
\end{pgfscope}%
\begin{pgfscope}%
\pgftext[x=0.776388in,y=1.730693in,left,base]{\sffamily\fontsize{10.000000}{12.000000}\selectfont \(\displaystyle 40\)}%
\end{pgfscope}%
\begin{pgfscope}%
\pgfsetbuttcap%
\pgfsetroundjoin%
\definecolor{currentfill}{rgb}{0.000000,0.000000,0.000000}%
\pgfsetfillcolor{currentfill}%
\pgfsetlinewidth{0.803000pt}%
\definecolor{currentstroke}{rgb}{0.000000,0.000000,0.000000}%
\pgfsetstrokecolor{currentstroke}%
\pgfsetdash{}{0pt}%
\pgfsys@defobject{currentmarker}{\pgfqpoint{-0.048611in}{0.000000in}}{\pgfqpoint{0.000000in}{0.000000in}}{%
\pgfpathmoveto{\pgfqpoint{0.000000in}{0.000000in}}%
\pgfpathlineto{\pgfqpoint{-0.048611in}{0.000000in}}%
\pgfusepath{stroke,fill}%
}%
\begin{pgfscope}%
\pgfsys@transformshift{1.012500in}{1.990240in}%
\pgfsys@useobject{currentmarker}{}%
\end{pgfscope}%
\end{pgfscope}%
\begin{pgfscope}%
\pgftext[x=0.776388in,y=1.937479in,left,base]{\sffamily\fontsize{10.000000}{12.000000}\selectfont \(\displaystyle 50\)}%
\end{pgfscope}%
\begin{pgfscope}%
\pgfsetbuttcap%
\pgfsetroundjoin%
\definecolor{currentfill}{rgb}{0.000000,0.000000,0.000000}%
\pgfsetfillcolor{currentfill}%
\pgfsetlinewidth{0.803000pt}%
\definecolor{currentstroke}{rgb}{0.000000,0.000000,0.000000}%
\pgfsetstrokecolor{currentstroke}%
\pgfsetdash{}{0pt}%
\pgfsys@defobject{currentmarker}{\pgfqpoint{-0.048611in}{0.000000in}}{\pgfqpoint{0.000000in}{0.000000in}}{%
\pgfpathmoveto{\pgfqpoint{0.000000in}{0.000000in}}%
\pgfpathlineto{\pgfqpoint{-0.048611in}{0.000000in}}%
\pgfusepath{stroke,fill}%
}%
\begin{pgfscope}%
\pgfsys@transformshift{1.012500in}{2.197026in}%
\pgfsys@useobject{currentmarker}{}%
\end{pgfscope}%
\end{pgfscope}%
\begin{pgfscope}%
\pgftext[x=0.776388in,y=2.144264in,left,base]{\sffamily\fontsize{10.000000}{12.000000}\selectfont \(\displaystyle 60\)}%
\end{pgfscope}%
\begin{pgfscope}%
\pgfsetbuttcap%
\pgfsetroundjoin%
\definecolor{currentfill}{rgb}{0.000000,0.000000,0.000000}%
\pgfsetfillcolor{currentfill}%
\pgfsetlinewidth{0.803000pt}%
\definecolor{currentstroke}{rgb}{0.000000,0.000000,0.000000}%
\pgfsetstrokecolor{currentstroke}%
\pgfsetdash{}{0pt}%
\pgfsys@defobject{currentmarker}{\pgfqpoint{-0.048611in}{0.000000in}}{\pgfqpoint{0.000000in}{0.000000in}}{%
\pgfpathmoveto{\pgfqpoint{0.000000in}{0.000000in}}%
\pgfpathlineto{\pgfqpoint{-0.048611in}{0.000000in}}%
\pgfusepath{stroke,fill}%
}%
\begin{pgfscope}%
\pgfsys@transformshift{1.012500in}{2.403812in}%
\pgfsys@useobject{currentmarker}{}%
\end{pgfscope}%
\end{pgfscope}%
\begin{pgfscope}%
\pgftext[x=0.776388in,y=2.351050in,left,base]{\sffamily\fontsize{10.000000}{12.000000}\selectfont \(\displaystyle 70\)}%
\end{pgfscope}%
\begin{pgfscope}%
\pgfsetbuttcap%
\pgfsetroundjoin%
\definecolor{currentfill}{rgb}{0.000000,0.000000,0.000000}%
\pgfsetfillcolor{currentfill}%
\pgfsetlinewidth{0.803000pt}%
\definecolor{currentstroke}{rgb}{0.000000,0.000000,0.000000}%
\pgfsetstrokecolor{currentstroke}%
\pgfsetdash{}{0pt}%
\pgfsys@defobject{currentmarker}{\pgfqpoint{-0.048611in}{0.000000in}}{\pgfqpoint{0.000000in}{0.000000in}}{%
\pgfpathmoveto{\pgfqpoint{0.000000in}{0.000000in}}%
\pgfpathlineto{\pgfqpoint{-0.048611in}{0.000000in}}%
\pgfusepath{stroke,fill}%
}%
\begin{pgfscope}%
\pgfsys@transformshift{1.012500in}{2.610597in}%
\pgfsys@useobject{currentmarker}{}%
\end{pgfscope}%
\end{pgfscope}%
\begin{pgfscope}%
\pgftext[x=0.776388in,y=2.557836in,left,base]{\sffamily\fontsize{10.000000}{12.000000}\selectfont \(\displaystyle 80\)}%
\end{pgfscope}%
\begin{pgfscope}%
\pgfsetbuttcap%
\pgfsetroundjoin%
\definecolor{currentfill}{rgb}{0.000000,0.000000,0.000000}%
\pgfsetfillcolor{currentfill}%
\pgfsetlinewidth{0.803000pt}%
\definecolor{currentstroke}{rgb}{0.000000,0.000000,0.000000}%
\pgfsetstrokecolor{currentstroke}%
\pgfsetdash{}{0pt}%
\pgfsys@defobject{currentmarker}{\pgfqpoint{-0.048611in}{0.000000in}}{\pgfqpoint{0.000000in}{0.000000in}}{%
\pgfpathmoveto{\pgfqpoint{0.000000in}{0.000000in}}%
\pgfpathlineto{\pgfqpoint{-0.048611in}{0.000000in}}%
\pgfusepath{stroke,fill}%
}%
\begin{pgfscope}%
\pgfsys@transformshift{1.012500in}{2.817383in}%
\pgfsys@useobject{currentmarker}{}%
\end{pgfscope}%
\end{pgfscope}%
\begin{pgfscope}%
\pgftext[x=0.776388in,y=2.764621in,left,base]{\sffamily\fontsize{10.000000}{12.000000}\selectfont \(\displaystyle 90\)}%
\end{pgfscope}%
\begin{pgfscope}%
\pgfsetbuttcap%
\pgfsetroundjoin%
\definecolor{currentfill}{rgb}{0.000000,0.000000,0.000000}%
\pgfsetfillcolor{currentfill}%
\pgfsetlinewidth{0.803000pt}%
\definecolor{currentstroke}{rgb}{0.000000,0.000000,0.000000}%
\pgfsetstrokecolor{currentstroke}%
\pgfsetdash{}{0pt}%
\pgfsys@defobject{currentmarker}{\pgfqpoint{-0.048611in}{0.000000in}}{\pgfqpoint{0.000000in}{0.000000in}}{%
\pgfpathmoveto{\pgfqpoint{0.000000in}{0.000000in}}%
\pgfpathlineto{\pgfqpoint{-0.048611in}{0.000000in}}%
\pgfusepath{stroke,fill}%
}%
\begin{pgfscope}%
\pgfsys@transformshift{1.012500in}{3.024169in}%
\pgfsys@useobject{currentmarker}{}%
\end{pgfscope}%
\end{pgfscope}%
\begin{pgfscope}%
\pgftext[x=0.706944in,y=2.971407in,left,base]{\sffamily\fontsize{10.000000}{12.000000}\selectfont \(\displaystyle 100\)}%
\end{pgfscope}%
\begin{pgfscope}%
\pgfsetbuttcap%
\pgfsetroundjoin%
\definecolor{currentfill}{rgb}{0.000000,0.000000,0.000000}%
\pgfsetfillcolor{currentfill}%
\pgfsetlinewidth{0.803000pt}%
\definecolor{currentstroke}{rgb}{0.000000,0.000000,0.000000}%
\pgfsetstrokecolor{currentstroke}%
\pgfsetdash{}{0pt}%
\pgfsys@defobject{currentmarker}{\pgfqpoint{-0.048611in}{0.000000in}}{\pgfqpoint{0.000000in}{0.000000in}}{%
\pgfpathmoveto{\pgfqpoint{0.000000in}{0.000000in}}%
\pgfpathlineto{\pgfqpoint{-0.048611in}{0.000000in}}%
\pgfusepath{stroke,fill}%
}%
\begin{pgfscope}%
\pgfsys@transformshift{1.012500in}{3.230954in}%
\pgfsys@useobject{currentmarker}{}%
\end{pgfscope}%
\end{pgfscope}%
\begin{pgfscope}%
\pgftext[x=0.706944in,y=3.178193in,left,base]{\sffamily\fontsize{10.000000}{12.000000}\selectfont \(\displaystyle 110\)}%
\end{pgfscope}%
\begin{pgfscope}%
\pgfsetbuttcap%
\pgfsetroundjoin%
\definecolor{currentfill}{rgb}{0.000000,0.000000,0.000000}%
\pgfsetfillcolor{currentfill}%
\pgfsetlinewidth{0.803000pt}%
\definecolor{currentstroke}{rgb}{0.000000,0.000000,0.000000}%
\pgfsetstrokecolor{currentstroke}%
\pgfsetdash{}{0pt}%
\pgfsys@defobject{currentmarker}{\pgfqpoint{-0.048611in}{0.000000in}}{\pgfqpoint{0.000000in}{0.000000in}}{%
\pgfpathmoveto{\pgfqpoint{0.000000in}{0.000000in}}%
\pgfpathlineto{\pgfqpoint{-0.048611in}{0.000000in}}%
\pgfusepath{stroke,fill}%
}%
\begin{pgfscope}%
\pgfsys@transformshift{1.012500in}{3.437740in}%
\pgfsys@useobject{currentmarker}{}%
\end{pgfscope}%
\end{pgfscope}%
\begin{pgfscope}%
\pgftext[x=0.706944in,y=3.384979in,left,base]{\sffamily\fontsize{10.000000}{12.000000}\selectfont \(\displaystyle 120\)}%
\end{pgfscope}%
\begin{pgfscope}%
\pgfsetbuttcap%
\pgfsetroundjoin%
\definecolor{currentfill}{rgb}{0.000000,0.000000,0.000000}%
\pgfsetfillcolor{currentfill}%
\pgfsetlinewidth{0.803000pt}%
\definecolor{currentstroke}{rgb}{0.000000,0.000000,0.000000}%
\pgfsetstrokecolor{currentstroke}%
\pgfsetdash{}{0pt}%
\pgfsys@defobject{currentmarker}{\pgfqpoint{-0.048611in}{0.000000in}}{\pgfqpoint{0.000000in}{0.000000in}}{%
\pgfpathmoveto{\pgfqpoint{0.000000in}{0.000000in}}%
\pgfpathlineto{\pgfqpoint{-0.048611in}{0.000000in}}%
\pgfusepath{stroke,fill}%
}%
\begin{pgfscope}%
\pgfsys@transformshift{1.012500in}{3.644526in}%
\pgfsys@useobject{currentmarker}{}%
\end{pgfscope}%
\end{pgfscope}%
\begin{pgfscope}%
\pgftext[x=0.706944in,y=3.591764in,left,base]{\sffamily\fontsize{10.000000}{12.000000}\selectfont \(\displaystyle 130\)}%
\end{pgfscope}%
\begin{pgfscope}%
\pgfsetbuttcap%
\pgfsetroundjoin%
\definecolor{currentfill}{rgb}{0.000000,0.000000,0.000000}%
\pgfsetfillcolor{currentfill}%
\pgfsetlinewidth{0.803000pt}%
\definecolor{currentstroke}{rgb}{0.000000,0.000000,0.000000}%
\pgfsetstrokecolor{currentstroke}%
\pgfsetdash{}{0pt}%
\pgfsys@defobject{currentmarker}{\pgfqpoint{-0.048611in}{0.000000in}}{\pgfqpoint{0.000000in}{0.000000in}}{%
\pgfpathmoveto{\pgfqpoint{0.000000in}{0.000000in}}%
\pgfpathlineto{\pgfqpoint{-0.048611in}{0.000000in}}%
\pgfusepath{stroke,fill}%
}%
\begin{pgfscope}%
\pgfsys@transformshift{1.012500in}{3.851311in}%
\pgfsys@useobject{currentmarker}{}%
\end{pgfscope}%
\end{pgfscope}%
\begin{pgfscope}%
\pgftext[x=0.706944in,y=3.798550in,left,base]{\sffamily\fontsize{10.000000}{12.000000}\selectfont \(\displaystyle 140\)}%
\end{pgfscope}%
\begin{pgfscope}%
\pgfsetbuttcap%
\pgfsetroundjoin%
\definecolor{currentfill}{rgb}{0.000000,0.000000,0.000000}%
\pgfsetfillcolor{currentfill}%
\pgfsetlinewidth{0.803000pt}%
\definecolor{currentstroke}{rgb}{0.000000,0.000000,0.000000}%
\pgfsetstrokecolor{currentstroke}%
\pgfsetdash{}{0pt}%
\pgfsys@defobject{currentmarker}{\pgfqpoint{-0.048611in}{0.000000in}}{\pgfqpoint{0.000000in}{0.000000in}}{%
\pgfpathmoveto{\pgfqpoint{0.000000in}{0.000000in}}%
\pgfpathlineto{\pgfqpoint{-0.048611in}{0.000000in}}%
\pgfusepath{stroke,fill}%
}%
\begin{pgfscope}%
\pgfsys@transformshift{1.012500in}{4.058097in}%
\pgfsys@useobject{currentmarker}{}%
\end{pgfscope}%
\end{pgfscope}%
\begin{pgfscope}%
\pgftext[x=0.706944in,y=4.005336in,left,base]{\sffamily\fontsize{10.000000}{12.000000}\selectfont \(\displaystyle 150\)}%
\end{pgfscope}%
\begin{pgfscope}%
\pgfsetbuttcap%
\pgfsetroundjoin%
\definecolor{currentfill}{rgb}{0.000000,0.000000,0.000000}%
\pgfsetfillcolor{currentfill}%
\pgfsetlinewidth{0.803000pt}%
\definecolor{currentstroke}{rgb}{0.000000,0.000000,0.000000}%
\pgfsetstrokecolor{currentstroke}%
\pgfsetdash{}{0pt}%
\pgfsys@defobject{currentmarker}{\pgfqpoint{-0.048611in}{0.000000in}}{\pgfqpoint{0.000000in}{0.000000in}}{%
\pgfpathmoveto{\pgfqpoint{0.000000in}{0.000000in}}%
\pgfpathlineto{\pgfqpoint{-0.048611in}{0.000000in}}%
\pgfusepath{stroke,fill}%
}%
\begin{pgfscope}%
\pgfsys@transformshift{1.012500in}{4.264883in}%
\pgfsys@useobject{currentmarker}{}%
\end{pgfscope}%
\end{pgfscope}%
\begin{pgfscope}%
\pgftext[x=0.706944in,y=4.212121in,left,base]{\sffamily\fontsize{10.000000}{12.000000}\selectfont \(\displaystyle 160\)}%
\end{pgfscope}%
\begin{pgfscope}%
\pgfsetbuttcap%
\pgfsetroundjoin%
\definecolor{currentfill}{rgb}{0.000000,0.000000,0.000000}%
\pgfsetfillcolor{currentfill}%
\pgfsetlinewidth{0.803000pt}%
\definecolor{currentstroke}{rgb}{0.000000,0.000000,0.000000}%
\pgfsetstrokecolor{currentstroke}%
\pgfsetdash{}{0pt}%
\pgfsys@defobject{currentmarker}{\pgfqpoint{-0.048611in}{0.000000in}}{\pgfqpoint{0.000000in}{0.000000in}}{%
\pgfpathmoveto{\pgfqpoint{0.000000in}{0.000000in}}%
\pgfpathlineto{\pgfqpoint{-0.048611in}{0.000000in}}%
\pgfusepath{stroke,fill}%
}%
\begin{pgfscope}%
\pgfsys@transformshift{1.012500in}{4.471668in}%
\pgfsys@useobject{currentmarker}{}%
\end{pgfscope}%
\end{pgfscope}%
\begin{pgfscope}%
\pgftext[x=0.706944in,y=4.418907in,left,base]{\sffamily\fontsize{10.000000}{12.000000}\selectfont \(\displaystyle 170\)}%
\end{pgfscope}%
\begin{pgfscope}%
\pgfsetbuttcap%
\pgfsetroundjoin%
\definecolor{currentfill}{rgb}{0.000000,0.000000,0.000000}%
\pgfsetfillcolor{currentfill}%
\pgfsetlinewidth{0.803000pt}%
\definecolor{currentstroke}{rgb}{0.000000,0.000000,0.000000}%
\pgfsetstrokecolor{currentstroke}%
\pgfsetdash{}{0pt}%
\pgfsys@defobject{currentmarker}{\pgfqpoint{-0.048611in}{0.000000in}}{\pgfqpoint{0.000000in}{0.000000in}}{%
\pgfpathmoveto{\pgfqpoint{0.000000in}{0.000000in}}%
\pgfpathlineto{\pgfqpoint{-0.048611in}{0.000000in}}%
\pgfusepath{stroke,fill}%
}%
\begin{pgfscope}%
\pgfsys@transformshift{1.012500in}{4.678454in}%
\pgfsys@useobject{currentmarker}{}%
\end{pgfscope}%
\end{pgfscope}%
\begin{pgfscope}%
\pgftext[x=0.706944in,y=4.625693in,left,base]{\sffamily\fontsize{10.000000}{12.000000}\selectfont \(\displaystyle 180\)}%
\end{pgfscope}%
\begin{pgfscope}%
\pgfsetbuttcap%
\pgfsetroundjoin%
\definecolor{currentfill}{rgb}{0.000000,0.000000,0.000000}%
\pgfsetfillcolor{currentfill}%
\pgfsetlinewidth{0.803000pt}%
\definecolor{currentstroke}{rgb}{0.000000,0.000000,0.000000}%
\pgfsetstrokecolor{currentstroke}%
\pgfsetdash{}{0pt}%
\pgfsys@defobject{currentmarker}{\pgfqpoint{-0.048611in}{0.000000in}}{\pgfqpoint{0.000000in}{0.000000in}}{%
\pgfpathmoveto{\pgfqpoint{0.000000in}{0.000000in}}%
\pgfpathlineto{\pgfqpoint{-0.048611in}{0.000000in}}%
\pgfusepath{stroke,fill}%
}%
\begin{pgfscope}%
\pgfsys@transformshift{1.012500in}{4.885240in}%
\pgfsys@useobject{currentmarker}{}%
\end{pgfscope}%
\end{pgfscope}%
\begin{pgfscope}%
\pgftext[x=0.706944in,y=4.832478in,left,base]{\sffamily\fontsize{10.000000}{12.000000}\selectfont \(\displaystyle 190\)}%
\end{pgfscope}%
\begin{pgfscope}%
\pgfsetbuttcap%
\pgfsetroundjoin%
\definecolor{currentfill}{rgb}{0.000000,0.000000,0.000000}%
\pgfsetfillcolor{currentfill}%
\pgfsetlinewidth{0.803000pt}%
\definecolor{currentstroke}{rgb}{0.000000,0.000000,0.000000}%
\pgfsetstrokecolor{currentstroke}%
\pgfsetdash{}{0pt}%
\pgfsys@defobject{currentmarker}{\pgfqpoint{-0.048611in}{0.000000in}}{\pgfqpoint{0.000000in}{0.000000in}}{%
\pgfpathmoveto{\pgfqpoint{0.000000in}{0.000000in}}%
\pgfpathlineto{\pgfqpoint{-0.048611in}{0.000000in}}%
\pgfusepath{stroke,fill}%
}%
\begin{pgfscope}%
\pgfsys@transformshift{1.012500in}{5.092026in}%
\pgfsys@useobject{currentmarker}{}%
\end{pgfscope}%
\end{pgfscope}%
\begin{pgfscope}%
\pgftext[x=0.706944in,y=5.039264in,left,base]{\sffamily\fontsize{10.000000}{12.000000}\selectfont \(\displaystyle 200\)}%
\end{pgfscope}%
\begin{pgfscope}%
\pgfsetbuttcap%
\pgfsetroundjoin%
\definecolor{currentfill}{rgb}{0.000000,0.000000,0.000000}%
\pgfsetfillcolor{currentfill}%
\pgfsetlinewidth{0.803000pt}%
\definecolor{currentstroke}{rgb}{0.000000,0.000000,0.000000}%
\pgfsetstrokecolor{currentstroke}%
\pgfsetdash{}{0pt}%
\pgfsys@defobject{currentmarker}{\pgfqpoint{-0.048611in}{0.000000in}}{\pgfqpoint{0.000000in}{0.000000in}}{%
\pgfpathmoveto{\pgfqpoint{0.000000in}{0.000000in}}%
\pgfpathlineto{\pgfqpoint{-0.048611in}{0.000000in}}%
\pgfusepath{stroke,fill}%
}%
\begin{pgfscope}%
\pgfsys@transformshift{1.012500in}{5.298811in}%
\pgfsys@useobject{currentmarker}{}%
\end{pgfscope}%
\end{pgfscope}%
\begin{pgfscope}%
\pgftext[x=0.706944in,y=5.246050in,left,base]{\sffamily\fontsize{10.000000}{12.000000}\selectfont \(\displaystyle 210\)}%
\end{pgfscope}%
\begin{pgfscope}%
\pgfsetbuttcap%
\pgfsetroundjoin%
\definecolor{currentfill}{rgb}{0.000000,0.000000,0.000000}%
\pgfsetfillcolor{currentfill}%
\pgfsetlinewidth{0.803000pt}%
\definecolor{currentstroke}{rgb}{0.000000,0.000000,0.000000}%
\pgfsetstrokecolor{currentstroke}%
\pgfsetdash{}{0pt}%
\pgfsys@defobject{currentmarker}{\pgfqpoint{-0.048611in}{0.000000in}}{\pgfqpoint{0.000000in}{0.000000in}}{%
\pgfpathmoveto{\pgfqpoint{0.000000in}{0.000000in}}%
\pgfpathlineto{\pgfqpoint{-0.048611in}{0.000000in}}%
\pgfusepath{stroke,fill}%
}%
\begin{pgfscope}%
\pgfsys@transformshift{1.012500in}{5.505597in}%
\pgfsys@useobject{currentmarker}{}%
\end{pgfscope}%
\end{pgfscope}%
\begin{pgfscope}%
\pgftext[x=0.706944in,y=5.452835in,left,base]{\sffamily\fontsize{10.000000}{12.000000}\selectfont \(\displaystyle 220\)}%
\end{pgfscope}%
\begin{pgfscope}%
\pgfsetbuttcap%
\pgfsetroundjoin%
\definecolor{currentfill}{rgb}{0.000000,0.000000,0.000000}%
\pgfsetfillcolor{currentfill}%
\pgfsetlinewidth{0.803000pt}%
\definecolor{currentstroke}{rgb}{0.000000,0.000000,0.000000}%
\pgfsetstrokecolor{currentstroke}%
\pgfsetdash{}{0pt}%
\pgfsys@defobject{currentmarker}{\pgfqpoint{-0.048611in}{0.000000in}}{\pgfqpoint{0.000000in}{0.000000in}}{%
\pgfpathmoveto{\pgfqpoint{0.000000in}{0.000000in}}%
\pgfpathlineto{\pgfqpoint{-0.048611in}{0.000000in}}%
\pgfusepath{stroke,fill}%
}%
\begin{pgfscope}%
\pgfsys@transformshift{1.012500in}{5.712383in}%
\pgfsys@useobject{currentmarker}{}%
\end{pgfscope}%
\end{pgfscope}%
\begin{pgfscope}%
\pgftext[x=0.706944in,y=5.659621in,left,base]{\sffamily\fontsize{10.000000}{12.000000}\selectfont \(\displaystyle 230\)}%
\end{pgfscope}%
\begin{pgfscope}%
\pgftext[x=0.651388in,y=3.267000in,,bottom,rotate=90.000000]{\sffamily\fontsize{12.000000}{14.400000}\selectfont seconds}%
\end{pgfscope}%
\begin{pgfscope}%
\pgfpathrectangle{\pgfqpoint{1.012500in}{0.726000in}}{\pgfqpoint{6.277500in}{5.082000in}}%
\pgfusepath{clip}%
\pgfsetrectcap%
\pgfsetroundjoin%
\pgfsetlinewidth{1.505625pt}%
\definecolor{currentstroke}{rgb}{0.172549,0.243137,0.313725}%
\pgfsetstrokecolor{currentstroke}%
\pgfsetdash{}{0pt}%
\pgfpathmoveto{\pgfqpoint{1.312142in}{0.970076in}}%
\pgfpathlineto{\pgfqpoint{1.611784in}{1.009013in}}%
\pgfpathlineto{\pgfqpoint{1.911426in}{1.080293in}}%
\pgfpathlineto{\pgfqpoint{2.211068in}{1.184390in}}%
\pgfpathlineto{\pgfqpoint{2.510710in}{1.283974in}}%
\pgfpathlineto{\pgfqpoint{2.810352in}{1.427928in}}%
\pgfpathlineto{\pgfqpoint{3.109994in}{1.610385in}}%
\pgfpathlineto{\pgfqpoint{3.409636in}{1.797896in}}%
\pgfpathlineto{\pgfqpoint{3.709278in}{2.054201in}}%
\pgfpathlineto{\pgfqpoint{4.008920in}{2.319644in}}%
\pgfpathlineto{\pgfqpoint{4.308562in}{2.294087in}}%
\pgfpathlineto{\pgfqpoint{4.608204in}{2.474652in}}%
\pgfpathlineto{\pgfqpoint{4.907846in}{2.582923in}}%
\pgfpathlineto{\pgfqpoint{5.207488in}{2.934833in}}%
\pgfpathlineto{\pgfqpoint{5.507130in}{3.347313in}}%
\pgfpathlineto{\pgfqpoint{5.806772in}{3.850360in}}%
\pgfpathlineto{\pgfqpoint{6.106414in}{4.259382in}}%
\pgfpathlineto{\pgfqpoint{6.406056in}{4.379649in}}%
\pgfpathlineto{\pgfqpoint{6.705698in}{4.993616in}}%
\pgfpathlineto{\pgfqpoint{7.005340in}{5.577000in}}%
\pgfusepath{stroke}%
\end{pgfscope}%
\begin{pgfscope}%
\pgfpathrectangle{\pgfqpoint{1.012500in}{0.726000in}}{\pgfqpoint{6.277500in}{5.082000in}}%
\pgfusepath{clip}%
\pgfsetrectcap%
\pgfsetroundjoin%
\pgfsetlinewidth{1.505625pt}%
\definecolor{currentstroke}{rgb}{0.086275,0.627451,0.521569}%
\pgfsetstrokecolor{currentstroke}%
\pgfsetdash{}{0pt}%
\pgfpathmoveto{\pgfqpoint{1.312142in}{0.957000in}}%
\pgfpathlineto{\pgfqpoint{1.611784in}{0.958645in}}%
\pgfpathlineto{\pgfqpoint{1.911426in}{0.961737in}}%
\pgfpathlineto{\pgfqpoint{2.211068in}{0.966354in}}%
\pgfpathlineto{\pgfqpoint{2.510710in}{0.970532in}}%
\pgfpathlineto{\pgfqpoint{2.810352in}{0.976822in}}%
\pgfpathlineto{\pgfqpoint{3.109994in}{0.984729in}}%
\pgfpathlineto{\pgfqpoint{3.409636in}{0.992454in}}%
\pgfpathlineto{\pgfqpoint{3.709278in}{1.004108in}}%
\pgfpathlineto{\pgfqpoint{4.008920in}{1.016118in}}%
\pgfpathlineto{\pgfqpoint{4.308562in}{1.012648in}}%
\pgfpathlineto{\pgfqpoint{4.608204in}{1.019145in}}%
\pgfpathlineto{\pgfqpoint{4.907846in}{1.024302in}}%
\pgfpathlineto{\pgfqpoint{5.207488in}{1.037882in}}%
\pgfpathlineto{\pgfqpoint{5.507130in}{1.055243in}}%
\pgfpathlineto{\pgfqpoint{5.806772in}{1.078164in}}%
\pgfpathlineto{\pgfqpoint{6.106414in}{1.096758in}}%
\pgfpathlineto{\pgfqpoint{6.406056in}{1.100500in}}%
\pgfpathlineto{\pgfqpoint{6.705698in}{1.127999in}}%
\pgfpathlineto{\pgfqpoint{7.005340in}{1.157340in}}%
\pgfusepath{stroke}%
\end{pgfscope}%
\begin{pgfscope}%
\pgfsetrectcap%
\pgfsetmiterjoin%
\pgfsetlinewidth{0.803000pt}%
\definecolor{currentstroke}{rgb}{0.000000,0.000000,0.000000}%
\pgfsetstrokecolor{currentstroke}%
\pgfsetdash{}{0pt}%
\pgfpathmoveto{\pgfqpoint{1.012500in}{0.726000in}}%
\pgfpathlineto{\pgfqpoint{1.012500in}{5.808000in}}%
\pgfusepath{stroke}%
\end{pgfscope}%
\begin{pgfscope}%
\pgfsetrectcap%
\pgfsetmiterjoin%
\pgfsetlinewidth{0.803000pt}%
\definecolor{currentstroke}{rgb}{0.000000,0.000000,0.000000}%
\pgfsetstrokecolor{currentstroke}%
\pgfsetdash{}{0pt}%
\pgfpathmoveto{\pgfqpoint{7.290000in}{0.726000in}}%
\pgfpathlineto{\pgfqpoint{7.290000in}{5.808000in}}%
\pgfusepath{stroke}%
\end{pgfscope}%
\begin{pgfscope}%
\pgfsetrectcap%
\pgfsetmiterjoin%
\pgfsetlinewidth{0.803000pt}%
\definecolor{currentstroke}{rgb}{0.000000,0.000000,0.000000}%
\pgfsetstrokecolor{currentstroke}%
\pgfsetdash{}{0pt}%
\pgfpathmoveto{\pgfqpoint{1.012500in}{0.726000in}}%
\pgfpathlineto{\pgfqpoint{7.290000in}{0.726000in}}%
\pgfusepath{stroke}%
\end{pgfscope}%
\begin{pgfscope}%
\pgfsetrectcap%
\pgfsetmiterjoin%
\pgfsetlinewidth{0.803000pt}%
\definecolor{currentstroke}{rgb}{0.000000,0.000000,0.000000}%
\pgfsetstrokecolor{currentstroke}%
\pgfsetdash{}{0pt}%
\pgfpathmoveto{\pgfqpoint{1.012500in}{5.808000in}}%
\pgfpathlineto{\pgfqpoint{7.290000in}{5.808000in}}%
\pgfusepath{stroke}%
\end{pgfscope}%
\begin{pgfscope}%
\pgfsetbuttcap%
\pgfsetmiterjoin%
\definecolor{currentfill}{rgb}{1.000000,1.000000,1.000000}%
\pgfsetfillcolor{currentfill}%
\pgfsetfillopacity{0.800000}%
\pgfsetlinewidth{1.003750pt}%
\definecolor{currentstroke}{rgb}{0.800000,0.800000,0.800000}%
\pgfsetstrokecolor{currentstroke}%
\pgfsetstrokeopacity{0.800000}%
\pgfsetdash{}{0pt}%
\pgfpathmoveto{\pgfqpoint{1.109722in}{5.289174in}}%
\pgfpathlineto{\pgfqpoint{2.257970in}{5.289174in}}%
\pgfpathquadraticcurveto{\pgfqpoint{2.285748in}{5.289174in}}{\pgfqpoint{2.285748in}{5.316952in}}%
\pgfpathlineto{\pgfqpoint{2.285748in}{5.710778in}}%
\pgfpathquadraticcurveto{\pgfqpoint{2.285748in}{5.738556in}}{\pgfqpoint{2.257970in}{5.738556in}}%
\pgfpathlineto{\pgfqpoint{1.109722in}{5.738556in}}%
\pgfpathquadraticcurveto{\pgfqpoint{1.081944in}{5.738556in}}{\pgfqpoint{1.081944in}{5.710778in}}%
\pgfpathlineto{\pgfqpoint{1.081944in}{5.316952in}}%
\pgfpathquadraticcurveto{\pgfqpoint{1.081944in}{5.289174in}}{\pgfqpoint{1.109722in}{5.289174in}}%
\pgfpathclose%
\pgfusepath{stroke,fill}%
\end{pgfscope}%
\begin{pgfscope}%
\pgfsetrectcap%
\pgfsetroundjoin%
\pgfsetlinewidth{1.505625pt}%
\definecolor{currentstroke}{rgb}{0.172549,0.243137,0.313725}%
\pgfsetstrokecolor{currentstroke}%
\pgfsetdash{}{0pt}%
\pgfpathmoveto{\pgfqpoint{1.137500in}{5.626088in}}%
\pgfpathlineto{\pgfqpoint{1.415278in}{5.626088in}}%
\pgfusepath{stroke}%
\end{pgfscope}%
\begin{pgfscope}%
\pgftext[x=1.526389in,y=5.577477in,left,base]{\sffamily\fontsize{10.000000}{12.000000}\selectfont \textsc{Linclosure}}%
\end{pgfscope}%
\begin{pgfscope}%
\pgfsetrectcap%
\pgfsetroundjoin%
\pgfsetlinewidth{1.505625pt}%
\definecolor{currentstroke}{rgb}{0.086275,0.627451,0.521569}%
\pgfsetstrokecolor{currentstroke}%
\pgfsetdash{}{0pt}%
\pgfpathmoveto{\pgfqpoint{1.137500in}{5.422231in}}%
\pgfpathlineto{\pgfqpoint{1.415278in}{5.422231in}}%
\pgfusepath{stroke}%
\end{pgfscope}%
\begin{pgfscope}%
\pgftext[x=1.526389in,y=5.373620in,left,base]{\sffamily\fontsize{10.000000}{12.000000}\selectfont \textsc{Closure}}%
\end{pgfscope}%
\end{pgfpicture}%
\makeatother%
\endgroup%
}
}
\end{minipage}
~
\begin{minipage}{0.4\textwidth}
\subfloat[Some landmarks times, $|\Sg| = 100$]{
\begin{tabular}{| c || c | c |}
	\hline \rowcolor{clouds}
	$|\B|$ & \textsc{LinClosure} & \textsc{Closure} \\ \hline
	5000  & 15.846  & 0.688 \\ \hline
	10000 & 65.930  & 2.724 \\ \hline
	15000 & 115.627 & 4.784 \\ \hline 
	20000 & 223.453 & 9.721 \\ \hline 
\end{tabular}
}
\end{minipage}

\centering

\subfloat[Average time (in $s$), $|\B| = 100$]{
	\scalebox{0.45}{%% Creator: Matplotlib, PGF backend
%%
%% To include the figure in your LaTeX document, write
%%   \input{<filename>.pgf}
%%
%% Make sure the required packages are loaded in your preamble
%%   \usepackage{pgf}
%%
%% Figures using additional raster images can only be included by \input if
%% they are in the same directory as the main LaTeX file. For loading figures
%% from other directories you can use the `import` package
%%   \usepackage{import}
%% and then include the figures with
%%   \import{<path to file>}{<filename>.pgf}
%%
%% Matplotlib used the following preamble
%%   \usepackage{fontspec}
%%   \setmainfont{DejaVu Serif}
%%   \setsansfont{DejaVu Sans}
%%   \setmonofont{DejaVu Sans Mono}
%%
\begingroup%
\makeatletter%
\begin{pgfpicture}%
\pgfpathrectangle{\pgfpointorigin}{\pgfqpoint{7.540000in}{5.870000in}}%
\pgfusepath{use as bounding box, clip}%
\begin{pgfscope}%
\pgfsetbuttcap%
\pgfsetmiterjoin%
\definecolor{currentfill}{rgb}{1.000000,1.000000,1.000000}%
\pgfsetfillcolor{currentfill}%
\pgfsetlinewidth{0.000000pt}%
\definecolor{currentstroke}{rgb}{1.000000,1.000000,1.000000}%
\pgfsetstrokecolor{currentstroke}%
\pgfsetdash{}{0pt}%
\pgfpathmoveto{\pgfqpoint{0.000000in}{0.000000in}}%
\pgfpathlineto{\pgfqpoint{7.540000in}{0.000000in}}%
\pgfpathlineto{\pgfqpoint{7.540000in}{5.870000in}}%
\pgfpathlineto{\pgfqpoint{0.000000in}{5.870000in}}%
\pgfpathclose%
\pgfusepath{fill}%
\end{pgfscope}%
\begin{pgfscope}%
\pgfsetbuttcap%
\pgfsetmiterjoin%
\definecolor{currentfill}{rgb}{1.000000,1.000000,1.000000}%
\pgfsetfillcolor{currentfill}%
\pgfsetlinewidth{0.000000pt}%
\definecolor{currentstroke}{rgb}{0.000000,0.000000,0.000000}%
\pgfsetstrokecolor{currentstroke}%
\pgfsetstrokeopacity{0.000000}%
\pgfsetdash{}{0pt}%
\pgfpathmoveto{\pgfqpoint{0.942500in}{0.645700in}}%
\pgfpathlineto{\pgfqpoint{6.786000in}{0.645700in}}%
\pgfpathlineto{\pgfqpoint{6.786000in}{5.165600in}}%
\pgfpathlineto{\pgfqpoint{0.942500in}{5.165600in}}%
\pgfpathclose%
\pgfusepath{fill}%
\end{pgfscope}%
\begin{pgfscope}%
\pgfsetbuttcap%
\pgfsetroundjoin%
\definecolor{currentfill}{rgb}{0.000000,0.000000,0.000000}%
\pgfsetfillcolor{currentfill}%
\pgfsetlinewidth{0.803000pt}%
\definecolor{currentstroke}{rgb}{0.000000,0.000000,0.000000}%
\pgfsetstrokecolor{currentstroke}%
\pgfsetdash{}{0pt}%
\pgfsys@defobject{currentmarker}{\pgfqpoint{0.000000in}{-0.048611in}}{\pgfqpoint{0.000000in}{0.000000in}}{%
\pgfpathmoveto{\pgfqpoint{0.000000in}{0.000000in}}%
\pgfpathlineto{\pgfqpoint{0.000000in}{-0.048611in}}%
\pgfusepath{stroke,fill}%
}%
\begin{pgfscope}%
\pgfsys@transformshift{1.154454in}{0.645700in}%
\pgfsys@useobject{currentmarker}{}%
\end{pgfscope}%
\end{pgfscope}%
\begin{pgfscope}%
\pgftext[x=1.154454in,y=0.548478in,,top]{\sffamily\fontsize{10.000000}{12.000000}\selectfont \(\displaystyle 0\)}%
\end{pgfscope}%
\begin{pgfscope}%
\pgfsetbuttcap%
\pgfsetroundjoin%
\definecolor{currentfill}{rgb}{0.000000,0.000000,0.000000}%
\pgfsetfillcolor{currentfill}%
\pgfsetlinewidth{0.803000pt}%
\definecolor{currentstroke}{rgb}{0.000000,0.000000,0.000000}%
\pgfsetstrokecolor{currentstroke}%
\pgfsetdash{}{0pt}%
\pgfsys@defobject{currentmarker}{\pgfqpoint{0.000000in}{-0.048611in}}{\pgfqpoint{0.000000in}{0.000000in}}{%
\pgfpathmoveto{\pgfqpoint{0.000000in}{0.000000in}}%
\pgfpathlineto{\pgfqpoint{0.000000in}{-0.048611in}}%
\pgfusepath{stroke,fill}%
}%
\begin{pgfscope}%
\pgfsys@transformshift{1.691048in}{0.645700in}%
\pgfsys@useobject{currentmarker}{}%
\end{pgfscope}%
\end{pgfscope}%
\begin{pgfscope}%
\pgftext[x=1.691048in,y=0.548478in,,top]{\sffamily\fontsize{10.000000}{12.000000}\selectfont \(\displaystyle 10000\)}%
\end{pgfscope}%
\begin{pgfscope}%
\pgfsetbuttcap%
\pgfsetroundjoin%
\definecolor{currentfill}{rgb}{0.000000,0.000000,0.000000}%
\pgfsetfillcolor{currentfill}%
\pgfsetlinewidth{0.803000pt}%
\definecolor{currentstroke}{rgb}{0.000000,0.000000,0.000000}%
\pgfsetstrokecolor{currentstroke}%
\pgfsetdash{}{0pt}%
\pgfsys@defobject{currentmarker}{\pgfqpoint{0.000000in}{-0.048611in}}{\pgfqpoint{0.000000in}{0.000000in}}{%
\pgfpathmoveto{\pgfqpoint{0.000000in}{0.000000in}}%
\pgfpathlineto{\pgfqpoint{0.000000in}{-0.048611in}}%
\pgfusepath{stroke,fill}%
}%
\begin{pgfscope}%
\pgfsys@transformshift{2.227641in}{0.645700in}%
\pgfsys@useobject{currentmarker}{}%
\end{pgfscope}%
\end{pgfscope}%
\begin{pgfscope}%
\pgftext[x=2.227641in,y=0.548478in,,top]{\sffamily\fontsize{10.000000}{12.000000}\selectfont \(\displaystyle 20000\)}%
\end{pgfscope}%
\begin{pgfscope}%
\pgfsetbuttcap%
\pgfsetroundjoin%
\definecolor{currentfill}{rgb}{0.000000,0.000000,0.000000}%
\pgfsetfillcolor{currentfill}%
\pgfsetlinewidth{0.803000pt}%
\definecolor{currentstroke}{rgb}{0.000000,0.000000,0.000000}%
\pgfsetstrokecolor{currentstroke}%
\pgfsetdash{}{0pt}%
\pgfsys@defobject{currentmarker}{\pgfqpoint{0.000000in}{-0.048611in}}{\pgfqpoint{0.000000in}{0.000000in}}{%
\pgfpathmoveto{\pgfqpoint{0.000000in}{0.000000in}}%
\pgfpathlineto{\pgfqpoint{0.000000in}{-0.048611in}}%
\pgfusepath{stroke,fill}%
}%
\begin{pgfscope}%
\pgfsys@transformshift{2.764234in}{0.645700in}%
\pgfsys@useobject{currentmarker}{}%
\end{pgfscope}%
\end{pgfscope}%
\begin{pgfscope}%
\pgftext[x=2.764234in,y=0.548478in,,top]{\sffamily\fontsize{10.000000}{12.000000}\selectfont \(\displaystyle 30000\)}%
\end{pgfscope}%
\begin{pgfscope}%
\pgfsetbuttcap%
\pgfsetroundjoin%
\definecolor{currentfill}{rgb}{0.000000,0.000000,0.000000}%
\pgfsetfillcolor{currentfill}%
\pgfsetlinewidth{0.803000pt}%
\definecolor{currentstroke}{rgb}{0.000000,0.000000,0.000000}%
\pgfsetstrokecolor{currentstroke}%
\pgfsetdash{}{0pt}%
\pgfsys@defobject{currentmarker}{\pgfqpoint{0.000000in}{-0.048611in}}{\pgfqpoint{0.000000in}{0.000000in}}{%
\pgfpathmoveto{\pgfqpoint{0.000000in}{0.000000in}}%
\pgfpathlineto{\pgfqpoint{0.000000in}{-0.048611in}}%
\pgfusepath{stroke,fill}%
}%
\begin{pgfscope}%
\pgfsys@transformshift{3.300827in}{0.645700in}%
\pgfsys@useobject{currentmarker}{}%
\end{pgfscope}%
\end{pgfscope}%
\begin{pgfscope}%
\pgftext[x=3.300827in,y=0.548478in,,top]{\sffamily\fontsize{10.000000}{12.000000}\selectfont \(\displaystyle 40000\)}%
\end{pgfscope}%
\begin{pgfscope}%
\pgfsetbuttcap%
\pgfsetroundjoin%
\definecolor{currentfill}{rgb}{0.000000,0.000000,0.000000}%
\pgfsetfillcolor{currentfill}%
\pgfsetlinewidth{0.803000pt}%
\definecolor{currentstroke}{rgb}{0.000000,0.000000,0.000000}%
\pgfsetstrokecolor{currentstroke}%
\pgfsetdash{}{0pt}%
\pgfsys@defobject{currentmarker}{\pgfqpoint{0.000000in}{-0.048611in}}{\pgfqpoint{0.000000in}{0.000000in}}{%
\pgfpathmoveto{\pgfqpoint{0.000000in}{0.000000in}}%
\pgfpathlineto{\pgfqpoint{0.000000in}{-0.048611in}}%
\pgfusepath{stroke,fill}%
}%
\begin{pgfscope}%
\pgfsys@transformshift{3.837420in}{0.645700in}%
\pgfsys@useobject{currentmarker}{}%
\end{pgfscope}%
\end{pgfscope}%
\begin{pgfscope}%
\pgftext[x=3.837420in,y=0.548478in,,top]{\sffamily\fontsize{10.000000}{12.000000}\selectfont \(\displaystyle 50000\)}%
\end{pgfscope}%
\begin{pgfscope}%
\pgfsetbuttcap%
\pgfsetroundjoin%
\definecolor{currentfill}{rgb}{0.000000,0.000000,0.000000}%
\pgfsetfillcolor{currentfill}%
\pgfsetlinewidth{0.803000pt}%
\definecolor{currentstroke}{rgb}{0.000000,0.000000,0.000000}%
\pgfsetstrokecolor{currentstroke}%
\pgfsetdash{}{0pt}%
\pgfsys@defobject{currentmarker}{\pgfqpoint{0.000000in}{-0.048611in}}{\pgfqpoint{0.000000in}{0.000000in}}{%
\pgfpathmoveto{\pgfqpoint{0.000000in}{0.000000in}}%
\pgfpathlineto{\pgfqpoint{0.000000in}{-0.048611in}}%
\pgfusepath{stroke,fill}%
}%
\begin{pgfscope}%
\pgfsys@transformshift{4.374014in}{0.645700in}%
\pgfsys@useobject{currentmarker}{}%
\end{pgfscope}%
\end{pgfscope}%
\begin{pgfscope}%
\pgftext[x=4.374014in,y=0.548478in,,top]{\sffamily\fontsize{10.000000}{12.000000}\selectfont \(\displaystyle 60000\)}%
\end{pgfscope}%
\begin{pgfscope}%
\pgfsetbuttcap%
\pgfsetroundjoin%
\definecolor{currentfill}{rgb}{0.000000,0.000000,0.000000}%
\pgfsetfillcolor{currentfill}%
\pgfsetlinewidth{0.803000pt}%
\definecolor{currentstroke}{rgb}{0.000000,0.000000,0.000000}%
\pgfsetstrokecolor{currentstroke}%
\pgfsetdash{}{0pt}%
\pgfsys@defobject{currentmarker}{\pgfqpoint{0.000000in}{-0.048611in}}{\pgfqpoint{0.000000in}{0.000000in}}{%
\pgfpathmoveto{\pgfqpoint{0.000000in}{0.000000in}}%
\pgfpathlineto{\pgfqpoint{0.000000in}{-0.048611in}}%
\pgfusepath{stroke,fill}%
}%
\begin{pgfscope}%
\pgfsys@transformshift{4.910607in}{0.645700in}%
\pgfsys@useobject{currentmarker}{}%
\end{pgfscope}%
\end{pgfscope}%
\begin{pgfscope}%
\pgftext[x=4.910607in,y=0.548478in,,top]{\sffamily\fontsize{10.000000}{12.000000}\selectfont \(\displaystyle 70000\)}%
\end{pgfscope}%
\begin{pgfscope}%
\pgfsetbuttcap%
\pgfsetroundjoin%
\definecolor{currentfill}{rgb}{0.000000,0.000000,0.000000}%
\pgfsetfillcolor{currentfill}%
\pgfsetlinewidth{0.803000pt}%
\definecolor{currentstroke}{rgb}{0.000000,0.000000,0.000000}%
\pgfsetstrokecolor{currentstroke}%
\pgfsetdash{}{0pt}%
\pgfsys@defobject{currentmarker}{\pgfqpoint{0.000000in}{-0.048611in}}{\pgfqpoint{0.000000in}{0.000000in}}{%
\pgfpathmoveto{\pgfqpoint{0.000000in}{0.000000in}}%
\pgfpathlineto{\pgfqpoint{0.000000in}{-0.048611in}}%
\pgfusepath{stroke,fill}%
}%
\begin{pgfscope}%
\pgfsys@transformshift{5.447200in}{0.645700in}%
\pgfsys@useobject{currentmarker}{}%
\end{pgfscope}%
\end{pgfscope}%
\begin{pgfscope}%
\pgftext[x=5.447200in,y=0.548478in,,top]{\sffamily\fontsize{10.000000}{12.000000}\selectfont \(\displaystyle 80000\)}%
\end{pgfscope}%
\begin{pgfscope}%
\pgfsetbuttcap%
\pgfsetroundjoin%
\definecolor{currentfill}{rgb}{0.000000,0.000000,0.000000}%
\pgfsetfillcolor{currentfill}%
\pgfsetlinewidth{0.803000pt}%
\definecolor{currentstroke}{rgb}{0.000000,0.000000,0.000000}%
\pgfsetstrokecolor{currentstroke}%
\pgfsetdash{}{0pt}%
\pgfsys@defobject{currentmarker}{\pgfqpoint{0.000000in}{-0.048611in}}{\pgfqpoint{0.000000in}{0.000000in}}{%
\pgfpathmoveto{\pgfqpoint{0.000000in}{0.000000in}}%
\pgfpathlineto{\pgfqpoint{0.000000in}{-0.048611in}}%
\pgfusepath{stroke,fill}%
}%
\begin{pgfscope}%
\pgfsys@transformshift{5.983793in}{0.645700in}%
\pgfsys@useobject{currentmarker}{}%
\end{pgfscope}%
\end{pgfscope}%
\begin{pgfscope}%
\pgftext[x=5.983793in,y=0.548478in,,top]{\sffamily\fontsize{10.000000}{12.000000}\selectfont \(\displaystyle 90000\)}%
\end{pgfscope}%
\begin{pgfscope}%
\pgfsetbuttcap%
\pgfsetroundjoin%
\definecolor{currentfill}{rgb}{0.000000,0.000000,0.000000}%
\pgfsetfillcolor{currentfill}%
\pgfsetlinewidth{0.803000pt}%
\definecolor{currentstroke}{rgb}{0.000000,0.000000,0.000000}%
\pgfsetstrokecolor{currentstroke}%
\pgfsetdash{}{0pt}%
\pgfsys@defobject{currentmarker}{\pgfqpoint{0.000000in}{-0.048611in}}{\pgfqpoint{0.000000in}{0.000000in}}{%
\pgfpathmoveto{\pgfqpoint{0.000000in}{0.000000in}}%
\pgfpathlineto{\pgfqpoint{0.000000in}{-0.048611in}}%
\pgfusepath{stroke,fill}%
}%
\begin{pgfscope}%
\pgfsys@transformshift{6.520386in}{0.645700in}%
\pgfsys@useobject{currentmarker}{}%
\end{pgfscope}%
\end{pgfscope}%
\begin{pgfscope}%
\pgftext[x=6.520386in,y=0.548478in,,top]{\sffamily\fontsize{10.000000}{12.000000}\selectfont \(\displaystyle 100000\)}%
\end{pgfscope}%
\begin{pgfscope}%
\pgftext[x=3.864250in,y=0.358509in,,top]{\sffamily\fontsize{14.000000}{16.800000}\selectfont \(\displaystyle |\Sigma|\)}%
\end{pgfscope}%
\begin{pgfscope}%
\pgfsetbuttcap%
\pgfsetroundjoin%
\definecolor{currentfill}{rgb}{0.000000,0.000000,0.000000}%
\pgfsetfillcolor{currentfill}%
\pgfsetlinewidth{0.803000pt}%
\definecolor{currentstroke}{rgb}{0.000000,0.000000,0.000000}%
\pgfsetstrokecolor{currentstroke}%
\pgfsetdash{}{0pt}%
\pgfsys@defobject{currentmarker}{\pgfqpoint{-0.048611in}{0.000000in}}{\pgfqpoint{0.000000in}{0.000000in}}{%
\pgfpathmoveto{\pgfqpoint{0.000000in}{0.000000in}}%
\pgfpathlineto{\pgfqpoint{-0.048611in}{0.000000in}}%
\pgfusepath{stroke,fill}%
}%
\begin{pgfscope}%
\pgfsys@transformshift{0.942500in}{0.851150in}%
\pgfsys@useobject{currentmarker}{}%
\end{pgfscope}%
\end{pgfscope}%
\begin{pgfscope}%
\pgftext[x=0.667808in,y=0.798388in,left,base]{\sffamily\fontsize{10.000000}{12.000000}\selectfont \(\displaystyle 0.0\)}%
\end{pgfscope}%
\begin{pgfscope}%
\pgfsetbuttcap%
\pgfsetroundjoin%
\definecolor{currentfill}{rgb}{0.000000,0.000000,0.000000}%
\pgfsetfillcolor{currentfill}%
\pgfsetlinewidth{0.803000pt}%
\definecolor{currentstroke}{rgb}{0.000000,0.000000,0.000000}%
\pgfsetstrokecolor{currentstroke}%
\pgfsetdash{}{0pt}%
\pgfsys@defobject{currentmarker}{\pgfqpoint{-0.048611in}{0.000000in}}{\pgfqpoint{0.000000in}{0.000000in}}{%
\pgfpathmoveto{\pgfqpoint{0.000000in}{0.000000in}}%
\pgfpathlineto{\pgfqpoint{-0.048611in}{0.000000in}}%
\pgfusepath{stroke,fill}%
}%
\begin{pgfscope}%
\pgfsys@transformshift{0.942500in}{1.458065in}%
\pgfsys@useobject{currentmarker}{}%
\end{pgfscope}%
\end{pgfscope}%
\begin{pgfscope}%
\pgftext[x=0.667808in,y=1.405303in,left,base]{\sffamily\fontsize{10.000000}{12.000000}\selectfont \(\displaystyle 0.1\)}%
\end{pgfscope}%
\begin{pgfscope}%
\pgfsetbuttcap%
\pgfsetroundjoin%
\definecolor{currentfill}{rgb}{0.000000,0.000000,0.000000}%
\pgfsetfillcolor{currentfill}%
\pgfsetlinewidth{0.803000pt}%
\definecolor{currentstroke}{rgb}{0.000000,0.000000,0.000000}%
\pgfsetstrokecolor{currentstroke}%
\pgfsetdash{}{0pt}%
\pgfsys@defobject{currentmarker}{\pgfqpoint{-0.048611in}{0.000000in}}{\pgfqpoint{0.000000in}{0.000000in}}{%
\pgfpathmoveto{\pgfqpoint{0.000000in}{0.000000in}}%
\pgfpathlineto{\pgfqpoint{-0.048611in}{0.000000in}}%
\pgfusepath{stroke,fill}%
}%
\begin{pgfscope}%
\pgfsys@transformshift{0.942500in}{2.064979in}%
\pgfsys@useobject{currentmarker}{}%
\end{pgfscope}%
\end{pgfscope}%
\begin{pgfscope}%
\pgftext[x=0.667808in,y=2.012218in,left,base]{\sffamily\fontsize{10.000000}{12.000000}\selectfont \(\displaystyle 0.2\)}%
\end{pgfscope}%
\begin{pgfscope}%
\pgfsetbuttcap%
\pgfsetroundjoin%
\definecolor{currentfill}{rgb}{0.000000,0.000000,0.000000}%
\pgfsetfillcolor{currentfill}%
\pgfsetlinewidth{0.803000pt}%
\definecolor{currentstroke}{rgb}{0.000000,0.000000,0.000000}%
\pgfsetstrokecolor{currentstroke}%
\pgfsetdash{}{0pt}%
\pgfsys@defobject{currentmarker}{\pgfqpoint{-0.048611in}{0.000000in}}{\pgfqpoint{0.000000in}{0.000000in}}{%
\pgfpathmoveto{\pgfqpoint{0.000000in}{0.000000in}}%
\pgfpathlineto{\pgfqpoint{-0.048611in}{0.000000in}}%
\pgfusepath{stroke,fill}%
}%
\begin{pgfscope}%
\pgfsys@transformshift{0.942500in}{2.671894in}%
\pgfsys@useobject{currentmarker}{}%
\end{pgfscope}%
\end{pgfscope}%
\begin{pgfscope}%
\pgftext[x=0.667808in,y=2.619132in,left,base]{\sffamily\fontsize{10.000000}{12.000000}\selectfont \(\displaystyle 0.3\)}%
\end{pgfscope}%
\begin{pgfscope}%
\pgfsetbuttcap%
\pgfsetroundjoin%
\definecolor{currentfill}{rgb}{0.000000,0.000000,0.000000}%
\pgfsetfillcolor{currentfill}%
\pgfsetlinewidth{0.803000pt}%
\definecolor{currentstroke}{rgb}{0.000000,0.000000,0.000000}%
\pgfsetstrokecolor{currentstroke}%
\pgfsetdash{}{0pt}%
\pgfsys@defobject{currentmarker}{\pgfqpoint{-0.048611in}{0.000000in}}{\pgfqpoint{0.000000in}{0.000000in}}{%
\pgfpathmoveto{\pgfqpoint{0.000000in}{0.000000in}}%
\pgfpathlineto{\pgfqpoint{-0.048611in}{0.000000in}}%
\pgfusepath{stroke,fill}%
}%
\begin{pgfscope}%
\pgfsys@transformshift{0.942500in}{3.278808in}%
\pgfsys@useobject{currentmarker}{}%
\end{pgfscope}%
\end{pgfscope}%
\begin{pgfscope}%
\pgftext[x=0.667808in,y=3.226047in,left,base]{\sffamily\fontsize{10.000000}{12.000000}\selectfont \(\displaystyle 0.4\)}%
\end{pgfscope}%
\begin{pgfscope}%
\pgfsetbuttcap%
\pgfsetroundjoin%
\definecolor{currentfill}{rgb}{0.000000,0.000000,0.000000}%
\pgfsetfillcolor{currentfill}%
\pgfsetlinewidth{0.803000pt}%
\definecolor{currentstroke}{rgb}{0.000000,0.000000,0.000000}%
\pgfsetstrokecolor{currentstroke}%
\pgfsetdash{}{0pt}%
\pgfsys@defobject{currentmarker}{\pgfqpoint{-0.048611in}{0.000000in}}{\pgfqpoint{0.000000in}{0.000000in}}{%
\pgfpathmoveto{\pgfqpoint{0.000000in}{0.000000in}}%
\pgfpathlineto{\pgfqpoint{-0.048611in}{0.000000in}}%
\pgfusepath{stroke,fill}%
}%
\begin{pgfscope}%
\pgfsys@transformshift{0.942500in}{3.885723in}%
\pgfsys@useobject{currentmarker}{}%
\end{pgfscope}%
\end{pgfscope}%
\begin{pgfscope}%
\pgftext[x=0.667808in,y=3.832961in,left,base]{\sffamily\fontsize{10.000000}{12.000000}\selectfont \(\displaystyle 0.5\)}%
\end{pgfscope}%
\begin{pgfscope}%
\pgfsetbuttcap%
\pgfsetroundjoin%
\definecolor{currentfill}{rgb}{0.000000,0.000000,0.000000}%
\pgfsetfillcolor{currentfill}%
\pgfsetlinewidth{0.803000pt}%
\definecolor{currentstroke}{rgb}{0.000000,0.000000,0.000000}%
\pgfsetstrokecolor{currentstroke}%
\pgfsetdash{}{0pt}%
\pgfsys@defobject{currentmarker}{\pgfqpoint{-0.048611in}{0.000000in}}{\pgfqpoint{0.000000in}{0.000000in}}{%
\pgfpathmoveto{\pgfqpoint{0.000000in}{0.000000in}}%
\pgfpathlineto{\pgfqpoint{-0.048611in}{0.000000in}}%
\pgfusepath{stroke,fill}%
}%
\begin{pgfscope}%
\pgfsys@transformshift{0.942500in}{4.492638in}%
\pgfsys@useobject{currentmarker}{}%
\end{pgfscope}%
\end{pgfscope}%
\begin{pgfscope}%
\pgftext[x=0.667808in,y=4.439876in,left,base]{\sffamily\fontsize{10.000000}{12.000000}\selectfont \(\displaystyle 0.6\)}%
\end{pgfscope}%
\begin{pgfscope}%
\pgfsetbuttcap%
\pgfsetroundjoin%
\definecolor{currentfill}{rgb}{0.000000,0.000000,0.000000}%
\pgfsetfillcolor{currentfill}%
\pgfsetlinewidth{0.803000pt}%
\definecolor{currentstroke}{rgb}{0.000000,0.000000,0.000000}%
\pgfsetstrokecolor{currentstroke}%
\pgfsetdash{}{0pt}%
\pgfsys@defobject{currentmarker}{\pgfqpoint{-0.048611in}{0.000000in}}{\pgfqpoint{0.000000in}{0.000000in}}{%
\pgfpathmoveto{\pgfqpoint{0.000000in}{0.000000in}}%
\pgfpathlineto{\pgfqpoint{-0.048611in}{0.000000in}}%
\pgfusepath{stroke,fill}%
}%
\begin{pgfscope}%
\pgfsys@transformshift{0.942500in}{5.099552in}%
\pgfsys@useobject{currentmarker}{}%
\end{pgfscope}%
\end{pgfscope}%
\begin{pgfscope}%
\pgftext[x=0.667808in,y=5.046791in,left,base]{\sffamily\fontsize{10.000000}{12.000000}\selectfont \(\displaystyle 0.7\)}%
\end{pgfscope}%
\begin{pgfscope}%
\pgftext[x=0.612252in,y=2.905650in,,bottom,rotate=90.000000]{\sffamily\fontsize{14.000000}{16.800000}\selectfont seconds}%
\end{pgfscope}%
\begin{pgfscope}%
\pgfpathrectangle{\pgfqpoint{0.942500in}{0.645700in}}{\pgfqpoint{5.843500in}{4.519900in}}%
\pgfusepath{clip}%
\pgfsetrectcap%
\pgfsetroundjoin%
\pgfsetlinewidth{1.505625pt}%
\definecolor{currentstroke}{rgb}{0.172549,0.243137,0.313725}%
\pgfsetstrokecolor{currentstroke}%
\pgfsetdash{}{0pt}%
\pgfpathmoveto{\pgfqpoint{1.208114in}{0.905203in}}%
\pgfpathlineto{\pgfqpoint{1.261773in}{0.923221in}}%
\pgfpathlineto{\pgfqpoint{1.315432in}{0.971585in}}%
\pgfpathlineto{\pgfqpoint{1.369092in}{0.981068in}}%
\pgfpathlineto{\pgfqpoint{1.422751in}{0.990550in}}%
\pgfpathlineto{\pgfqpoint{1.476410in}{1.016155in}}%
\pgfpathlineto{\pgfqpoint{1.530070in}{1.055984in}}%
\pgfpathlineto{\pgfqpoint{1.583729in}{1.117624in}}%
\pgfpathlineto{\pgfqpoint{1.637388in}{1.210558in}}%
\pgfpathlineto{\pgfqpoint{1.691048in}{1.192539in}}%
\pgfpathlineto{\pgfqpoint{1.744707in}{1.230472in}}%
\pgfpathlineto{\pgfqpoint{1.798366in}{1.283577in}}%
\pgfpathlineto{\pgfqpoint{1.852025in}{1.272197in}}%
\pgfpathlineto{\pgfqpoint{1.905685in}{1.323405in}}%
\pgfpathlineto{\pgfqpoint{1.959344in}{1.362286in}}%
\pgfpathlineto{\pgfqpoint{2.013003in}{1.427719in}}%
\pgfpathlineto{\pgfqpoint{2.066663in}{1.492204in}}%
\pgfpathlineto{\pgfqpoint{2.120322in}{1.373665in}}%
\pgfpathlineto{\pgfqpoint{2.173981in}{1.511170in}}%
\pgfpathlineto{\pgfqpoint{2.227641in}{1.645832in}}%
\pgfpathlineto{\pgfqpoint{2.281300in}{1.708417in}}%
\pgfpathlineto{\pgfqpoint{2.334959in}{1.647725in}}%
\pgfpathlineto{\pgfqpoint{2.388619in}{1.843079in}}%
\pgfpathlineto{\pgfqpoint{2.442278in}{1.797560in}}%
\pgfpathlineto{\pgfqpoint{2.495937in}{1.828853in}}%
\pgfpathlineto{\pgfqpoint{2.549597in}{1.833593in}}%
\pgfpathlineto{\pgfqpoint{2.603256in}{1.819367in}}%
\pgfpathlineto{\pgfqpoint{2.656915in}{1.878165in}}%
\pgfpathlineto{\pgfqpoint{2.710575in}{1.903764in}}%
\pgfpathlineto{\pgfqpoint{2.764234in}{2.129464in}}%
\pgfpathlineto{\pgfqpoint{2.817893in}{2.030840in}}%
\pgfpathlineto{\pgfqpoint{2.871553in}{1.985322in}}%
\pgfpathlineto{\pgfqpoint{2.925212in}{2.154123in}}%
\pgfpathlineto{\pgfqpoint{2.978871in}{1.979635in}}%
\pgfpathlineto{\pgfqpoint{3.032531in}{2.195842in}}%
\pgfpathlineto{\pgfqpoint{3.086190in}{2.341884in}}%
\pgfpathlineto{\pgfqpoint{3.139849in}{2.296365in}}%
\pgfpathlineto{\pgfqpoint{3.193508in}{2.153170in}}%
\pgfpathlineto{\pgfqpoint{3.247168in}{2.231881in}}%
\pgfpathlineto{\pgfqpoint{3.300827in}{2.299212in}}%
\pgfpathlineto{\pgfqpoint{3.354486in}{2.284986in}}%
\pgfpathlineto{\pgfqpoint{3.408146in}{2.583703in}}%
\pgfpathlineto{\pgfqpoint{3.461805in}{2.566631in}}%
\pgfpathlineto{\pgfqpoint{3.515464in}{2.549564in}}%
\pgfpathlineto{\pgfqpoint{3.569124in}{2.410162in}}%
\pgfpathlineto{\pgfqpoint{3.622783in}{2.683273in}}%
\pgfpathlineto{\pgfqpoint{3.676442in}{2.818882in}}%
\pgfpathlineto{\pgfqpoint{3.730102in}{2.570424in}}%
\pgfpathlineto{\pgfqpoint{3.783761in}{2.774311in}}%
\pgfpathlineto{\pgfqpoint{3.837420in}{2.675687in}}%
\pgfpathlineto{\pgfqpoint{3.891080in}{2.923193in}}%
\pgfpathlineto{\pgfqpoint{3.944739in}{2.565684in}}%
\pgfpathlineto{\pgfqpoint{3.998398in}{2.493613in}}%
\pgfpathlineto{\pgfqpoint{4.052058in}{2.771464in}}%
\pgfpathlineto{\pgfqpoint{4.105717in}{2.909920in}}%
\pgfpathlineto{\pgfqpoint{4.159376in}{2.854915in}}%
\pgfpathlineto{\pgfqpoint{4.213036in}{3.134666in}}%
\pgfpathlineto{\pgfqpoint{4.266695in}{3.005697in}}%
\pgfpathlineto{\pgfqpoint{4.320354in}{3.428637in}}%
\pgfpathlineto{\pgfqpoint{4.374014in}{2.966818in}}%
\pgfpathlineto{\pgfqpoint{4.427673in}{3.023716in}}%
\pgfpathlineto{\pgfqpoint{4.481332in}{2.867247in}}%
\pgfpathlineto{\pgfqpoint{4.534992in}{3.247516in}}%
\pgfpathlineto{\pgfqpoint{4.588651in}{3.240876in}}%
\pgfpathlineto{\pgfqpoint{4.642310in}{3.637264in}}%
\pgfpathlineto{\pgfqpoint{4.695969in}{3.583212in}}%
\pgfpathlineto{\pgfqpoint{4.749629in}{3.306308in}}%
\pgfpathlineto{\pgfqpoint{4.803288in}{3.376485in}}%
\pgfpathlineto{\pgfqpoint{4.856947in}{3.670456in}}%
\pgfpathlineto{\pgfqpoint{4.910607in}{3.412518in}}%
\pgfpathlineto{\pgfqpoint{4.964266in}{4.152195in}}%
\pgfpathlineto{\pgfqpoint{5.017925in}{3.719768in}}%
\pgfpathlineto{\pgfqpoint{5.071585in}{3.714081in}}%
\pgfpathlineto{\pgfqpoint{5.125244in}{3.485542in}}%
\pgfpathlineto{\pgfqpoint{5.178903in}{3.827878in}}%
\pgfpathlineto{\pgfqpoint{5.232563in}{3.828825in}}%
\pgfpathlineto{\pgfqpoint{5.286222in}{3.398292in}}%
\pgfpathlineto{\pgfqpoint{5.339881in}{3.772873in}}%
\pgfpathlineto{\pgfqpoint{5.393541in}{4.019432in}}%
\pgfpathlineto{\pgfqpoint{5.447200in}{4.089610in}}%
\pgfpathlineto{\pgfqpoint{5.500859in}{3.490282in}}%
\pgfpathlineto{\pgfqpoint{5.554519in}{3.920809in}}%
\pgfpathlineto{\pgfqpoint{5.608178in}{3.929348in}}%
\pgfpathlineto{\pgfqpoint{5.661837in}{4.102883in}}%
\pgfpathlineto{\pgfqpoint{5.715497in}{4.464192in}}%
\pgfpathlineto{\pgfqpoint{5.769156in}{4.688938in}}%
\pgfpathlineto{\pgfqpoint{5.822815in}{4.080124in}}%
\pgfpathlineto{\pgfqpoint{5.876475in}{4.103830in}}%
\pgfpathlineto{\pgfqpoint{5.930134in}{4.137969in}}%
\pgfpathlineto{\pgfqpoint{5.983793in}{4.613074in}}%
\pgfpathlineto{\pgfqpoint{6.037452in}{4.231852in}}%
\pgfpathlineto{\pgfqpoint{6.091112in}{4.516344in}}%
\pgfpathlineto{\pgfqpoint{6.144771in}{4.283064in}}%
\pgfpathlineto{\pgfqpoint{6.198430in}{4.618761in}}%
\pgfpathlineto{\pgfqpoint{6.252090in}{4.633933in}}%
\pgfpathlineto{\pgfqpoint{6.305749in}{4.509704in}}%
\pgfpathlineto{\pgfqpoint{6.359408in}{4.289698in}}%
\pgfpathlineto{\pgfqpoint{6.413068in}{4.632987in}}%
\pgfpathlineto{\pgfqpoint{6.466727in}{4.960150in}}%
\pgfpathlineto{\pgfqpoint{6.520386in}{4.445219in}}%
\pgfusepath{stroke}%
\end{pgfscope}%
\begin{pgfscope}%
\pgfpathrectangle{\pgfqpoint{0.942500in}{0.645700in}}{\pgfqpoint{5.843500in}{4.519900in}}%
\pgfusepath{clip}%
\pgfsetrectcap%
\pgfsetroundjoin%
\pgfsetlinewidth{1.505625pt}%
\definecolor{currentstroke}{rgb}{0.905882,0.298039,0.235294}%
\pgfsetstrokecolor{currentstroke}%
\pgfsetdash{}{0pt}%
\pgfpathmoveto{\pgfqpoint{1.208114in}{0.889082in}}%
\pgfpathlineto{\pgfqpoint{1.261773in}{0.927014in}}%
\pgfpathlineto{\pgfqpoint{1.315432in}{0.964946in}}%
\pgfpathlineto{\pgfqpoint{1.369092in}{1.002879in}}%
\pgfpathlineto{\pgfqpoint{1.422751in}{1.040811in}}%
\pgfpathlineto{\pgfqpoint{1.476410in}{1.078743in}}%
\pgfpathlineto{\pgfqpoint{1.530070in}{1.116675in}}%
\pgfpathlineto{\pgfqpoint{1.583729in}{1.154607in}}%
\pgfpathlineto{\pgfqpoint{1.637388in}{1.192539in}}%
\pgfpathlineto{\pgfqpoint{1.691048in}{1.230472in}}%
\pgfpathlineto{\pgfqpoint{1.744707in}{1.268404in}}%
\pgfpathlineto{\pgfqpoint{1.798366in}{1.306336in}}%
\pgfpathlineto{\pgfqpoint{1.852025in}{1.344268in}}%
\pgfpathlineto{\pgfqpoint{1.905685in}{1.382200in}}%
\pgfpathlineto{\pgfqpoint{1.959344in}{1.420132in}}%
\pgfpathlineto{\pgfqpoint{2.013003in}{1.458065in}}%
\pgfpathlineto{\pgfqpoint{2.066663in}{1.495997in}}%
\pgfpathlineto{\pgfqpoint{2.120322in}{1.533929in}}%
\pgfpathlineto{\pgfqpoint{2.173981in}{1.571861in}}%
\pgfpathlineto{\pgfqpoint{2.227641in}{1.609793in}}%
\pgfpathlineto{\pgfqpoint{2.281300in}{1.647725in}}%
\pgfpathlineto{\pgfqpoint{2.334959in}{1.685658in}}%
\pgfpathlineto{\pgfqpoint{2.388619in}{1.723590in}}%
\pgfpathlineto{\pgfqpoint{2.442278in}{1.761522in}}%
\pgfpathlineto{\pgfqpoint{2.495937in}{1.799454in}}%
\pgfpathlineto{\pgfqpoint{2.549597in}{1.837386in}}%
\pgfpathlineto{\pgfqpoint{2.603256in}{1.875318in}}%
\pgfpathlineto{\pgfqpoint{2.656915in}{1.913251in}}%
\pgfpathlineto{\pgfqpoint{2.710575in}{1.951183in}}%
\pgfpathlineto{\pgfqpoint{2.764234in}{1.989115in}}%
\pgfpathlineto{\pgfqpoint{2.817893in}{2.027047in}}%
\pgfpathlineto{\pgfqpoint{2.871553in}{2.064979in}}%
\pgfpathlineto{\pgfqpoint{2.925212in}{2.102911in}}%
\pgfpathlineto{\pgfqpoint{2.978871in}{2.140844in}}%
\pgfpathlineto{\pgfqpoint{3.032531in}{2.178776in}}%
\pgfpathlineto{\pgfqpoint{3.086190in}{2.216708in}}%
\pgfpathlineto{\pgfqpoint{3.139849in}{2.254640in}}%
\pgfpathlineto{\pgfqpoint{3.193508in}{2.292572in}}%
\pgfpathlineto{\pgfqpoint{3.247168in}{2.330504in}}%
\pgfpathlineto{\pgfqpoint{3.300827in}{2.368437in}}%
\pgfpathlineto{\pgfqpoint{3.354486in}{2.406369in}}%
\pgfpathlineto{\pgfqpoint{3.408146in}{2.444301in}}%
\pgfpathlineto{\pgfqpoint{3.461805in}{2.482233in}}%
\pgfpathlineto{\pgfqpoint{3.515464in}{2.520165in}}%
\pgfpathlineto{\pgfqpoint{3.569124in}{2.558097in}}%
\pgfpathlineto{\pgfqpoint{3.622783in}{2.596029in}}%
\pgfpathlineto{\pgfqpoint{3.676442in}{2.633962in}}%
\pgfpathlineto{\pgfqpoint{3.730102in}{2.671894in}}%
\pgfpathlineto{\pgfqpoint{3.783761in}{2.709826in}}%
\pgfpathlineto{\pgfqpoint{3.837420in}{2.747758in}}%
\pgfpathlineto{\pgfqpoint{3.891080in}{2.785690in}}%
\pgfpathlineto{\pgfqpoint{3.944739in}{2.823622in}}%
\pgfpathlineto{\pgfqpoint{3.998398in}{2.861555in}}%
\pgfpathlineto{\pgfqpoint{4.052058in}{2.899487in}}%
\pgfpathlineto{\pgfqpoint{4.105717in}{2.937419in}}%
\pgfpathlineto{\pgfqpoint{4.159376in}{2.975351in}}%
\pgfpathlineto{\pgfqpoint{4.213036in}{3.013283in}}%
\pgfpathlineto{\pgfqpoint{4.266695in}{3.051215in}}%
\pgfpathlineto{\pgfqpoint{4.320354in}{3.089148in}}%
\pgfpathlineto{\pgfqpoint{4.374014in}{3.127080in}}%
\pgfpathlineto{\pgfqpoint{4.427673in}{3.165012in}}%
\pgfpathlineto{\pgfqpoint{4.481332in}{3.202944in}}%
\pgfpathlineto{\pgfqpoint{4.534992in}{3.240876in}}%
\pgfpathlineto{\pgfqpoint{4.588651in}{3.278808in}}%
\pgfpathlineto{\pgfqpoint{4.642310in}{3.316741in}}%
\pgfpathlineto{\pgfqpoint{4.695969in}{3.354673in}}%
\pgfpathlineto{\pgfqpoint{4.749629in}{3.392605in}}%
\pgfpathlineto{\pgfqpoint{4.803288in}{3.430537in}}%
\pgfpathlineto{\pgfqpoint{4.856947in}{3.468469in}}%
\pgfpathlineto{\pgfqpoint{4.910607in}{3.506401in}}%
\pgfpathlineto{\pgfqpoint{4.964266in}{3.544334in}}%
\pgfpathlineto{\pgfqpoint{5.017925in}{3.582266in}}%
\pgfpathlineto{\pgfqpoint{5.071585in}{3.620198in}}%
\pgfpathlineto{\pgfqpoint{5.125244in}{3.658130in}}%
\pgfpathlineto{\pgfqpoint{5.178903in}{3.696062in}}%
\pgfpathlineto{\pgfqpoint{5.232563in}{3.733994in}}%
\pgfpathlineto{\pgfqpoint{5.286222in}{3.771927in}}%
\pgfpathlineto{\pgfqpoint{5.339881in}{3.809859in}}%
\pgfpathlineto{\pgfqpoint{5.393541in}{3.847791in}}%
\pgfpathlineto{\pgfqpoint{5.447200in}{3.885723in}}%
\pgfpathlineto{\pgfqpoint{5.500859in}{3.923655in}}%
\pgfpathlineto{\pgfqpoint{5.554519in}{3.961587in}}%
\pgfpathlineto{\pgfqpoint{5.608178in}{3.999519in}}%
\pgfpathlineto{\pgfqpoint{5.661837in}{4.037452in}}%
\pgfpathlineto{\pgfqpoint{5.715497in}{4.075384in}}%
\pgfpathlineto{\pgfqpoint{5.769156in}{4.113316in}}%
\pgfpathlineto{\pgfqpoint{5.822815in}{4.151248in}}%
\pgfpathlineto{\pgfqpoint{5.876475in}{4.189180in}}%
\pgfpathlineto{\pgfqpoint{5.930134in}{4.227112in}}%
\pgfpathlineto{\pgfqpoint{5.983793in}{4.265045in}}%
\pgfpathlineto{\pgfqpoint{6.037452in}{4.302977in}}%
\pgfpathlineto{\pgfqpoint{6.091112in}{4.340909in}}%
\pgfpathlineto{\pgfqpoint{6.144771in}{4.378841in}}%
\pgfpathlineto{\pgfqpoint{6.198430in}{4.416773in}}%
\pgfpathlineto{\pgfqpoint{6.252090in}{4.454705in}}%
\pgfpathlineto{\pgfqpoint{6.305749in}{4.492638in}}%
\pgfpathlineto{\pgfqpoint{6.359408in}{4.530570in}}%
\pgfpathlineto{\pgfqpoint{6.413068in}{4.568502in}}%
\pgfpathlineto{\pgfqpoint{6.466727in}{4.606434in}}%
\pgfpathlineto{\pgfqpoint{6.520386in}{4.644366in}}%
\pgfusepath{stroke}%
\end{pgfscope}%
\begin{pgfscope}%
\pgfpathrectangle{\pgfqpoint{0.942500in}{0.645700in}}{\pgfqpoint{5.843500in}{4.519900in}}%
\pgfusepath{clip}%
\pgfsetrectcap%
\pgfsetroundjoin%
\pgfsetlinewidth{1.505625pt}%
\definecolor{currentstroke}{rgb}{0.086275,0.627451,0.521569}%
\pgfsetstrokecolor{currentstroke}%
\pgfsetdash{}{0pt}%
\pgfpathmoveto{\pgfqpoint{1.208114in}{0.851150in}}%
\pgfpathlineto{\pgfqpoint{1.261773in}{0.851150in}}%
\pgfpathlineto{\pgfqpoint{1.315432in}{0.852098in}}%
\pgfpathlineto{\pgfqpoint{1.369092in}{0.852098in}}%
\pgfpathlineto{\pgfqpoint{1.422751in}{0.852098in}}%
\pgfpathlineto{\pgfqpoint{1.476410in}{0.853995in}}%
\pgfpathlineto{\pgfqpoint{1.530070in}{0.853047in}}%
\pgfpathlineto{\pgfqpoint{1.583729in}{0.852098in}}%
\pgfpathlineto{\pgfqpoint{1.637388in}{0.853995in}}%
\pgfpathlineto{\pgfqpoint{1.691048in}{0.852098in}}%
\pgfpathlineto{\pgfqpoint{1.744707in}{0.852098in}}%
\pgfpathlineto{\pgfqpoint{1.798366in}{0.854943in}}%
\pgfpathlineto{\pgfqpoint{1.852025in}{0.852098in}}%
\pgfpathlineto{\pgfqpoint{1.905685in}{0.852098in}}%
\pgfpathlineto{\pgfqpoint{1.959344in}{0.852098in}}%
\pgfpathlineto{\pgfqpoint{2.013003in}{0.859685in}}%
\pgfpathlineto{\pgfqpoint{2.066663in}{0.853995in}}%
\pgfpathlineto{\pgfqpoint{2.120322in}{0.853995in}}%
\pgfpathlineto{\pgfqpoint{2.173981in}{0.853995in}}%
\pgfpathlineto{\pgfqpoint{2.227641in}{0.851150in}}%
\pgfpathlineto{\pgfqpoint{2.281300in}{0.855892in}}%
\pgfpathlineto{\pgfqpoint{2.334959in}{0.853047in}}%
\pgfpathlineto{\pgfqpoint{2.388619in}{0.855892in}}%
\pgfpathlineto{\pgfqpoint{2.442278in}{0.854943in}}%
\pgfpathlineto{\pgfqpoint{2.495937in}{0.853047in}}%
\pgfpathlineto{\pgfqpoint{2.549597in}{0.852098in}}%
\pgfpathlineto{\pgfqpoint{2.603256in}{0.858736in}}%
\pgfpathlineto{\pgfqpoint{2.656915in}{0.853995in}}%
\pgfpathlineto{\pgfqpoint{2.710575in}{0.853995in}}%
\pgfpathlineto{\pgfqpoint{2.764234in}{0.853995in}}%
\pgfpathlineto{\pgfqpoint{2.817893in}{0.860633in}}%
\pgfpathlineto{\pgfqpoint{2.871553in}{0.856840in}}%
\pgfpathlineto{\pgfqpoint{2.925212in}{0.855892in}}%
\pgfpathlineto{\pgfqpoint{2.978871in}{0.857788in}}%
\pgfpathlineto{\pgfqpoint{3.032531in}{0.853995in}}%
\pgfpathlineto{\pgfqpoint{3.086190in}{0.854943in}}%
\pgfpathlineto{\pgfqpoint{3.139849in}{0.854943in}}%
\pgfpathlineto{\pgfqpoint{3.193508in}{0.854943in}}%
\pgfpathlineto{\pgfqpoint{3.247168in}{0.855892in}}%
\pgfpathlineto{\pgfqpoint{3.300827in}{0.853047in}}%
\pgfpathlineto{\pgfqpoint{3.354486in}{0.854943in}}%
\pgfpathlineto{\pgfqpoint{3.408146in}{0.856840in}}%
\pgfpathlineto{\pgfqpoint{3.461805in}{0.853995in}}%
\pgfpathlineto{\pgfqpoint{3.515464in}{0.857788in}}%
\pgfpathlineto{\pgfqpoint{3.569124in}{0.853047in}}%
\pgfpathlineto{\pgfqpoint{3.622783in}{0.853995in}}%
\pgfpathlineto{\pgfqpoint{3.676442in}{0.853995in}}%
\pgfpathlineto{\pgfqpoint{3.730102in}{0.856840in}}%
\pgfpathlineto{\pgfqpoint{3.783761in}{0.858736in}}%
\pgfpathlineto{\pgfqpoint{3.837420in}{0.853995in}}%
\pgfpathlineto{\pgfqpoint{3.891080in}{0.857788in}}%
\pgfpathlineto{\pgfqpoint{3.944739in}{0.853047in}}%
\pgfpathlineto{\pgfqpoint{3.998398in}{0.852098in}}%
\pgfpathlineto{\pgfqpoint{4.052058in}{0.856840in}}%
\pgfpathlineto{\pgfqpoint{4.105717in}{0.854943in}}%
\pgfpathlineto{\pgfqpoint{4.159376in}{0.853995in}}%
\pgfpathlineto{\pgfqpoint{4.213036in}{0.855892in}}%
\pgfpathlineto{\pgfqpoint{4.266695in}{0.855892in}}%
\pgfpathlineto{\pgfqpoint{4.320354in}{0.854943in}}%
\pgfpathlineto{\pgfqpoint{4.374014in}{0.858736in}}%
\pgfpathlineto{\pgfqpoint{4.427673in}{0.855892in}}%
\pgfpathlineto{\pgfqpoint{4.481332in}{0.857788in}}%
\pgfpathlineto{\pgfqpoint{4.534992in}{0.853995in}}%
\pgfpathlineto{\pgfqpoint{4.588651in}{0.858736in}}%
\pgfpathlineto{\pgfqpoint{4.642310in}{0.855892in}}%
\pgfpathlineto{\pgfqpoint{4.695969in}{0.860633in}}%
\pgfpathlineto{\pgfqpoint{4.749629in}{0.857788in}}%
\pgfpathlineto{\pgfqpoint{4.803288in}{0.856840in}}%
\pgfpathlineto{\pgfqpoint{4.856947in}{0.859685in}}%
\pgfpathlineto{\pgfqpoint{4.910607in}{0.859685in}}%
\pgfpathlineto{\pgfqpoint{4.964266in}{0.857788in}}%
\pgfpathlineto{\pgfqpoint{5.017925in}{0.859685in}}%
\pgfpathlineto{\pgfqpoint{5.071585in}{0.856840in}}%
\pgfpathlineto{\pgfqpoint{5.125244in}{0.859685in}}%
\pgfpathlineto{\pgfqpoint{5.178903in}{0.867271in}}%
\pgfpathlineto{\pgfqpoint{5.232563in}{0.858736in}}%
\pgfpathlineto{\pgfqpoint{5.286222in}{0.854943in}}%
\pgfpathlineto{\pgfqpoint{5.339881in}{0.863478in}}%
\pgfpathlineto{\pgfqpoint{5.393541in}{0.858736in}}%
\pgfpathlineto{\pgfqpoint{5.447200in}{0.861581in}}%
\pgfpathlineto{\pgfqpoint{5.500859in}{0.858736in}}%
\pgfpathlineto{\pgfqpoint{5.554519in}{0.856840in}}%
\pgfpathlineto{\pgfqpoint{5.608178in}{0.858736in}}%
\pgfpathlineto{\pgfqpoint{5.661837in}{0.865375in}}%
\pgfpathlineto{\pgfqpoint{5.715497in}{0.861581in}}%
\pgfpathlineto{\pgfqpoint{5.769156in}{0.864426in}}%
\pgfpathlineto{\pgfqpoint{5.822815in}{0.861581in}}%
\pgfpathlineto{\pgfqpoint{5.876475in}{0.860633in}}%
\pgfpathlineto{\pgfqpoint{5.930134in}{0.859685in}}%
\pgfpathlineto{\pgfqpoint{5.983793in}{0.857788in}}%
\pgfpathlineto{\pgfqpoint{6.037452in}{0.861581in}}%
\pgfpathlineto{\pgfqpoint{6.091112in}{0.857788in}}%
\pgfpathlineto{\pgfqpoint{6.144771in}{0.867271in}}%
\pgfpathlineto{\pgfqpoint{6.198430in}{0.856840in}}%
\pgfpathlineto{\pgfqpoint{6.252090in}{0.855892in}}%
\pgfpathlineto{\pgfqpoint{6.305749in}{0.864426in}}%
\pgfpathlineto{\pgfqpoint{6.359408in}{0.862530in}}%
\pgfpathlineto{\pgfqpoint{6.413068in}{0.857788in}}%
\pgfpathlineto{\pgfqpoint{6.466727in}{0.865375in}}%
\pgfpathlineto{\pgfqpoint{6.520386in}{0.860633in}}%
\pgfusepath{stroke}%
\end{pgfscope}%
\begin{pgfscope}%
\pgfsetrectcap%
\pgfsetmiterjoin%
\pgfsetlinewidth{0.803000pt}%
\definecolor{currentstroke}{rgb}{0.000000,0.000000,0.000000}%
\pgfsetstrokecolor{currentstroke}%
\pgfsetdash{}{0pt}%
\pgfpathmoveto{\pgfqpoint{0.942500in}{0.645700in}}%
\pgfpathlineto{\pgfqpoint{0.942500in}{5.165600in}}%
\pgfusepath{stroke}%
\end{pgfscope}%
\begin{pgfscope}%
\pgfsetrectcap%
\pgfsetmiterjoin%
\pgfsetlinewidth{0.803000pt}%
\definecolor{currentstroke}{rgb}{0.000000,0.000000,0.000000}%
\pgfsetstrokecolor{currentstroke}%
\pgfsetdash{}{0pt}%
\pgfpathmoveto{\pgfqpoint{6.786000in}{0.645700in}}%
\pgfpathlineto{\pgfqpoint{6.786000in}{5.165600in}}%
\pgfusepath{stroke}%
\end{pgfscope}%
\begin{pgfscope}%
\pgfsetrectcap%
\pgfsetmiterjoin%
\pgfsetlinewidth{0.803000pt}%
\definecolor{currentstroke}{rgb}{0.000000,0.000000,0.000000}%
\pgfsetstrokecolor{currentstroke}%
\pgfsetdash{}{0pt}%
\pgfpathmoveto{\pgfqpoint{0.942500in}{0.645700in}}%
\pgfpathlineto{\pgfqpoint{6.786000in}{0.645700in}}%
\pgfusepath{stroke}%
\end{pgfscope}%
\begin{pgfscope}%
\pgfsetrectcap%
\pgfsetmiterjoin%
\pgfsetlinewidth{0.803000pt}%
\definecolor{currentstroke}{rgb}{0.000000,0.000000,0.000000}%
\pgfsetstrokecolor{currentstroke}%
\pgfsetdash{}{0pt}%
\pgfpathmoveto{\pgfqpoint{0.942500in}{5.165600in}}%
\pgfpathlineto{\pgfqpoint{6.786000in}{5.165600in}}%
\pgfusepath{stroke}%
\end{pgfscope}%
\begin{pgfscope}%
\pgfsetbuttcap%
\pgfsetmiterjoin%
\definecolor{currentfill}{rgb}{1.000000,1.000000,1.000000}%
\pgfsetfillcolor{currentfill}%
\pgfsetfillopacity{0.800000}%
\pgfsetlinewidth{1.003750pt}%
\definecolor{currentstroke}{rgb}{0.800000,0.800000,0.800000}%
\pgfsetstrokecolor{currentstroke}%
\pgfsetstrokeopacity{0.800000}%
\pgfsetdash{}{0pt}%
\pgfpathmoveto{\pgfqpoint{1.039722in}{4.442917in}}%
\pgfpathlineto{\pgfqpoint{2.973289in}{4.442917in}}%
\pgfpathquadraticcurveto{\pgfqpoint{3.001067in}{4.442917in}}{\pgfqpoint{3.001067in}{4.470695in}}%
\pgfpathlineto{\pgfqpoint{3.001067in}{5.068378in}}%
\pgfpathquadraticcurveto{\pgfqpoint{3.001067in}{5.096156in}}{\pgfqpoint{2.973289in}{5.096156in}}%
\pgfpathlineto{\pgfqpoint{1.039722in}{5.096156in}}%
\pgfpathquadraticcurveto{\pgfqpoint{1.011944in}{5.096156in}}{\pgfqpoint{1.011944in}{5.068378in}}%
\pgfpathlineto{\pgfqpoint{1.011944in}{4.470695in}}%
\pgfpathquadraticcurveto{\pgfqpoint{1.011944in}{4.442917in}}{\pgfqpoint{1.039722in}{4.442917in}}%
\pgfpathclose%
\pgfusepath{stroke,fill}%
\end{pgfscope}%
\begin{pgfscope}%
\pgfsetrectcap%
\pgfsetroundjoin%
\pgfsetlinewidth{1.505625pt}%
\definecolor{currentstroke}{rgb}{0.172549,0.243137,0.313725}%
\pgfsetstrokecolor{currentstroke}%
\pgfsetdash{}{0pt}%
\pgfpathmoveto{\pgfqpoint{1.067500in}{4.983688in}}%
\pgfpathlineto{\pgfqpoint{1.345278in}{4.983688in}}%
\pgfusepath{stroke}%
\end{pgfscope}%
\begin{pgfscope}%
\pgftext[x=1.456389in,y=4.935077in,left,base]{\sffamily\fontsize{10.000000}{12.000000}\selectfont \textsc{Linclosure}}%
\end{pgfscope}%
\begin{pgfscope}%
\pgfsetrectcap%
\pgfsetroundjoin%
\pgfsetlinewidth{1.505625pt}%
\definecolor{currentstroke}{rgb}{0.905882,0.298039,0.235294}%
\pgfsetstrokecolor{currentstroke}%
\pgfsetdash{}{0pt}%
\pgfpathmoveto{\pgfqpoint{1.067500in}{4.779831in}}%
\pgfpathlineto{\pgfqpoint{1.345278in}{4.779831in}}%
\pgfusepath{stroke}%
\end{pgfscope}%
\begin{pgfscope}%
\pgftext[x=1.456389in,y=4.731220in,left,base]{\sffamily\fontsize{10.000000}{12.000000}\selectfont Linear Approximation}%
\end{pgfscope}%
\begin{pgfscope}%
\pgfsetrectcap%
\pgfsetroundjoin%
\pgfsetlinewidth{1.505625pt}%
\definecolor{currentstroke}{rgb}{0.086275,0.627451,0.521569}%
\pgfsetstrokecolor{currentstroke}%
\pgfsetdash{}{0pt}%
\pgfpathmoveto{\pgfqpoint{1.067500in}{4.575974in}}%
\pgfpathlineto{\pgfqpoint{1.345278in}{4.575974in}}%
\pgfusepath{stroke}%
\end{pgfscope}%
\begin{pgfscope}%
\pgftext[x=1.456389in,y=4.527363in,left,base]{\sffamily\fontsize{10.000000}{12.000000}\selectfont \textsc{Closure}}%
\end{pgfscope}%
\end{pgfpicture}%
\makeatother%
\endgroup%
}
}

\caption{Comparison of closure operators for \textsc{MinCover}}
\label{fig:MinCover-Prune}
\end{figure}

\vspace{1.2em}

The top left figure represents the time spent in \textsc{MinCover} when $|\Sg|$ is fixed to 100, and $|\B|$ goes from 1000 to 20000 by steps of 1000. The top right one, is $|\B| = 100$ and $|\Sg|$ running from 1000 to 100000 by step of 2000. In both cases We associated an array (right under) with some particular values, being useful to see evolution of \textsc{Closure}, flattened by \textsc{LinClosure}. The reason for this limitation is time spent in test relatively to the information it brings. As one can see, there is a huge difference of speed between \textsc{MinCoverLin} and \textsc{MinCoverClo} even if they differ only in the first loop. For each test case, we iterate the algorithm over 100 randomly generated basis so as to explore somehow the space of implication theories. Hence, we see for instance that when the number of implications $|\B|$ reaches 20000, with $|\Sg| = 100$, \textsc{MinCover} with \textsc{LinClosure} needs roughly 225 seconds to run on average for one execution. Assume we would like to go up to $|\B| = 100000$. Then, we would need at least $225 \times 100 \times 80 = 1800000s$ being approximately 3 weeks of test. Of course, we could change the range function to reduce the number of step, but still in our case we want to compare efficiency of \textsc{Closure} against \textsc{LinClosure}. According to the results, we admitted this range was sufficient to see the gap between those two closure operator. Interestingly, \textsc{LinClosure} permits to see practical complexity of \textsc{MinCover}: while the evolution of $|\B|$ draws a somehow quadratic curve of time, $|\Sg|$ matches a linear growth in spite of noisy points. This matches the theoretical complexity $O(|\B||\I|)$ required by \textsc{MinCover}. One could argue that the algorithms we tested both used \textsc{Closure} in the
second loop, altering the overhaul complexity. Indeed, but as we said, \textsc{LinClosure} performs much worse than \textsc{Closure} so that the practical complexity is driven by \textsc{LinClosure}. For \textsc{MinCover}
we can conclude that \textsc{Closure} is a better choice.

\subsection{\textsc{DuquenneMinimization}}

\subsection{\textsc{MaierMinimization}}

\begin{figure}[ht]
	
\begin{minipage}{0.55\textwidth}
\subfloat[Graphical representation of average time (in $s$)]{
	\scalebox{0.55}{%% Creator: Matplotlib, PGF backend
%%
%% To include the figure in your LaTeX document, write
%%   \input{<filename>.pgf}
%%
%% Make sure the required packages are loaded in your preamble
%%   \usepackage{pgf}
%%
%% Figures using additional raster images can only be included by \input if
%% they are in the same directory as the main LaTeX file. For loading figures
%% from other directories you can use the `import` package
%%   \usepackage{import}
%% and then include the figures with
%%   \import{<path to file>}{<filename>.pgf}
%%
%% Matplotlib used the following preamble
%%   \usepackage{fontspec}
%%   \setmainfont{DejaVu Serif}
%%   \setsansfont{DejaVu Sans}
%%   \setmonofont{DejaVu Sans Mono}
%%
\begingroup%
\makeatletter%
\begin{pgfpicture}%
\pgfpathrectangle{\pgfpointorigin}{\pgfqpoint{6.820000in}{5.320000in}}%
\pgfusepath{use as bounding box, clip}%
\begin{pgfscope}%
\pgfsetbuttcap%
\pgfsetmiterjoin%
\definecolor{currentfill}{rgb}{1.000000,1.000000,1.000000}%
\pgfsetfillcolor{currentfill}%
\pgfsetlinewidth{0.000000pt}%
\definecolor{currentstroke}{rgb}{1.000000,1.000000,1.000000}%
\pgfsetstrokecolor{currentstroke}%
\pgfsetdash{}{0pt}%
\pgfpathmoveto{\pgfqpoint{0.000000in}{0.000000in}}%
\pgfpathlineto{\pgfqpoint{6.820000in}{0.000000in}}%
\pgfpathlineto{\pgfqpoint{6.820000in}{5.320000in}}%
\pgfpathlineto{\pgfqpoint{0.000000in}{5.320000in}}%
\pgfpathclose%
\pgfusepath{fill}%
\end{pgfscope}%
\begin{pgfscope}%
\pgfsetbuttcap%
\pgfsetmiterjoin%
\definecolor{currentfill}{rgb}{1.000000,1.000000,1.000000}%
\pgfsetfillcolor{currentfill}%
\pgfsetlinewidth{0.000000pt}%
\definecolor{currentstroke}{rgb}{0.000000,0.000000,0.000000}%
\pgfsetstrokecolor{currentstroke}%
\pgfsetstrokeopacity{0.000000}%
\pgfsetdash{}{0pt}%
\pgfpathmoveto{\pgfqpoint{0.852500in}{0.585200in}}%
\pgfpathlineto{\pgfqpoint{6.138000in}{0.585200in}}%
\pgfpathlineto{\pgfqpoint{6.138000in}{4.681600in}}%
\pgfpathlineto{\pgfqpoint{0.852500in}{4.681600in}}%
\pgfpathclose%
\pgfusepath{fill}%
\end{pgfscope}%
\begin{pgfscope}%
\pgfsetbuttcap%
\pgfsetroundjoin%
\definecolor{currentfill}{rgb}{0.000000,0.000000,0.000000}%
\pgfsetfillcolor{currentfill}%
\pgfsetlinewidth{0.803000pt}%
\definecolor{currentstroke}{rgb}{0.000000,0.000000,0.000000}%
\pgfsetstrokecolor{currentstroke}%
\pgfsetdash{}{0pt}%
\pgfsys@defobject{currentmarker}{\pgfqpoint{0.000000in}{-0.048611in}}{\pgfqpoint{0.000000in}{0.000000in}}{%
\pgfpathmoveto{\pgfqpoint{0.000000in}{0.000000in}}%
\pgfpathlineto{\pgfqpoint{0.000000in}{-0.048611in}}%
\pgfusepath{stroke,fill}%
}%
\begin{pgfscope}%
\pgfsys@transformshift{1.044215in}{0.585200in}%
\pgfsys@useobject{currentmarker}{}%
\end{pgfscope}%
\end{pgfscope}%
\begin{pgfscope}%
\pgftext[x=1.044215in,y=0.487978in,,top]{\sffamily\fontsize{10.000000}{12.000000}\selectfont \(\displaystyle 0\)}%
\end{pgfscope}%
\begin{pgfscope}%
\pgfsetbuttcap%
\pgfsetroundjoin%
\definecolor{currentfill}{rgb}{0.000000,0.000000,0.000000}%
\pgfsetfillcolor{currentfill}%
\pgfsetlinewidth{0.803000pt}%
\definecolor{currentstroke}{rgb}{0.000000,0.000000,0.000000}%
\pgfsetstrokecolor{currentstroke}%
\pgfsetdash{}{0pt}%
\pgfsys@defobject{currentmarker}{\pgfqpoint{0.000000in}{-0.048611in}}{\pgfqpoint{0.000000in}{0.000000in}}{%
\pgfpathmoveto{\pgfqpoint{0.000000in}{0.000000in}}%
\pgfpathlineto{\pgfqpoint{0.000000in}{-0.048611in}}%
\pgfusepath{stroke,fill}%
}%
\begin{pgfscope}%
\pgfsys@transformshift{1.529568in}{0.585200in}%
\pgfsys@useobject{currentmarker}{}%
\end{pgfscope}%
\end{pgfscope}%
\begin{pgfscope}%
\pgftext[x=1.529568in,y=0.487978in,,top]{\sffamily\fontsize{10.000000}{12.000000}\selectfont \(\displaystyle 10000\)}%
\end{pgfscope}%
\begin{pgfscope}%
\pgfsetbuttcap%
\pgfsetroundjoin%
\definecolor{currentfill}{rgb}{0.000000,0.000000,0.000000}%
\pgfsetfillcolor{currentfill}%
\pgfsetlinewidth{0.803000pt}%
\definecolor{currentstroke}{rgb}{0.000000,0.000000,0.000000}%
\pgfsetstrokecolor{currentstroke}%
\pgfsetdash{}{0pt}%
\pgfsys@defobject{currentmarker}{\pgfqpoint{0.000000in}{-0.048611in}}{\pgfqpoint{0.000000in}{0.000000in}}{%
\pgfpathmoveto{\pgfqpoint{0.000000in}{0.000000in}}%
\pgfpathlineto{\pgfqpoint{0.000000in}{-0.048611in}}%
\pgfusepath{stroke,fill}%
}%
\begin{pgfscope}%
\pgfsys@transformshift{2.014922in}{0.585200in}%
\pgfsys@useobject{currentmarker}{}%
\end{pgfscope}%
\end{pgfscope}%
\begin{pgfscope}%
\pgftext[x=2.014922in,y=0.487978in,,top]{\sffamily\fontsize{10.000000}{12.000000}\selectfont \(\displaystyle 20000\)}%
\end{pgfscope}%
\begin{pgfscope}%
\pgfsetbuttcap%
\pgfsetroundjoin%
\definecolor{currentfill}{rgb}{0.000000,0.000000,0.000000}%
\pgfsetfillcolor{currentfill}%
\pgfsetlinewidth{0.803000pt}%
\definecolor{currentstroke}{rgb}{0.000000,0.000000,0.000000}%
\pgfsetstrokecolor{currentstroke}%
\pgfsetdash{}{0pt}%
\pgfsys@defobject{currentmarker}{\pgfqpoint{0.000000in}{-0.048611in}}{\pgfqpoint{0.000000in}{0.000000in}}{%
\pgfpathmoveto{\pgfqpoint{0.000000in}{0.000000in}}%
\pgfpathlineto{\pgfqpoint{0.000000in}{-0.048611in}}%
\pgfusepath{stroke,fill}%
}%
\begin{pgfscope}%
\pgfsys@transformshift{2.500275in}{0.585200in}%
\pgfsys@useobject{currentmarker}{}%
\end{pgfscope}%
\end{pgfscope}%
\begin{pgfscope}%
\pgftext[x=2.500275in,y=0.487978in,,top]{\sffamily\fontsize{10.000000}{12.000000}\selectfont \(\displaystyle 30000\)}%
\end{pgfscope}%
\begin{pgfscope}%
\pgfsetbuttcap%
\pgfsetroundjoin%
\definecolor{currentfill}{rgb}{0.000000,0.000000,0.000000}%
\pgfsetfillcolor{currentfill}%
\pgfsetlinewidth{0.803000pt}%
\definecolor{currentstroke}{rgb}{0.000000,0.000000,0.000000}%
\pgfsetstrokecolor{currentstroke}%
\pgfsetdash{}{0pt}%
\pgfsys@defobject{currentmarker}{\pgfqpoint{0.000000in}{-0.048611in}}{\pgfqpoint{0.000000in}{0.000000in}}{%
\pgfpathmoveto{\pgfqpoint{0.000000in}{0.000000in}}%
\pgfpathlineto{\pgfqpoint{0.000000in}{-0.048611in}}%
\pgfusepath{stroke,fill}%
}%
\begin{pgfscope}%
\pgfsys@transformshift{2.985629in}{0.585200in}%
\pgfsys@useobject{currentmarker}{}%
\end{pgfscope}%
\end{pgfscope}%
\begin{pgfscope}%
\pgftext[x=2.985629in,y=0.487978in,,top]{\sffamily\fontsize{10.000000}{12.000000}\selectfont \(\displaystyle 40000\)}%
\end{pgfscope}%
\begin{pgfscope}%
\pgfsetbuttcap%
\pgfsetroundjoin%
\definecolor{currentfill}{rgb}{0.000000,0.000000,0.000000}%
\pgfsetfillcolor{currentfill}%
\pgfsetlinewidth{0.803000pt}%
\definecolor{currentstroke}{rgb}{0.000000,0.000000,0.000000}%
\pgfsetstrokecolor{currentstroke}%
\pgfsetdash{}{0pt}%
\pgfsys@defobject{currentmarker}{\pgfqpoint{0.000000in}{-0.048611in}}{\pgfqpoint{0.000000in}{0.000000in}}{%
\pgfpathmoveto{\pgfqpoint{0.000000in}{0.000000in}}%
\pgfpathlineto{\pgfqpoint{0.000000in}{-0.048611in}}%
\pgfusepath{stroke,fill}%
}%
\begin{pgfscope}%
\pgfsys@transformshift{3.470982in}{0.585200in}%
\pgfsys@useobject{currentmarker}{}%
\end{pgfscope}%
\end{pgfscope}%
\begin{pgfscope}%
\pgftext[x=3.470982in,y=0.487978in,,top]{\sffamily\fontsize{10.000000}{12.000000}\selectfont \(\displaystyle 50000\)}%
\end{pgfscope}%
\begin{pgfscope}%
\pgfsetbuttcap%
\pgfsetroundjoin%
\definecolor{currentfill}{rgb}{0.000000,0.000000,0.000000}%
\pgfsetfillcolor{currentfill}%
\pgfsetlinewidth{0.803000pt}%
\definecolor{currentstroke}{rgb}{0.000000,0.000000,0.000000}%
\pgfsetstrokecolor{currentstroke}%
\pgfsetdash{}{0pt}%
\pgfsys@defobject{currentmarker}{\pgfqpoint{0.000000in}{-0.048611in}}{\pgfqpoint{0.000000in}{0.000000in}}{%
\pgfpathmoveto{\pgfqpoint{0.000000in}{0.000000in}}%
\pgfpathlineto{\pgfqpoint{0.000000in}{-0.048611in}}%
\pgfusepath{stroke,fill}%
}%
\begin{pgfscope}%
\pgfsys@transformshift{3.956336in}{0.585200in}%
\pgfsys@useobject{currentmarker}{}%
\end{pgfscope}%
\end{pgfscope}%
\begin{pgfscope}%
\pgftext[x=3.956336in,y=0.487978in,,top]{\sffamily\fontsize{10.000000}{12.000000}\selectfont \(\displaystyle 60000\)}%
\end{pgfscope}%
\begin{pgfscope}%
\pgfsetbuttcap%
\pgfsetroundjoin%
\definecolor{currentfill}{rgb}{0.000000,0.000000,0.000000}%
\pgfsetfillcolor{currentfill}%
\pgfsetlinewidth{0.803000pt}%
\definecolor{currentstroke}{rgb}{0.000000,0.000000,0.000000}%
\pgfsetstrokecolor{currentstroke}%
\pgfsetdash{}{0pt}%
\pgfsys@defobject{currentmarker}{\pgfqpoint{0.000000in}{-0.048611in}}{\pgfqpoint{0.000000in}{0.000000in}}{%
\pgfpathmoveto{\pgfqpoint{0.000000in}{0.000000in}}%
\pgfpathlineto{\pgfqpoint{0.000000in}{-0.048611in}}%
\pgfusepath{stroke,fill}%
}%
\begin{pgfscope}%
\pgfsys@transformshift{4.441689in}{0.585200in}%
\pgfsys@useobject{currentmarker}{}%
\end{pgfscope}%
\end{pgfscope}%
\begin{pgfscope}%
\pgftext[x=4.441689in,y=0.487978in,,top]{\sffamily\fontsize{10.000000}{12.000000}\selectfont \(\displaystyle 70000\)}%
\end{pgfscope}%
\begin{pgfscope}%
\pgfsetbuttcap%
\pgfsetroundjoin%
\definecolor{currentfill}{rgb}{0.000000,0.000000,0.000000}%
\pgfsetfillcolor{currentfill}%
\pgfsetlinewidth{0.803000pt}%
\definecolor{currentstroke}{rgb}{0.000000,0.000000,0.000000}%
\pgfsetstrokecolor{currentstroke}%
\pgfsetdash{}{0pt}%
\pgfsys@defobject{currentmarker}{\pgfqpoint{0.000000in}{-0.048611in}}{\pgfqpoint{0.000000in}{0.000000in}}{%
\pgfpathmoveto{\pgfqpoint{0.000000in}{0.000000in}}%
\pgfpathlineto{\pgfqpoint{0.000000in}{-0.048611in}}%
\pgfusepath{stroke,fill}%
}%
\begin{pgfscope}%
\pgfsys@transformshift{4.927043in}{0.585200in}%
\pgfsys@useobject{currentmarker}{}%
\end{pgfscope}%
\end{pgfscope}%
\begin{pgfscope}%
\pgftext[x=4.927043in,y=0.487978in,,top]{\sffamily\fontsize{10.000000}{12.000000}\selectfont \(\displaystyle 80000\)}%
\end{pgfscope}%
\begin{pgfscope}%
\pgfsetbuttcap%
\pgfsetroundjoin%
\definecolor{currentfill}{rgb}{0.000000,0.000000,0.000000}%
\pgfsetfillcolor{currentfill}%
\pgfsetlinewidth{0.803000pt}%
\definecolor{currentstroke}{rgb}{0.000000,0.000000,0.000000}%
\pgfsetstrokecolor{currentstroke}%
\pgfsetdash{}{0pt}%
\pgfsys@defobject{currentmarker}{\pgfqpoint{0.000000in}{-0.048611in}}{\pgfqpoint{0.000000in}{0.000000in}}{%
\pgfpathmoveto{\pgfqpoint{0.000000in}{0.000000in}}%
\pgfpathlineto{\pgfqpoint{0.000000in}{-0.048611in}}%
\pgfusepath{stroke,fill}%
}%
\begin{pgfscope}%
\pgfsys@transformshift{5.412396in}{0.585200in}%
\pgfsys@useobject{currentmarker}{}%
\end{pgfscope}%
\end{pgfscope}%
\begin{pgfscope}%
\pgftext[x=5.412396in,y=0.487978in,,top]{\sffamily\fontsize{10.000000}{12.000000}\selectfont \(\displaystyle 90000\)}%
\end{pgfscope}%
\begin{pgfscope}%
\pgfsetbuttcap%
\pgfsetroundjoin%
\definecolor{currentfill}{rgb}{0.000000,0.000000,0.000000}%
\pgfsetfillcolor{currentfill}%
\pgfsetlinewidth{0.803000pt}%
\definecolor{currentstroke}{rgb}{0.000000,0.000000,0.000000}%
\pgfsetstrokecolor{currentstroke}%
\pgfsetdash{}{0pt}%
\pgfsys@defobject{currentmarker}{\pgfqpoint{0.000000in}{-0.048611in}}{\pgfqpoint{0.000000in}{0.000000in}}{%
\pgfpathmoveto{\pgfqpoint{0.000000in}{0.000000in}}%
\pgfpathlineto{\pgfqpoint{0.000000in}{-0.048611in}}%
\pgfusepath{stroke,fill}%
}%
\begin{pgfscope}%
\pgfsys@transformshift{5.897750in}{0.585200in}%
\pgfsys@useobject{currentmarker}{}%
\end{pgfscope}%
\end{pgfscope}%
\begin{pgfscope}%
\pgftext[x=5.897750in,y=0.487978in,,top]{\sffamily\fontsize{10.000000}{12.000000}\selectfont \(\displaystyle 100000\)}%
\end{pgfscope}%
\begin{pgfscope}%
\pgftext[x=3.495250in,y=0.298009in,,top]{\sffamily\fontsize{14.000000}{16.800000}\selectfont \(\displaystyle |\Sigma|\)}%
\end{pgfscope}%
\begin{pgfscope}%
\pgfsetbuttcap%
\pgfsetroundjoin%
\definecolor{currentfill}{rgb}{0.000000,0.000000,0.000000}%
\pgfsetfillcolor{currentfill}%
\pgfsetlinewidth{0.803000pt}%
\definecolor{currentstroke}{rgb}{0.000000,0.000000,0.000000}%
\pgfsetstrokecolor{currentstroke}%
\pgfsetdash{}{0pt}%
\pgfsys@defobject{currentmarker}{\pgfqpoint{-0.048611in}{0.000000in}}{\pgfqpoint{0.000000in}{0.000000in}}{%
\pgfpathmoveto{\pgfqpoint{0.000000in}{0.000000in}}%
\pgfpathlineto{\pgfqpoint{-0.048611in}{0.000000in}}%
\pgfusepath{stroke,fill}%
}%
\begin{pgfscope}%
\pgfsys@transformshift{0.852500in}{0.766739in}%
\pgfsys@useobject{currentmarker}{}%
\end{pgfscope}%
\end{pgfscope}%
\begin{pgfscope}%
\pgftext[x=0.577808in,y=0.713978in,left,base]{\sffamily\fontsize{10.000000}{12.000000}\selectfont \(\displaystyle 0.0\)}%
\end{pgfscope}%
\begin{pgfscope}%
\pgfsetbuttcap%
\pgfsetroundjoin%
\definecolor{currentfill}{rgb}{0.000000,0.000000,0.000000}%
\pgfsetfillcolor{currentfill}%
\pgfsetlinewidth{0.803000pt}%
\definecolor{currentstroke}{rgb}{0.000000,0.000000,0.000000}%
\pgfsetstrokecolor{currentstroke}%
\pgfsetdash{}{0pt}%
\pgfsys@defobject{currentmarker}{\pgfqpoint{-0.048611in}{0.000000in}}{\pgfqpoint{0.000000in}{0.000000in}}{%
\pgfpathmoveto{\pgfqpoint{0.000000in}{0.000000in}}%
\pgfpathlineto{\pgfqpoint{-0.048611in}{0.000000in}}%
\pgfusepath{stroke,fill}%
}%
\begin{pgfscope}%
\pgfsys@transformshift{0.852500in}{1.192872in}%
\pgfsys@useobject{currentmarker}{}%
\end{pgfscope}%
\end{pgfscope}%
\begin{pgfscope}%
\pgftext[x=0.577808in,y=1.140110in,left,base]{\sffamily\fontsize{10.000000}{12.000000}\selectfont \(\displaystyle 0.4\)}%
\end{pgfscope}%
\begin{pgfscope}%
\pgfsetbuttcap%
\pgfsetroundjoin%
\definecolor{currentfill}{rgb}{0.000000,0.000000,0.000000}%
\pgfsetfillcolor{currentfill}%
\pgfsetlinewidth{0.803000pt}%
\definecolor{currentstroke}{rgb}{0.000000,0.000000,0.000000}%
\pgfsetstrokecolor{currentstroke}%
\pgfsetdash{}{0pt}%
\pgfsys@defobject{currentmarker}{\pgfqpoint{-0.048611in}{0.000000in}}{\pgfqpoint{0.000000in}{0.000000in}}{%
\pgfpathmoveto{\pgfqpoint{0.000000in}{0.000000in}}%
\pgfpathlineto{\pgfqpoint{-0.048611in}{0.000000in}}%
\pgfusepath{stroke,fill}%
}%
\begin{pgfscope}%
\pgfsys@transformshift{0.852500in}{1.619005in}%
\pgfsys@useobject{currentmarker}{}%
\end{pgfscope}%
\end{pgfscope}%
\begin{pgfscope}%
\pgftext[x=0.577808in,y=1.566243in,left,base]{\sffamily\fontsize{10.000000}{12.000000}\selectfont \(\displaystyle 0.8\)}%
\end{pgfscope}%
\begin{pgfscope}%
\pgfsetbuttcap%
\pgfsetroundjoin%
\definecolor{currentfill}{rgb}{0.000000,0.000000,0.000000}%
\pgfsetfillcolor{currentfill}%
\pgfsetlinewidth{0.803000pt}%
\definecolor{currentstroke}{rgb}{0.000000,0.000000,0.000000}%
\pgfsetstrokecolor{currentstroke}%
\pgfsetdash{}{0pt}%
\pgfsys@defobject{currentmarker}{\pgfqpoint{-0.048611in}{0.000000in}}{\pgfqpoint{0.000000in}{0.000000in}}{%
\pgfpathmoveto{\pgfqpoint{0.000000in}{0.000000in}}%
\pgfpathlineto{\pgfqpoint{-0.048611in}{0.000000in}}%
\pgfusepath{stroke,fill}%
}%
\begin{pgfscope}%
\pgfsys@transformshift{0.852500in}{2.045137in}%
\pgfsys@useobject{currentmarker}{}%
\end{pgfscope}%
\end{pgfscope}%
\begin{pgfscope}%
\pgftext[x=0.577808in,y=1.992376in,left,base]{\sffamily\fontsize{10.000000}{12.000000}\selectfont \(\displaystyle 1.2\)}%
\end{pgfscope}%
\begin{pgfscope}%
\pgfsetbuttcap%
\pgfsetroundjoin%
\definecolor{currentfill}{rgb}{0.000000,0.000000,0.000000}%
\pgfsetfillcolor{currentfill}%
\pgfsetlinewidth{0.803000pt}%
\definecolor{currentstroke}{rgb}{0.000000,0.000000,0.000000}%
\pgfsetstrokecolor{currentstroke}%
\pgfsetdash{}{0pt}%
\pgfsys@defobject{currentmarker}{\pgfqpoint{-0.048611in}{0.000000in}}{\pgfqpoint{0.000000in}{0.000000in}}{%
\pgfpathmoveto{\pgfqpoint{0.000000in}{0.000000in}}%
\pgfpathlineto{\pgfqpoint{-0.048611in}{0.000000in}}%
\pgfusepath{stroke,fill}%
}%
\begin{pgfscope}%
\pgfsys@transformshift{0.852500in}{2.471270in}%
\pgfsys@useobject{currentmarker}{}%
\end{pgfscope}%
\end{pgfscope}%
\begin{pgfscope}%
\pgftext[x=0.577808in,y=2.418508in,left,base]{\sffamily\fontsize{10.000000}{12.000000}\selectfont \(\displaystyle 1.6\)}%
\end{pgfscope}%
\begin{pgfscope}%
\pgfsetbuttcap%
\pgfsetroundjoin%
\definecolor{currentfill}{rgb}{0.000000,0.000000,0.000000}%
\pgfsetfillcolor{currentfill}%
\pgfsetlinewidth{0.803000pt}%
\definecolor{currentstroke}{rgb}{0.000000,0.000000,0.000000}%
\pgfsetstrokecolor{currentstroke}%
\pgfsetdash{}{0pt}%
\pgfsys@defobject{currentmarker}{\pgfqpoint{-0.048611in}{0.000000in}}{\pgfqpoint{0.000000in}{0.000000in}}{%
\pgfpathmoveto{\pgfqpoint{0.000000in}{0.000000in}}%
\pgfpathlineto{\pgfqpoint{-0.048611in}{0.000000in}}%
\pgfusepath{stroke,fill}%
}%
\begin{pgfscope}%
\pgfsys@transformshift{0.852500in}{2.897403in}%
\pgfsys@useobject{currentmarker}{}%
\end{pgfscope}%
\end{pgfscope}%
\begin{pgfscope}%
\pgftext[x=0.577808in,y=2.844641in,left,base]{\sffamily\fontsize{10.000000}{12.000000}\selectfont \(\displaystyle 2.0\)}%
\end{pgfscope}%
\begin{pgfscope}%
\pgfsetbuttcap%
\pgfsetroundjoin%
\definecolor{currentfill}{rgb}{0.000000,0.000000,0.000000}%
\pgfsetfillcolor{currentfill}%
\pgfsetlinewidth{0.803000pt}%
\definecolor{currentstroke}{rgb}{0.000000,0.000000,0.000000}%
\pgfsetstrokecolor{currentstroke}%
\pgfsetdash{}{0pt}%
\pgfsys@defobject{currentmarker}{\pgfqpoint{-0.048611in}{0.000000in}}{\pgfqpoint{0.000000in}{0.000000in}}{%
\pgfpathmoveto{\pgfqpoint{0.000000in}{0.000000in}}%
\pgfpathlineto{\pgfqpoint{-0.048611in}{0.000000in}}%
\pgfusepath{stroke,fill}%
}%
\begin{pgfscope}%
\pgfsys@transformshift{0.852500in}{3.323535in}%
\pgfsys@useobject{currentmarker}{}%
\end{pgfscope}%
\end{pgfscope}%
\begin{pgfscope}%
\pgftext[x=0.577808in,y=3.270774in,left,base]{\sffamily\fontsize{10.000000}{12.000000}\selectfont \(\displaystyle 2.4\)}%
\end{pgfscope}%
\begin{pgfscope}%
\pgfsetbuttcap%
\pgfsetroundjoin%
\definecolor{currentfill}{rgb}{0.000000,0.000000,0.000000}%
\pgfsetfillcolor{currentfill}%
\pgfsetlinewidth{0.803000pt}%
\definecolor{currentstroke}{rgb}{0.000000,0.000000,0.000000}%
\pgfsetstrokecolor{currentstroke}%
\pgfsetdash{}{0pt}%
\pgfsys@defobject{currentmarker}{\pgfqpoint{-0.048611in}{0.000000in}}{\pgfqpoint{0.000000in}{0.000000in}}{%
\pgfpathmoveto{\pgfqpoint{0.000000in}{0.000000in}}%
\pgfpathlineto{\pgfqpoint{-0.048611in}{0.000000in}}%
\pgfusepath{stroke,fill}%
}%
\begin{pgfscope}%
\pgfsys@transformshift{0.852500in}{3.749668in}%
\pgfsys@useobject{currentmarker}{}%
\end{pgfscope}%
\end{pgfscope}%
\begin{pgfscope}%
\pgftext[x=0.577808in,y=3.696906in,left,base]{\sffamily\fontsize{10.000000}{12.000000}\selectfont \(\displaystyle 2.8\)}%
\end{pgfscope}%
\begin{pgfscope}%
\pgfsetbuttcap%
\pgfsetroundjoin%
\definecolor{currentfill}{rgb}{0.000000,0.000000,0.000000}%
\pgfsetfillcolor{currentfill}%
\pgfsetlinewidth{0.803000pt}%
\definecolor{currentstroke}{rgb}{0.000000,0.000000,0.000000}%
\pgfsetstrokecolor{currentstroke}%
\pgfsetdash{}{0pt}%
\pgfsys@defobject{currentmarker}{\pgfqpoint{-0.048611in}{0.000000in}}{\pgfqpoint{0.000000in}{0.000000in}}{%
\pgfpathmoveto{\pgfqpoint{0.000000in}{0.000000in}}%
\pgfpathlineto{\pgfqpoint{-0.048611in}{0.000000in}}%
\pgfusepath{stroke,fill}%
}%
\begin{pgfscope}%
\pgfsys@transformshift{0.852500in}{4.175801in}%
\pgfsys@useobject{currentmarker}{}%
\end{pgfscope}%
\end{pgfscope}%
\begin{pgfscope}%
\pgftext[x=0.577808in,y=4.123039in,left,base]{\sffamily\fontsize{10.000000}{12.000000}\selectfont \(\displaystyle 3.2\)}%
\end{pgfscope}%
\begin{pgfscope}%
\pgftext[x=0.522252in,y=2.633400in,,bottom,rotate=90.000000]{\sffamily\fontsize{14.000000}{16.800000}\selectfont seconds}%
\end{pgfscope}%
\begin{pgfscope}%
\pgfpathrectangle{\pgfqpoint{0.852500in}{0.585200in}}{\pgfqpoint{5.285500in}{4.096400in}}%
\pgfusepath{clip}%
\pgfsetrectcap%
\pgfsetroundjoin%
\pgfsetlinewidth{1.505625pt}%
\definecolor{currentstroke}{rgb}{0.172549,0.243137,0.313725}%
\pgfsetstrokecolor{currentstroke}%
\pgfsetdash{}{0pt}%
\pgfpathmoveto{\pgfqpoint{1.092750in}{0.869111in}}%
\pgfpathlineto{\pgfqpoint{1.189821in}{0.908894in}}%
\pgfpathlineto{\pgfqpoint{1.286891in}{0.954837in}}%
\pgfpathlineto{\pgfqpoint{1.383962in}{1.007438in}}%
\pgfpathlineto{\pgfqpoint{1.481033in}{1.050384in}}%
\pgfpathlineto{\pgfqpoint{1.578104in}{1.194204in}}%
\pgfpathlineto{\pgfqpoint{1.675174in}{1.164407in}}%
\pgfpathlineto{\pgfqpoint{1.772245in}{1.350174in}}%
\pgfpathlineto{\pgfqpoint{1.869316in}{1.348011in}}%
\pgfpathlineto{\pgfqpoint{1.966386in}{1.402941in}}%
\pgfpathlineto{\pgfqpoint{2.063457in}{1.516466in}}%
\pgfpathlineto{\pgfqpoint{2.160528in}{1.599030in}}%
\pgfpathlineto{\pgfqpoint{2.257598in}{1.699736in}}%
\pgfpathlineto{\pgfqpoint{2.354669in}{1.835736in}}%
\pgfpathlineto{\pgfqpoint{2.451740in}{1.747344in}}%
\pgfpathlineto{\pgfqpoint{2.548811in}{1.975560in}}%
\pgfpathlineto{\pgfqpoint{2.645881in}{1.925788in}}%
\pgfpathlineto{\pgfqpoint{2.742952in}{1.877678in}}%
\pgfpathlineto{\pgfqpoint{2.840023in}{2.161993in}}%
\pgfpathlineto{\pgfqpoint{2.937093in}{2.348586in}}%
\pgfpathlineto{\pgfqpoint{3.034164in}{2.025663in}}%
\pgfpathlineto{\pgfqpoint{3.131235in}{2.312141in}}%
\pgfpathlineto{\pgfqpoint{3.228306in}{2.479260in}}%
\pgfpathlineto{\pgfqpoint{3.325376in}{2.485918in}}%
\pgfpathlineto{\pgfqpoint{3.422447in}{2.632732in}}%
\pgfpathlineto{\pgfqpoint{3.519518in}{2.832982in}}%
\pgfpathlineto{\pgfqpoint{3.616588in}{2.699986in}}%
\pgfpathlineto{\pgfqpoint{3.713659in}{2.930694in}}%
\pgfpathlineto{\pgfqpoint{3.810730in}{2.894569in}}%
\pgfpathlineto{\pgfqpoint{3.907801in}{2.882253in}}%
\pgfpathlineto{\pgfqpoint{4.004871in}{3.019916in}}%
\pgfpathlineto{\pgfqpoint{4.101942in}{3.047710in}}%
\pgfpathlineto{\pgfqpoint{4.199013in}{3.060867in}}%
\pgfpathlineto{\pgfqpoint{4.296083in}{3.296731in}}%
\pgfpathlineto{\pgfqpoint{4.393154in}{3.245468in}}%
\pgfpathlineto{\pgfqpoint{4.490225in}{3.546925in}}%
\pgfpathlineto{\pgfqpoint{4.587295in}{3.552411in}}%
\pgfpathlineto{\pgfqpoint{4.684366in}{3.346173in}}%
\pgfpathlineto{\pgfqpoint{4.781437in}{3.497152in}}%
\pgfpathlineto{\pgfqpoint{4.878508in}{3.587205in}}%
\pgfpathlineto{\pgfqpoint{4.975578in}{3.958409in}}%
\pgfpathlineto{\pgfqpoint{5.072649in}{3.784621in}}%
\pgfpathlineto{\pgfqpoint{5.169720in}{4.248712in}}%
\pgfpathlineto{\pgfqpoint{5.266790in}{3.966559in}}%
\pgfpathlineto{\pgfqpoint{5.363861in}{4.038469in}}%
\pgfpathlineto{\pgfqpoint{5.460932in}{4.361403in}}%
\pgfpathlineto{\pgfqpoint{5.558003in}{4.354574in}}%
\pgfpathlineto{\pgfqpoint{5.655073in}{4.345923in}}%
\pgfpathlineto{\pgfqpoint{5.752144in}{4.224571in}}%
\pgfpathlineto{\pgfqpoint{5.849215in}{4.394193in}}%
\pgfpathlineto{\pgfqpoint{5.897750in}{4.210253in}}%
\pgfusepath{stroke}%
\end{pgfscope}%
\begin{pgfscope}%
\pgfpathrectangle{\pgfqpoint{0.852500in}{0.585200in}}{\pgfqpoint{5.285500in}{4.096400in}}%
\pgfusepath{clip}%
\pgfsetrectcap%
\pgfsetroundjoin%
\pgfsetlinewidth{1.505625pt}%
\definecolor{currentstroke}{rgb}{0.905882,0.298039,0.235294}%
\pgfsetstrokecolor{currentstroke}%
\pgfsetdash{}{0pt}%
\pgfpathmoveto{\pgfqpoint{1.092750in}{0.804026in}}%
\pgfpathlineto{\pgfqpoint{1.189821in}{0.878599in}}%
\pgfpathlineto{\pgfqpoint{1.286891in}{0.953172in}}%
\pgfpathlineto{\pgfqpoint{1.383962in}{1.027745in}}%
\pgfpathlineto{\pgfqpoint{1.481033in}{1.102319in}}%
\pgfpathlineto{\pgfqpoint{1.578104in}{1.176892in}}%
\pgfpathlineto{\pgfqpoint{1.675174in}{1.251465in}}%
\pgfpathlineto{\pgfqpoint{1.772245in}{1.326038in}}%
\pgfpathlineto{\pgfqpoint{1.869316in}{1.400612in}}%
\pgfpathlineto{\pgfqpoint{1.966386in}{1.475185in}}%
\pgfpathlineto{\pgfqpoint{2.063457in}{1.549758in}}%
\pgfpathlineto{\pgfqpoint{2.160528in}{1.624331in}}%
\pgfpathlineto{\pgfqpoint{2.257598in}{1.698904in}}%
\pgfpathlineto{\pgfqpoint{2.354669in}{1.773478in}}%
\pgfpathlineto{\pgfqpoint{2.451740in}{1.848051in}}%
\pgfpathlineto{\pgfqpoint{2.548811in}{1.922624in}}%
\pgfpathlineto{\pgfqpoint{2.645881in}{1.997197in}}%
\pgfpathlineto{\pgfqpoint{2.742952in}{2.071770in}}%
\pgfpathlineto{\pgfqpoint{2.840023in}{2.146344in}}%
\pgfpathlineto{\pgfqpoint{2.937093in}{2.220917in}}%
\pgfpathlineto{\pgfqpoint{3.034164in}{2.295490in}}%
\pgfpathlineto{\pgfqpoint{3.131235in}{2.370063in}}%
\pgfpathlineto{\pgfqpoint{3.228306in}{2.444637in}}%
\pgfpathlineto{\pgfqpoint{3.325376in}{2.519210in}}%
\pgfpathlineto{\pgfqpoint{3.422447in}{2.593783in}}%
\pgfpathlineto{\pgfqpoint{3.519518in}{2.668356in}}%
\pgfpathlineto{\pgfqpoint{3.616588in}{2.742929in}}%
\pgfpathlineto{\pgfqpoint{3.713659in}{2.817503in}}%
\pgfpathlineto{\pgfqpoint{3.810730in}{2.892076in}}%
\pgfpathlineto{\pgfqpoint{3.907801in}{2.966649in}}%
\pgfpathlineto{\pgfqpoint{4.004871in}{3.041222in}}%
\pgfpathlineto{\pgfqpoint{4.101942in}{3.115795in}}%
\pgfpathlineto{\pgfqpoint{4.199013in}{3.190369in}}%
\pgfpathlineto{\pgfqpoint{4.296083in}{3.264942in}}%
\pgfpathlineto{\pgfqpoint{4.393154in}{3.339515in}}%
\pgfpathlineto{\pgfqpoint{4.490225in}{3.414088in}}%
\pgfpathlineto{\pgfqpoint{4.587295in}{3.488662in}}%
\pgfpathlineto{\pgfqpoint{4.684366in}{3.563235in}}%
\pgfpathlineto{\pgfqpoint{4.781437in}{3.637808in}}%
\pgfpathlineto{\pgfqpoint{4.878508in}{3.712381in}}%
\pgfpathlineto{\pgfqpoint{4.975578in}{3.786954in}}%
\pgfpathlineto{\pgfqpoint{5.072649in}{3.861528in}}%
\pgfpathlineto{\pgfqpoint{5.169720in}{3.936101in}}%
\pgfpathlineto{\pgfqpoint{5.266790in}{4.010674in}}%
\pgfpathlineto{\pgfqpoint{5.363861in}{4.085247in}}%
\pgfpathlineto{\pgfqpoint{5.460932in}{4.159821in}}%
\pgfpathlineto{\pgfqpoint{5.558003in}{4.234394in}}%
\pgfpathlineto{\pgfqpoint{5.655073in}{4.308967in}}%
\pgfpathlineto{\pgfqpoint{5.752144in}{4.383540in}}%
\pgfpathlineto{\pgfqpoint{5.849215in}{4.458113in}}%
\pgfpathlineto{\pgfqpoint{5.897750in}{4.495400in}}%
\pgfusepath{stroke}%
\end{pgfscope}%
\begin{pgfscope}%
\pgfpathrectangle{\pgfqpoint{0.852500in}{0.585200in}}{\pgfqpoint{5.285500in}{4.096400in}}%
\pgfusepath{clip}%
\pgfsetrectcap%
\pgfsetroundjoin%
\pgfsetlinewidth{1.505625pt}%
\definecolor{currentstroke}{rgb}{0.086275,0.627451,0.521569}%
\pgfsetstrokecolor{currentstroke}%
\pgfsetdash{}{0pt}%
\pgfpathmoveto{\pgfqpoint{1.092750in}{0.771400in}}%
\pgfpathlineto{\pgfqpoint{1.189821in}{0.772732in}}%
\pgfpathlineto{\pgfqpoint{1.286891in}{0.773397in}}%
\pgfpathlineto{\pgfqpoint{1.383962in}{0.772066in}}%
\pgfpathlineto{\pgfqpoint{1.481033in}{0.773897in}}%
\pgfpathlineto{\pgfqpoint{1.578104in}{0.774896in}}%
\pgfpathlineto{\pgfqpoint{1.675174in}{0.772399in}}%
\pgfpathlineto{\pgfqpoint{1.772245in}{0.774563in}}%
\pgfpathlineto{\pgfqpoint{1.869316in}{0.775062in}}%
\pgfpathlineto{\pgfqpoint{1.966386in}{0.775561in}}%
\pgfpathlineto{\pgfqpoint{2.063457in}{0.774396in}}%
\pgfpathlineto{\pgfqpoint{2.160528in}{0.775062in}}%
\pgfpathlineto{\pgfqpoint{2.257598in}{0.776227in}}%
\pgfpathlineto{\pgfqpoint{2.354669in}{0.775728in}}%
\pgfpathlineto{\pgfqpoint{2.451740in}{0.777725in}}%
\pgfpathlineto{\pgfqpoint{2.548811in}{0.777559in}}%
\pgfpathlineto{\pgfqpoint{2.645881in}{0.776394in}}%
\pgfpathlineto{\pgfqpoint{2.742952in}{0.776061in}}%
\pgfpathlineto{\pgfqpoint{2.840023in}{0.776893in}}%
\pgfpathlineto{\pgfqpoint{2.937093in}{0.776893in}}%
\pgfpathlineto{\pgfqpoint{3.034164in}{0.776061in}}%
\pgfpathlineto{\pgfqpoint{3.131235in}{0.777892in}}%
\pgfpathlineto{\pgfqpoint{3.228306in}{0.777559in}}%
\pgfpathlineto{\pgfqpoint{3.325376in}{0.777725in}}%
\pgfpathlineto{\pgfqpoint{3.422447in}{0.778058in}}%
\pgfpathlineto{\pgfqpoint{3.519518in}{0.778225in}}%
\pgfpathlineto{\pgfqpoint{3.616588in}{0.779723in}}%
\pgfpathlineto{\pgfqpoint{3.713659in}{0.778558in}}%
\pgfpathlineto{\pgfqpoint{3.810730in}{0.778391in}}%
\pgfpathlineto{\pgfqpoint{3.907801in}{0.781387in}}%
\pgfpathlineto{\pgfqpoint{4.004871in}{0.780056in}}%
\pgfpathlineto{\pgfqpoint{4.101942in}{0.780056in}}%
\pgfpathlineto{\pgfqpoint{4.199013in}{0.780722in}}%
\pgfpathlineto{\pgfqpoint{4.296083in}{0.780888in}}%
\pgfpathlineto{\pgfqpoint{4.393154in}{0.780555in}}%
\pgfpathlineto{\pgfqpoint{4.490225in}{0.783551in}}%
\pgfpathlineto{\pgfqpoint{4.587295in}{0.782053in}}%
\pgfpathlineto{\pgfqpoint{4.684366in}{0.779723in}}%
\pgfpathlineto{\pgfqpoint{4.781437in}{0.782386in}}%
\pgfpathlineto{\pgfqpoint{4.878508in}{0.781554in}}%
\pgfpathlineto{\pgfqpoint{4.975578in}{0.783052in}}%
\pgfpathlineto{\pgfqpoint{5.072649in}{0.782886in}}%
\pgfpathlineto{\pgfqpoint{5.169720in}{0.783385in}}%
\pgfpathlineto{\pgfqpoint{5.266790in}{0.782886in}}%
\pgfpathlineto{\pgfqpoint{5.363861in}{0.782220in}}%
\pgfpathlineto{\pgfqpoint{5.460932in}{0.784717in}}%
\pgfpathlineto{\pgfqpoint{5.558003in}{0.785382in}}%
\pgfpathlineto{\pgfqpoint{5.655073in}{0.784717in}}%
\pgfpathlineto{\pgfqpoint{5.752144in}{0.782886in}}%
\pgfpathlineto{\pgfqpoint{5.849215in}{0.785216in}}%
\pgfpathlineto{\pgfqpoint{5.897750in}{0.784051in}}%
\pgfusepath{stroke}%
\end{pgfscope}%
\begin{pgfscope}%
\pgfsetrectcap%
\pgfsetmiterjoin%
\pgfsetlinewidth{0.803000pt}%
\definecolor{currentstroke}{rgb}{0.000000,0.000000,0.000000}%
\pgfsetstrokecolor{currentstroke}%
\pgfsetdash{}{0pt}%
\pgfpathmoveto{\pgfqpoint{0.852500in}{0.585200in}}%
\pgfpathlineto{\pgfqpoint{0.852500in}{4.681600in}}%
\pgfusepath{stroke}%
\end{pgfscope}%
\begin{pgfscope}%
\pgfsetrectcap%
\pgfsetmiterjoin%
\pgfsetlinewidth{0.803000pt}%
\definecolor{currentstroke}{rgb}{0.000000,0.000000,0.000000}%
\pgfsetstrokecolor{currentstroke}%
\pgfsetdash{}{0pt}%
\pgfpathmoveto{\pgfqpoint{6.138000in}{0.585200in}}%
\pgfpathlineto{\pgfqpoint{6.138000in}{4.681600in}}%
\pgfusepath{stroke}%
\end{pgfscope}%
\begin{pgfscope}%
\pgfsetrectcap%
\pgfsetmiterjoin%
\pgfsetlinewidth{0.803000pt}%
\definecolor{currentstroke}{rgb}{0.000000,0.000000,0.000000}%
\pgfsetstrokecolor{currentstroke}%
\pgfsetdash{}{0pt}%
\pgfpathmoveto{\pgfqpoint{0.852500in}{0.585200in}}%
\pgfpathlineto{\pgfqpoint{6.138000in}{0.585200in}}%
\pgfusepath{stroke}%
\end{pgfscope}%
\begin{pgfscope}%
\pgfsetrectcap%
\pgfsetmiterjoin%
\pgfsetlinewidth{0.803000pt}%
\definecolor{currentstroke}{rgb}{0.000000,0.000000,0.000000}%
\pgfsetstrokecolor{currentstroke}%
\pgfsetdash{}{0pt}%
\pgfpathmoveto{\pgfqpoint{0.852500in}{4.681600in}}%
\pgfpathlineto{\pgfqpoint{6.138000in}{4.681600in}}%
\pgfusepath{stroke}%
\end{pgfscope}%
\begin{pgfscope}%
\pgfsetbuttcap%
\pgfsetmiterjoin%
\definecolor{currentfill}{rgb}{1.000000,1.000000,1.000000}%
\pgfsetfillcolor{currentfill}%
\pgfsetfillopacity{0.800000}%
\pgfsetlinewidth{1.003750pt}%
\definecolor{currentstroke}{rgb}{0.800000,0.800000,0.800000}%
\pgfsetstrokecolor{currentstroke}%
\pgfsetstrokeopacity{0.800000}%
\pgfsetdash{}{0pt}%
\pgfpathmoveto{\pgfqpoint{0.949722in}{3.958917in}}%
\pgfpathlineto{\pgfqpoint{2.883289in}{3.958917in}}%
\pgfpathquadraticcurveto{\pgfqpoint{2.911067in}{3.958917in}}{\pgfqpoint{2.911067in}{3.986695in}}%
\pgfpathlineto{\pgfqpoint{2.911067in}{4.584378in}}%
\pgfpathquadraticcurveto{\pgfqpoint{2.911067in}{4.612156in}}{\pgfqpoint{2.883289in}{4.612156in}}%
\pgfpathlineto{\pgfqpoint{0.949722in}{4.612156in}}%
\pgfpathquadraticcurveto{\pgfqpoint{0.921944in}{4.612156in}}{\pgfqpoint{0.921944in}{4.584378in}}%
\pgfpathlineto{\pgfqpoint{0.921944in}{3.986695in}}%
\pgfpathquadraticcurveto{\pgfqpoint{0.921944in}{3.958917in}}{\pgfqpoint{0.949722in}{3.958917in}}%
\pgfpathclose%
\pgfusepath{stroke,fill}%
\end{pgfscope}%
\begin{pgfscope}%
\pgfsetrectcap%
\pgfsetroundjoin%
\pgfsetlinewidth{1.505625pt}%
\definecolor{currentstroke}{rgb}{0.172549,0.243137,0.313725}%
\pgfsetstrokecolor{currentstroke}%
\pgfsetdash{}{0pt}%
\pgfpathmoveto{\pgfqpoint{0.977500in}{4.499688in}}%
\pgfpathlineto{\pgfqpoint{1.255278in}{4.499688in}}%
\pgfusepath{stroke}%
\end{pgfscope}%
\begin{pgfscope}%
\pgftext[x=1.366389in,y=4.451077in,left,base]{\sffamily\fontsize{10.000000}{12.000000}\selectfont \textsc{Linclosure}}%
\end{pgfscope}%
\begin{pgfscope}%
\pgfsetrectcap%
\pgfsetroundjoin%
\pgfsetlinewidth{1.505625pt}%
\definecolor{currentstroke}{rgb}{0.905882,0.298039,0.235294}%
\pgfsetstrokecolor{currentstroke}%
\pgfsetdash{}{0pt}%
\pgfpathmoveto{\pgfqpoint{0.977500in}{4.295831in}}%
\pgfpathlineto{\pgfqpoint{1.255278in}{4.295831in}}%
\pgfusepath{stroke}%
\end{pgfscope}%
\begin{pgfscope}%
\pgftext[x=1.366389in,y=4.247220in,left,base]{\sffamily\fontsize{10.000000}{12.000000}\selectfont Linear Approximation}%
\end{pgfscope}%
\begin{pgfscope}%
\pgfsetrectcap%
\pgfsetroundjoin%
\pgfsetlinewidth{1.505625pt}%
\definecolor{currentstroke}{rgb}{0.086275,0.627451,0.521569}%
\pgfsetstrokecolor{currentstroke}%
\pgfsetdash{}{0pt}%
\pgfpathmoveto{\pgfqpoint{0.977500in}{4.091974in}}%
\pgfpathlineto{\pgfqpoint{1.255278in}{4.091974in}}%
\pgfusepath{stroke}%
\end{pgfscope}%
\begin{pgfscope}%
\pgftext[x=1.366389in,y=4.043362in,left,base]{\sffamily\fontsize{10.000000}{12.000000}\selectfont \textsc{Closure}}%
\end{pgfscope}%
\end{pgfpicture}%
\makeatother%
\endgroup%
}
}
\end{minipage}
~
\begin{minipage}{0.4\textwidth}
\subfloat[Some landmarks times]{
	\begin{tabular}{| c || c |}
		\hline \rowcolor{clouds}
		$|\Sigma|$ & \textsc{Closure} \\ \hline
		1000   &  0.004 \\ \hline
		5000   &  0.006 \\ \hline
		25000  &  0.009 \\ \hline
		45000  &  0.010 \\ \hline 
		65000  &  0.013 \\ \hline 
		85000  &  0.016 \\ \hline
		100000 &  0.016 \\ \hline
	\end{tabular}
}
\end{minipage}

\caption{Average execution time of \textsc{MaierMinimization} when $|\B| = 500$}
\label{fig:Maier-BFixed}
\end{figure}

\subsection{\textsc{BercziMinimization}}

The algorithm issued by Berczi and al in \cite{berczi_directed_2017} (algoritm \ref{alg:Berczi-min}) can be fine-tuned at first sight without considering closure operators. Indeed, recall that we compute only closure of premises under
$\I$ being our input basis, and $\I_c$ our output one. Furthermore, the algorithm suggests to compute the closure of some premises under $\I$ several 
times, which is extensive and redundant. Furthermore, because we build $\I_c$ only by adding implications, all the closure previously computed can only grow.
Therefore we can improve \textsc{BercziMinimization} by the next means:
\begin{itemize}
	\item[-] compute all $\I(A), A \in \B(\I)$, and store them in a list: $C_{\I}$,
	\item[-] keep a list of growing $\I_c(A), A \in \B(\I)$: $C_{\I_c}$.
\end{itemize}
\noindent To illustrate, we can observe the pseudo-code \textsc{BercziImp}. This algorithm does not strictly reflect its implementation of course, but it
presents the two ideas we were talking about previously. Observe that the closures under $\I$ are the most complicated to compute (because $|\B(\I_c)| \leq |\B(\I)|$ even though we repeatedly add implications to $\I_c$) whence
the interest of avoiding useless computations especially for $\I$.

\begin{algorithm}
	\KwIn{$\I$: an implication theory}
	\KwOut{$\I_c$: the DQ-basis of $\I$}
	
	\BlankLine
	\BlankLine
	
	$\I_c := \emptyset$ \;
	$C_{\I} := \emptyset, \; C_{\I_c} := \emptyset $ \;
	
	\ForEach{$A \imp B \in \I$}{
		$C_{\I}[A] = \I(A)$ \;
		$C_{\I_c}[A] = A$ \;
	}

	\BlankLine
	
	\While{$\exists A \in \B(\I)$ s.t $C_{\I_c}[A] \neq C_{\I}[A]$}{
		\ForEach{$A \imp B \in \I$}{
			\If{$C_{\I}[A] \neq C_{\I_c}[A]$}{
				$C_{\I_c}[A] = \I_c(C_{\I_c}[A])$ \;
			}
		}
	
		\BlankLine
	
		$A_P, P := \min\{A, C_{\I_c}[A]: \;  C_{\I_c}[A] \neq C_{\I}[A] \}$ \;
		$\I_c := \I_c \cup \{P \imp C_{\I}[A_P] \}$ \
		
	}

	\BlankLine
	
	return $\I_c$ \;
	
	\caption{\textsc{BercziImp}}
	\label{alg:Berczi-imp}
\end{algorithm}

When getting the minimum, note that we get both the minimum $\I_c$-closed set not closed in $\I$ and its associated premise (or equivalently, the associated $\I$-closure). Because notations may be a bit heavy, let us illustrate the meaning of $C_{\I}$ and $C_{\I_c}$ in an example.


\paragraph{Example} As usual, let us retake our small example:
\begin{itemize}
	\item $\Sg = \{a, \ b, \ c, \ d, \ e, \ f \}$,
	\item $\I =$ \{\textit{ab $\imp$ cde, cd $\imp$ f, c $\imp$ a, d $\imp$ b, 
		abcd $\imp$ ef} \} 
\end{itemize}
\noindent Table \ref{tab:berczi-tra} is a trace of the algorithm, using $C_{\I}$ and $C_{\I_c}$. The first column contains premises of $\I$, the second one elements of $C_{\I}$ (that is closures of premises of $\I$ under $\I$) and
the third one, $C_{\I_c}$ (closures of $\B(\I)$ under $\I_c$). The last column
says whether $C_{\I}[A] = C_{\I_c}[A]$.

\begin{table}[ht]
\centering
\subfloat[after initialization step and first loop]{
	\begin{tabular}{| c | c | c | c |}
		\hline \rowcolor{clouds}
		$\B(\I)$ & $C_{\I}$ & $C_{\I_c}$ & $=$ \\ \hline
		$ab$     & $abcdef$ & $ab$ & $\times$ \\ \hline
		$cd$     & $abcdef$ & $cd$ & $\times$  \\ \hline \rowcolor{emerald!40!white}
		$c$      & $ca$     & $c$ & $\times$ \\ \hline 
		$d$      & $db$     & $d$ & $ \times$ \\ \hline
		$abcd$   & $abcdef$ & $abcd$ & $\times$ \\ \hline
	\end{tabular}
	\begin{tabular}{c}
		$\I_c$ before \\ 
		$\emptyset$   \\ \\
		$\I_c$ after \\
		$c \imp ca$
	\end{tabular}	
}\quad
\subfloat[second loop]{
	\begin{tabular}{| c | c | c | c |}
		\hline \rowcolor{clouds}
		$\B(\I)$ & $C_{\I}$ & $C_{\I_c}$ & $=$ \\ \hline
		$ab$     & $abcdef$ & $ab$ & $\times$ \\ \hline
		$cd$     & $abcdef$ & $acd$ & $\times$ \\ \hline
		$c$      & $ca$     & $ca$ & $\lor$ \\ \hline \rowcolor{emerald!40!white}
		$d$      & $db$     & $d$ & $\times$ \\ \hline
		$abcd$   & $abcdef$ & $abcd$ & $\times$ \\ \hline
	\end{tabular}
	\begin{tabular}{c}
		$\I_c$ before \\ 
		$c \imp ca$   \\ \\
		$\I_c$ after \\
		$c \imp ca, \ d \imp bd$
	\end{tabular}	
}

\subfloat[third loop]{
	\begin{tabular}{| c | c | c | c |}
		\hline \rowcolor{clouds}
		$\B(\I)$ & $C_{\I}$ & $C_{\I_c}$ & $=$ \\ \hline \rowcolor{emerald!40!white}
		$ab$     & $abcdef$ & $ab$  & $\times$ \\ \hline
		$cd$     & $abcdef$ & $abcd$ & $\times$ \\ \hline
		$c$      & $ca$     & $ca$ & $\lor$ \\ \hline 
		$d$      & $db$     & $db$  & $\lor$ \\ \hline
		$abcd$   & $abcdef$ & $abcd$ & $\times$\\ \hline
	\end{tabular}
	\begin{tabular}{c}
		$\I_c$ before \\ 
		$c \imp ca, \ d \imp bd$   \\ \\
		$\I_c$ after \\
		$c \imp ca, \ d \imp bd, \ ab \imp abcdef$
	\end{tabular}	
}

\subfloat[final loop]{
	\begin{tabular}{| c | c | c | c |}
		\hline \rowcolor{clouds}
		$\B(\I)$ & $C_{\I}$ & $C_{\I_c}$ & $=$ \\ \hline 
		$ab$     & $abcdef$ & $abcdef$ & $\lor$ \\ \hline 
		$cd$     & $abcdef$ & $abcdef$ & $\lor$ \\ \hline 
		$c$      & $ca$     & $ca$ & $\lor$ \\ \hline 
		$d$      & $db$     & $db$ & $\lor$ \\ \hline
		$abcd$   & $abcdef$ & $abcdef$ & $\lor$ \\ \hline
	\end{tabular}
	\begin{tabular}{c}
		Resulting $\I_c$  \\ 
		$c \imp ca, \ d \imp bd, \ ab \imp abcdef$
	\end{tabular}	
}
\caption{Example of execution of \textsc{BercziImp} using $C_{\I}$ and $C_{\I_c}$}
\label{tab:berczi-tra}
\end{table}

At each step, the highlighted row is the set of premises and closure satisfying $\min \{ C_{\I_c}[A]: \ C_{\I_c}[A] \neq C_{\I}[A] \}$, or this row contains the
minimal (inclusion-wise) $\I_c$-closed set not closed in $\I$. As wished, the
only some closures of $\I_c$ are updated at each step, instead of re-computing all of them every time.

\vspace{1.2em}

 More than fine-tuning the principle, we can optimize the use of \textsc{LinClosure} on two aspects:
\begin{itemize}
	\item[(i)] in the loop where we initialize lists, we can use the same
	pruning as in the right-closing step in \textsc{MinCover},
	\item[(ii)] when computing closures of $\I_c$, because $\I_c$ is only
	increasing in implications, updating list and counters can be done in $O(|\Sg|)$,	hence better than the initialization step of $O(|\I|)$.
\end{itemize}
\noindent Consequently, we will again compare for this algorithm two versions, given as \textsc{BercziImpClo} and \textsc{BercziImpLin} (\ref{alg:BercziImpClo}, \ref{alg:BercziImpLin}).

\vspace{1.2em}

\begin{algorithm}[H]
	\TitleOfAlgo{\textsc{BercziImpClo}}
	\KwIn{$\I$: an implication theory}
	\KwOut{$\I_c$: the DQ-basis of $\I$}
	
	\BlankLine
	\BlankLine
	
	$\I_c := \emptyset$ \;
	$C_{\I} := \emptyset, \; C_{\I_c} := \emptyset $ \;
	
	\ForEach{$A \imp B \in \I$}{
		$C_{\I}[A] = \textsc{Closure}(\I, A)$ \;
		$C_{\I_c}[A] = A$ \;
	}
	
	\BlankLine
	
	\While{$\exists A \in \B(\I)$ s.t $C_{\I_c}[A] \neq C_{\I}[A]$}{
		\ForEach{$A \imp B \in \I$}{
			\If{$C_{\I}[A] \neq C_{\I_c}[A]$}{
				$C_{\I_c}[A] = \textsc{Closure}(\I_c, C_{\I_c}[A])$ \;
			}
		}
		
		\BlankLine
		
		$A_P, P := \min\{A, C_{\I_c}[A]: \;  C_{\I_c}[A] \neq C_{\I}[A] \}$ \;
		$\I_c := \I_c \cup \{P \imp C_{\I}[A_P] \}$ \;
		
	}
	
	\BlankLine
	
	return $\I_c$ \;
	

	\label{alg:BercziImpClo}
\end{algorithm}

\begin{algorithm}[H]
	\TitleOfAlgo{\textsc{BercziImpLin}}
	\KwIn{$\I$: an implication theory}
	\KwOut{$\I_c$: the DQ-basis of $\I$}
	
	\BlankLine
	\BlankLine
	
	$\I_c := \emptyset$ \;
	$C_{\I} := \emptyset, \; C_{\I_c} := \emptyset $ \;
	
	\textsc{LinClosureInit}($\I$) \;
	
	\ForEach{$A \imp B \in \I$}{
		$C_{\I}[A] = \textsc{LinClosure}(\I, A)$ \;
		$C_{\I_c}[A] = A$ \;
	}
	
	\BlankLine
	
	\While{$\exists A \in \B(\I)$ s.t $C_{\I_c}[A] \neq C_{\I}[A]$}{
		\ForEach{$A \imp B \in \I$}{
			\If{$C_{\I}[A] \neq C_{\I_c}[A]$}{
				$C_{\I_c}[A] = \textsc{LinClosure}(\I_c, C_{\I_c}[A])$ \;
			}
		}
		
		\BlankLine
		
		$A_P, P := \min\{A, C_{\I_c}[A]: \; C_{\I_c}[A] \neq C_{\I}[A] \}$ \;
		$\I_c := \I_c \cup \{P \imp C_{\I}[A_P] \}$ \;
		\textsc{LinClosureAddImp}($\I_c, P \imp C_{\I}[A_P]$)\;
		
	}
	
	\BlankLine
	
	return $\I_c$ \;
	
	\label{alg:BercziImpLin}
\end{algorithm}

\vspace{1.2em}

In fact, \textsc{BercziImpClo} does differ from \textsc{BercziImp} only by the closure notations. However, when dealing with \textsc{LinClosure} in \textsc{BercziImpLin}, we use first \textsc{LinClosureInit} to instanciate 
\textit{list} and \textit{count} for \textsc{LinClosure} in $\I$. Hence a call
to this procedure performs only the effective closure computations. Then, 
\textsc{LinClosureAddImp} just updates \textit{list} and \textit{count} for $\I_c$ by adding a new implication in them. Note that before the first call to
this subroutine, \textit{list} and \textit{count} for $I_c$ are empty, hence
\textsc{LinClosure}($\I_c$, $X$) returns $X$.

\vspace{1.2em}

We may observe tests of \textsc{BercziMinimization} (or more precisely, \textsc{BercziImp}) in graphs and tables of figure \ref{fig:Berczi-Prune}. As in previous cases we represent two range-based test on one parameter, when the other is fixed. On the left part we have a graph and a table of tests with fixed $|\Sg| = 100$. On the other side, we fixed $|\B|$ to $100$ also (and not 1000 as in previous cases!). We lowered the boundary $|\B|$ because of the complexity of the algorithm. As one can observe if we recall previous tests, even if $|\B| = 100$ the results are of the same order as previous tests where $|\B| = 1000$. Or course we should consider this quick comparison as no more but a quick glance to previous results since tests are not ran on the same basis (because of randomness). In any event, in spite of the noisy behaviour when $|\Sg|$ tends to $100000$, we can observe a linear growth of the execution time in accordance with the theoretical complexity of \textsc{BercziMinimization}.


\begin{figure}[ht]
	
\begin{minipage}{0.5\textwidth}
\subfloat[Average time (in $s$), $|\Sg| = 100$]{
	\scalebox{0.46}{%% Creator: Matplotlib, PGF backend
%%
%% To include the figure in your LaTeX document, write
%%   \input{<filename>.pgf}
%%
%% Make sure the required packages are loaded in your preamble
%%   \usepackage{pgf}
%%
%% Figures using additional raster images can only be included by \input if
%% they are in the same directory as the main LaTeX file. For loading figures
%% from other directories you can use the `import` package
%%   \usepackage{import}
%% and then include the figures with
%%   \import{<path to file>}{<filename>.pgf}
%%
%% Matplotlib used the following preamble
%%   \usepackage{fontspec}
%%   \setmainfont{DejaVu Serif}
%%   \setsansfont{DejaVu Sans}
%%   \setmonofont{DejaVu Sans Mono}
%%
\begingroup%
\makeatletter%
\begin{pgfpicture}%
\pgfpathrectangle{\pgfpointorigin}{\pgfqpoint{7.670000in}{6.260000in}}%
\pgfusepath{use as bounding box, clip}%
\begin{pgfscope}%
\pgfsetbuttcap%
\pgfsetmiterjoin%
\definecolor{currentfill}{rgb}{1.000000,1.000000,1.000000}%
\pgfsetfillcolor{currentfill}%
\pgfsetlinewidth{0.000000pt}%
\definecolor{currentstroke}{rgb}{1.000000,1.000000,1.000000}%
\pgfsetstrokecolor{currentstroke}%
\pgfsetdash{}{0pt}%
\pgfpathmoveto{\pgfqpoint{0.000000in}{0.000000in}}%
\pgfpathlineto{\pgfqpoint{7.670000in}{0.000000in}}%
\pgfpathlineto{\pgfqpoint{7.670000in}{6.260000in}}%
\pgfpathlineto{\pgfqpoint{0.000000in}{6.260000in}}%
\pgfpathclose%
\pgfusepath{fill}%
\end{pgfscope}%
\begin{pgfscope}%
\pgfsetbuttcap%
\pgfsetmiterjoin%
\definecolor{currentfill}{rgb}{1.000000,1.000000,1.000000}%
\pgfsetfillcolor{currentfill}%
\pgfsetlinewidth{0.000000pt}%
\definecolor{currentstroke}{rgb}{0.000000,0.000000,0.000000}%
\pgfsetstrokecolor{currentstroke}%
\pgfsetstrokeopacity{0.000000}%
\pgfsetdash{}{0pt}%
\pgfpathmoveto{\pgfqpoint{0.958750in}{0.688600in}}%
\pgfpathlineto{\pgfqpoint{6.903000in}{0.688600in}}%
\pgfpathlineto{\pgfqpoint{6.903000in}{5.508800in}}%
\pgfpathlineto{\pgfqpoint{0.958750in}{5.508800in}}%
\pgfpathclose%
\pgfusepath{fill}%
\end{pgfscope}%
\begin{pgfscope}%
\pgfsetbuttcap%
\pgfsetroundjoin%
\definecolor{currentfill}{rgb}{0.000000,0.000000,0.000000}%
\pgfsetfillcolor{currentfill}%
\pgfsetlinewidth{0.803000pt}%
\definecolor{currentstroke}{rgb}{0.000000,0.000000,0.000000}%
\pgfsetstrokecolor{currentstroke}%
\pgfsetdash{}{0pt}%
\pgfsys@defobject{currentmarker}{\pgfqpoint{0.000000in}{-0.048611in}}{\pgfqpoint{0.000000in}{0.000000in}}{%
\pgfpathmoveto{\pgfqpoint{0.000000in}{0.000000in}}%
\pgfpathlineto{\pgfqpoint{0.000000in}{-0.048611in}}%
\pgfusepath{stroke,fill}%
}%
\begin{pgfscope}%
\pgfsys@transformshift{0.958750in}{0.688600in}%
\pgfsys@useobject{currentmarker}{}%
\end{pgfscope}%
\end{pgfscope}%
\begin{pgfscope}%
\pgftext[x=0.958750in,y=0.591378in,,top]{\sffamily\fontsize{10.000000}{12.000000}\selectfont \(\displaystyle 0\)}%
\end{pgfscope}%
\begin{pgfscope}%
\pgfsetbuttcap%
\pgfsetroundjoin%
\definecolor{currentfill}{rgb}{0.000000,0.000000,0.000000}%
\pgfsetfillcolor{currentfill}%
\pgfsetlinewidth{0.803000pt}%
\definecolor{currentstroke}{rgb}{0.000000,0.000000,0.000000}%
\pgfsetstrokecolor{currentstroke}%
\pgfsetdash{}{0pt}%
\pgfsys@defobject{currentmarker}{\pgfqpoint{0.000000in}{-0.048611in}}{\pgfqpoint{0.000000in}{0.000000in}}{%
\pgfpathmoveto{\pgfqpoint{0.000000in}{0.000000in}}%
\pgfpathlineto{\pgfqpoint{0.000000in}{-0.048611in}}%
\pgfusepath{stroke,fill}%
}%
\begin{pgfscope}%
\pgfsys@transformshift{1.526220in}{0.688600in}%
\pgfsys@useobject{currentmarker}{}%
\end{pgfscope}%
\end{pgfscope}%
\begin{pgfscope}%
\pgftext[x=1.526220in,y=0.591378in,,top]{\sffamily\fontsize{10.000000}{12.000000}\selectfont \(\displaystyle 2000\)}%
\end{pgfscope}%
\begin{pgfscope}%
\pgfsetbuttcap%
\pgfsetroundjoin%
\definecolor{currentfill}{rgb}{0.000000,0.000000,0.000000}%
\pgfsetfillcolor{currentfill}%
\pgfsetlinewidth{0.803000pt}%
\definecolor{currentstroke}{rgb}{0.000000,0.000000,0.000000}%
\pgfsetstrokecolor{currentstroke}%
\pgfsetdash{}{0pt}%
\pgfsys@defobject{currentmarker}{\pgfqpoint{0.000000in}{-0.048611in}}{\pgfqpoint{0.000000in}{0.000000in}}{%
\pgfpathmoveto{\pgfqpoint{0.000000in}{0.000000in}}%
\pgfpathlineto{\pgfqpoint{0.000000in}{-0.048611in}}%
\pgfusepath{stroke,fill}%
}%
\begin{pgfscope}%
\pgfsys@transformshift{2.093690in}{0.688600in}%
\pgfsys@useobject{currentmarker}{}%
\end{pgfscope}%
\end{pgfscope}%
\begin{pgfscope}%
\pgftext[x=2.093690in,y=0.591378in,,top]{\sffamily\fontsize{10.000000}{12.000000}\selectfont \(\displaystyle 4000\)}%
\end{pgfscope}%
\begin{pgfscope}%
\pgfsetbuttcap%
\pgfsetroundjoin%
\definecolor{currentfill}{rgb}{0.000000,0.000000,0.000000}%
\pgfsetfillcolor{currentfill}%
\pgfsetlinewidth{0.803000pt}%
\definecolor{currentstroke}{rgb}{0.000000,0.000000,0.000000}%
\pgfsetstrokecolor{currentstroke}%
\pgfsetdash{}{0pt}%
\pgfsys@defobject{currentmarker}{\pgfqpoint{0.000000in}{-0.048611in}}{\pgfqpoint{0.000000in}{0.000000in}}{%
\pgfpathmoveto{\pgfqpoint{0.000000in}{0.000000in}}%
\pgfpathlineto{\pgfqpoint{0.000000in}{-0.048611in}}%
\pgfusepath{stroke,fill}%
}%
\begin{pgfscope}%
\pgfsys@transformshift{2.661161in}{0.688600in}%
\pgfsys@useobject{currentmarker}{}%
\end{pgfscope}%
\end{pgfscope}%
\begin{pgfscope}%
\pgftext[x=2.661161in,y=0.591378in,,top]{\sffamily\fontsize{10.000000}{12.000000}\selectfont \(\displaystyle 6000\)}%
\end{pgfscope}%
\begin{pgfscope}%
\pgfsetbuttcap%
\pgfsetroundjoin%
\definecolor{currentfill}{rgb}{0.000000,0.000000,0.000000}%
\pgfsetfillcolor{currentfill}%
\pgfsetlinewidth{0.803000pt}%
\definecolor{currentstroke}{rgb}{0.000000,0.000000,0.000000}%
\pgfsetstrokecolor{currentstroke}%
\pgfsetdash{}{0pt}%
\pgfsys@defobject{currentmarker}{\pgfqpoint{0.000000in}{-0.048611in}}{\pgfqpoint{0.000000in}{0.000000in}}{%
\pgfpathmoveto{\pgfqpoint{0.000000in}{0.000000in}}%
\pgfpathlineto{\pgfqpoint{0.000000in}{-0.048611in}}%
\pgfusepath{stroke,fill}%
}%
\begin{pgfscope}%
\pgfsys@transformshift{3.228631in}{0.688600in}%
\pgfsys@useobject{currentmarker}{}%
\end{pgfscope}%
\end{pgfscope}%
\begin{pgfscope}%
\pgftext[x=3.228631in,y=0.591378in,,top]{\sffamily\fontsize{10.000000}{12.000000}\selectfont \(\displaystyle 8000\)}%
\end{pgfscope}%
\begin{pgfscope}%
\pgfsetbuttcap%
\pgfsetroundjoin%
\definecolor{currentfill}{rgb}{0.000000,0.000000,0.000000}%
\pgfsetfillcolor{currentfill}%
\pgfsetlinewidth{0.803000pt}%
\definecolor{currentstroke}{rgb}{0.000000,0.000000,0.000000}%
\pgfsetstrokecolor{currentstroke}%
\pgfsetdash{}{0pt}%
\pgfsys@defobject{currentmarker}{\pgfqpoint{0.000000in}{-0.048611in}}{\pgfqpoint{0.000000in}{0.000000in}}{%
\pgfpathmoveto{\pgfqpoint{0.000000in}{0.000000in}}%
\pgfpathlineto{\pgfqpoint{0.000000in}{-0.048611in}}%
\pgfusepath{stroke,fill}%
}%
\begin{pgfscope}%
\pgfsys@transformshift{3.796101in}{0.688600in}%
\pgfsys@useobject{currentmarker}{}%
\end{pgfscope}%
\end{pgfscope}%
\begin{pgfscope}%
\pgftext[x=3.796101in,y=0.591378in,,top]{\sffamily\fontsize{10.000000}{12.000000}\selectfont \(\displaystyle 10000\)}%
\end{pgfscope}%
\begin{pgfscope}%
\pgfsetbuttcap%
\pgfsetroundjoin%
\definecolor{currentfill}{rgb}{0.000000,0.000000,0.000000}%
\pgfsetfillcolor{currentfill}%
\pgfsetlinewidth{0.803000pt}%
\definecolor{currentstroke}{rgb}{0.000000,0.000000,0.000000}%
\pgfsetstrokecolor{currentstroke}%
\pgfsetdash{}{0pt}%
\pgfsys@defobject{currentmarker}{\pgfqpoint{0.000000in}{-0.048611in}}{\pgfqpoint{0.000000in}{0.000000in}}{%
\pgfpathmoveto{\pgfqpoint{0.000000in}{0.000000in}}%
\pgfpathlineto{\pgfqpoint{0.000000in}{-0.048611in}}%
\pgfusepath{stroke,fill}%
}%
\begin{pgfscope}%
\pgfsys@transformshift{4.363571in}{0.688600in}%
\pgfsys@useobject{currentmarker}{}%
\end{pgfscope}%
\end{pgfscope}%
\begin{pgfscope}%
\pgftext[x=4.363571in,y=0.591378in,,top]{\sffamily\fontsize{10.000000}{12.000000}\selectfont \(\displaystyle 12000\)}%
\end{pgfscope}%
\begin{pgfscope}%
\pgfsetbuttcap%
\pgfsetroundjoin%
\definecolor{currentfill}{rgb}{0.000000,0.000000,0.000000}%
\pgfsetfillcolor{currentfill}%
\pgfsetlinewidth{0.803000pt}%
\definecolor{currentstroke}{rgb}{0.000000,0.000000,0.000000}%
\pgfsetstrokecolor{currentstroke}%
\pgfsetdash{}{0pt}%
\pgfsys@defobject{currentmarker}{\pgfqpoint{0.000000in}{-0.048611in}}{\pgfqpoint{0.000000in}{0.000000in}}{%
\pgfpathmoveto{\pgfqpoint{0.000000in}{0.000000in}}%
\pgfpathlineto{\pgfqpoint{0.000000in}{-0.048611in}}%
\pgfusepath{stroke,fill}%
}%
\begin{pgfscope}%
\pgfsys@transformshift{4.931041in}{0.688600in}%
\pgfsys@useobject{currentmarker}{}%
\end{pgfscope}%
\end{pgfscope}%
\begin{pgfscope}%
\pgftext[x=4.931041in,y=0.591378in,,top]{\sffamily\fontsize{10.000000}{12.000000}\selectfont \(\displaystyle 14000\)}%
\end{pgfscope}%
\begin{pgfscope}%
\pgfsetbuttcap%
\pgfsetroundjoin%
\definecolor{currentfill}{rgb}{0.000000,0.000000,0.000000}%
\pgfsetfillcolor{currentfill}%
\pgfsetlinewidth{0.803000pt}%
\definecolor{currentstroke}{rgb}{0.000000,0.000000,0.000000}%
\pgfsetstrokecolor{currentstroke}%
\pgfsetdash{}{0pt}%
\pgfsys@defobject{currentmarker}{\pgfqpoint{0.000000in}{-0.048611in}}{\pgfqpoint{0.000000in}{0.000000in}}{%
\pgfpathmoveto{\pgfqpoint{0.000000in}{0.000000in}}%
\pgfpathlineto{\pgfqpoint{0.000000in}{-0.048611in}}%
\pgfusepath{stroke,fill}%
}%
\begin{pgfscope}%
\pgfsys@transformshift{5.498511in}{0.688600in}%
\pgfsys@useobject{currentmarker}{}%
\end{pgfscope}%
\end{pgfscope}%
\begin{pgfscope}%
\pgftext[x=5.498511in,y=0.591378in,,top]{\sffamily\fontsize{10.000000}{12.000000}\selectfont \(\displaystyle 16000\)}%
\end{pgfscope}%
\begin{pgfscope}%
\pgfsetbuttcap%
\pgfsetroundjoin%
\definecolor{currentfill}{rgb}{0.000000,0.000000,0.000000}%
\pgfsetfillcolor{currentfill}%
\pgfsetlinewidth{0.803000pt}%
\definecolor{currentstroke}{rgb}{0.000000,0.000000,0.000000}%
\pgfsetstrokecolor{currentstroke}%
\pgfsetdash{}{0pt}%
\pgfsys@defobject{currentmarker}{\pgfqpoint{0.000000in}{-0.048611in}}{\pgfqpoint{0.000000in}{0.000000in}}{%
\pgfpathmoveto{\pgfqpoint{0.000000in}{0.000000in}}%
\pgfpathlineto{\pgfqpoint{0.000000in}{-0.048611in}}%
\pgfusepath{stroke,fill}%
}%
\begin{pgfscope}%
\pgfsys@transformshift{6.065982in}{0.688600in}%
\pgfsys@useobject{currentmarker}{}%
\end{pgfscope}%
\end{pgfscope}%
\begin{pgfscope}%
\pgftext[x=6.065982in,y=0.591378in,,top]{\sffamily\fontsize{10.000000}{12.000000}\selectfont \(\displaystyle 18000\)}%
\end{pgfscope}%
\begin{pgfscope}%
\pgfsetbuttcap%
\pgfsetroundjoin%
\definecolor{currentfill}{rgb}{0.000000,0.000000,0.000000}%
\pgfsetfillcolor{currentfill}%
\pgfsetlinewidth{0.803000pt}%
\definecolor{currentstroke}{rgb}{0.000000,0.000000,0.000000}%
\pgfsetstrokecolor{currentstroke}%
\pgfsetdash{}{0pt}%
\pgfsys@defobject{currentmarker}{\pgfqpoint{0.000000in}{-0.048611in}}{\pgfqpoint{0.000000in}{0.000000in}}{%
\pgfpathmoveto{\pgfqpoint{0.000000in}{0.000000in}}%
\pgfpathlineto{\pgfqpoint{0.000000in}{-0.048611in}}%
\pgfusepath{stroke,fill}%
}%
\begin{pgfscope}%
\pgfsys@transformshift{6.633452in}{0.688600in}%
\pgfsys@useobject{currentmarker}{}%
\end{pgfscope}%
\end{pgfscope}%
\begin{pgfscope}%
\pgftext[x=6.633452in,y=0.591378in,,top]{\sffamily\fontsize{10.000000}{12.000000}\selectfont \(\displaystyle 20000\)}%
\end{pgfscope}%
\begin{pgfscope}%
\pgftext[x=3.930875in,y=0.401409in,,top]{\sffamily\fontsize{12.000000}{14.400000}\selectfont \(\displaystyle |\mathcal{B}|\)}%
\end{pgfscope}%
\begin{pgfscope}%
\pgfsetbuttcap%
\pgfsetroundjoin%
\definecolor{currentfill}{rgb}{0.000000,0.000000,0.000000}%
\pgfsetfillcolor{currentfill}%
\pgfsetlinewidth{0.803000pt}%
\definecolor{currentstroke}{rgb}{0.000000,0.000000,0.000000}%
\pgfsetstrokecolor{currentstroke}%
\pgfsetdash{}{0pt}%
\pgfsys@defobject{currentmarker}{\pgfqpoint{-0.048611in}{0.000000in}}{\pgfqpoint{0.000000in}{0.000000in}}{%
\pgfpathmoveto{\pgfqpoint{0.000000in}{0.000000in}}%
\pgfpathlineto{\pgfqpoint{-0.048611in}{0.000000in}}%
\pgfusepath{stroke,fill}%
}%
\begin{pgfscope}%
\pgfsys@transformshift{0.958750in}{0.907295in}%
\pgfsys@useobject{currentmarker}{}%
\end{pgfscope}%
\end{pgfscope}%
\begin{pgfscope}%
\pgftext[x=0.792083in,y=0.854533in,left,base]{\sffamily\fontsize{10.000000}{12.000000}\selectfont \(\displaystyle 0\)}%
\end{pgfscope}%
\begin{pgfscope}%
\pgfsetbuttcap%
\pgfsetroundjoin%
\definecolor{currentfill}{rgb}{0.000000,0.000000,0.000000}%
\pgfsetfillcolor{currentfill}%
\pgfsetlinewidth{0.803000pt}%
\definecolor{currentstroke}{rgb}{0.000000,0.000000,0.000000}%
\pgfsetstrokecolor{currentstroke}%
\pgfsetdash{}{0pt}%
\pgfsys@defobject{currentmarker}{\pgfqpoint{-0.048611in}{0.000000in}}{\pgfqpoint{0.000000in}{0.000000in}}{%
\pgfpathmoveto{\pgfqpoint{0.000000in}{0.000000in}}%
\pgfpathlineto{\pgfqpoint{-0.048611in}{0.000000in}}%
\pgfusepath{stroke,fill}%
}%
\begin{pgfscope}%
\pgfsys@transformshift{0.958750in}{1.083654in}%
\pgfsys@useobject{currentmarker}{}%
\end{pgfscope}%
\end{pgfscope}%
\begin{pgfscope}%
\pgftext[x=0.722638in,y=1.030893in,left,base]{\sffamily\fontsize{10.000000}{12.000000}\selectfont \(\displaystyle 10\)}%
\end{pgfscope}%
\begin{pgfscope}%
\pgfsetbuttcap%
\pgfsetroundjoin%
\definecolor{currentfill}{rgb}{0.000000,0.000000,0.000000}%
\pgfsetfillcolor{currentfill}%
\pgfsetlinewidth{0.803000pt}%
\definecolor{currentstroke}{rgb}{0.000000,0.000000,0.000000}%
\pgfsetstrokecolor{currentstroke}%
\pgfsetdash{}{0pt}%
\pgfsys@defobject{currentmarker}{\pgfqpoint{-0.048611in}{0.000000in}}{\pgfqpoint{0.000000in}{0.000000in}}{%
\pgfpathmoveto{\pgfqpoint{0.000000in}{0.000000in}}%
\pgfpathlineto{\pgfqpoint{-0.048611in}{0.000000in}}%
\pgfusepath{stroke,fill}%
}%
\begin{pgfscope}%
\pgfsys@transformshift{0.958750in}{1.260014in}%
\pgfsys@useobject{currentmarker}{}%
\end{pgfscope}%
\end{pgfscope}%
\begin{pgfscope}%
\pgftext[x=0.722638in,y=1.207252in,left,base]{\sffamily\fontsize{10.000000}{12.000000}\selectfont \(\displaystyle 20\)}%
\end{pgfscope}%
\begin{pgfscope}%
\pgfsetbuttcap%
\pgfsetroundjoin%
\definecolor{currentfill}{rgb}{0.000000,0.000000,0.000000}%
\pgfsetfillcolor{currentfill}%
\pgfsetlinewidth{0.803000pt}%
\definecolor{currentstroke}{rgb}{0.000000,0.000000,0.000000}%
\pgfsetstrokecolor{currentstroke}%
\pgfsetdash{}{0pt}%
\pgfsys@defobject{currentmarker}{\pgfqpoint{-0.048611in}{0.000000in}}{\pgfqpoint{0.000000in}{0.000000in}}{%
\pgfpathmoveto{\pgfqpoint{0.000000in}{0.000000in}}%
\pgfpathlineto{\pgfqpoint{-0.048611in}{0.000000in}}%
\pgfusepath{stroke,fill}%
}%
\begin{pgfscope}%
\pgfsys@transformshift{0.958750in}{1.436373in}%
\pgfsys@useobject{currentmarker}{}%
\end{pgfscope}%
\end{pgfscope}%
\begin{pgfscope}%
\pgftext[x=0.722638in,y=1.383611in,left,base]{\sffamily\fontsize{10.000000}{12.000000}\selectfont \(\displaystyle 30\)}%
\end{pgfscope}%
\begin{pgfscope}%
\pgfsetbuttcap%
\pgfsetroundjoin%
\definecolor{currentfill}{rgb}{0.000000,0.000000,0.000000}%
\pgfsetfillcolor{currentfill}%
\pgfsetlinewidth{0.803000pt}%
\definecolor{currentstroke}{rgb}{0.000000,0.000000,0.000000}%
\pgfsetstrokecolor{currentstroke}%
\pgfsetdash{}{0pt}%
\pgfsys@defobject{currentmarker}{\pgfqpoint{-0.048611in}{0.000000in}}{\pgfqpoint{0.000000in}{0.000000in}}{%
\pgfpathmoveto{\pgfqpoint{0.000000in}{0.000000in}}%
\pgfpathlineto{\pgfqpoint{-0.048611in}{0.000000in}}%
\pgfusepath{stroke,fill}%
}%
\begin{pgfscope}%
\pgfsys@transformshift{0.958750in}{1.612732in}%
\pgfsys@useobject{currentmarker}{}%
\end{pgfscope}%
\end{pgfscope}%
\begin{pgfscope}%
\pgftext[x=0.722638in,y=1.559971in,left,base]{\sffamily\fontsize{10.000000}{12.000000}\selectfont \(\displaystyle 40\)}%
\end{pgfscope}%
\begin{pgfscope}%
\pgfsetbuttcap%
\pgfsetroundjoin%
\definecolor{currentfill}{rgb}{0.000000,0.000000,0.000000}%
\pgfsetfillcolor{currentfill}%
\pgfsetlinewidth{0.803000pt}%
\definecolor{currentstroke}{rgb}{0.000000,0.000000,0.000000}%
\pgfsetstrokecolor{currentstroke}%
\pgfsetdash{}{0pt}%
\pgfsys@defobject{currentmarker}{\pgfqpoint{-0.048611in}{0.000000in}}{\pgfqpoint{0.000000in}{0.000000in}}{%
\pgfpathmoveto{\pgfqpoint{0.000000in}{0.000000in}}%
\pgfpathlineto{\pgfqpoint{-0.048611in}{0.000000in}}%
\pgfusepath{stroke,fill}%
}%
\begin{pgfscope}%
\pgfsys@transformshift{0.958750in}{1.789091in}%
\pgfsys@useobject{currentmarker}{}%
\end{pgfscope}%
\end{pgfscope}%
\begin{pgfscope}%
\pgftext[x=0.722638in,y=1.736330in,left,base]{\sffamily\fontsize{10.000000}{12.000000}\selectfont \(\displaystyle 50\)}%
\end{pgfscope}%
\begin{pgfscope}%
\pgfsetbuttcap%
\pgfsetroundjoin%
\definecolor{currentfill}{rgb}{0.000000,0.000000,0.000000}%
\pgfsetfillcolor{currentfill}%
\pgfsetlinewidth{0.803000pt}%
\definecolor{currentstroke}{rgb}{0.000000,0.000000,0.000000}%
\pgfsetstrokecolor{currentstroke}%
\pgfsetdash{}{0pt}%
\pgfsys@defobject{currentmarker}{\pgfqpoint{-0.048611in}{0.000000in}}{\pgfqpoint{0.000000in}{0.000000in}}{%
\pgfpathmoveto{\pgfqpoint{0.000000in}{0.000000in}}%
\pgfpathlineto{\pgfqpoint{-0.048611in}{0.000000in}}%
\pgfusepath{stroke,fill}%
}%
\begin{pgfscope}%
\pgfsys@transformshift{0.958750in}{1.965451in}%
\pgfsys@useobject{currentmarker}{}%
\end{pgfscope}%
\end{pgfscope}%
\begin{pgfscope}%
\pgftext[x=0.722638in,y=1.912689in,left,base]{\sffamily\fontsize{10.000000}{12.000000}\selectfont \(\displaystyle 60\)}%
\end{pgfscope}%
\begin{pgfscope}%
\pgfsetbuttcap%
\pgfsetroundjoin%
\definecolor{currentfill}{rgb}{0.000000,0.000000,0.000000}%
\pgfsetfillcolor{currentfill}%
\pgfsetlinewidth{0.803000pt}%
\definecolor{currentstroke}{rgb}{0.000000,0.000000,0.000000}%
\pgfsetstrokecolor{currentstroke}%
\pgfsetdash{}{0pt}%
\pgfsys@defobject{currentmarker}{\pgfqpoint{-0.048611in}{0.000000in}}{\pgfqpoint{0.000000in}{0.000000in}}{%
\pgfpathmoveto{\pgfqpoint{0.000000in}{0.000000in}}%
\pgfpathlineto{\pgfqpoint{-0.048611in}{0.000000in}}%
\pgfusepath{stroke,fill}%
}%
\begin{pgfscope}%
\pgfsys@transformshift{0.958750in}{2.141810in}%
\pgfsys@useobject{currentmarker}{}%
\end{pgfscope}%
\end{pgfscope}%
\begin{pgfscope}%
\pgftext[x=0.722638in,y=2.089048in,left,base]{\sffamily\fontsize{10.000000}{12.000000}\selectfont \(\displaystyle 70\)}%
\end{pgfscope}%
\begin{pgfscope}%
\pgfsetbuttcap%
\pgfsetroundjoin%
\definecolor{currentfill}{rgb}{0.000000,0.000000,0.000000}%
\pgfsetfillcolor{currentfill}%
\pgfsetlinewidth{0.803000pt}%
\definecolor{currentstroke}{rgb}{0.000000,0.000000,0.000000}%
\pgfsetstrokecolor{currentstroke}%
\pgfsetdash{}{0pt}%
\pgfsys@defobject{currentmarker}{\pgfqpoint{-0.048611in}{0.000000in}}{\pgfqpoint{0.000000in}{0.000000in}}{%
\pgfpathmoveto{\pgfqpoint{0.000000in}{0.000000in}}%
\pgfpathlineto{\pgfqpoint{-0.048611in}{0.000000in}}%
\pgfusepath{stroke,fill}%
}%
\begin{pgfscope}%
\pgfsys@transformshift{0.958750in}{2.318169in}%
\pgfsys@useobject{currentmarker}{}%
\end{pgfscope}%
\end{pgfscope}%
\begin{pgfscope}%
\pgftext[x=0.722638in,y=2.265408in,left,base]{\sffamily\fontsize{10.000000}{12.000000}\selectfont \(\displaystyle 80\)}%
\end{pgfscope}%
\begin{pgfscope}%
\pgfsetbuttcap%
\pgfsetroundjoin%
\definecolor{currentfill}{rgb}{0.000000,0.000000,0.000000}%
\pgfsetfillcolor{currentfill}%
\pgfsetlinewidth{0.803000pt}%
\definecolor{currentstroke}{rgb}{0.000000,0.000000,0.000000}%
\pgfsetstrokecolor{currentstroke}%
\pgfsetdash{}{0pt}%
\pgfsys@defobject{currentmarker}{\pgfqpoint{-0.048611in}{0.000000in}}{\pgfqpoint{0.000000in}{0.000000in}}{%
\pgfpathmoveto{\pgfqpoint{0.000000in}{0.000000in}}%
\pgfpathlineto{\pgfqpoint{-0.048611in}{0.000000in}}%
\pgfusepath{stroke,fill}%
}%
\begin{pgfscope}%
\pgfsys@transformshift{0.958750in}{2.494529in}%
\pgfsys@useobject{currentmarker}{}%
\end{pgfscope}%
\end{pgfscope}%
\begin{pgfscope}%
\pgftext[x=0.722638in,y=2.441767in,left,base]{\sffamily\fontsize{10.000000}{12.000000}\selectfont \(\displaystyle 90\)}%
\end{pgfscope}%
\begin{pgfscope}%
\pgfsetbuttcap%
\pgfsetroundjoin%
\definecolor{currentfill}{rgb}{0.000000,0.000000,0.000000}%
\pgfsetfillcolor{currentfill}%
\pgfsetlinewidth{0.803000pt}%
\definecolor{currentstroke}{rgb}{0.000000,0.000000,0.000000}%
\pgfsetstrokecolor{currentstroke}%
\pgfsetdash{}{0pt}%
\pgfsys@defobject{currentmarker}{\pgfqpoint{-0.048611in}{0.000000in}}{\pgfqpoint{0.000000in}{0.000000in}}{%
\pgfpathmoveto{\pgfqpoint{0.000000in}{0.000000in}}%
\pgfpathlineto{\pgfqpoint{-0.048611in}{0.000000in}}%
\pgfusepath{stroke,fill}%
}%
\begin{pgfscope}%
\pgfsys@transformshift{0.958750in}{2.670888in}%
\pgfsys@useobject{currentmarker}{}%
\end{pgfscope}%
\end{pgfscope}%
\begin{pgfscope}%
\pgftext[x=0.653194in,y=2.618126in,left,base]{\sffamily\fontsize{10.000000}{12.000000}\selectfont \(\displaystyle 100\)}%
\end{pgfscope}%
\begin{pgfscope}%
\pgfsetbuttcap%
\pgfsetroundjoin%
\definecolor{currentfill}{rgb}{0.000000,0.000000,0.000000}%
\pgfsetfillcolor{currentfill}%
\pgfsetlinewidth{0.803000pt}%
\definecolor{currentstroke}{rgb}{0.000000,0.000000,0.000000}%
\pgfsetstrokecolor{currentstroke}%
\pgfsetdash{}{0pt}%
\pgfsys@defobject{currentmarker}{\pgfqpoint{-0.048611in}{0.000000in}}{\pgfqpoint{0.000000in}{0.000000in}}{%
\pgfpathmoveto{\pgfqpoint{0.000000in}{0.000000in}}%
\pgfpathlineto{\pgfqpoint{-0.048611in}{0.000000in}}%
\pgfusepath{stroke,fill}%
}%
\begin{pgfscope}%
\pgfsys@transformshift{0.958750in}{2.847247in}%
\pgfsys@useobject{currentmarker}{}%
\end{pgfscope}%
\end{pgfscope}%
\begin{pgfscope}%
\pgftext[x=0.653194in,y=2.794486in,left,base]{\sffamily\fontsize{10.000000}{12.000000}\selectfont \(\displaystyle 110\)}%
\end{pgfscope}%
\begin{pgfscope}%
\pgfsetbuttcap%
\pgfsetroundjoin%
\definecolor{currentfill}{rgb}{0.000000,0.000000,0.000000}%
\pgfsetfillcolor{currentfill}%
\pgfsetlinewidth{0.803000pt}%
\definecolor{currentstroke}{rgb}{0.000000,0.000000,0.000000}%
\pgfsetstrokecolor{currentstroke}%
\pgfsetdash{}{0pt}%
\pgfsys@defobject{currentmarker}{\pgfqpoint{-0.048611in}{0.000000in}}{\pgfqpoint{0.000000in}{0.000000in}}{%
\pgfpathmoveto{\pgfqpoint{0.000000in}{0.000000in}}%
\pgfpathlineto{\pgfqpoint{-0.048611in}{0.000000in}}%
\pgfusepath{stroke,fill}%
}%
\begin{pgfscope}%
\pgfsys@transformshift{0.958750in}{3.023606in}%
\pgfsys@useobject{currentmarker}{}%
\end{pgfscope}%
\end{pgfscope}%
\begin{pgfscope}%
\pgftext[x=0.653194in,y=2.970845in,left,base]{\sffamily\fontsize{10.000000}{12.000000}\selectfont \(\displaystyle 120\)}%
\end{pgfscope}%
\begin{pgfscope}%
\pgfsetbuttcap%
\pgfsetroundjoin%
\definecolor{currentfill}{rgb}{0.000000,0.000000,0.000000}%
\pgfsetfillcolor{currentfill}%
\pgfsetlinewidth{0.803000pt}%
\definecolor{currentstroke}{rgb}{0.000000,0.000000,0.000000}%
\pgfsetstrokecolor{currentstroke}%
\pgfsetdash{}{0pt}%
\pgfsys@defobject{currentmarker}{\pgfqpoint{-0.048611in}{0.000000in}}{\pgfqpoint{0.000000in}{0.000000in}}{%
\pgfpathmoveto{\pgfqpoint{0.000000in}{0.000000in}}%
\pgfpathlineto{\pgfqpoint{-0.048611in}{0.000000in}}%
\pgfusepath{stroke,fill}%
}%
\begin{pgfscope}%
\pgfsys@transformshift{0.958750in}{3.199966in}%
\pgfsys@useobject{currentmarker}{}%
\end{pgfscope}%
\end{pgfscope}%
\begin{pgfscope}%
\pgftext[x=0.653194in,y=3.147204in,left,base]{\sffamily\fontsize{10.000000}{12.000000}\selectfont \(\displaystyle 130\)}%
\end{pgfscope}%
\begin{pgfscope}%
\pgfsetbuttcap%
\pgfsetroundjoin%
\definecolor{currentfill}{rgb}{0.000000,0.000000,0.000000}%
\pgfsetfillcolor{currentfill}%
\pgfsetlinewidth{0.803000pt}%
\definecolor{currentstroke}{rgb}{0.000000,0.000000,0.000000}%
\pgfsetstrokecolor{currentstroke}%
\pgfsetdash{}{0pt}%
\pgfsys@defobject{currentmarker}{\pgfqpoint{-0.048611in}{0.000000in}}{\pgfqpoint{0.000000in}{0.000000in}}{%
\pgfpathmoveto{\pgfqpoint{0.000000in}{0.000000in}}%
\pgfpathlineto{\pgfqpoint{-0.048611in}{0.000000in}}%
\pgfusepath{stroke,fill}%
}%
\begin{pgfscope}%
\pgfsys@transformshift{0.958750in}{3.376325in}%
\pgfsys@useobject{currentmarker}{}%
\end{pgfscope}%
\end{pgfscope}%
\begin{pgfscope}%
\pgftext[x=0.653194in,y=3.323564in,left,base]{\sffamily\fontsize{10.000000}{12.000000}\selectfont \(\displaystyle 140\)}%
\end{pgfscope}%
\begin{pgfscope}%
\pgfsetbuttcap%
\pgfsetroundjoin%
\definecolor{currentfill}{rgb}{0.000000,0.000000,0.000000}%
\pgfsetfillcolor{currentfill}%
\pgfsetlinewidth{0.803000pt}%
\definecolor{currentstroke}{rgb}{0.000000,0.000000,0.000000}%
\pgfsetstrokecolor{currentstroke}%
\pgfsetdash{}{0pt}%
\pgfsys@defobject{currentmarker}{\pgfqpoint{-0.048611in}{0.000000in}}{\pgfqpoint{0.000000in}{0.000000in}}{%
\pgfpathmoveto{\pgfqpoint{0.000000in}{0.000000in}}%
\pgfpathlineto{\pgfqpoint{-0.048611in}{0.000000in}}%
\pgfusepath{stroke,fill}%
}%
\begin{pgfscope}%
\pgfsys@transformshift{0.958750in}{3.552684in}%
\pgfsys@useobject{currentmarker}{}%
\end{pgfscope}%
\end{pgfscope}%
\begin{pgfscope}%
\pgftext[x=0.653194in,y=3.499923in,left,base]{\sffamily\fontsize{10.000000}{12.000000}\selectfont \(\displaystyle 150\)}%
\end{pgfscope}%
\begin{pgfscope}%
\pgfsetbuttcap%
\pgfsetroundjoin%
\definecolor{currentfill}{rgb}{0.000000,0.000000,0.000000}%
\pgfsetfillcolor{currentfill}%
\pgfsetlinewidth{0.803000pt}%
\definecolor{currentstroke}{rgb}{0.000000,0.000000,0.000000}%
\pgfsetstrokecolor{currentstroke}%
\pgfsetdash{}{0pt}%
\pgfsys@defobject{currentmarker}{\pgfqpoint{-0.048611in}{0.000000in}}{\pgfqpoint{0.000000in}{0.000000in}}{%
\pgfpathmoveto{\pgfqpoint{0.000000in}{0.000000in}}%
\pgfpathlineto{\pgfqpoint{-0.048611in}{0.000000in}}%
\pgfusepath{stroke,fill}%
}%
\begin{pgfscope}%
\pgfsys@transformshift{0.958750in}{3.729044in}%
\pgfsys@useobject{currentmarker}{}%
\end{pgfscope}%
\end{pgfscope}%
\begin{pgfscope}%
\pgftext[x=0.653194in,y=3.676282in,left,base]{\sffamily\fontsize{10.000000}{12.000000}\selectfont \(\displaystyle 160\)}%
\end{pgfscope}%
\begin{pgfscope}%
\pgfsetbuttcap%
\pgfsetroundjoin%
\definecolor{currentfill}{rgb}{0.000000,0.000000,0.000000}%
\pgfsetfillcolor{currentfill}%
\pgfsetlinewidth{0.803000pt}%
\definecolor{currentstroke}{rgb}{0.000000,0.000000,0.000000}%
\pgfsetstrokecolor{currentstroke}%
\pgfsetdash{}{0pt}%
\pgfsys@defobject{currentmarker}{\pgfqpoint{-0.048611in}{0.000000in}}{\pgfqpoint{0.000000in}{0.000000in}}{%
\pgfpathmoveto{\pgfqpoint{0.000000in}{0.000000in}}%
\pgfpathlineto{\pgfqpoint{-0.048611in}{0.000000in}}%
\pgfusepath{stroke,fill}%
}%
\begin{pgfscope}%
\pgfsys@transformshift{0.958750in}{3.905403in}%
\pgfsys@useobject{currentmarker}{}%
\end{pgfscope}%
\end{pgfscope}%
\begin{pgfscope}%
\pgftext[x=0.653194in,y=3.852641in,left,base]{\sffamily\fontsize{10.000000}{12.000000}\selectfont \(\displaystyle 170\)}%
\end{pgfscope}%
\begin{pgfscope}%
\pgfsetbuttcap%
\pgfsetroundjoin%
\definecolor{currentfill}{rgb}{0.000000,0.000000,0.000000}%
\pgfsetfillcolor{currentfill}%
\pgfsetlinewidth{0.803000pt}%
\definecolor{currentstroke}{rgb}{0.000000,0.000000,0.000000}%
\pgfsetstrokecolor{currentstroke}%
\pgfsetdash{}{0pt}%
\pgfsys@defobject{currentmarker}{\pgfqpoint{-0.048611in}{0.000000in}}{\pgfqpoint{0.000000in}{0.000000in}}{%
\pgfpathmoveto{\pgfqpoint{0.000000in}{0.000000in}}%
\pgfpathlineto{\pgfqpoint{-0.048611in}{0.000000in}}%
\pgfusepath{stroke,fill}%
}%
\begin{pgfscope}%
\pgfsys@transformshift{0.958750in}{4.081762in}%
\pgfsys@useobject{currentmarker}{}%
\end{pgfscope}%
\end{pgfscope}%
\begin{pgfscope}%
\pgftext[x=0.653194in,y=4.029001in,left,base]{\sffamily\fontsize{10.000000}{12.000000}\selectfont \(\displaystyle 180\)}%
\end{pgfscope}%
\begin{pgfscope}%
\pgfsetbuttcap%
\pgfsetroundjoin%
\definecolor{currentfill}{rgb}{0.000000,0.000000,0.000000}%
\pgfsetfillcolor{currentfill}%
\pgfsetlinewidth{0.803000pt}%
\definecolor{currentstroke}{rgb}{0.000000,0.000000,0.000000}%
\pgfsetstrokecolor{currentstroke}%
\pgfsetdash{}{0pt}%
\pgfsys@defobject{currentmarker}{\pgfqpoint{-0.048611in}{0.000000in}}{\pgfqpoint{0.000000in}{0.000000in}}{%
\pgfpathmoveto{\pgfqpoint{0.000000in}{0.000000in}}%
\pgfpathlineto{\pgfqpoint{-0.048611in}{0.000000in}}%
\pgfusepath{stroke,fill}%
}%
\begin{pgfscope}%
\pgfsys@transformshift{0.958750in}{4.258122in}%
\pgfsys@useobject{currentmarker}{}%
\end{pgfscope}%
\end{pgfscope}%
\begin{pgfscope}%
\pgftext[x=0.653194in,y=4.205360in,left,base]{\sffamily\fontsize{10.000000}{12.000000}\selectfont \(\displaystyle 190\)}%
\end{pgfscope}%
\begin{pgfscope}%
\pgfsetbuttcap%
\pgfsetroundjoin%
\definecolor{currentfill}{rgb}{0.000000,0.000000,0.000000}%
\pgfsetfillcolor{currentfill}%
\pgfsetlinewidth{0.803000pt}%
\definecolor{currentstroke}{rgb}{0.000000,0.000000,0.000000}%
\pgfsetstrokecolor{currentstroke}%
\pgfsetdash{}{0pt}%
\pgfsys@defobject{currentmarker}{\pgfqpoint{-0.048611in}{0.000000in}}{\pgfqpoint{0.000000in}{0.000000in}}{%
\pgfpathmoveto{\pgfqpoint{0.000000in}{0.000000in}}%
\pgfpathlineto{\pgfqpoint{-0.048611in}{0.000000in}}%
\pgfusepath{stroke,fill}%
}%
\begin{pgfscope}%
\pgfsys@transformshift{0.958750in}{4.434481in}%
\pgfsys@useobject{currentmarker}{}%
\end{pgfscope}%
\end{pgfscope}%
\begin{pgfscope}%
\pgftext[x=0.653194in,y=4.381719in,left,base]{\sffamily\fontsize{10.000000}{12.000000}\selectfont \(\displaystyle 200\)}%
\end{pgfscope}%
\begin{pgfscope}%
\pgfsetbuttcap%
\pgfsetroundjoin%
\definecolor{currentfill}{rgb}{0.000000,0.000000,0.000000}%
\pgfsetfillcolor{currentfill}%
\pgfsetlinewidth{0.803000pt}%
\definecolor{currentstroke}{rgb}{0.000000,0.000000,0.000000}%
\pgfsetstrokecolor{currentstroke}%
\pgfsetdash{}{0pt}%
\pgfsys@defobject{currentmarker}{\pgfqpoint{-0.048611in}{0.000000in}}{\pgfqpoint{0.000000in}{0.000000in}}{%
\pgfpathmoveto{\pgfqpoint{0.000000in}{0.000000in}}%
\pgfpathlineto{\pgfqpoint{-0.048611in}{0.000000in}}%
\pgfusepath{stroke,fill}%
}%
\begin{pgfscope}%
\pgfsys@transformshift{0.958750in}{4.610840in}%
\pgfsys@useobject{currentmarker}{}%
\end{pgfscope}%
\end{pgfscope}%
\begin{pgfscope}%
\pgftext[x=0.653194in,y=4.558079in,left,base]{\sffamily\fontsize{10.000000}{12.000000}\selectfont \(\displaystyle 210\)}%
\end{pgfscope}%
\begin{pgfscope}%
\pgfsetbuttcap%
\pgfsetroundjoin%
\definecolor{currentfill}{rgb}{0.000000,0.000000,0.000000}%
\pgfsetfillcolor{currentfill}%
\pgfsetlinewidth{0.803000pt}%
\definecolor{currentstroke}{rgb}{0.000000,0.000000,0.000000}%
\pgfsetstrokecolor{currentstroke}%
\pgfsetdash{}{0pt}%
\pgfsys@defobject{currentmarker}{\pgfqpoint{-0.048611in}{0.000000in}}{\pgfqpoint{0.000000in}{0.000000in}}{%
\pgfpathmoveto{\pgfqpoint{0.000000in}{0.000000in}}%
\pgfpathlineto{\pgfqpoint{-0.048611in}{0.000000in}}%
\pgfusepath{stroke,fill}%
}%
\begin{pgfscope}%
\pgfsys@transformshift{0.958750in}{4.787199in}%
\pgfsys@useobject{currentmarker}{}%
\end{pgfscope}%
\end{pgfscope}%
\begin{pgfscope}%
\pgftext[x=0.653194in,y=4.734438in,left,base]{\sffamily\fontsize{10.000000}{12.000000}\selectfont \(\displaystyle 220\)}%
\end{pgfscope}%
\begin{pgfscope}%
\pgfsetbuttcap%
\pgfsetroundjoin%
\definecolor{currentfill}{rgb}{0.000000,0.000000,0.000000}%
\pgfsetfillcolor{currentfill}%
\pgfsetlinewidth{0.803000pt}%
\definecolor{currentstroke}{rgb}{0.000000,0.000000,0.000000}%
\pgfsetstrokecolor{currentstroke}%
\pgfsetdash{}{0pt}%
\pgfsys@defobject{currentmarker}{\pgfqpoint{-0.048611in}{0.000000in}}{\pgfqpoint{0.000000in}{0.000000in}}{%
\pgfpathmoveto{\pgfqpoint{0.000000in}{0.000000in}}%
\pgfpathlineto{\pgfqpoint{-0.048611in}{0.000000in}}%
\pgfusepath{stroke,fill}%
}%
\begin{pgfscope}%
\pgfsys@transformshift{0.958750in}{4.963559in}%
\pgfsys@useobject{currentmarker}{}%
\end{pgfscope}%
\end{pgfscope}%
\begin{pgfscope}%
\pgftext[x=0.653194in,y=4.910797in,left,base]{\sffamily\fontsize{10.000000}{12.000000}\selectfont \(\displaystyle 230\)}%
\end{pgfscope}%
\begin{pgfscope}%
\pgfsetbuttcap%
\pgfsetroundjoin%
\definecolor{currentfill}{rgb}{0.000000,0.000000,0.000000}%
\pgfsetfillcolor{currentfill}%
\pgfsetlinewidth{0.803000pt}%
\definecolor{currentstroke}{rgb}{0.000000,0.000000,0.000000}%
\pgfsetstrokecolor{currentstroke}%
\pgfsetdash{}{0pt}%
\pgfsys@defobject{currentmarker}{\pgfqpoint{-0.048611in}{0.000000in}}{\pgfqpoint{0.000000in}{0.000000in}}{%
\pgfpathmoveto{\pgfqpoint{0.000000in}{0.000000in}}%
\pgfpathlineto{\pgfqpoint{-0.048611in}{0.000000in}}%
\pgfusepath{stroke,fill}%
}%
\begin{pgfscope}%
\pgfsys@transformshift{0.958750in}{5.139918in}%
\pgfsys@useobject{currentmarker}{}%
\end{pgfscope}%
\end{pgfscope}%
\begin{pgfscope}%
\pgftext[x=0.653194in,y=5.087157in,left,base]{\sffamily\fontsize{10.000000}{12.000000}\selectfont \(\displaystyle 240\)}%
\end{pgfscope}%
\begin{pgfscope}%
\pgfsetbuttcap%
\pgfsetroundjoin%
\definecolor{currentfill}{rgb}{0.000000,0.000000,0.000000}%
\pgfsetfillcolor{currentfill}%
\pgfsetlinewidth{0.803000pt}%
\definecolor{currentstroke}{rgb}{0.000000,0.000000,0.000000}%
\pgfsetstrokecolor{currentstroke}%
\pgfsetdash{}{0pt}%
\pgfsys@defobject{currentmarker}{\pgfqpoint{-0.048611in}{0.000000in}}{\pgfqpoint{0.000000in}{0.000000in}}{%
\pgfpathmoveto{\pgfqpoint{0.000000in}{0.000000in}}%
\pgfpathlineto{\pgfqpoint{-0.048611in}{0.000000in}}%
\pgfusepath{stroke,fill}%
}%
\begin{pgfscope}%
\pgfsys@transformshift{0.958750in}{5.316277in}%
\pgfsys@useobject{currentmarker}{}%
\end{pgfscope}%
\end{pgfscope}%
\begin{pgfscope}%
\pgftext[x=0.653194in,y=5.263516in,left,base]{\sffamily\fontsize{10.000000}{12.000000}\selectfont \(\displaystyle 250\)}%
\end{pgfscope}%
\begin{pgfscope}%
\pgftext[x=0.597638in,y=3.098700in,,bottom,rotate=90.000000]{\sffamily\fontsize{12.000000}{14.400000}\selectfont seconds}%
\end{pgfscope}%
\begin{pgfscope}%
\pgfpathrectangle{\pgfqpoint{0.958750in}{0.688600in}}{\pgfqpoint{5.944250in}{4.820200in}}%
\pgfusepath{clip}%
\pgfsetrectcap%
\pgfsetroundjoin%
\pgfsetlinewidth{1.505625pt}%
\definecolor{currentstroke}{rgb}{0.172549,0.243137,0.313725}%
\pgfsetstrokecolor{currentstroke}%
\pgfsetdash{}{0pt}%
\pgfpathmoveto{\pgfqpoint{1.242485in}{0.917821in}}%
\pgfpathlineto{\pgfqpoint{1.526220in}{0.947604in}}%
\pgfpathlineto{\pgfqpoint{1.809955in}{1.015130in}}%
\pgfpathlineto{\pgfqpoint{2.093690in}{1.104544in}}%
\pgfpathlineto{\pgfqpoint{2.377425in}{1.268440in}}%
\pgfpathlineto{\pgfqpoint{2.661161in}{1.420010in}}%
\pgfpathlineto{\pgfqpoint{2.944896in}{1.337347in}}%
\pgfpathlineto{\pgfqpoint{3.228631in}{1.467241in}}%
\pgfpathlineto{\pgfqpoint{3.512366in}{1.640115in}}%
\pgfpathlineto{\pgfqpoint{3.796101in}{1.873299in}}%
\pgfpathlineto{\pgfqpoint{4.079836in}{2.215172in}}%
\pgfpathlineto{\pgfqpoint{4.363571in}{2.495566in}}%
\pgfpathlineto{\pgfqpoint{4.647306in}{2.430999in}}%
\pgfpathlineto{\pgfqpoint{4.931041in}{2.450918in}}%
\pgfpathlineto{\pgfqpoint{5.214776in}{2.692792in}}%
\pgfpathlineto{\pgfqpoint{5.498511in}{3.374738in}}%
\pgfpathlineto{\pgfqpoint{5.782246in}{4.222850in}}%
\pgfpathlineto{\pgfqpoint{6.065982in}{4.653643in}}%
\pgfpathlineto{\pgfqpoint{6.349717in}{4.760393in}}%
\pgfpathlineto{\pgfqpoint{6.633452in}{5.289700in}}%
\pgfusepath{stroke}%
\end{pgfscope}%
\begin{pgfscope}%
\pgfpathrectangle{\pgfqpoint{0.958750in}{0.688600in}}{\pgfqpoint{5.944250in}{4.820200in}}%
\pgfusepath{clip}%
\pgfsetrectcap%
\pgfsetroundjoin%
\pgfsetlinewidth{1.505625pt}%
\definecolor{currentstroke}{rgb}{0.086275,0.627451,0.521569}%
\pgfsetstrokecolor{currentstroke}%
\pgfsetdash{}{0pt}%
\pgfpathmoveto{\pgfqpoint{1.242485in}{0.907700in}}%
\pgfpathlineto{\pgfqpoint{1.526220in}{0.908692in}}%
\pgfpathlineto{\pgfqpoint{1.809955in}{0.910773in}}%
\pgfpathlineto{\pgfqpoint{2.093690in}{0.913310in}}%
\pgfpathlineto{\pgfqpoint{2.377425in}{0.918191in}}%
\pgfpathlineto{\pgfqpoint{2.661161in}{0.923426in}}%
\pgfpathlineto{\pgfqpoint{2.944896in}{0.920042in}}%
\pgfpathlineto{\pgfqpoint{3.228631in}{0.924664in}}%
\pgfpathlineto{\pgfqpoint{3.512366in}{0.928998in}}%
\pgfpathlineto{\pgfqpoint{3.796101in}{0.936557in}}%
\pgfpathlineto{\pgfqpoint{4.079836in}{0.947899in}}%
\pgfpathlineto{\pgfqpoint{4.363571in}{0.956411in}}%
\pgfpathlineto{\pgfqpoint{4.647306in}{0.953947in}}%
\pgfpathlineto{\pgfqpoint{4.931041in}{0.952572in}}%
\pgfpathlineto{\pgfqpoint{5.214776in}{0.959652in}}%
\pgfpathlineto{\pgfqpoint{5.498511in}{0.982402in}}%
\pgfpathlineto{\pgfqpoint{5.782246in}{1.014238in}}%
\pgfpathlineto{\pgfqpoint{6.065982in}{1.030540in}}%
\pgfpathlineto{\pgfqpoint{6.349717in}{1.028498in}}%
\pgfpathlineto{\pgfqpoint{6.633452in}{1.049330in}}%
\pgfusepath{stroke}%
\end{pgfscope}%
\begin{pgfscope}%
\pgfsetrectcap%
\pgfsetmiterjoin%
\pgfsetlinewidth{0.803000pt}%
\definecolor{currentstroke}{rgb}{0.000000,0.000000,0.000000}%
\pgfsetstrokecolor{currentstroke}%
\pgfsetdash{}{0pt}%
\pgfpathmoveto{\pgfqpoint{0.958750in}{0.688600in}}%
\pgfpathlineto{\pgfqpoint{0.958750in}{5.508800in}}%
\pgfusepath{stroke}%
\end{pgfscope}%
\begin{pgfscope}%
\pgfsetrectcap%
\pgfsetmiterjoin%
\pgfsetlinewidth{0.803000pt}%
\definecolor{currentstroke}{rgb}{0.000000,0.000000,0.000000}%
\pgfsetstrokecolor{currentstroke}%
\pgfsetdash{}{0pt}%
\pgfpathmoveto{\pgfqpoint{6.903000in}{0.688600in}}%
\pgfpathlineto{\pgfqpoint{6.903000in}{5.508800in}}%
\pgfusepath{stroke}%
\end{pgfscope}%
\begin{pgfscope}%
\pgfsetrectcap%
\pgfsetmiterjoin%
\pgfsetlinewidth{0.803000pt}%
\definecolor{currentstroke}{rgb}{0.000000,0.000000,0.000000}%
\pgfsetstrokecolor{currentstroke}%
\pgfsetdash{}{0pt}%
\pgfpathmoveto{\pgfqpoint{0.958750in}{0.688600in}}%
\pgfpathlineto{\pgfqpoint{6.903000in}{0.688600in}}%
\pgfusepath{stroke}%
\end{pgfscope}%
\begin{pgfscope}%
\pgfsetrectcap%
\pgfsetmiterjoin%
\pgfsetlinewidth{0.803000pt}%
\definecolor{currentstroke}{rgb}{0.000000,0.000000,0.000000}%
\pgfsetstrokecolor{currentstroke}%
\pgfsetdash{}{0pt}%
\pgfpathmoveto{\pgfqpoint{0.958750in}{5.508800in}}%
\pgfpathlineto{\pgfqpoint{6.903000in}{5.508800in}}%
\pgfusepath{stroke}%
\end{pgfscope}%
\begin{pgfscope}%
\pgfsetbuttcap%
\pgfsetmiterjoin%
\definecolor{currentfill}{rgb}{1.000000,1.000000,1.000000}%
\pgfsetfillcolor{currentfill}%
\pgfsetfillopacity{0.800000}%
\pgfsetlinewidth{1.003750pt}%
\definecolor{currentstroke}{rgb}{0.800000,0.800000,0.800000}%
\pgfsetstrokecolor{currentstroke}%
\pgfsetstrokeopacity{0.800000}%
\pgfsetdash{}{0pt}%
\pgfpathmoveto{\pgfqpoint{1.055972in}{4.989974in}}%
\pgfpathlineto{\pgfqpoint{2.204220in}{4.989974in}}%
\pgfpathquadraticcurveto{\pgfqpoint{2.231998in}{4.989974in}}{\pgfqpoint{2.231998in}{5.017752in}}%
\pgfpathlineto{\pgfqpoint{2.231998in}{5.411578in}}%
\pgfpathquadraticcurveto{\pgfqpoint{2.231998in}{5.439356in}}{\pgfqpoint{2.204220in}{5.439356in}}%
\pgfpathlineto{\pgfqpoint{1.055972in}{5.439356in}}%
\pgfpathquadraticcurveto{\pgfqpoint{1.028194in}{5.439356in}}{\pgfqpoint{1.028194in}{5.411578in}}%
\pgfpathlineto{\pgfqpoint{1.028194in}{5.017752in}}%
\pgfpathquadraticcurveto{\pgfqpoint{1.028194in}{4.989974in}}{\pgfqpoint{1.055972in}{4.989974in}}%
\pgfpathclose%
\pgfusepath{stroke,fill}%
\end{pgfscope}%
\begin{pgfscope}%
\pgfsetrectcap%
\pgfsetroundjoin%
\pgfsetlinewidth{1.505625pt}%
\definecolor{currentstroke}{rgb}{0.172549,0.243137,0.313725}%
\pgfsetstrokecolor{currentstroke}%
\pgfsetdash{}{0pt}%
\pgfpathmoveto{\pgfqpoint{1.083750in}{5.326888in}}%
\pgfpathlineto{\pgfqpoint{1.361528in}{5.326888in}}%
\pgfusepath{stroke}%
\end{pgfscope}%
\begin{pgfscope}%
\pgftext[x=1.472639in,y=5.278277in,left,base]{\sffamily\fontsize{10.000000}{12.000000}\selectfont \textsc{Linclosure}}%
\end{pgfscope}%
\begin{pgfscope}%
\pgfsetrectcap%
\pgfsetroundjoin%
\pgfsetlinewidth{1.505625pt}%
\definecolor{currentstroke}{rgb}{0.086275,0.627451,0.521569}%
\pgfsetstrokecolor{currentstroke}%
\pgfsetdash{}{0pt}%
\pgfpathmoveto{\pgfqpoint{1.083750in}{5.123031in}}%
\pgfpathlineto{\pgfqpoint{1.361528in}{5.123031in}}%
\pgfusepath{stroke}%
\end{pgfscope}%
\begin{pgfscope}%
\pgftext[x=1.472639in,y=5.074420in,left,base]{\sffamily\fontsize{10.000000}{12.000000}\selectfont \textsc{Closure}}%
\end{pgfscope}%
\end{pgfpicture}%
\makeatother%
\endgroup%
}
}
\end{minipage}
~
\begin{minipage}{0.5\textwidth}
\subfloat[Average time (in $s$), $|\B| = 100$]{
	\scalebox{0.46}{%% Creator: Matplotlib, PGF backend
%%
%% To include the figure in your LaTeX document, write
%%   \input{<filename>.pgf}
%%
%% Make sure the required packages are loaded in your preamble
%%   \usepackage{pgf}
%%
%% Figures using additional raster images can only be included by \input if
%% they are in the same directory as the main LaTeX file. For loading figures
%% from other directories you can use the `import` package
%%   \usepackage{import}
%% and then include the figures with
%%   \import{<path to file>}{<filename>.pgf}
%%
%% Matplotlib used the following preamble
%%   \usepackage{fontspec}
%%   \setmainfont{DejaVu Serif}
%%   \setsansfont{DejaVu Sans}
%%   \setmonofont{DejaVu Sans Mono}
%%
\begingroup%
\makeatletter%
\begin{pgfpicture}%
\pgfpathrectangle{\pgfpointorigin}{\pgfqpoint{7.190000in}{5.660000in}}%
\pgfusepath{use as bounding box, clip}%
\begin{pgfscope}%
\pgfsetbuttcap%
\pgfsetmiterjoin%
\definecolor{currentfill}{rgb}{1.000000,1.000000,1.000000}%
\pgfsetfillcolor{currentfill}%
\pgfsetlinewidth{0.000000pt}%
\definecolor{currentstroke}{rgb}{1.000000,1.000000,1.000000}%
\pgfsetstrokecolor{currentstroke}%
\pgfsetdash{}{0pt}%
\pgfpathmoveto{\pgfqpoint{0.000000in}{0.000000in}}%
\pgfpathlineto{\pgfqpoint{7.190000in}{0.000000in}}%
\pgfpathlineto{\pgfqpoint{7.190000in}{5.660000in}}%
\pgfpathlineto{\pgfqpoint{0.000000in}{5.660000in}}%
\pgfpathclose%
\pgfusepath{fill}%
\end{pgfscope}%
\begin{pgfscope}%
\pgfsetbuttcap%
\pgfsetmiterjoin%
\definecolor{currentfill}{rgb}{1.000000,1.000000,1.000000}%
\pgfsetfillcolor{currentfill}%
\pgfsetlinewidth{0.000000pt}%
\definecolor{currentstroke}{rgb}{0.000000,0.000000,0.000000}%
\pgfsetstrokecolor{currentstroke}%
\pgfsetstrokeopacity{0.000000}%
\pgfsetdash{}{0pt}%
\pgfpathmoveto{\pgfqpoint{0.898750in}{0.622600in}}%
\pgfpathlineto{\pgfqpoint{6.471000in}{0.622600in}}%
\pgfpathlineto{\pgfqpoint{6.471000in}{4.980800in}}%
\pgfpathlineto{\pgfqpoint{0.898750in}{4.980800in}}%
\pgfpathclose%
\pgfusepath{fill}%
\end{pgfscope}%
\begin{pgfscope}%
\pgfsetbuttcap%
\pgfsetroundjoin%
\definecolor{currentfill}{rgb}{0.000000,0.000000,0.000000}%
\pgfsetfillcolor{currentfill}%
\pgfsetlinewidth{0.803000pt}%
\definecolor{currentstroke}{rgb}{0.000000,0.000000,0.000000}%
\pgfsetstrokecolor{currentstroke}%
\pgfsetdash{}{0pt}%
\pgfsys@defobject{currentmarker}{\pgfqpoint{0.000000in}{-0.048611in}}{\pgfqpoint{0.000000in}{0.000000in}}{%
\pgfpathmoveto{\pgfqpoint{0.000000in}{0.000000in}}%
\pgfpathlineto{\pgfqpoint{0.000000in}{-0.048611in}}%
\pgfusepath{stroke,fill}%
}%
\begin{pgfscope}%
\pgfsys@transformshift{1.100866in}{0.622600in}%
\pgfsys@useobject{currentmarker}{}%
\end{pgfscope}%
\end{pgfscope}%
\begin{pgfscope}%
\pgftext[x=1.100866in,y=0.525378in,,top]{\sffamily\fontsize{10.000000}{12.000000}\selectfont \(\displaystyle 0\)}%
\end{pgfscope}%
\begin{pgfscope}%
\pgfsetbuttcap%
\pgfsetroundjoin%
\definecolor{currentfill}{rgb}{0.000000,0.000000,0.000000}%
\pgfsetfillcolor{currentfill}%
\pgfsetlinewidth{0.803000pt}%
\definecolor{currentstroke}{rgb}{0.000000,0.000000,0.000000}%
\pgfsetstrokecolor{currentstroke}%
\pgfsetdash{}{0pt}%
\pgfsys@defobject{currentmarker}{\pgfqpoint{0.000000in}{-0.048611in}}{\pgfqpoint{0.000000in}{0.000000in}}{%
\pgfpathmoveto{\pgfqpoint{0.000000in}{0.000000in}}%
\pgfpathlineto{\pgfqpoint{0.000000in}{-0.048611in}}%
\pgfusepath{stroke,fill}%
}%
\begin{pgfscope}%
\pgfsys@transformshift{1.612551in}{0.622600in}%
\pgfsys@useobject{currentmarker}{}%
\end{pgfscope}%
\end{pgfscope}%
\begin{pgfscope}%
\pgftext[x=1.612551in,y=0.525378in,,top]{\sffamily\fontsize{10.000000}{12.000000}\selectfont \(\displaystyle 10000\)}%
\end{pgfscope}%
\begin{pgfscope}%
\pgfsetbuttcap%
\pgfsetroundjoin%
\definecolor{currentfill}{rgb}{0.000000,0.000000,0.000000}%
\pgfsetfillcolor{currentfill}%
\pgfsetlinewidth{0.803000pt}%
\definecolor{currentstroke}{rgb}{0.000000,0.000000,0.000000}%
\pgfsetstrokecolor{currentstroke}%
\pgfsetdash{}{0pt}%
\pgfsys@defobject{currentmarker}{\pgfqpoint{0.000000in}{-0.048611in}}{\pgfqpoint{0.000000in}{0.000000in}}{%
\pgfpathmoveto{\pgfqpoint{0.000000in}{0.000000in}}%
\pgfpathlineto{\pgfqpoint{0.000000in}{-0.048611in}}%
\pgfusepath{stroke,fill}%
}%
\begin{pgfscope}%
\pgfsys@transformshift{2.124236in}{0.622600in}%
\pgfsys@useobject{currentmarker}{}%
\end{pgfscope}%
\end{pgfscope}%
\begin{pgfscope}%
\pgftext[x=2.124236in,y=0.525378in,,top]{\sffamily\fontsize{10.000000}{12.000000}\selectfont \(\displaystyle 20000\)}%
\end{pgfscope}%
\begin{pgfscope}%
\pgfsetbuttcap%
\pgfsetroundjoin%
\definecolor{currentfill}{rgb}{0.000000,0.000000,0.000000}%
\pgfsetfillcolor{currentfill}%
\pgfsetlinewidth{0.803000pt}%
\definecolor{currentstroke}{rgb}{0.000000,0.000000,0.000000}%
\pgfsetstrokecolor{currentstroke}%
\pgfsetdash{}{0pt}%
\pgfsys@defobject{currentmarker}{\pgfqpoint{0.000000in}{-0.048611in}}{\pgfqpoint{0.000000in}{0.000000in}}{%
\pgfpathmoveto{\pgfqpoint{0.000000in}{0.000000in}}%
\pgfpathlineto{\pgfqpoint{0.000000in}{-0.048611in}}%
\pgfusepath{stroke,fill}%
}%
\begin{pgfscope}%
\pgfsys@transformshift{2.635921in}{0.622600in}%
\pgfsys@useobject{currentmarker}{}%
\end{pgfscope}%
\end{pgfscope}%
\begin{pgfscope}%
\pgftext[x=2.635921in,y=0.525378in,,top]{\sffamily\fontsize{10.000000}{12.000000}\selectfont \(\displaystyle 30000\)}%
\end{pgfscope}%
\begin{pgfscope}%
\pgfsetbuttcap%
\pgfsetroundjoin%
\definecolor{currentfill}{rgb}{0.000000,0.000000,0.000000}%
\pgfsetfillcolor{currentfill}%
\pgfsetlinewidth{0.803000pt}%
\definecolor{currentstroke}{rgb}{0.000000,0.000000,0.000000}%
\pgfsetstrokecolor{currentstroke}%
\pgfsetdash{}{0pt}%
\pgfsys@defobject{currentmarker}{\pgfqpoint{0.000000in}{-0.048611in}}{\pgfqpoint{0.000000in}{0.000000in}}{%
\pgfpathmoveto{\pgfqpoint{0.000000in}{0.000000in}}%
\pgfpathlineto{\pgfqpoint{0.000000in}{-0.048611in}}%
\pgfusepath{stroke,fill}%
}%
\begin{pgfscope}%
\pgfsys@transformshift{3.147606in}{0.622600in}%
\pgfsys@useobject{currentmarker}{}%
\end{pgfscope}%
\end{pgfscope}%
\begin{pgfscope}%
\pgftext[x=3.147606in,y=0.525378in,,top]{\sffamily\fontsize{10.000000}{12.000000}\selectfont \(\displaystyle 40000\)}%
\end{pgfscope}%
\begin{pgfscope}%
\pgfsetbuttcap%
\pgfsetroundjoin%
\definecolor{currentfill}{rgb}{0.000000,0.000000,0.000000}%
\pgfsetfillcolor{currentfill}%
\pgfsetlinewidth{0.803000pt}%
\definecolor{currentstroke}{rgb}{0.000000,0.000000,0.000000}%
\pgfsetstrokecolor{currentstroke}%
\pgfsetdash{}{0pt}%
\pgfsys@defobject{currentmarker}{\pgfqpoint{0.000000in}{-0.048611in}}{\pgfqpoint{0.000000in}{0.000000in}}{%
\pgfpathmoveto{\pgfqpoint{0.000000in}{0.000000in}}%
\pgfpathlineto{\pgfqpoint{0.000000in}{-0.048611in}}%
\pgfusepath{stroke,fill}%
}%
\begin{pgfscope}%
\pgfsys@transformshift{3.659291in}{0.622600in}%
\pgfsys@useobject{currentmarker}{}%
\end{pgfscope}%
\end{pgfscope}%
\begin{pgfscope}%
\pgftext[x=3.659291in,y=0.525378in,,top]{\sffamily\fontsize{10.000000}{12.000000}\selectfont \(\displaystyle 50000\)}%
\end{pgfscope}%
\begin{pgfscope}%
\pgfsetbuttcap%
\pgfsetroundjoin%
\definecolor{currentfill}{rgb}{0.000000,0.000000,0.000000}%
\pgfsetfillcolor{currentfill}%
\pgfsetlinewidth{0.803000pt}%
\definecolor{currentstroke}{rgb}{0.000000,0.000000,0.000000}%
\pgfsetstrokecolor{currentstroke}%
\pgfsetdash{}{0pt}%
\pgfsys@defobject{currentmarker}{\pgfqpoint{0.000000in}{-0.048611in}}{\pgfqpoint{0.000000in}{0.000000in}}{%
\pgfpathmoveto{\pgfqpoint{0.000000in}{0.000000in}}%
\pgfpathlineto{\pgfqpoint{0.000000in}{-0.048611in}}%
\pgfusepath{stroke,fill}%
}%
\begin{pgfscope}%
\pgfsys@transformshift{4.170976in}{0.622600in}%
\pgfsys@useobject{currentmarker}{}%
\end{pgfscope}%
\end{pgfscope}%
\begin{pgfscope}%
\pgftext[x=4.170976in,y=0.525378in,,top]{\sffamily\fontsize{10.000000}{12.000000}\selectfont \(\displaystyle 60000\)}%
\end{pgfscope}%
\begin{pgfscope}%
\pgfsetbuttcap%
\pgfsetroundjoin%
\definecolor{currentfill}{rgb}{0.000000,0.000000,0.000000}%
\pgfsetfillcolor{currentfill}%
\pgfsetlinewidth{0.803000pt}%
\definecolor{currentstroke}{rgb}{0.000000,0.000000,0.000000}%
\pgfsetstrokecolor{currentstroke}%
\pgfsetdash{}{0pt}%
\pgfsys@defobject{currentmarker}{\pgfqpoint{0.000000in}{-0.048611in}}{\pgfqpoint{0.000000in}{0.000000in}}{%
\pgfpathmoveto{\pgfqpoint{0.000000in}{0.000000in}}%
\pgfpathlineto{\pgfqpoint{0.000000in}{-0.048611in}}%
\pgfusepath{stroke,fill}%
}%
\begin{pgfscope}%
\pgfsys@transformshift{4.682661in}{0.622600in}%
\pgfsys@useobject{currentmarker}{}%
\end{pgfscope}%
\end{pgfscope}%
\begin{pgfscope}%
\pgftext[x=4.682661in,y=0.525378in,,top]{\sffamily\fontsize{10.000000}{12.000000}\selectfont \(\displaystyle 70000\)}%
\end{pgfscope}%
\begin{pgfscope}%
\pgfsetbuttcap%
\pgfsetroundjoin%
\definecolor{currentfill}{rgb}{0.000000,0.000000,0.000000}%
\pgfsetfillcolor{currentfill}%
\pgfsetlinewidth{0.803000pt}%
\definecolor{currentstroke}{rgb}{0.000000,0.000000,0.000000}%
\pgfsetstrokecolor{currentstroke}%
\pgfsetdash{}{0pt}%
\pgfsys@defobject{currentmarker}{\pgfqpoint{0.000000in}{-0.048611in}}{\pgfqpoint{0.000000in}{0.000000in}}{%
\pgfpathmoveto{\pgfqpoint{0.000000in}{0.000000in}}%
\pgfpathlineto{\pgfqpoint{0.000000in}{-0.048611in}}%
\pgfusepath{stroke,fill}%
}%
\begin{pgfscope}%
\pgfsys@transformshift{5.194346in}{0.622600in}%
\pgfsys@useobject{currentmarker}{}%
\end{pgfscope}%
\end{pgfscope}%
\begin{pgfscope}%
\pgftext[x=5.194346in,y=0.525378in,,top]{\sffamily\fontsize{10.000000}{12.000000}\selectfont \(\displaystyle 80000\)}%
\end{pgfscope}%
\begin{pgfscope}%
\pgfsetbuttcap%
\pgfsetroundjoin%
\definecolor{currentfill}{rgb}{0.000000,0.000000,0.000000}%
\pgfsetfillcolor{currentfill}%
\pgfsetlinewidth{0.803000pt}%
\definecolor{currentstroke}{rgb}{0.000000,0.000000,0.000000}%
\pgfsetstrokecolor{currentstroke}%
\pgfsetdash{}{0pt}%
\pgfsys@defobject{currentmarker}{\pgfqpoint{0.000000in}{-0.048611in}}{\pgfqpoint{0.000000in}{0.000000in}}{%
\pgfpathmoveto{\pgfqpoint{0.000000in}{0.000000in}}%
\pgfpathlineto{\pgfqpoint{0.000000in}{-0.048611in}}%
\pgfusepath{stroke,fill}%
}%
\begin{pgfscope}%
\pgfsys@transformshift{5.706031in}{0.622600in}%
\pgfsys@useobject{currentmarker}{}%
\end{pgfscope}%
\end{pgfscope}%
\begin{pgfscope}%
\pgftext[x=5.706031in,y=0.525378in,,top]{\sffamily\fontsize{10.000000}{12.000000}\selectfont \(\displaystyle 90000\)}%
\end{pgfscope}%
\begin{pgfscope}%
\pgfsetbuttcap%
\pgfsetroundjoin%
\definecolor{currentfill}{rgb}{0.000000,0.000000,0.000000}%
\pgfsetfillcolor{currentfill}%
\pgfsetlinewidth{0.803000pt}%
\definecolor{currentstroke}{rgb}{0.000000,0.000000,0.000000}%
\pgfsetstrokecolor{currentstroke}%
\pgfsetdash{}{0pt}%
\pgfsys@defobject{currentmarker}{\pgfqpoint{0.000000in}{-0.048611in}}{\pgfqpoint{0.000000in}{0.000000in}}{%
\pgfpathmoveto{\pgfqpoint{0.000000in}{0.000000in}}%
\pgfpathlineto{\pgfqpoint{0.000000in}{-0.048611in}}%
\pgfusepath{stroke,fill}%
}%
\begin{pgfscope}%
\pgfsys@transformshift{6.217716in}{0.622600in}%
\pgfsys@useobject{currentmarker}{}%
\end{pgfscope}%
\end{pgfscope}%
\begin{pgfscope}%
\pgftext[x=6.217716in,y=0.525378in,,top]{\sffamily\fontsize{10.000000}{12.000000}\selectfont \(\displaystyle 100000\)}%
\end{pgfscope}%
\begin{pgfscope}%
\pgftext[x=3.684875in,y=0.335409in,,top]{\sffamily\fontsize{14.000000}{16.800000}\selectfont \(\displaystyle |\Sigma|\)}%
\end{pgfscope}%
\begin{pgfscope}%
\pgfsetbuttcap%
\pgfsetroundjoin%
\definecolor{currentfill}{rgb}{0.000000,0.000000,0.000000}%
\pgfsetfillcolor{currentfill}%
\pgfsetlinewidth{0.803000pt}%
\definecolor{currentstroke}{rgb}{0.000000,0.000000,0.000000}%
\pgfsetstrokecolor{currentstroke}%
\pgfsetdash{}{0pt}%
\pgfsys@defobject{currentmarker}{\pgfqpoint{-0.048611in}{0.000000in}}{\pgfqpoint{0.000000in}{0.000000in}}{%
\pgfpathmoveto{\pgfqpoint{0.000000in}{0.000000in}}%
\pgfpathlineto{\pgfqpoint{-0.048611in}{0.000000in}}%
\pgfusepath{stroke,fill}%
}%
\begin{pgfscope}%
\pgfsys@transformshift{0.898750in}{0.819937in}%
\pgfsys@useobject{currentmarker}{}%
\end{pgfscope}%
\end{pgfscope}%
\begin{pgfscope}%
\pgftext[x=0.624058in,y=0.767175in,left,base]{\sffamily\fontsize{10.000000}{12.000000}\selectfont \(\displaystyle 0.0\)}%
\end{pgfscope}%
\begin{pgfscope}%
\pgfsetbuttcap%
\pgfsetroundjoin%
\definecolor{currentfill}{rgb}{0.000000,0.000000,0.000000}%
\pgfsetfillcolor{currentfill}%
\pgfsetlinewidth{0.803000pt}%
\definecolor{currentstroke}{rgb}{0.000000,0.000000,0.000000}%
\pgfsetstrokecolor{currentstroke}%
\pgfsetdash{}{0pt}%
\pgfsys@defobject{currentmarker}{\pgfqpoint{-0.048611in}{0.000000in}}{\pgfqpoint{0.000000in}{0.000000in}}{%
\pgfpathmoveto{\pgfqpoint{0.000000in}{0.000000in}}%
\pgfpathlineto{\pgfqpoint{-0.048611in}{0.000000in}}%
\pgfusepath{stroke,fill}%
}%
\begin{pgfscope}%
\pgfsys@transformshift{0.898750in}{1.064192in}%
\pgfsys@useobject{currentmarker}{}%
\end{pgfscope}%
\end{pgfscope}%
\begin{pgfscope}%
\pgftext[x=0.624058in,y=1.011431in,left,base]{\sffamily\fontsize{10.000000}{12.000000}\selectfont \(\displaystyle 0.4\)}%
\end{pgfscope}%
\begin{pgfscope}%
\pgfsetbuttcap%
\pgfsetroundjoin%
\definecolor{currentfill}{rgb}{0.000000,0.000000,0.000000}%
\pgfsetfillcolor{currentfill}%
\pgfsetlinewidth{0.803000pt}%
\definecolor{currentstroke}{rgb}{0.000000,0.000000,0.000000}%
\pgfsetstrokecolor{currentstroke}%
\pgfsetdash{}{0pt}%
\pgfsys@defobject{currentmarker}{\pgfqpoint{-0.048611in}{0.000000in}}{\pgfqpoint{0.000000in}{0.000000in}}{%
\pgfpathmoveto{\pgfqpoint{0.000000in}{0.000000in}}%
\pgfpathlineto{\pgfqpoint{-0.048611in}{0.000000in}}%
\pgfusepath{stroke,fill}%
}%
\begin{pgfscope}%
\pgfsys@transformshift{0.898750in}{1.308448in}%
\pgfsys@useobject{currentmarker}{}%
\end{pgfscope}%
\end{pgfscope}%
\begin{pgfscope}%
\pgftext[x=0.624058in,y=1.255687in,left,base]{\sffamily\fontsize{10.000000}{12.000000}\selectfont \(\displaystyle 0.8\)}%
\end{pgfscope}%
\begin{pgfscope}%
\pgfsetbuttcap%
\pgfsetroundjoin%
\definecolor{currentfill}{rgb}{0.000000,0.000000,0.000000}%
\pgfsetfillcolor{currentfill}%
\pgfsetlinewidth{0.803000pt}%
\definecolor{currentstroke}{rgb}{0.000000,0.000000,0.000000}%
\pgfsetstrokecolor{currentstroke}%
\pgfsetdash{}{0pt}%
\pgfsys@defobject{currentmarker}{\pgfqpoint{-0.048611in}{0.000000in}}{\pgfqpoint{0.000000in}{0.000000in}}{%
\pgfpathmoveto{\pgfqpoint{0.000000in}{0.000000in}}%
\pgfpathlineto{\pgfqpoint{-0.048611in}{0.000000in}}%
\pgfusepath{stroke,fill}%
}%
\begin{pgfscope}%
\pgfsys@transformshift{0.898750in}{1.552704in}%
\pgfsys@useobject{currentmarker}{}%
\end{pgfscope}%
\end{pgfscope}%
\begin{pgfscope}%
\pgftext[x=0.624058in,y=1.499943in,left,base]{\sffamily\fontsize{10.000000}{12.000000}\selectfont \(\displaystyle 1.2\)}%
\end{pgfscope}%
\begin{pgfscope}%
\pgfsetbuttcap%
\pgfsetroundjoin%
\definecolor{currentfill}{rgb}{0.000000,0.000000,0.000000}%
\pgfsetfillcolor{currentfill}%
\pgfsetlinewidth{0.803000pt}%
\definecolor{currentstroke}{rgb}{0.000000,0.000000,0.000000}%
\pgfsetstrokecolor{currentstroke}%
\pgfsetdash{}{0pt}%
\pgfsys@defobject{currentmarker}{\pgfqpoint{-0.048611in}{0.000000in}}{\pgfqpoint{0.000000in}{0.000000in}}{%
\pgfpathmoveto{\pgfqpoint{0.000000in}{0.000000in}}%
\pgfpathlineto{\pgfqpoint{-0.048611in}{0.000000in}}%
\pgfusepath{stroke,fill}%
}%
\begin{pgfscope}%
\pgfsys@transformshift{0.898750in}{1.796960in}%
\pgfsys@useobject{currentmarker}{}%
\end{pgfscope}%
\end{pgfscope}%
\begin{pgfscope}%
\pgftext[x=0.624058in,y=1.744198in,left,base]{\sffamily\fontsize{10.000000}{12.000000}\selectfont \(\displaystyle 1.6\)}%
\end{pgfscope}%
\begin{pgfscope}%
\pgfsetbuttcap%
\pgfsetroundjoin%
\definecolor{currentfill}{rgb}{0.000000,0.000000,0.000000}%
\pgfsetfillcolor{currentfill}%
\pgfsetlinewidth{0.803000pt}%
\definecolor{currentstroke}{rgb}{0.000000,0.000000,0.000000}%
\pgfsetstrokecolor{currentstroke}%
\pgfsetdash{}{0pt}%
\pgfsys@defobject{currentmarker}{\pgfqpoint{-0.048611in}{0.000000in}}{\pgfqpoint{0.000000in}{0.000000in}}{%
\pgfpathmoveto{\pgfqpoint{0.000000in}{0.000000in}}%
\pgfpathlineto{\pgfqpoint{-0.048611in}{0.000000in}}%
\pgfusepath{stroke,fill}%
}%
\begin{pgfscope}%
\pgfsys@transformshift{0.898750in}{2.041216in}%
\pgfsys@useobject{currentmarker}{}%
\end{pgfscope}%
\end{pgfscope}%
\begin{pgfscope}%
\pgftext[x=0.624058in,y=1.988454in,left,base]{\sffamily\fontsize{10.000000}{12.000000}\selectfont \(\displaystyle 2.0\)}%
\end{pgfscope}%
\begin{pgfscope}%
\pgfsetbuttcap%
\pgfsetroundjoin%
\definecolor{currentfill}{rgb}{0.000000,0.000000,0.000000}%
\pgfsetfillcolor{currentfill}%
\pgfsetlinewidth{0.803000pt}%
\definecolor{currentstroke}{rgb}{0.000000,0.000000,0.000000}%
\pgfsetstrokecolor{currentstroke}%
\pgfsetdash{}{0pt}%
\pgfsys@defobject{currentmarker}{\pgfqpoint{-0.048611in}{0.000000in}}{\pgfqpoint{0.000000in}{0.000000in}}{%
\pgfpathmoveto{\pgfqpoint{0.000000in}{0.000000in}}%
\pgfpathlineto{\pgfqpoint{-0.048611in}{0.000000in}}%
\pgfusepath{stroke,fill}%
}%
\begin{pgfscope}%
\pgfsys@transformshift{0.898750in}{2.285471in}%
\pgfsys@useobject{currentmarker}{}%
\end{pgfscope}%
\end{pgfscope}%
\begin{pgfscope}%
\pgftext[x=0.624058in,y=2.232710in,left,base]{\sffamily\fontsize{10.000000}{12.000000}\selectfont \(\displaystyle 2.4\)}%
\end{pgfscope}%
\begin{pgfscope}%
\pgfsetbuttcap%
\pgfsetroundjoin%
\definecolor{currentfill}{rgb}{0.000000,0.000000,0.000000}%
\pgfsetfillcolor{currentfill}%
\pgfsetlinewidth{0.803000pt}%
\definecolor{currentstroke}{rgb}{0.000000,0.000000,0.000000}%
\pgfsetstrokecolor{currentstroke}%
\pgfsetdash{}{0pt}%
\pgfsys@defobject{currentmarker}{\pgfqpoint{-0.048611in}{0.000000in}}{\pgfqpoint{0.000000in}{0.000000in}}{%
\pgfpathmoveto{\pgfqpoint{0.000000in}{0.000000in}}%
\pgfpathlineto{\pgfqpoint{-0.048611in}{0.000000in}}%
\pgfusepath{stroke,fill}%
}%
\begin{pgfscope}%
\pgfsys@transformshift{0.898750in}{2.529727in}%
\pgfsys@useobject{currentmarker}{}%
\end{pgfscope}%
\end{pgfscope}%
\begin{pgfscope}%
\pgftext[x=0.624058in,y=2.476966in,left,base]{\sffamily\fontsize{10.000000}{12.000000}\selectfont \(\displaystyle 2.8\)}%
\end{pgfscope}%
\begin{pgfscope}%
\pgfsetbuttcap%
\pgfsetroundjoin%
\definecolor{currentfill}{rgb}{0.000000,0.000000,0.000000}%
\pgfsetfillcolor{currentfill}%
\pgfsetlinewidth{0.803000pt}%
\definecolor{currentstroke}{rgb}{0.000000,0.000000,0.000000}%
\pgfsetstrokecolor{currentstroke}%
\pgfsetdash{}{0pt}%
\pgfsys@defobject{currentmarker}{\pgfqpoint{-0.048611in}{0.000000in}}{\pgfqpoint{0.000000in}{0.000000in}}{%
\pgfpathmoveto{\pgfqpoint{0.000000in}{0.000000in}}%
\pgfpathlineto{\pgfqpoint{-0.048611in}{0.000000in}}%
\pgfusepath{stroke,fill}%
}%
\begin{pgfscope}%
\pgfsys@transformshift{0.898750in}{2.773983in}%
\pgfsys@useobject{currentmarker}{}%
\end{pgfscope}%
\end{pgfscope}%
\begin{pgfscope}%
\pgftext[x=0.624058in,y=2.721222in,left,base]{\sffamily\fontsize{10.000000}{12.000000}\selectfont \(\displaystyle 3.2\)}%
\end{pgfscope}%
\begin{pgfscope}%
\pgfsetbuttcap%
\pgfsetroundjoin%
\definecolor{currentfill}{rgb}{0.000000,0.000000,0.000000}%
\pgfsetfillcolor{currentfill}%
\pgfsetlinewidth{0.803000pt}%
\definecolor{currentstroke}{rgb}{0.000000,0.000000,0.000000}%
\pgfsetstrokecolor{currentstroke}%
\pgfsetdash{}{0pt}%
\pgfsys@defobject{currentmarker}{\pgfqpoint{-0.048611in}{0.000000in}}{\pgfqpoint{0.000000in}{0.000000in}}{%
\pgfpathmoveto{\pgfqpoint{0.000000in}{0.000000in}}%
\pgfpathlineto{\pgfqpoint{-0.048611in}{0.000000in}}%
\pgfusepath{stroke,fill}%
}%
\begin{pgfscope}%
\pgfsys@transformshift{0.898750in}{3.018239in}%
\pgfsys@useobject{currentmarker}{}%
\end{pgfscope}%
\end{pgfscope}%
\begin{pgfscope}%
\pgftext[x=0.624058in,y=2.965477in,left,base]{\sffamily\fontsize{10.000000}{12.000000}\selectfont \(\displaystyle 3.6\)}%
\end{pgfscope}%
\begin{pgfscope}%
\pgfsetbuttcap%
\pgfsetroundjoin%
\definecolor{currentfill}{rgb}{0.000000,0.000000,0.000000}%
\pgfsetfillcolor{currentfill}%
\pgfsetlinewidth{0.803000pt}%
\definecolor{currentstroke}{rgb}{0.000000,0.000000,0.000000}%
\pgfsetstrokecolor{currentstroke}%
\pgfsetdash{}{0pt}%
\pgfsys@defobject{currentmarker}{\pgfqpoint{-0.048611in}{0.000000in}}{\pgfqpoint{0.000000in}{0.000000in}}{%
\pgfpathmoveto{\pgfqpoint{0.000000in}{0.000000in}}%
\pgfpathlineto{\pgfqpoint{-0.048611in}{0.000000in}}%
\pgfusepath{stroke,fill}%
}%
\begin{pgfscope}%
\pgfsys@transformshift{0.898750in}{3.262495in}%
\pgfsys@useobject{currentmarker}{}%
\end{pgfscope}%
\end{pgfscope}%
\begin{pgfscope}%
\pgftext[x=0.624058in,y=3.209733in,left,base]{\sffamily\fontsize{10.000000}{12.000000}\selectfont \(\displaystyle 4.0\)}%
\end{pgfscope}%
\begin{pgfscope}%
\pgfsetbuttcap%
\pgfsetroundjoin%
\definecolor{currentfill}{rgb}{0.000000,0.000000,0.000000}%
\pgfsetfillcolor{currentfill}%
\pgfsetlinewidth{0.803000pt}%
\definecolor{currentstroke}{rgb}{0.000000,0.000000,0.000000}%
\pgfsetstrokecolor{currentstroke}%
\pgfsetdash{}{0pt}%
\pgfsys@defobject{currentmarker}{\pgfqpoint{-0.048611in}{0.000000in}}{\pgfqpoint{0.000000in}{0.000000in}}{%
\pgfpathmoveto{\pgfqpoint{0.000000in}{0.000000in}}%
\pgfpathlineto{\pgfqpoint{-0.048611in}{0.000000in}}%
\pgfusepath{stroke,fill}%
}%
\begin{pgfscope}%
\pgfsys@transformshift{0.898750in}{3.506750in}%
\pgfsys@useobject{currentmarker}{}%
\end{pgfscope}%
\end{pgfscope}%
\begin{pgfscope}%
\pgftext[x=0.624058in,y=3.453989in,left,base]{\sffamily\fontsize{10.000000}{12.000000}\selectfont \(\displaystyle 4.4\)}%
\end{pgfscope}%
\begin{pgfscope}%
\pgfsetbuttcap%
\pgfsetroundjoin%
\definecolor{currentfill}{rgb}{0.000000,0.000000,0.000000}%
\pgfsetfillcolor{currentfill}%
\pgfsetlinewidth{0.803000pt}%
\definecolor{currentstroke}{rgb}{0.000000,0.000000,0.000000}%
\pgfsetstrokecolor{currentstroke}%
\pgfsetdash{}{0pt}%
\pgfsys@defobject{currentmarker}{\pgfqpoint{-0.048611in}{0.000000in}}{\pgfqpoint{0.000000in}{0.000000in}}{%
\pgfpathmoveto{\pgfqpoint{0.000000in}{0.000000in}}%
\pgfpathlineto{\pgfqpoint{-0.048611in}{0.000000in}}%
\pgfusepath{stroke,fill}%
}%
\begin{pgfscope}%
\pgfsys@transformshift{0.898750in}{3.751006in}%
\pgfsys@useobject{currentmarker}{}%
\end{pgfscope}%
\end{pgfscope}%
\begin{pgfscope}%
\pgftext[x=0.624058in,y=3.698245in,left,base]{\sffamily\fontsize{10.000000}{12.000000}\selectfont \(\displaystyle 4.8\)}%
\end{pgfscope}%
\begin{pgfscope}%
\pgfsetbuttcap%
\pgfsetroundjoin%
\definecolor{currentfill}{rgb}{0.000000,0.000000,0.000000}%
\pgfsetfillcolor{currentfill}%
\pgfsetlinewidth{0.803000pt}%
\definecolor{currentstroke}{rgb}{0.000000,0.000000,0.000000}%
\pgfsetstrokecolor{currentstroke}%
\pgfsetdash{}{0pt}%
\pgfsys@defobject{currentmarker}{\pgfqpoint{-0.048611in}{0.000000in}}{\pgfqpoint{0.000000in}{0.000000in}}{%
\pgfpathmoveto{\pgfqpoint{0.000000in}{0.000000in}}%
\pgfpathlineto{\pgfqpoint{-0.048611in}{0.000000in}}%
\pgfusepath{stroke,fill}%
}%
\begin{pgfscope}%
\pgfsys@transformshift{0.898750in}{3.995262in}%
\pgfsys@useobject{currentmarker}{}%
\end{pgfscope}%
\end{pgfscope}%
\begin{pgfscope}%
\pgftext[x=0.624058in,y=3.942501in,left,base]{\sffamily\fontsize{10.000000}{12.000000}\selectfont \(\displaystyle 5.2\)}%
\end{pgfscope}%
\begin{pgfscope}%
\pgfsetbuttcap%
\pgfsetroundjoin%
\definecolor{currentfill}{rgb}{0.000000,0.000000,0.000000}%
\pgfsetfillcolor{currentfill}%
\pgfsetlinewidth{0.803000pt}%
\definecolor{currentstroke}{rgb}{0.000000,0.000000,0.000000}%
\pgfsetstrokecolor{currentstroke}%
\pgfsetdash{}{0pt}%
\pgfsys@defobject{currentmarker}{\pgfqpoint{-0.048611in}{0.000000in}}{\pgfqpoint{0.000000in}{0.000000in}}{%
\pgfpathmoveto{\pgfqpoint{0.000000in}{0.000000in}}%
\pgfpathlineto{\pgfqpoint{-0.048611in}{0.000000in}}%
\pgfusepath{stroke,fill}%
}%
\begin{pgfscope}%
\pgfsys@transformshift{0.898750in}{4.239518in}%
\pgfsys@useobject{currentmarker}{}%
\end{pgfscope}%
\end{pgfscope}%
\begin{pgfscope}%
\pgftext[x=0.624058in,y=4.186756in,left,base]{\sffamily\fontsize{10.000000}{12.000000}\selectfont \(\displaystyle 5.6\)}%
\end{pgfscope}%
\begin{pgfscope}%
\pgfsetbuttcap%
\pgfsetroundjoin%
\definecolor{currentfill}{rgb}{0.000000,0.000000,0.000000}%
\pgfsetfillcolor{currentfill}%
\pgfsetlinewidth{0.803000pt}%
\definecolor{currentstroke}{rgb}{0.000000,0.000000,0.000000}%
\pgfsetstrokecolor{currentstroke}%
\pgfsetdash{}{0pt}%
\pgfsys@defobject{currentmarker}{\pgfqpoint{-0.048611in}{0.000000in}}{\pgfqpoint{0.000000in}{0.000000in}}{%
\pgfpathmoveto{\pgfqpoint{0.000000in}{0.000000in}}%
\pgfpathlineto{\pgfqpoint{-0.048611in}{0.000000in}}%
\pgfusepath{stroke,fill}%
}%
\begin{pgfscope}%
\pgfsys@transformshift{0.898750in}{4.483774in}%
\pgfsys@useobject{currentmarker}{}%
\end{pgfscope}%
\end{pgfscope}%
\begin{pgfscope}%
\pgftext[x=0.624058in,y=4.431012in,left,base]{\sffamily\fontsize{10.000000}{12.000000}\selectfont \(\displaystyle 6.0\)}%
\end{pgfscope}%
\begin{pgfscope}%
\pgfsetbuttcap%
\pgfsetroundjoin%
\definecolor{currentfill}{rgb}{0.000000,0.000000,0.000000}%
\pgfsetfillcolor{currentfill}%
\pgfsetlinewidth{0.803000pt}%
\definecolor{currentstroke}{rgb}{0.000000,0.000000,0.000000}%
\pgfsetstrokecolor{currentstroke}%
\pgfsetdash{}{0pt}%
\pgfsys@defobject{currentmarker}{\pgfqpoint{-0.048611in}{0.000000in}}{\pgfqpoint{0.000000in}{0.000000in}}{%
\pgfpathmoveto{\pgfqpoint{0.000000in}{0.000000in}}%
\pgfpathlineto{\pgfqpoint{-0.048611in}{0.000000in}}%
\pgfusepath{stroke,fill}%
}%
\begin{pgfscope}%
\pgfsys@transformshift{0.898750in}{4.728029in}%
\pgfsys@useobject{currentmarker}{}%
\end{pgfscope}%
\end{pgfscope}%
\begin{pgfscope}%
\pgftext[x=0.624058in,y=4.675268in,left,base]{\sffamily\fontsize{10.000000}{12.000000}\selectfont \(\displaystyle 6.4\)}%
\end{pgfscope}%
\begin{pgfscope}%
\pgftext[x=0.568502in,y=2.801700in,,bottom,rotate=90.000000]{\sffamily\fontsize{14.000000}{16.800000}\selectfont seconds}%
\end{pgfscope}%
\begin{pgfscope}%
\pgfpathrectangle{\pgfqpoint{0.898750in}{0.622600in}}{\pgfqpoint{5.572250in}{4.358200in}}%
\pgfusepath{clip}%
\pgfsetrectcap%
\pgfsetroundjoin%
\pgfsetlinewidth{1.505625pt}%
\definecolor{currentstroke}{rgb}{0.172549,0.243137,0.313725}%
\pgfsetstrokecolor{currentstroke}%
\pgfsetdash{}{0pt}%
\pgfpathmoveto{\pgfqpoint{1.152034in}{0.850373in}}%
\pgfpathlineto{\pgfqpoint{1.254371in}{0.889587in}}%
\pgfpathlineto{\pgfqpoint{1.356708in}{0.945976in}}%
\pgfpathlineto{\pgfqpoint{1.459045in}{0.986813in}}%
\pgfpathlineto{\pgfqpoint{1.561382in}{1.052266in}}%
\pgfpathlineto{\pgfqpoint{1.663719in}{1.146438in}}%
\pgfpathlineto{\pgfqpoint{1.766056in}{1.240228in}}%
\pgfpathlineto{\pgfqpoint{1.868393in}{1.301483in}}%
\pgfpathlineto{\pgfqpoint{1.970730in}{1.399854in}}%
\pgfpathlineto{\pgfqpoint{2.073067in}{1.393270in}}%
\pgfpathlineto{\pgfqpoint{2.175404in}{1.530568in}}%
\pgfpathlineto{\pgfqpoint{2.277741in}{1.577227in}}%
\pgfpathlineto{\pgfqpoint{2.380078in}{1.614147in}}%
\pgfpathlineto{\pgfqpoint{2.482415in}{1.883310in}}%
\pgfpathlineto{\pgfqpoint{2.584752in}{1.811560in}}%
\pgfpathlineto{\pgfqpoint{2.687089in}{1.867947in}}%
\pgfpathlineto{\pgfqpoint{2.789426in}{2.043890in}}%
\pgfpathlineto{\pgfqpoint{2.891763in}{1.939599in}}%
\pgfpathlineto{\pgfqpoint{2.994100in}{2.170311in}}%
\pgfpathlineto{\pgfqpoint{3.096437in}{2.116880in}}%
\pgfpathlineto{\pgfqpoint{3.198774in}{2.226032in}}%
\pgfpathlineto{\pgfqpoint{3.301111in}{2.059248in}}%
\pgfpathlineto{\pgfqpoint{3.403448in}{2.423726in}}%
\pgfpathlineto{\pgfqpoint{3.505785in}{2.266578in}}%
\pgfpathlineto{\pgfqpoint{3.608122in}{2.830089in}}%
\pgfpathlineto{\pgfqpoint{3.710459in}{2.716351in}}%
\pgfpathlineto{\pgfqpoint{3.812796in}{2.659964in}}%
\pgfpathlineto{\pgfqpoint{3.915133in}{2.796595in}}%
\pgfpathlineto{\pgfqpoint{4.017470in}{2.713017in}}%
\pgfpathlineto{\pgfqpoint{4.119807in}{2.766252in}}%
\pgfpathlineto{\pgfqpoint{4.222144in}{2.888380in}}%
\pgfpathlineto{\pgfqpoint{4.324481in}{3.087125in}}%
\pgfpathlineto{\pgfqpoint{4.426818in}{2.771980in}}%
\pgfpathlineto{\pgfqpoint{4.529155in}{3.188168in}}%
\pgfpathlineto{\pgfqpoint{4.631492in}{3.557989in}}%
\pgfpathlineto{\pgfqpoint{4.733829in}{3.660558in}}%
\pgfpathlineto{\pgfqpoint{4.836166in}{3.384238in}}%
\pgfpathlineto{\pgfqpoint{4.938503in}{3.342928in}}%
\pgfpathlineto{\pgfqpoint{5.040840in}{3.681735in}}%
\pgfpathlineto{\pgfqpoint{5.143177in}{3.444062in}}%
\pgfpathlineto{\pgfqpoint{5.245514in}{3.651112in}}%
\pgfpathlineto{\pgfqpoint{5.347851in}{3.691182in}}%
\pgfpathlineto{\pgfqpoint{5.450188in}{3.869318in}}%
\pgfpathlineto{\pgfqpoint{5.552525in}{4.210225in}}%
\pgfpathlineto{\pgfqpoint{5.654862in}{4.269952in}}%
\pgfpathlineto{\pgfqpoint{5.757199in}{4.270428in}}%
\pgfpathlineto{\pgfqpoint{5.859536in}{4.345427in}}%
\pgfpathlineto{\pgfqpoint{5.961873in}{4.350483in}}%
\pgfpathlineto{\pgfqpoint{6.064210in}{4.523087in}}%
\pgfpathlineto{\pgfqpoint{6.166547in}{4.687098in}}%
\pgfpathlineto{\pgfqpoint{6.217716in}{4.782700in}}%
\pgfusepath{stroke}%
\end{pgfscope}%
\begin{pgfscope}%
\pgfpathrectangle{\pgfqpoint{0.898750in}{0.622600in}}{\pgfqpoint{5.572250in}{4.358200in}}%
\pgfusepath{clip}%
\pgfsetrectcap%
\pgfsetroundjoin%
\pgfsetlinewidth{1.505625pt}%
\definecolor{currentstroke}{rgb}{0.905882,0.298039,0.235294}%
\pgfsetstrokecolor{currentstroke}%
\pgfsetdash{}{0pt}%
\pgfpathmoveto{\pgfqpoint{1.152034in}{0.856575in}}%
\pgfpathlineto{\pgfqpoint{1.254371in}{0.929852in}}%
\pgfpathlineto{\pgfqpoint{1.356708in}{1.003129in}}%
\pgfpathlineto{\pgfqpoint{1.459045in}{1.076405in}}%
\pgfpathlineto{\pgfqpoint{1.561382in}{1.149682in}}%
\pgfpathlineto{\pgfqpoint{1.663719in}{1.222959in}}%
\pgfpathlineto{\pgfqpoint{1.766056in}{1.296236in}}%
\pgfpathlineto{\pgfqpoint{1.868393in}{1.369512in}}%
\pgfpathlineto{\pgfqpoint{1.970730in}{1.442789in}}%
\pgfpathlineto{\pgfqpoint{2.073067in}{1.516066in}}%
\pgfpathlineto{\pgfqpoint{2.175404in}{1.589342in}}%
\pgfpathlineto{\pgfqpoint{2.277741in}{1.662619in}}%
\pgfpathlineto{\pgfqpoint{2.380078in}{1.735896in}}%
\pgfpathlineto{\pgfqpoint{2.482415in}{1.809173in}}%
\pgfpathlineto{\pgfqpoint{2.584752in}{1.882449in}}%
\pgfpathlineto{\pgfqpoint{2.687089in}{1.955726in}}%
\pgfpathlineto{\pgfqpoint{2.789426in}{2.029003in}}%
\pgfpathlineto{\pgfqpoint{2.891763in}{2.102280in}}%
\pgfpathlineto{\pgfqpoint{2.994100in}{2.175556in}}%
\pgfpathlineto{\pgfqpoint{3.096437in}{2.248833in}}%
\pgfpathlineto{\pgfqpoint{3.198774in}{2.322110in}}%
\pgfpathlineto{\pgfqpoint{3.301111in}{2.395387in}}%
\pgfpathlineto{\pgfqpoint{3.403448in}{2.468663in}}%
\pgfpathlineto{\pgfqpoint{3.505785in}{2.541940in}}%
\pgfpathlineto{\pgfqpoint{3.608122in}{2.615217in}}%
\pgfpathlineto{\pgfqpoint{3.710459in}{2.688494in}}%
\pgfpathlineto{\pgfqpoint{3.812796in}{2.761770in}}%
\pgfpathlineto{\pgfqpoint{3.915133in}{2.835047in}}%
\pgfpathlineto{\pgfqpoint{4.017470in}{2.908324in}}%
\pgfpathlineto{\pgfqpoint{4.119807in}{2.981601in}}%
\pgfpathlineto{\pgfqpoint{4.222144in}{3.054877in}}%
\pgfpathlineto{\pgfqpoint{4.324481in}{3.128154in}}%
\pgfpathlineto{\pgfqpoint{4.426818in}{3.201431in}}%
\pgfpathlineto{\pgfqpoint{4.529155in}{3.274707in}}%
\pgfpathlineto{\pgfqpoint{4.631492in}{3.347984in}}%
\pgfpathlineto{\pgfqpoint{4.733829in}{3.421261in}}%
\pgfpathlineto{\pgfqpoint{4.836166in}{3.494538in}}%
\pgfpathlineto{\pgfqpoint{4.938503in}{3.567814in}}%
\pgfpathlineto{\pgfqpoint{5.040840in}{3.641091in}}%
\pgfpathlineto{\pgfqpoint{5.143177in}{3.714368in}}%
\pgfpathlineto{\pgfqpoint{5.245514in}{3.787645in}}%
\pgfpathlineto{\pgfqpoint{5.347851in}{3.860921in}}%
\pgfpathlineto{\pgfqpoint{5.450188in}{3.934198in}}%
\pgfpathlineto{\pgfqpoint{5.552525in}{4.007475in}}%
\pgfpathlineto{\pgfqpoint{5.654862in}{4.080752in}}%
\pgfpathlineto{\pgfqpoint{5.757199in}{4.154028in}}%
\pgfpathlineto{\pgfqpoint{5.859536in}{4.227305in}}%
\pgfpathlineto{\pgfqpoint{5.961873in}{4.300582in}}%
\pgfpathlineto{\pgfqpoint{6.064210in}{4.373859in}}%
\pgfpathlineto{\pgfqpoint{6.166547in}{4.447135in}}%
\pgfpathlineto{\pgfqpoint{6.217716in}{4.483774in}}%
\pgfusepath{stroke}%
\end{pgfscope}%
\begin{pgfscope}%
\pgfpathrectangle{\pgfqpoint{0.898750in}{0.622600in}}{\pgfqpoint{5.572250in}{4.358200in}}%
\pgfusepath{clip}%
\pgfsetrectcap%
\pgfsetroundjoin%
\pgfsetlinewidth{1.505625pt}%
\definecolor{currentstroke}{rgb}{0.086275,0.627451,0.521569}%
\pgfsetstrokecolor{currentstroke}%
\pgfsetdash{}{0pt}%
\pgfpathmoveto{\pgfqpoint{1.152034in}{0.820700in}}%
\pgfpathlineto{\pgfqpoint{1.254371in}{0.821463in}}%
\pgfpathlineto{\pgfqpoint{1.356708in}{0.822513in}}%
\pgfpathlineto{\pgfqpoint{1.459045in}{0.822894in}}%
\pgfpathlineto{\pgfqpoint{1.561382in}{0.822990in}}%
\pgfpathlineto{\pgfqpoint{1.663719in}{0.824803in}}%
\pgfpathlineto{\pgfqpoint{1.766056in}{0.826806in}}%
\pgfpathlineto{\pgfqpoint{1.868393in}{0.826425in}}%
\pgfpathlineto{\pgfqpoint{1.970730in}{0.827188in}}%
\pgfpathlineto{\pgfqpoint{2.073067in}{0.827856in}}%
\pgfpathlineto{\pgfqpoint{2.175404in}{0.828905in}}%
\pgfpathlineto{\pgfqpoint{2.277741in}{0.829573in}}%
\pgfpathlineto{\pgfqpoint{2.380078in}{0.830432in}}%
\pgfpathlineto{\pgfqpoint{2.482415in}{0.832722in}}%
\pgfpathlineto{\pgfqpoint{2.584752in}{0.832722in}}%
\pgfpathlineto{\pgfqpoint{2.687089in}{0.832054in}}%
\pgfpathlineto{\pgfqpoint{2.789426in}{0.834630in}}%
\pgfpathlineto{\pgfqpoint{2.891763in}{0.832913in}}%
\pgfpathlineto{\pgfqpoint{2.994100in}{0.835489in}}%
\pgfpathlineto{\pgfqpoint{3.096437in}{0.835107in}}%
\pgfpathlineto{\pgfqpoint{3.198774in}{0.836348in}}%
\pgfpathlineto{\pgfqpoint{3.301111in}{0.835012in}}%
\pgfpathlineto{\pgfqpoint{3.403448in}{0.838828in}}%
\pgfpathlineto{\pgfqpoint{3.505785in}{0.835871in}}%
\pgfpathlineto{\pgfqpoint{3.608122in}{0.843981in}}%
\pgfpathlineto{\pgfqpoint{3.710459in}{0.841309in}}%
\pgfpathlineto{\pgfqpoint{3.812796in}{0.839973in}}%
\pgfpathlineto{\pgfqpoint{3.915133in}{0.841214in}}%
\pgfpathlineto{\pgfqpoint{4.017470in}{0.840737in}}%
\pgfpathlineto{\pgfqpoint{4.119807in}{0.841214in}}%
\pgfpathlineto{\pgfqpoint{4.222144in}{0.842454in}}%
\pgfpathlineto{\pgfqpoint{4.324481in}{0.844362in}}%
\pgfpathlineto{\pgfqpoint{4.426818in}{0.840927in}}%
\pgfpathlineto{\pgfqpoint{4.529155in}{0.845126in}}%
\pgfpathlineto{\pgfqpoint{4.631492in}{0.848847in}}%
\pgfpathlineto{\pgfqpoint{4.733829in}{0.851232in}}%
\pgfpathlineto{\pgfqpoint{4.836166in}{0.848274in}}%
\pgfpathlineto{\pgfqpoint{4.938503in}{0.847415in}}%
\pgfpathlineto{\pgfqpoint{5.040840in}{0.850469in}}%
\pgfpathlineto{\pgfqpoint{5.143177in}{0.848274in}}%
\pgfpathlineto{\pgfqpoint{5.245514in}{0.850564in}}%
\pgfpathlineto{\pgfqpoint{5.347851in}{0.850278in}}%
\pgfpathlineto{\pgfqpoint{5.450188in}{0.851423in}}%
\pgfpathlineto{\pgfqpoint{5.552525in}{0.855335in}}%
\pgfpathlineto{\pgfqpoint{5.654862in}{0.856480in}}%
\pgfpathlineto{\pgfqpoint{5.757199in}{0.855430in}}%
\pgfpathlineto{\pgfqpoint{5.859536in}{0.855812in}}%
\pgfpathlineto{\pgfqpoint{5.961873in}{0.857911in}}%
\pgfpathlineto{\pgfqpoint{6.064210in}{0.858197in}}%
\pgfpathlineto{\pgfqpoint{6.166547in}{0.860105in}}%
\pgfpathlineto{\pgfqpoint{6.217716in}{0.859915in}}%
\pgfusepath{stroke}%
\end{pgfscope}%
\begin{pgfscope}%
\pgfsetrectcap%
\pgfsetmiterjoin%
\pgfsetlinewidth{0.803000pt}%
\definecolor{currentstroke}{rgb}{0.000000,0.000000,0.000000}%
\pgfsetstrokecolor{currentstroke}%
\pgfsetdash{}{0pt}%
\pgfpathmoveto{\pgfqpoint{0.898750in}{0.622600in}}%
\pgfpathlineto{\pgfqpoint{0.898750in}{4.980800in}}%
\pgfusepath{stroke}%
\end{pgfscope}%
\begin{pgfscope}%
\pgfsetrectcap%
\pgfsetmiterjoin%
\pgfsetlinewidth{0.803000pt}%
\definecolor{currentstroke}{rgb}{0.000000,0.000000,0.000000}%
\pgfsetstrokecolor{currentstroke}%
\pgfsetdash{}{0pt}%
\pgfpathmoveto{\pgfqpoint{6.471000in}{0.622600in}}%
\pgfpathlineto{\pgfqpoint{6.471000in}{4.980800in}}%
\pgfusepath{stroke}%
\end{pgfscope}%
\begin{pgfscope}%
\pgfsetrectcap%
\pgfsetmiterjoin%
\pgfsetlinewidth{0.803000pt}%
\definecolor{currentstroke}{rgb}{0.000000,0.000000,0.000000}%
\pgfsetstrokecolor{currentstroke}%
\pgfsetdash{}{0pt}%
\pgfpathmoveto{\pgfqpoint{0.898750in}{0.622600in}}%
\pgfpathlineto{\pgfqpoint{6.471000in}{0.622600in}}%
\pgfusepath{stroke}%
\end{pgfscope}%
\begin{pgfscope}%
\pgfsetrectcap%
\pgfsetmiterjoin%
\pgfsetlinewidth{0.803000pt}%
\definecolor{currentstroke}{rgb}{0.000000,0.000000,0.000000}%
\pgfsetstrokecolor{currentstroke}%
\pgfsetdash{}{0pt}%
\pgfpathmoveto{\pgfqpoint{0.898750in}{4.980800in}}%
\pgfpathlineto{\pgfqpoint{6.471000in}{4.980800in}}%
\pgfusepath{stroke}%
\end{pgfscope}%
\begin{pgfscope}%
\pgfsetbuttcap%
\pgfsetmiterjoin%
\definecolor{currentfill}{rgb}{1.000000,1.000000,1.000000}%
\pgfsetfillcolor{currentfill}%
\pgfsetfillopacity{0.800000}%
\pgfsetlinewidth{1.003750pt}%
\definecolor{currentstroke}{rgb}{0.800000,0.800000,0.800000}%
\pgfsetstrokecolor{currentstroke}%
\pgfsetstrokeopacity{0.800000}%
\pgfsetdash{}{0pt}%
\pgfpathmoveto{\pgfqpoint{0.995972in}{4.258117in}}%
\pgfpathlineto{\pgfqpoint{2.929539in}{4.258117in}}%
\pgfpathquadraticcurveto{\pgfqpoint{2.957317in}{4.258117in}}{\pgfqpoint{2.957317in}{4.285895in}}%
\pgfpathlineto{\pgfqpoint{2.957317in}{4.883578in}}%
\pgfpathquadraticcurveto{\pgfqpoint{2.957317in}{4.911356in}}{\pgfqpoint{2.929539in}{4.911356in}}%
\pgfpathlineto{\pgfqpoint{0.995972in}{4.911356in}}%
\pgfpathquadraticcurveto{\pgfqpoint{0.968194in}{4.911356in}}{\pgfqpoint{0.968194in}{4.883578in}}%
\pgfpathlineto{\pgfqpoint{0.968194in}{4.285895in}}%
\pgfpathquadraticcurveto{\pgfqpoint{0.968194in}{4.258117in}}{\pgfqpoint{0.995972in}{4.258117in}}%
\pgfpathclose%
\pgfusepath{stroke,fill}%
\end{pgfscope}%
\begin{pgfscope}%
\pgfsetrectcap%
\pgfsetroundjoin%
\pgfsetlinewidth{1.505625pt}%
\definecolor{currentstroke}{rgb}{0.172549,0.243137,0.313725}%
\pgfsetstrokecolor{currentstroke}%
\pgfsetdash{}{0pt}%
\pgfpathmoveto{\pgfqpoint{1.023750in}{4.798888in}}%
\pgfpathlineto{\pgfqpoint{1.301528in}{4.798888in}}%
\pgfusepath{stroke}%
\end{pgfscope}%
\begin{pgfscope}%
\pgftext[x=1.412639in,y=4.750277in,left,base]{\sffamily\fontsize{10.000000}{12.000000}\selectfont \textsc{Linclosure}}%
\end{pgfscope}%
\begin{pgfscope}%
\pgfsetrectcap%
\pgfsetroundjoin%
\pgfsetlinewidth{1.505625pt}%
\definecolor{currentstroke}{rgb}{0.905882,0.298039,0.235294}%
\pgfsetstrokecolor{currentstroke}%
\pgfsetdash{}{0pt}%
\pgfpathmoveto{\pgfqpoint{1.023750in}{4.595031in}}%
\pgfpathlineto{\pgfqpoint{1.301528in}{4.595031in}}%
\pgfusepath{stroke}%
\end{pgfscope}%
\begin{pgfscope}%
\pgftext[x=1.412639in,y=4.546420in,left,base]{\sffamily\fontsize{10.000000}{12.000000}\selectfont Linear Approximation}%
\end{pgfscope}%
\begin{pgfscope}%
\pgfsetrectcap%
\pgfsetroundjoin%
\pgfsetlinewidth{1.505625pt}%
\definecolor{currentstroke}{rgb}{0.086275,0.627451,0.521569}%
\pgfsetstrokecolor{currentstroke}%
\pgfsetdash{}{0pt}%
\pgfpathmoveto{\pgfqpoint{1.023750in}{4.391174in}}%
\pgfpathlineto{\pgfqpoint{1.301528in}{4.391174in}}%
\pgfusepath{stroke}%
\end{pgfscope}%
\begin{pgfscope}%
\pgftext[x=1.412639in,y=4.342562in,left,base]{\sffamily\fontsize{10.000000}{12.000000}\selectfont \textsc{Closure}}%
\end{pgfscope}%
\end{pgfpicture}%
\makeatother%
\endgroup%
}
}
\end{minipage}


\begin{minipage}{0.5\textwidth}
	\centering
\subfloat[Some landmarks times, $|\Sg| = 100$]{
	\begin{tabular}{| c || c | c |}
		\hline \rowcolor{clouds}
		$|\B|$ & \textsc{LinClosure} & \textsc{Closure} \\ \hline
		1000  & 0.494  & 0.019 \\ \hline
		4000  & 7.387  & 0.237 \\ \hline
		7000  & 23.460 & 0.673 \\ \hline 
		10000 & 44.054 & 1.252 \\ \hline 
	\end{tabular}
}
\end{minipage}
~
\begin{minipage}{0.5\textwidth}
	\centering
\subfloat[Some landmarks times, $|\B| = 100$]{
	\begin{tabular}{| c || c | c |}
		\hline \rowcolor{clouds}
		$|\Sg|$ & \textsc{Closure} \\ \hline
		1000   & 0.001 \\ \hline
		5000   & 0.004 \\ \hline
		25000  & 0.017 \\ \hline 
		45000  & 0.031 \\ \hline 
		65000  & 0.034 \\ \hline
		85000  & 0.052 \\ \hline
		100000 & 0.065 \\ \hline
	\end{tabular}
}
\end{minipage}

\caption{Comparison of closure operators for \textsc{BercziMinimization}}
\label{fig:Berczi-Prune}
\end{figure}

\vspace{1.2em}

Regarding the evolution of time when $|\Sg|$ is fixed we denote a highly chaotic
behaviour. However, we may perceive a polynomial growth of the time required by
\textsc{BercziMinimization} when used with \textsc{LinClosure} matching theoretical expectations. If we keep on throwing an eye on previous results,
it seems like \textsc{BercziMinimization} performs better than \textsc{MinCover}
for some values, notably in the table, for $|\B| = 10000$ and $15000$. In fact,
those values can somehow be considered as outliers in the graph we gave (they are convexities in the curve). Because the mean measure is sensitive to outlier
basis, we can suppose that in those particular cases we generated some highly redundant basis decreasing the cost of operations. As we will see with real datasets in joint comparison, the difference between \textsc{MinCover} and \textsc{BercziMinimization} is clearly in favour of \textsc{MinCover}. Again, the difference between \textsc{Closure} and \textsc{LinClosure} is clearly visible and we can take \textsc{Closure} as our fastest closure operator.

\subsection{\textsc{AFPMinimization}}


\vspace{1.2em}

During all this section we have been testing our minimization procedures with two closure algorithms: \textsc{Closure} and \textsc{LinCLosure}. From all of
our tests it seemed like \textsc{LinClosure} performs much worse than the 
other operator. This is in accordance with conclusions and hypothesis drawn in
\cite{bazhanov_optimizations_2014} about efficiency of \textsc{LinClosure}. However, those tests allowed us to have a glimpse of the correspondence between
theoretical and practical complexity. Still, we may be able to find various more combinations of closure operations to test to see which one is the best. For us, in the meantime of the master thesis and regarding the performance of \textsc{LinClosure} it seemed that our tests were sufficient. Now that we found out which version of each algorithm was the fastest, we can move forward to 
further tests and joint comparison.


\section{Joint comparison}

In this section we consider the algorithms pruned previously and we compare them
with the real datasets we exposed in the first section of this chapter (see table \ref{tab:real-DS}). For the recall they are datasets taken from the UCI 
repository. For all of those 28 basis, we ran the 5 minimization procedure
we reviewed and implemented. The value recorded is the execution time in seconds
for minimizing the given basis. See the results in \ref{tab:real-exe}.

\vspace{1.2em}

First, one can observe the wide range of performances. On the one hand,
\textsc{MinCover} and \textsc{DuquenneMinimization} do not exceed 100 seconds of execution time, while \textsc{BercziMinimization} and \textsc{AFPMinimization} can require up to several hours. The most striking
case is for Breast Wisconsin case, with proper and generators basis. This is
probably because of the number of implications in those cases. Anyway, while
the two first algorithms require about 30 to 90 seconds for minimization, while
\textsc{BercziMinimization} needs roughly one hour and \textsc{AFP} up to ten or eleven hours. This stresses on the intractability of \textsc{AFP} in practice, and possibly the Angluin algorithm when we are prevented from the speed of an oracle.

\vspace{1.2em}

Regarding the difference between \textsc{MinCover} and \textsc{DuquenneMinimization}, observe that in general, the algorithm by Duquenne is faster on non-minimal basis and slightly slower on already minimal ones. In both cases, one hypothesis to explain those gaps could be the second
loop of \textsc{DuquenneMinimization} allowing for a limitation of closure computations. As we already discussed, the number of closure computations in 
\textsc{MinCover} is likely to be constant whatever the case is, even though
those operations become lighter with the reduction of the input basis. In \textsc{DuquenneMinimization} however, apart from first left-saturation and
redundancy elimination allowing for some savings in closure computations, we stop an iteration of the second loop whenever some quasi-closed premise is not
pseudo-closed. This break point permits to omit closure computations and in practice, thanks to lectic ordering, we may not need to go over all implications of the output basis to check whether an implication is redundant or not. The only case when the second loop should require more computations is when the base is already minimal, because for each step of the second loop, we have to go over all implications of the growing output basis. As remarked in most of execution times, when the basis is already minimal \textsc{DuquenneMinimization} performs somehow slighter worse than 
\textsc{MinCover}, which could be explained by the previous hypothesis.

\begin{table}[ht]
	\centering
\begin{tabular}{| c | c || c | c | c | c | c |}
	\hline \rowcolor{clouds}
	\multicolumn{2}{c}{$\I$} & \textsc{MinCover} & \textsc{Duquenne} & \textsc{Maier} & \textsc{Berczi} & \textsc{AFP} \\ \hline
	
	\multirow{4}{*}{Zoo} & DQ & < 0.001 & < 0.001 & < 0.001 & < 0.001 & 0.016 \\
	& min & < 0.001 & < 0.001 & 0.015 & < 0.001 & 0.016 \\
	& proper & 0.007 & < 0.001 & < 0.001 & 0.016 & 0.063 \\
	& mingen & 0.009 & 0.016 & < 0.001 & 0.047 & 0.094 \\ \hline
	
	\multirow{4}{*}{Flare} & DQ & 0.110 & 0.203 & 0.328 & 27.922 & 115.656 \\
	& min & 0.162 & 0.219 & 0.406 & 27.750 & 116.594 \\
	& proper & 2.166 & 0.953 & 1.250 & 88.375 & 524.031 \\
	& mingen & 19.382 & 9.906 & 11.344 & 160.328 & 2810.620 \\ \hline
	
	\multirow{4}{*}{Breast Cancer} & DQ & 0.120 & 0.140 & 0.234 & 33.047 & 90.031 \\
	& min & 0.147 & 0.156 & 0.297 & 26.578 & 89.516 \\
	& proper & 2.712 & 1.031 & 1.219 & 93.266 & 429.844 \\
	& mingen & 2.850 & 1.578 & 1.703 & 102.562 & 598.172 \\ \hline
	
	\multirow{4}{*}{Breast Wisconsin} & DQ & 1.333 & 1.703 & 2.8125 & 1005.750 & 3109.920 \\
	& min & 1.801 & 2.016 & 3.531 & 949.953 & 3140.940 \\
	& proper & 88.188 & 33.688 & 46.312 & 3675.910 & 40521.0 \\
	& mingen & 51.378 & 32.859 & 33.843 & 2772.980 & 38310.200 \\ \hline
	
	\multirow{4}{*}{Operative} & DQ & 0.004 & < 0.001 & 0.015 & 0.219 & 0.734 \\
	& min & 0.005 & < 0.001 & 0.031 & 0.219 & 0.719 \\
	& proper & 0.048 & 0.016 & 0.016 & 0.594 & 2.422 \\
	& mingen & 0.085 & 0.063 & 0.109 & 0.813 & 4.063 \\ \hline
	
	\multirow{4}{*}{SPECT} & DQ & 0.040 & 0.078 & 0.141 & 10.328 & 23.609 \\
	& min & 0.055 & 0.094 & 0.203 & 8.156 & 22.906 \\
	& proper & 1.270 & 0.484 & 0.531 & 51.063 & 118.531 \\
	& mingen & 32.580 & 13.375 & 13.609 & 194.875 & 930.578 \\ \hline
	
	\multirow{4}{*}{Vote} & DQ & 0.007 & 0.016 & 0.032 & 0.484 & 1.579 \\ 
	& min & 0.009 & 0.016 & 0.047 & 0.469 & 1.516 \\
	& proper & 0.090 & 0.047 & 0.078 & 1.625 & 6.203 \\
	& mingen & 0.682 & 0.313 & 0.485 & 4.109 & 22.875 \\ \hline
	
	
\end{tabular} 
\caption{Comparison of the algorithms on real datasets (execution in $s$)}
\label{tab:real-exe}
\end{table}

\vspace{1.2em}

One can also denote that \textsc{MaierMinimization} seems to be placed in between \textsc{DuquenneMinimization} and \textsc{MinCover} except in the minimal cases. Our hypothesis is the same as for the difference between 
\textsc{DuquenneMinimization} and \textsc{MinCover}: in fact, the first loop of Maier's algorithm is to get rid of redundant implications, therefore when a 
basis is highly redundant, first removing them allow to spare numerous closure
computations and reduces closure cost. This is not the case in \textsc{MinCover}
where we compute right closure of all implications. Still, the rest of the algorithm seems to be slower than the second loop of \textsc{DuquenneMinimization}. This may be because to determine equivalence classes and removing direct determinations we need to use closure operators
twice for each implications, while with the algorithm by Duquenne, we only compute closure when it is needed after an increasing number of set operations though. The interesting point which seems to go in the direction of our explanation is the behaviour under already minimal basis: Maier's algorithm requires an asymptotically higher number of closure computations than both 
\textsc{DuquenneMinimization} and \textsc{MinCover} making it the heaviest 
procedure when it comes at minimizing a basis already minimal.

\vspace{1.2em}

We would like to remind that this explanation holds within the scope of our
tests. We cannot assume it in general and it should be tested in further work on other datasets. Let us say this is a valid hypothesis. In this case, one
should be first worried about removing as much  redundant implications as possible to lighten the burden of subsequent closure computations, whether it is for getting the canonical basis or not.


\section{Perspectives and difficulties}

So far we have developed and explained our work during this master thesis. However, there are two points we should mention at last being in a sense out of 
the scope of the algorithms, namely difficulties we encountered and perspectives
our study leave for future work. 

\vspace{1.2em}

First, let us discuss difficulties. Apart from having to deal with various mathematical frameworks we concentrated to closure systems, some algorithms
or articles appeared to be very demanding. This the case for instance with the work of Ausiello and al. which took us about one month of work before concluding to a potential mistake. Often, those technical difficulties to understand some articles as much as knowing whether to study or not some frameworks found solutions in discussions with the academic supervisor. Tests have also been a problem since most of them are time-consuming (up to several days) relatively to the time we had to work on this thesis. Even though tests were performed in parallel with other work, having only one computer slows down tests because of side tasks. Still, we got some results allowing for hypothesis
on the behaviour of several algorithms leading to hints and tracks to follow for future work.

\vspace{1.2em}

Speaking about future work, let us talk about perspectives. We provided both theoretical and practical review. The two approaches benefit from possible improvement or in-depth look. First on a theoretical point of view, the algorithm \textsc{AFPMinimization} derived from query learning still requires
a proof of correctness, even if we observed a dreadful efficiency in practice. Furthermore, we may still investigate Ausiello to sharpen our insight of redundancy elimination and removing superfluous nodes. An interesting question
drawing a link between theoretical and practical study would be to get more
understanding of redundancy so as to regulate its generation under testing and
improving algorithms by pointing out special kinds of redundancies. Talking about testing, the main perspective on a practical side would be first to run tests on new real datasets to see whether our hypothesis hold or not. One may also be interested in testing more combinations of closure algorithms or finding
more optimizations in each minimization procedure.


\paragraph{Conclusion}



% Conclusion
\input{Chapter/Conclusion.tex}
\addcontentsline{toc}{chapter}{Conclusion}

In this master thesis, we were interested in the problem of implication theories minimization which is: having a system of implications, trying to remove as much elements as possible without altering the knowledge described by
the system. More precisely, we had to review existing algorithms and implement them to see how do they perform under practical computations.

\vspace{1.2em}

To answer this problematic, we first tried to find in literature existing algorithms matching our minimization task. Then we brought to the field of closure spaces and implications some algorithms to study them. In this review, we gave elements of proof and complexity analysis. Because of some difficulties, errors or similarities in algorithms we found, we can still improve our insight on some of them as a perspective of our study.

\vspace{1.2em}

Consequently to this review, and using a code previously developed for FCA and closure computing purposes, we implemented the algorithms we worked on in C++. Next, after setting up a testing framework, we ran tests on implemented procedure. Experiments gave results on the behaviour of algorithms
and their efficiency in practice. We exhibited  possible improvements in the implementation of each procedure and accordingly, difference of speed
for closure operators. We also depicted some practical conclusions on which algorithm to use depending on the case or at least, some step to consider when trying to think of another way to minimize systems with a closure-based approach.

\vspace{1.2em}

Finally, our work opens some possibilities for future research and experiments. One could for instance be interested in studying a better way to handle the structure (notably in terms of redundancy) of randomly generated data as compared to real life datasets. More generally, we could be interested in extending our tests to larger datasets and a broader range of improvements for closure operators to strengthen our hypothesis.




\newpage
\pagenumbering{roman}
\setcounter{page}{8}

\bibliographystyle{acm}
	\bibliography{Biblio.bib}

\newpage



\end{document}